%!TEX TS-program = xelatex
%!TEX encoding = UTF-8 Unicode
%\TeXXeTstate=1
\documentclass[12pt,a4paper]{book}
\usepackage{fontspec}

\defaultfontfeatures{Mapping=tex-text}
\setromanfont[Scale=1]{Palatino}
\setmonofont[Scale=0.8]{Monaco}

\usepackage{parskip}
\begin{document}
\sloppy
\raggedbottom
\newcommand{\mdsh}[1]{\mbox{#1}\linebreak[1]}
\pagestyle{empty}
\newcommand{\nodate}{\date{}}\nodate
\title{House of Mirth}
\author{Edith Wharton}
\maketitle
\frontmatter
\pagestyle{myheadings}
\newpage

\mainmatter
\chapter*{\raggedright Chapter 1}

\addcontentsline{toc}{chapter}{Chapter 1}

\markboth{House of Mirth}{Chapter 1}





Selden paused in surprise. In the afternoon rush of the Grand
Central Station his eyes had been refreshed by the sight of Miss
Lily Bart.





It was a Monday in early September, and he was returning to his
work from a hurried dip into the country; but what was Miss Bart
doing in town at that season? If she had appeared to be catching
a train, he might have inferred that he had come on her in the
act of transition between one and another of the country-houses
which disputed her presence after the close of the Newport
season; but her desultory air perplexed him. She stood apart from
the crowd, letting it drift by her to the platform or the street,
and wearing an air of irresolution which might, as he surmised,
be the mask of a very definite purpose. It struck him at once
that she was waiting for some one, but he hardly knew why the
idea arrested him. There was nothing new about Lily Bart, yet he
could never see her without a faint movement of interest: it was
characteristic of her that she always roused speculation, that
her simplest acts seemed the result of far-reaching intentions.





An impulse of curiosity made him turn out of his direct line to
the door, and stroll past her. He knew that if she did not wish
to be seen she would contrive to elude him; and it amused him to
think of putting her skill to the test.





``Mr.\ Selden---what good luck!''





She came forward smiling, eager almost, in her resolve to
intercept him. One or two persons, in brushing past them,
lingered to look; for Miss Bart was a figure to arrest even the
suburban traveller rushing to his last train.





Selden had never seen her more radiant. Her vivid head, relieved
against the dull tints of the crowd, made her more conspicuous
than in a ball-room, and under her dark hat and veil she regained
the girlish smoothness, the purity of tint, that she was
beginning to lose after eleven years of late hours and
indefatigable dancing. Was it really eleven years, Selden found
himself wondering, and had she indeed reached the
nine-and-twentieth birthday with which her rivals credited her?





``What luck!''\ she repeated. ``How nice of you to come to my
rescue!''





He responded joyfully that to do so was his mission in life, and
asked what form the rescue was to take.





``Oh, almost any---even to sitting on a bench and talking to me. 
One sits out a cotillion---why not sit out a train? It isn't a bit
hotter here than in Mrs.\ Van Osburgh's conservatory---and some of
the women are not a bit uglier.''  She broke off, laughing, to
explain that she had come up to town from Tuxedo, on her way to
the Gus Trenors'\ at Bellomont, and had missed the three-fifteen
train to Rhinebeck.  ``And there isn't another till half-past
five.'' She consulted the little jewelled watch among her laces. 
``Just two hours to wait. And I don't know what to do with myself. 
My maid came up this morning to do some shopping for me, and was
to go on to Bellomont at one o'clock, and my aunt's house is
closed, and I don't know a soul in town.'' She glanced plaintively
about the station. ``It \textit{is} hotter than Mrs.\ Van Osburgh's, after
all. If you can spare the time, do take me somewhere for a breath
of air.''





He declared himself entirely at her disposal: the adventure
struck him as diverting. As a spectator, he had always enjoyed
Lily Bart; and his course lay so far out of her orbit that it
amused him to be drawn for a moment into the sudden intimacy
which her proposal implied.





``Shall we go over to Sherry's for a cup of tea?''





She smiled assentingly, and then made a slight grimace.





``So many people come up to town on a Monday---one is sure to meet
a lot of bores. I'm as old as the hills, of course, and it ought
not to make any difference; but if \textit{I'm} old enough, you're not,''
she objected gaily. ``I'm dying for tea---but isn't there a quieter
place?''





He answered her smile, which rested on him vividly. Her
discretions interested him almost as much as her imprudences: he
was so sure that both were part of the same carefully-elaborated
plan. In judging Miss Bart, he had always made use of the
``argument from design.''





``The resources of New York are rather meagre,''\ he said; ``but I'll
find a hansom first, and then we'll invent something.'' He led her
through the throng of returning holiday-makers, past sallow-faced
girls in preposterous hats, and flat-chested women struggling
with paper bundles and palm-leaf fans. Was it possible that she
belonged to the same race? The dinginess, the crudity of this
average section of womanhood made him feel how highly
specialized she was.





A rapid shower had cooled the air, and clouds still hung
refreshingly over the moist street.





``How delicious! Let us walk a little,''\ she said as they emerged
from the station.





They turned into Madison Avenue and began to stroll northward. As
she moved beside him, with her long light step, Selden was
conscious of taking a luxurious pleasure in her nearness: in the
modelling of her little ear, the crisp upward wave of her
hair---was it ever so slightly brightened by art?---and the thick
planting of her straight black lashes. Everything about her was
at once vigorous and exquisite, at once strong and fine. He had a
confused sense that she must have cost a great deal to make, that
a great many dull and ugly people must, in some mysterious way,
have been sacrificed to produce her. He was aware that the
qualities distinguishing her from the herd of her sex were
chiefly external: as though a fine glaze of beauty and
fastidiousness had been applied to vulgar clay. Yet the analogy
left him unsatisfied, for a coarse texture will not take a high
finish; and was it not possible that the material was fine, but
that circumstance had fashioned it into a futile shape?





As he reached this point in his speculations the sun came out,
and her lifted parasol cut off his enjoyment. A moment or two
later she paused with a sigh.





``Oh, dear, I'm so hot and thirsty---and what a hideous place New
York is!''\ She looked despairingly up and down the dreary
thoroughfare. ``Other cities put on their best clothes in summer,
but New York seems to sit in its shirtsleeves.'' Her eyes wandered
down one of the side-streets. ``Someone has had the humanity to
plant a few trees over there. Let us go into the shade.''





``I am glad my street meets with your approval,''\ said Selden as
they turned the corner.





``Your street? Do you live here?''





She glanced with interest along the new brick and limestone
house-fronts, fantastically varied in obedience to the American
craving for novelty, but fresh and inviting with their awnings
and flower-boxes.





``Ah, yes---to be sure: \textit{The} \textit{Benedick}. What a nice-looking building! 
I don't think I've ever seen it before.'' She looked across at the
flat-house with its marble porch and pseudo-Georgian fa\c{c}ade. 
``Which are your windows? Those with the awnings down?''





``On the top floor---yes.''





``And that nice little balcony is yours? How cool it looks up
there!''





He paused a moment. ``Come up and see,''\ he suggested. ``I can give
you a cup of tea in no time---and you won't meet any bores.''





Her colour deepened---she still had the art of blushing at the
right time---but she took the suggestion as lightly as it was
made.





``Why not? It's too tempting---I'll take the risk,''\ she declared.





``Oh, I'm not dangerous,''\ he said in the same key. In truth, he
had never liked her as well as at that moment. He knew she had
accepted without afterthought: he could never be a factor in her
calculations, and there was a surprise, a refreshment almost, in
the spontaneity of her consent.





On the threshold he paused a moment, feeling for his latchkey.





``There's no one here; but I have a servant who is supposed to
come in the mornings, and it's just possible he may have put out
the tea-things and provided some cake.''





He ushered her into a slip of a hall hung with old prints. She
noticed the letters and notes heaped on the table among his
gloves and sticks; then she found herself in a small library,
dark but cheerful, with its walls of books, a pleasantly faded
Turkey rug, a littered desk and, as he had foretold, a tea-tray
on a low table near the window. A breeze had sprung up, swaying
inward the muslin curtains, and bringing a fresh scent of
mignonette and petunias from the flower-box on the balcony.





Lily sank with a sigh into one of the shabby leather chairs.





``How delicious to have a place like this all to one's self! What
a miserable thing it is to be a woman.'' She leaned back in a
luxury of discontent.





Selden was rummaging in a cupboard for the cake.





``Even women,''\ he said, ``have been known to enjoy the privileges
of a flat.''





``Oh, governesses---or widows. But not girls---not poor, miserable,
marriageable girls!''





``I even know a girl who lives in a flat.''





She sat up in surprise. ``You do?''





``I do,''\ he assured her, emerging from the cupboard with the
sought-for cake.





``Oh, I know---you mean Gerty Farish.'' She smiled a little
unkindly. ``But I said \textit{marriageable}---and besides, she has a horrid
little place, and no maid, and such queer things to eat. Her cook
does the washing and the food tastes of soap. I should hate that,
you know.''





``You shouldn't dine with her on wash-days,''\ said Selden, cutting
the cake.





They both laughed, and he knelt by the table to light the lamp
under the kettle, while she measured out the tea into a little
tea-pot of green glaze. As he watched her hand, polished as a bit
of old ivory, with its slender pink nails, and the sapphire
bracelet slipping over her wrist, he was struck with the irony of
suggesting to her such a life as his cousin Gertrude Farish had
chosen. She was so evidently the victim of the civilization which
had produced her, that the links of her bracelet seemed like
manacles chaining her to her fate.





She seemed to read his thought. ``It was horrid of me to say that
of Gerty,''\ she said with charming compunction. ``I forgot she was
your cousin. But we're so different, you know: she likes being
good, and I like being happy. And besides, she is free and I am
not. If I were, I daresay I could manage to be happy even in her
flat. It must be pure bliss to arrange the furniture just as one
likes, and give all the horrors to the ash-man. If I could only
do over my aunt's drawing-room I know I should be a better
woman.''





``Is it so very bad?''\ he asked sympathetically.





She smiled at him across the tea-pot which she was holding up to
be filled.





``That shows how seldom you come there. Why don't you come
oftener?''





``When I do come, it's not to look at Mrs.\ Peniston's furniture.''





``Nonsense,''\ she said. ``You don't come at all---and yet we get on
so well when we meet.''





``Perhaps that's the reason,''\ he answered promptly. ``I'm afraid I
haven't any cream, you know---shall you mind a slice of lemon
instead?''





``I shall like it better.'' She waited while he cut the lemon and
dropped a thin disk into her cup. ``But that is not the reason,''
she insisted.





``The reason for what?''





``For your never coming.'' She leaned forward with a shade of
perplexity in her charming eyes. ``I wish I knew---I wish I could
make you out. Of course I know there are men who don't like
me---one can tell that at a glance. And there are others who are
afraid of me: they think I want to marry them.'' She smiled up at
him frankly. ``But I don't think you dislike me---and you can't
possibly think I want to marry you.''





``No---I absolve you of that,''\ he agreed.





``Well, then----?''





He had carried his cup to the fireplace, and stood leaning
against the chimney-piece and looking down on her with an air of
indolent amusement. The provocation in her eyes increased his
amusement---he had not supposed she would waste her powder on such
small game; but perhaps she was only keeping her hand in; or
perhaps a girl of her type had no conversation but of the
personal kind. At any rate, she was amazingly pretty, and he had
asked her to tea and must live up to his obligations.





``Well, then,''\ he said with a plunge, ``perhaps \textit{that's} the reason.''





``What?''





``The fact that you don't want to marry me. Perhaps I don't regard
it as such a strong inducement to go and see you.'' He felt
a slight shiver down his spine as he ventured this, but her laugh
reassured him.





``Dear Mr.\ Selden, that wasn't worthy of you. It's stupid of you
to make love to me, and it isn't like you to be stupid.'' She
leaned back, sipping her tea with an air so enchantingly judicial
that, if they had been in her aunt's drawing-room, he might
almost have tried to disprove her deduction.





``Don't you see,''\ she continued, ``that there are men enough to say
pleasant things to me, and that what I want is a friend who won't
be afraid to say disagreeable ones when I need them? Sometimes I
have fancied you might be that friend---I don't know why, except
that you are neither a prig nor a bounder, and that I shouldn't
have to pretend with you or be on my guard against you.'' Her
voice had dropped to a note of seriousness, and she sat gazing up
at him with the troubled gravity of a child.





``You don't know how much I need such a friend,''\ she said. ``My
aunt is full of copy-book axioms, but they were all meant to
apply to conduct in the early fifties. I always feel that to live
up to them would include wearing book-muslin with gigot sleeves. 
And the other women---my best friends---well, they use me or abuse
me; but they don't care a straw what happens to me. I've been
about too long---people are getting tired of me; they are
beginning to say I ought to marry.''





There was a moment's pause, during which Selden meditated one or
two replies calculated to add a momentary zest to the situation;
but he rejected them in favour of the simple question: ``Well, why
don't you?''





She coloured and laughed. ``Ah, I see you \textit{are} a friend after all,
and that is one of the disagreeable things I was asking for.''





``It wasn't meant to be disagreeable,''\ he returned amicably. 
``Isn't marriage your vocation? Isn't it what you're all brought
up for?''





She sighed. ``I suppose so. What else is there?''





``Exactly. And so why not take the plunge and have it over?''





She shrugged her shoulders. ``You speak as if I ought to marry the
first man who came along.''





``I didn't mean to imply that you are as hard put to it as
that. But there must be some one with the requisite
qualifications.''





She shook her head wearily. ``I threw away one or two good chances
when I first came out---I suppose every girl does; and you know I
am horribly poor---and very expensive. I must have a great deal of
money.''





Selden had turned to reach for a cigarette-box on the
mantelpiece.





``What's become of Dillworth?''\ he asked.





``Oh, his mother was frightened---she was afraid I should have all
the family jewels reset. And she wanted me to promise that I
wouldn't do over the drawing-room.''





``The very thing you are marrying for!''





``Exactly. So she packed him off to India.''





``Hard luck---but you can do better than Dillworth.''





He offered the box, and she took out three or four cigarettes,
putting one between her lips and slipping the others into a
little gold case attached to her long pearl chain.





``Have I time? Just a whiff, then.'' She leaned forward, holding
the tip of her cigarette to his. As she did so, he noted, with a
purely impersonal enjoyment, how evenly the black lashes were set
in her smooth white lids, and how the purplish shade beneath them
melted into the pure pallour of the cheek.





She began to saunter about the room, examining the bookshelves
between the puffs of her cigarette-smoke. Some of the volumes had
the ripe tints of good tooling and old morocco, and her eyes
lingered on them caressingly, not with the appreciation of the
expert, but with the pleasure in agreeable tones and textures
that was one of her inmost susceptibilities. Suddenly her
expression changed from desultory enjoyment to active conjecture,
and she turned to Selden with a question.





``You collect, don't you---you know about first editions and
things?''





``As much as a man may who has no money to spend. Now and then I
pick up something in the rubbish heap; and I go and look on at
the big sales.''





She had again addressed herself to the shelves, but her eyes now
swept them inattentively, and he saw that she was preoccupied
with a new idea.





``And Americana---do you collect Americana?''





Selden stared and laughed.





``No, that's rather out of my line. I'm not really a collector,
you see; I simply like to have good editions of the books I am
fond of.''





She made a slight grimace. ``And Americana are horribly dull, I
suppose?''





``I should fancy so---except to the historian. But your real
collector values a thing for its rarity. I don't suppose the
buyers of Americana sit up reading them all night---old Jefferson
Gryce certainly didn't.''





She was listening with keen attention. ``And yet they fetch
fabulous prices, don't they? It seems so odd to want to pay a lot
for an ugly badly-printed book that one is never going to read! 
And I suppose most of the owners of Americana are not historians
either?''





``No; very few of the historians can afford to buy them. They have
to use those in the public libraries or in private collections. 
It seems to be the mere rarity that attracts the average
collector.''





He had seated himself on an arm of the chair near which she was
standing, and she continued to question him, asking which were
the rarest volumes, whether the Jefferson Gryce collection was
really considered the finest in the world, and what was the
largest price ever fetched by a single volume.





It was so pleasant to sit there looking up at her, as she lifted
now one book and then another from the shelves, fluttering the
pages between her fingers, while her drooping profile was
outlined against the warm background of old bindings, that he
talked on without pausing to wonder at her sudden interest in so
unsuggestive a subject. But he could never be long with her
without trying to find a reason for what she was doing, and as
she replaced his first edition of La Bruyere and turned away from
the bookcases, he began to ask himself what she had been driving
at. Her next question was not of a nature to enlighten him. She
paused before him with a smile which seemed at once designed to
admit him to her familiarity, and to remind him of the
restrictions it imposed.





``Don't you ever mind,''\ she asked suddenly, ``not being rich enough
to buy all the books you want?''





He followed her glance about the room, with its worn furniture
and shabby walls.





``Don't I just? Do you take me for a saint on a pillar?''





``And having to work---do you mind that?''





``Oh, the work itself is not so bad---I'm rather fond of the law.''





``No; but the being tied down: the routine---don't you ever want to
get away, to see new places and people?''





``Horribly---especially when I see all my friends rushing to the
steamer.''





She drew a sympathetic breath. ``But do you mind enough---to marry
to get out of it?''





Selden broke into a laugh. ``God forbid!''\ he declared.





She rose with a sigh, tossing her cigarette into the grate.





``Ah, there's the difference---a girl must, a man may if he
chooses.'' She surveyed him critically. ``Your coat's a little
shabby---but who cares? It doesn't keep people from asking you to
dine. If I were shabby no one would have me: a woman is asked out
as much for her clothes as for herself. The clothes are the
background, the frame, if you like: they don't make success, but
they are a part of it. Who wants a dingy woman? We are expected
to be pretty and well-dressed till we drop---and if we can't keep
it up alone, we have to go into partnership.''





Selden glanced at her with amusement: it was impossible, even
with her lovely eyes imploring him, to take a sentimental view of
her case.





``Ah, well, there must be plenty of capital on the look-out for
such an investment. Perhaps you'll meet your fate tonight at the
Trenors'.''





She returned his look interrogatively.





``I thought you might be going there---oh, not in that capacity! 
But there are to be a lot of your set---Gwen Van Osburgh, the
Wetheralls, Lady Cressida Raith---and the George Dorsets.''





She paused a moment before the last name, and shot a query
through her lashes; but he remained imperturbable.





``Mrs.\ Trenor asked me; but I can't get away till the end of the
week; and those big parties bore me.''





``Ah, so they do me,''\ she exclaimed.





``Then why go?''





``It's part of the business---you forget! And besides, if I didn't,
I should be playing bezique with my aunt at Richfield Springs.''





``That's almost as bad as marrying Dillworth,''\ he agreed, and they
both laughed for pure pleasure in their sudden intimacy.





She glanced at the clock.





``Dear me! I must be off. It's after five.''





She paused before the mantelpiece, studying herself in the mirror
while she adjusted her veil. The attitude revealed the long slope
of her slender sides, which gave a kind of wild-wood grace to her
outline---as though she were a captured dryad subdued to the
conventions of the drawing-room; and Selden reflected that it was
the same streak of sylvan freedom in her nature that lent such
savour to her artificiality.





He followed her across the room to the entrance-hall; but on the
threshold she held out her hand with a gesture of leave-taking.





``It's been delightful; and now you will have to return my visit.''





``But don't you want me to see you to the station?''





``No; good bye here, please.''





She let her hand lie in his a moment, smiling up at him adorably.





``Good bye, then---and good luck at Bellomont!''\ he said, opening
the door for her.





On the landing she paused to look about her. There were a
thousand chances to one against her meeting anybody, but one
could never tell, and she always paid for her rare indiscretions
by a violent reaction of prudence. There was no one in sight,
however, but a char-woman who was scrubbing the stairs. Her own
stout person and its surrounding implements took up so much room
that Lily, to pass her, had to gather up her skirts and brush
against the wall. As she did so, the woman paused in her work and
looked up curiously, resting her clenched red fists on the
wet cloth she had just drawn from her pail. She had a broad
sallow face, slightly pitted with small-pox, and thin
straw-coloured hair through which her scalp shone unpleasantly.





``I beg your pardon,''\ said Lily, intending by her politeness to
convey a criticism of the other's manner.





The woman, without answering, pushed her pail aside, and
continued to stare as Miss Bart swept by with a murmur of silken
linings. Lily felt herself flushing under the look. What did the
creature suppose? Could one never do the simplest, the most
harmless thing, without subjecting one's self to some odious
conjecture? Half way down the next flight, she smiled to think
that a char-woman's stare should so perturb her. The poor thing
was probably dazzled by such an unwonted apparition. But \textit{were}
such apparitions unwonted on Selden's stairs? Miss Bart was not
familiar with the moral code of bachelors'\ flat-houses, and her
colour rose again as it occurred to her that the woman's
persistent gaze implied a groping among past associations. But
she put aside the thought with a smile at her own fears, and
hastened downward, wondering if she should find a cab short of
Fifth Avenue.





Under the Georgian porch she paused again, scanning the street
for a hansom. None was in sight, but as she reached the sidewalk
she ran against a small glossy-looking man with a gardenia in his
coat, who raised his hat with a surprised exclamation.





``Miss Bart? Well---of all people! This \textit{is} luck,''\ he declared; and
she caught a twinkle of amused curiosity between his screwed-up
lids.





``Oh, Mr.\ Rosedale---how are you?''\ she said, perceiving that the
irrepressible annoyance on her face was reflected in the sudden
intimacy of his smile.





Mr.\ Rosedale stood scanning her with interest and approval. He
was a plump rosy man of the blond Jewish type, with smart London
clothes fitting him like upholstery, and small sidelong eyes
which gave him the air of appraising people as if they were
bric-a-brac. He glanced up interrogatively at the porch of the
Benedick.





``Been up to town for a little shopping, I suppose?''\ he said, in a
tone which had the familiarity of a touch.





Miss Bart shrank from it slightly, and then flung herself into
precipitate explanations.





``Yes---I came up to see my dress-maker. I am just on my way to
catch the train to the Trenors'.''





``Ah---your dress-maker; just so,''\ he said blandly. ``I didn't know
there were any dress-makers in the Benedick.''





``The Benedick?''\ She looked gently puzzled. ``Is that the name of
this building?''





``Yes, that's the name: I believe it's an old word for bachelor,
isn't it? I happen to own the building---that's the way I know.'' 
His smile deepened as he added with increasing assurance: ``But
you must let me take you to the station. The Trenors are at
Bellomont, of course? You've barely time to catch the five-forty. 
The dress-maker kept you waiting, I suppose.''





Lily stiffened under the pleasantry.





``Oh, thanks,''\ she stammered; and at that moment her eye caught a
hansom drifting down Madison Avenue, and she hailed it with a
desperate gesture.





``You're very kind; but I couldn't think of troubling you,''\ she
said, extending her hand to Mr.\ Rosedale; and heedless of his
protestations, she sprang into the rescuing vehicle, and called
out a breathless order to the driver.





\chapter*{\raggedright Chapter 2}

\addcontentsline{toc}{chapter}{Chapter 2}

\markboth{House of Mirth}{Chapter 2}





In the hansom she leaned back with a sigh. Why must a girl pay so
dearly for her least escape from routine? Why could one never do
a natural thing without having to screen it behind a structure of
artifice? She had yielded to a passing impulse in going to
Lawrence Selden's rooms, and it was so seldom that she could
allow herself the luxury of an impulse! This one, at any rate,
was going to cost her rather more than she could afford. She was
vexed to see that, in spite of so many years of vigilance, she
had blundered twice within five minutes. That stupid story about
her dress-maker was bad enough---it would have been so simple to
tell Rosedale that she had been taking tea with Selden! The mere
statement of the fact would have rendered it innocuous. But,
after having let herself be surprised in a falsehood, it was
doubly stupid to snub the witness of her discomfiture. If she had
had the presence of mind to let Rosedale drive her to the
station, the concession might have purchased his silence. He had
his race's accuracy in the appraisal of values, and to be seen
walking down the platform at the crowded afternoon hour in the
company of Miss Lily Bart would have been money in his pocket, as
he might himself have phrased it. He knew, of course, that there
would be a large house-party at Bellomont, and the possibility of
being taken for one of Mrs.\ Trenor's guests was doubtless
included in his calculations. Mr.\ Rosedale was still at a stage
in his social ascent when it was of importance to produce such
impressions.





The provoking part was that Lily knew all this---knew how easy it
would have been to silence him on the spot, and how difficult it
might be to do so afterward. Mr.\ Simon Rosedale was a man who
made it his business to know everything about every one, whose
idea of showing himself to be at home in society was to display
an inconvenient familiarity with the habits of those with whom he
wished to be thought intimate. Lily was sure that within
twenty-four hours the story of her visiting her dress-maker at
the Benedick would be in active circulation among Mr.\ Rosedale's
acquaintances. The worst of it was that she had always snubbed
and ignored him. On his first appearance---when her
improvident cousin, Jack Stepney, had obtained for him (in return
for favours too easily guessed)\ a card to one of the vast
impersonal Van Osburgh ``crushes''---Rosedale, with that mixture of
artistic sensibility and business astuteness which characterizes
his race, had instantly gravitated toward Miss Bart. She
understood his motives, for her own course was guided by as nice
calculations. Training and experience had taught her to be
hospitable to newcomers, since the most unpromising might be
useful later on, and there were plenty of available OUBLIETTES to
swallow them if they were not. But some intuitive repugnance,
getting the better of years of social discipline, had made her
push Mr.\ Rosedale into his \textit{oubliette} without a trial. He had left
behind only the ripple of amusement which his speedy despatch had
caused among her friends; and though later (to shift the
metaphor)\ he reappeared lower down the stream, it was only in
fleeting glimpses, with long submergences between.





Hitherto Lily had been undisturbed by scruples. In her little set
Mr.\ Rosedale had been pronounced ``impossible,''\ and Jack Stepney
roundly snubbed for his attempt to pay his debts in dinner
invitations. Even Mrs.\ Trenor, whose taste for variety had led
her into some hazardous experiments, resisted Jack's attempts to
disguise Mr.\ Rosedale as a novelty, and declared that he was the
same little Jew who had been served up and rejected at the social
board a dozen times within her memory; and while Judy Trenor was
obdurate there was small chance of Mr.\ Rosedale's penetrating
beyond the outer limbo of the Van Osburgh crushes. Jack gave up
the contest with a laughing ``You'll see,''\ and, sticking manfully
to his guns, showed himself with Rosedale at the fashionable
restaurants, in company with the personally vivid if socially
obscure ladies who are available for such purposes. But the
attempt had hitherto been vain, and as Rosedale undoubtedly paid
for the dinners, the laugh remained with his debtor.





Mr.\ Rosedale, it will be seen, was thus far not a factor to be
feared---unless one put one's self in his power. And this was
precisely what Miss Bart had done. Her clumsy fib had let him see
that she had something to conceal; and she was sure he had a
score to settle with her. Something in his smile told her
he had not forgotten. She turned from the thought with a little
shiver, but it hung on her all the way to the station, and dogged
her down the platform with the persistency of Mr.\ Rosedale
himself.





She had just time to take her seat before the train started; but
having arranged herself in her corner with the instinctive
feeling for effect which never forsook her, she glanced about in
the hope of seeing some other member of the Trenors'\ party. She
wanted to get away from herself, and conversation was the only
means of escape that she knew.





Her search was rewarded by the discovery of a very blond young
man with a soft reddish beard, who, at the other end of the
carriage, appeared to be dissembling himself behind an unfolded
newspaper. Lily's eye brightened, and a faint smile relaxed the
drawn lines of her mouth. She had known that Mr.\ Percy Gryce was
to be at Bellomont, but she had not counted on the luck of having
him to herself in the train; and the fact banished all perturbing
thoughts of Mr.\ Rosedale. Perhaps, after all, the day was to end
more favourably than it had begun.





She began to cut the pages of a novel, tranquilly studying her
prey through downcast lashes while she organized a method of
attack. Something in his attitude of conscious absorption told
her that he was aware of her presence: no one had ever been quite
so engrossed in an evening paper! She guessed that he was too shy
to come up to her, and that she would have to devise some means
of approach which should not appear to be an advance on her part. 
It amused her to think that any one as rich as Mr.\ Percy Gryce
should be shy; but she was gifted with treasures of indulgence
for such idiosyncrasies, and besides, his timidity might serve
her purpose better than too much assurance. She had the art of
giving self-confidence to the embarrassed, but she was not
equally sure of being able to embarrass the self-confident.





She waited till the train had emerged from the tunnel and was
racing between the ragged edges of the northern suburbs. Then, as
it lowered its speed near Yonkers, she rose from her seat and
drifted slowly down the carriage. As she passed Mr.\ Gryce, the
train gave a lurch, and he was aware of a slender hand gripping
the back of his chair. He rose with a start, his ingenuous
face looking as though it had been dipped in crimson: even the
reddish tint in his beard seemed to deepen. The train swayed
again, almost flinging Miss Bart into his arms.





She steadied herself with a laugh and drew back; but he was
enveloped in the scent of her dress, and his shoulder had felt
her fugitive touch.





``Oh, Mr.\ Gryce, is it you? I'm so sorry---I was trying to find the
porter and get some tea.''





She held out her hand as the train resumed its level rush, and
they stood exchanging a few words in the aisle. Yes---he was going
to Bellomont. He had heard she was to be of the party---he blushed
again as he admitted it. And was he to be there for a whole week? 
How delightful!





But at this point one or two belated passengers from the last
station forced their way into the carriage, and Lily had to
retreat to her seat.





``The chair next to mine is empty---do take it,''\ she said over her
shoulder; and Mr.\ Gryce, with considerable embarrassment,
succeeded in effecting an exchange which enabled him to transport
himself and his bags to her side.





``Ah---and here is the porter, and perhaps we can have some tea.''





She signalled to that official, and in a moment, with the ease
that seemed to attend the fulfilment of all her wishes, a little
table had been set up between the seats, and she had helped Mr.
Gryce to bestow his encumbering properties beneath it.





When the tea came he watched her in silent fascination while her
hands flitted above the tray, looking miraculously fine and
slender in contrast to the coarse china and lumpy bread. It
seemed wonderful to him that any one should perform with such
careless ease the difficult task of making tea in public in a
lurching train. He would never have dared to order it for
himself, lest he should attract the notice of his
fellow-passengers; but, secure in the shelter of her
conspicuousness, he sipped the inky draught with a delicious
sense of exhilaration.





Lily, with the flavour of Selden's caravan tea on her lips, had
no great fancy to drown it in the railway brew which seemed such
nectar to her companion; but, rightly judging that one of
the charms of tea is the fact of drinking it together, she
proceeded to give the last touch to Mr.\ Gryce's enjoyment by
smiling at him across her lifted cup.





``Is it quite right---I haven't made it too strong?''\ she asked
solicitously; and he replied with conviction that he had never
tasted better tea.





``I daresay it is true,''\ she reflected; and her imagination was
fired by the thought that Mr.\ Gryce, who might have sounded the
depths of the most complex self-indulgence, was perhaps actually
taking his first journey alone with a pretty woman.





It struck her as providential that she should be the instrument
of his initiation. Some girls would not have known how to manage
him. They would have over-emphasized the novelty of the
adventure, trying to make him feel in it the zest of an escapade. 
But Lily's methods were more delicate. She remembered that her
cousin Jack Stepney had once defined Mr.\ Gryce as the young man
who had promised his mother never to go out in the rain without
his overshoes; and acting on this hint, she resolved to impart a
gently domestic air to the scene, in the hope that her companion,
instead of feeling that he was doing something reckless or
unusual, would merely be led to dwell on the advantage of always
having a companion to make one's tea in the train.





But in spite of her efforts, conversation flagged after the tray
had been removed, and she was driven to take a fresh measurement
of Mr.\ Gryce's limitations. It was not, after all, opportunity
but imagination that he lacked: he had a mental palate which
would never learn to distinguish between railway tea and nectar. 
There was, however, one topic she could rely on: one spring that
she had only to touch to set his simple machinery in motion. She
had refrained from touching it because it was a last resource,
and she had relied on other arts to stimulate other sensations;
but as a settled look of dulness began to creep over his candid
features, she saw that extreme measures were necessary.





``And how,''\ she said, leaning forward, ``are you getting on with
your Americana?''





His eye became a degree less opaque: it was as though an
incipient film had been removed from it, and she felt the pride
of a skilful operator.





``I've got a few new things,''\ he said, suffused with pleasure, but
lowering his voice as though he feared his fellow-passengers
might be in league to despoil him.





She returned a sympathetic enquiry, and gradually he was drawn on
to talk of his latest purchases. It was the one subject which
enabled him to forget himself, or allowed him, rather, to
remember himself without constraint, because he was at home in
it, and could assert a superiority that there were few to
dispute. Hardly any of his acquaintances cared for Americana, or
knew anything about them; and the consciousness of this ignorance
threw Mr.\ Gryce's knowledge into agreeable relief. The only
difficulty was to introduce the topic and to keep it to the
front; most people showed no desire to have their ignorance
dispelled, and Mr.\ Gryce was like a merchant whose warehouses are
crammed with an unmarketable commodity.





But Miss Bart, it appeared, really did want to know about
Americana; and moreover, she was already sufficiently informed to
make the task of farther instruction as easy as it was agreeable. 
She questioned him intelligently, she heard him submissively;
and, prepared for the look of lassitude which usually crept over
his listeners'\ faces, he grew eloquent under her receptive gaze. 
The ``points''\ she had had the presence of mind to glean from
Selden, in anticipation of this very contingency, were serving
her to such good purpose that she began to think her visit to him
had been the luckiest incident of the day. She had once more
shown her talent for profiting by the unexpected, and dangerous
theories as to the advisability of yielding to impulse were
germinating under the surface of smiling attention which she
continued to present to her companion.





Mr.\ Gryce's sensations, if less definite, were equally agreeable. 
He felt the confused titillation with which the lower organisms
welcome the gratification of their needs, and all his senses
floundered in a vague well-being, through which Miss Bart's
personality was dimly but pleasantly perceptible.





Mr.\ Gryce's interest in Americana had not originated with
himself: it was impossible to think of him as evolving any taste
of his own. An uncle had left him a collection already noted
among bibliophiles; the existence of the collection was
the only fact that had ever shed glory on the name of Gryce, and
the nephew took as much pride in his inheritance as though it had
been his own work. Indeed, he gradually came to regard it as
such, and to feel a sense of personal complacency when he chanced
on any reference to the Gryce Americana. Anxious as he was to
avoid personal notice, he took, in the printed mention of his
name, a pleasure so exquisite and excessive that it seemed a
compensation for his shrinking from publicity.





To enjoy the sensation as often as possible, he subscribed to all
the reviews dealing with book-collecting in general, and American
history in particular, and as allusions to his library abounded
in the pages of these journals, which formed his only reading, he
came to regard himself as figuring prominently in the public eye,
and to enjoy the thought of the interest which would be excited
if the persons he met in the street, or sat among in travelling,
were suddenly to be told that he was the possessor of the Gryce
Americana.





Most timidities have such secret compensations, and Miss Bart was
discerning enough to know that the inner vanity is generally in
proportion to the outer self-depreciation. With a more confident
person she would not have dared to dwell so long on one topic, or
to show such exaggerated interest in it; but she had rightly
guessed that Mr.\ Gryce's egoism was a thirsty soil, requiring
constant nurture from without. Miss Bart had the gift of
following an undercurrent of thought while she appeared to be
sailing on the surface of conversation; and in this case her
mental excursion took the form of a rapid survey of Mr.\ Percy
Gryce's future as combined with her own. The Gryces were from
Albany, and but lately introduced to the metropolis, where the
mother and son had come, after old Jefferson Gryce's death, to
take possession of his house in Madison Avenue---an appalling
house, all brown stone without and black walnut within, with the
Gryce library in a fire-proof annex that looked like a mausoleum. 
Lily, however, knew all about them: young Mr.\ Gryce's arrival had
fluttered the maternal breasts of New York, and when a girl has
no mother to palpitate for her she must needs be on the alert for
herself. Lily, therefore, had not only contrived to put herself
in the young man's way, but had made the acquaintance of
Mrs.\ Gryce, a monumental woman with the voice of a pulpit orator
and a mind preoccupied with the iniquities of her servants, who
came sometimes to sit with Mrs.\ Peniston and learn from that lady
how she managed to prevent the kitchen-maid's smuggling groceries
out of the house. Mrs.\ Gryce had a kind of impersonal
benevolence: cases of individual need she regarded with
suspicion, but she subscribed to Institutions when their annual
reports showed an impressive surplus. Her domestic duties were
manifold, for they extended from furtive inspections of the
servants'\ bedrooms to unannounced descents to the cellar; but she
had never allowed herself many pleasures. Once, however, she had
had a special edition of the Sarum Rule printed in rubric and
presented to every clergyman in the diocese; and the gilt album
in which their letters of thanks were pasted formed the chief
ornament of her drawing-room table.





Percy had been brought up in the principles which so excellent a
woman was sure to inculcate. Every form of prudence and suspicion
had been grafted on a nature originally reluctant and cautious,
with the result that it would have seemed hardly needful for Mrs.
Gryce to extract his promise about the overshoes, so little
likely was he to hazard himself abroad in the rain. After
attaining his majority, and coming into the fortune which the
late Mr.\ Gryce had made out of a patent device for excluding
fresh air from hotels, the young man continued to live with his
mother in Albany; but on Jefferson Gryce's death, when another
large property passed into her son's hands, Mrs.\ Gryce thought
that what she called his ``interests''\ demanded his presence in New
York. She accordingly installed herself in the Madison Avenue
house, and Percy, whose sense of duty was not inferior to his
mother's, spent all his week days in the handsome Broad Street
office where a batch of pale men on small salaries had grown grey
in the management of the Gryce estate, and where he was initiated
with becoming reverence into every detail of the art of
accumulation.





As far as Lily could learn, this had hitherto been Mr.\ Gryce's
only occupation, and she might have been pardoned for thinking it
not too hard a task to interest a young man who had been kept on
such low diet. At any rate, she felt herself so completely
in command of the situation that she yielded to a sense of
security in which all fear of Mr.\ Rosedale, and of the
difficulties on which that fear was contingent, vanished beyond
the edge of thought.





The stopping of the train at Garrisons would not have distracted
her from these thoughts, had she not caught a sudden look of
distress in her companion's eye. His seat faced toward the door,
and she guessed that he had been perturbed by the approach of an
acquaintance; a fact confirmed by the turning of heads and
general sense of commotion which her own entrance into a
railway-carriage was apt to produce.





She knew the symptoms at once, and was not surprised to be hailed
by the high notes of a pretty woman, who entered the train
accompanied by a maid, a bull-terrier, and a footman staggering
under a load of bags and dressing-cases.





``Oh, Lily---are you going to Bellomont? Then you can't let me have
your seat, I suppose? But I \textit{must} have a seat in this
carriage---porter, you must find me a place at once. Can't some
one be put somewhere else? I want to be with my friends. Oh, how
do you do, Mr.\ Gryce? Do please make him understand that I must
have a seat next to you and Lily.''





Mrs.\ George Dorset, regardless of the mild efforts of a traveller
with a carpet-bag, who was doing his best to make room for her by
getting out of the train, stood in the middle of the aisle,
diffusing about her that general sense of exasperation which a
pretty woman on her travels not infrequently creates.





She was smaller and thinner than Lily Bart, with a restless
pliability of pose, as if she could have been crumpled up and run
through a ring, like the sinuous draperies she affected. Her
small pale face seemed the mere setting of a pair of dark
exaggerated eyes, of which the visionary gaze contrasted
curiously with her self-assertive tone and gestures; so that, as
one of her friends observed, she was like a disembodied spirit
who took up a great deal of room.





Having finally discovered that the seat adjoining Miss Bart's was
at her disposal, she possessed herself of it with a farther
displacement of her surroundings, explaining meanwhile that she
had come across from Mount Kisco in her motor-car that morning,
and had been kicking her heels for an hour at Garrisons, without
even the alleviation of a cigarette, her brute of a
husband having neglected to replenish her case before they parted
that morning.





``And at this hour of the day I don't suppose you've a single one
left, have you, Lily?''\ she plaintively concluded.





Miss Bart caught the startled glance of Mr.\ Percy Gryce, whose
own lips were never defiled by tobacco.





``What an absurd question, Bertha!''\ she exclaimed, blushing at the
thought of the store she had laid in at Lawrence Selden's.





``Why, don't you smoke? Since when have you given it up? What---you
never----And you don't either, Mr.\ Gryce? Ah, of course---how
stupid of me---I understand.''





And Mrs.\ Dorset leaned back against her travelling cushions with
a smile which made Lily wish there had been no vacant seat beside
her own.





\chapter*{\raggedright Chapter 3}

\addcontentsline{toc}{chapter}{Chapter 3}

\markboth{House of Mirth}{Chapter 3}





Bridge at Bellomont usually lasted till the small hours; and when
Lily went to bed that night she had played too long for her own
good.





Feeling no desire for the self-communion which awaited her in her
room, she lingered on the broad stairway, looking down into the
hall below, where the last card-players were grouped about the
tray of tall glasses and silver-collared decanters which the
butler had just placed on a low table near the fire.





The hall was arcaded, with a gallery supported on columns of pale
yellow marble. Tall clumps of flowering plants were grouped
against a background of dark foliage in the angles of the walls. 
On the crimson carpet a deer-hound and two or three spaniels
dozed luxuriously before the fire, and the light from the great
central lantern overhead shed a brightness on the women's hair
and struck sparks from their jewels as they moved.





There were moments when such scenes delighted Lily, when they
gratified her sense of beauty and her craving for the external
finish of life; there were others when they gave a sharper edge
to the meagreness of her own opportunities. This was one of the
moments when the sense of contrast was uppermost, and she turned
away impatiently as Mrs.\ George Dorset, glittering in serpentine
spangles, drew Percy Gryce in her wake to a confidential nook
beneath the gallery.





It was not that Miss Bart was afraid of losing her newly-acquired
hold over Mr.\ Gryce. Mrs.\ Dorset might startle or dazzle him, but
she had neither the skill nor the patience to effect his capture. 
She was too self-engrossed to penetrate the recesses of his
shyness, and besides, why should she care to give herself the
trouble? At most it might amuse her to make sport of his
simplicity for an evening---after that he would be merely a burden
to her, and knowing this, she was far too experienced to
encourage him. But the mere thought of that other woman, who
could take a man up and toss him aside as she willed, without
having to regard him as a possible factor in her plans, filled
Lily Bart with envy. She had been bored all the afternoon
by Percy Gryce---the mere thought seemed to waken an echo of his
droning voice---but she could not ignore him on the morrow, she
must follow up her success, must submit to more boredom, must be
ready with fresh compliances and adaptabilities, and all on the
bare chance that he might ultimately decide to do her the honour
of boring her for life.





It was a hateful fate---but how escape from it? What choice had
she? To be herself, or a Gerty Farish. As she entered her
bedroom, with its softly-shaded lights, her lace dressing-gown
lying across the silken bedspread, her little embroidered
slippers before the fire, a vase of carnations filling the air
with perfume, and the last novels and magazines lying uncut on a
table beside the reading-lamp, she had a vision of Miss Farish's
cramped flat, with its cheap conveniences and hideous
wall-papers. No; she was not made for mean and shabby
surroundings, for the squalid compromises of poverty. Her whole
being dilated in an atmosphere of luxury; it was the background
she required, the only climate she could breathe in. But the
luxury of others was not what she wanted. A few years ago it had
sufficed her: she had taken her daily meed of pleasure without
caring who provided it. Now she was beginning to chafe at the
obligations it imposed, to feel herself a mere pensioner on the
splendour which had once seemed to belong to her. There were even
moments when she was conscious of having to pay her way.





For a long time she had refused to play bridge. She knew she
could not afford it, and she was afraid of acquiring so expensive
a taste. She had seen the danger exemplified in more than one of
her associates---in young Ned Silverton, for instance, the
charming fair boy now seated in abject rapture at the elbow of
Mrs.\ Fisher, a striking divorcee with eyes and gowns as emphatic
as the head-lines of her ``case.'' Lily could remember when young
Silverton had stumbled into their circle, with the air of a
strayed Arcadian who has published chamung sonnets in his college
journal. Since then he had developed a taste for Mrs.\ Fisher and
bridge, and the latter at least had involved him in expenses from
which he had been more than once rescued by harassed maiden
sisters, who treasured the sonnets, and went without sugar in
their tea to keep their darling afloat. Ned's case was
familiar to Lily: she had seen his charming eyes---which had a
good deal more poetry in them than the sonnets---change from
surprise to amusement, and from amusement to anxiety, as he
passed under the spell of the terrible god of chance; and she was
afraid of discovering the same symptoms in her own case.





For in the last year she had found that her hostesses expected
her to take a place at the card-table. It was one of the taxes
she had to pay for their prolonged hospitality, and for the
dresses and trinkets which occasionally replenished her
insufficient wardrobe. And since she had played regularly the
passion had grown on her. Once or twice of late she had won a
large sum, and instead of keeping it against future losses, had
spent it in dress or jewelry; and the desire to atone for this
imprudence, combined with the increasing exhilaration of the
game, drove her to risk higher stakes at each fresh venture. She
tried to excuse herself on the plea that, in the Trenor set, if
one played at all one must either play high or be set down as
priggish or stingy; but she knew that the gambling passion was
upon her, and that in her present surroundings there was small
hope of resisting it.





Tonight the luck had been persistently bad, and the little gold
purse which hung among her trinkets was almost empty when she
returned to her room. She unlocked the wardrobe, and taking out
her jewel-case, looked under the tray for the roll of bills from
which she had replenished the purse before going down to dinner. 
Only twenty dollars were left: the discovery was so startling
that for a moment she fancied she must have been robbed. Then she
took paper and pencil, and seating herself at the writing-table,
tried to reckon up what she had spent during the day. Her head
was throbbing with fatigue, and she had to go over the figures
again and again; but at last it became clear to her that she had
lost three hundred dollars at cards. She took out her cheque-book
to see if her balance was larger than she remembered, but found
she had erred in the other direction. Then she returned to her
calculations; but figure as she would, she could not conjure back
the vanished three hundred dollars. It was the sum she had set
aside to pacify her dress-maker---unless she should decide to use
it as a sop to the jeweller. At any rate, she had so many
uses for it that its very insufficiency had caused her to play
high in the hope of doubling it. But of course she had lost---she
who needed every penny, while Bertha Dorset, whose husband
showered money on her, must have pocketed at least five hundred,
and Judy Trenor, who could have afforded to lose a thousand a
night, had left the table clutching such a heap of bills that she
had been unable to shake hands with her guests when they bade her
good night.





A world in which such things could be seemed a miserable place to
Lily Bart; but then she had never been able to understand the
laws of a universe which was so ready to leave her out of its
calculations.





She began to undress without ringing for her maid, whom she had
sent to bed. She had been long enough in bondage to other
people's pleasure to be considerate of those who depended on
hers, and in her bitter moods it sometimes struck her that she
and her maid were in the same position, except that the latter
received her wages more regularly.





As she sat before the mirror brushing her hair, her face looked
hollow and pale, and she was frightened by two little lines near
her mouth, faint flaws in the smooth curve of the cheek.





``Oh, I must stop worrying!''\ she exclaimed. ``Unless it's the
electric light----''\ she reflected, springing up from her seat and
lighting the candles on the dressing-table.





She turned out the wall-lights, and peered at herself between the
candle-flames. The white oval of her face swam out waveringly
from a background of shadows, the uncertain light blurring it
like a haze; but the two lines about the mouth remained.





Lily rose and undressed in haste.





``It is only because I am tired and have such odious things to
think about,''\ she kept repeating; and it seemed an added
injustice that petty cares should leave a trace on the beauty
which was her only defence against them.





But the odious things were there, and remained with her. She
returned wearily to the thought of Percy Gryce, as a wayfarer
picks up a heavy load and toils on after a brief rest. She was
almost sure she had ``landed''\ him: a few days'\ work and she would
win her reward. But the reward itself seemed upalatable
just then: she could get no zest from the thought of victory. It
would be a rest from worry, no more---and how little that would
have seemed to her a few years earlier! Her ambitions had shrunk
gradually in the desiccating air of failure. But why had she
failed? Was it her own fault or that of destiny?





She remembered how her mother, after they had lost their money,
used to say to her with a kind of fierce vindictiveness: ``But
you'll get it all back---you'll get it all back, with your face.''
.\ .\ . The remembrance roused a whole train of association, and
she lay in the darkness reconstructing the past out of which her
present had grown.





A house in which no one ever dined at home unless there was
``company''; a door-bell perpetually ringing; a hall-table showered
with square envelopes which were opened in haste, and oblong
envelopes which were allowed to gather dust in the depths of a
bronze jar; a series of French and English maids giving warning
amid a chaos of hurriedly-ransacked wardrobes and dress-closets;
an equally changing dynasty of nurses and footmen; quarrels in
the pantry, the kitchen and the drawing-room; precipitate trips
to Europe, and returns with gorged trunks and days of
interminable unpacking; semi-annual discussions as to where the
summer should be spent, grey interludes of economy and brilliant
reactions of expense---such was the setting of Lily Bart's first
memories.





Ruling the turbulent element called home was the vigorous and
determined figure of a mother still young enough to dance her
ball-dresses to rags, while the hazy outline of a neutral-tinted
father filled an intermediate space between the butler and the
man who came to wind the clocks. Even to the eyes of infancy,
Mrs.\ Hudson Bart had appeared young; but Lily could not recall
the time when her father had not been bald and slightly stooping,
with streaks of grey in his hair, and a tired walk. It was a
shock to her to learn afterward that he was but two years older
than her mother.





Lily seldom saw her father by daylight. All day he was ``down
town''; and in winter it was long after nightfall when she heard
his fagged step on the stairs and his hand on the school-room
door. He would kiss her in silence, and ask one or two questions
of the nurse or the governess; then Mrs.\ Bart's maid would
come to remind him that he was dining out, and he would hurry
away with a nod to Lily. In summer, when he joined them for a
Sunday at Newport or Southampton, he was even more effaced and
silent than in winter. It seemed to tire him to rest, and he
would sit for hours staring at the sea-line from a quiet corner
of the verandah, while the clatter of his wife's existence went
on unheeded a few feet off. Generally, however, Mrs.\ Bart and
Lily went to Europe for the summer, and before the steamer was
half way over Mr.\ Bart had dipped below the horizon. Sometimes
his daughter heard him denounced for having neglected to forward
Mrs.\ Bart's remittances; but for the most part he was never
mentioned or thought of till his patient stooping figure
presented itself on the New York dock as a buffer between the
magnitude of his wife's luggage and the restrictions of the
American custom-house.





In this desultory yet agitated fashion life went on through
Lily's teens: a zig-zag broken course down which the family craft
glided on a rapid current of amusement, tugged at by the
underflow of a perpetual need---the need of more money. Lily could
not recall the time when there had been money enough, and in some
vague way her father seemed always to blame for the deficiency. 
It could certainly not be the fault of Mrs.\ Bart, who was spoken
of by her friends as a ``wonderful manager.'' Mrs.\ Bart was famous
for the unlimited effect she produced on limited means; and to
the lady and her acquaintances there was something heroic in
living as though one were much richer than one's bank-book
denoted.





Lily was naturally proud of her mother's aptitude in this line: 
she had been brought up in the faith that, whatever it cost, one
must have a good cook, and be what Mrs.\ Bart called ``decently
dressed.'' Mrs.\ Bart's worst reproach to her husband was to ask
him if he expected her to ``live like a pig''; and his replying in
the negative was always regarded as a justification for cabling
to Paris for an extra dress or two, and telephoning to the
jeweller that he might, after all, send home the turquoise
bracelet which Mrs.\ Bart had looked at that morning.





Lily knew people who ``lived like pigs,''\ and their appearance and
surroundings justified her mother's repugnance to that
form of existence. They were mostly cousins, who inhabited dingy
houses with engravings from Cole's Voyage of Life on the
drawing-room walls, and slatternly parlour-maids who said ``I'll
go and see''\ to visitors calling at an hour when all right-minded
persons are conventionally if not actually out. The disgusting
part of it was that many of these cousins were rich, so that Lily
imbibed the idea that if people lived like pigs it was from
choice, and through the lack of any proper standard of conduct. 
This gave her a sense of reflected superiority, and she did not
need Mrs.\ Bart's comments on the family frumps and misers to
foster her naturally lively taste for splendour.





Lily was nineteen when circumstances caused her to revise her
view of the universe.





The previous year she had made a dazzling debut fringed by a
heavy thunder-cloud of bills. The light of the debut still
lingered on the horizon, but the cloud had thickened; and
suddenly it broke. The suddenness added to the horror; and there
were still times when Lily relived with painful vividness every
detail of the day on which the blow fell. She and her mother had
been seated at the luncheon-table, over the CHAUFROIX and cold
salmon of the previous night's dinner: it was one of Mrs.\ Bart's
few economies to consume in private the expensive remnants of her
hospitality. Lily was feeling the pleasant languor which is
youth's penalty for dancing till dawn; but her mother, in spite
of a few lines about the mouth, and under the yellow waves on her
temples, was as alert, determined and high in colour as if she
had risen from an untroubled sleep.





In the centre of the table, between the melting MARRONS GLACES
and candied cherries, a pyramid of American Beauties lifted their
vigorous stems; they held their heads as high as Mrs.\ Bart, but
their rose-colour had turned to a dissipated purple, and Lily's
sense of fitness was disturbed by their reappearance on the
luncheon-table.





``I really think, mother,''\ she said reproachfully, ``we might
afford a few fresh flowers for luncheon. Just some jonquils or
lilies-of-the-valley----''





Mrs.\ Bart stared. Her own fastidiousness had its eye fixed on the
world, and she did not care how the luncheon-table looked
when there was no one present at it but the family. But she
smiled at her daughter's innocence.





``Lilies-of-the-valley,''\ she said calmly, ``cost two dollars a
dozen at this season.''





Lily was not impressed. She knew very little of the value of
money.





``It would not take more than six dozen to fill that bowl,''\ she
argued.





``Six dozen what?''\ asked her father's voice in the doorway.





The two women looked up in surprise; though it was a Saturday,
the sight of Mr.\ Bart at luncheon was an unwonted one. But
neither his wife nor his daughter was sufficiently interested to
ask an explanation.





Mr.\ Bart dropped into a chair, and sat gazing absently at the
fragment of jellied salmon which the butler had placed before
him.





``I was only saying,''\ Lily began, ``that I hate to see faded
flowers at luncheon; and mother says a bunch of lilies-of-the-valley would not cost more than twelve dollars. Mayn't I tell the
florist to send a few every day?''





She leaned confidently toward her father: he seldom refused her
anything, and Mrs.\ Bart had taught her to plead with him when her
own entreaties failed.





Mr.\ Bart sat motionless, his gaze still fixed on the salmon, and
his lower jaw dropped; he looked even paler than usual, and his
thin hair lay in untidy streaks on his forehead. Suddenly he
looked at his daughter and laughed. The laugh was so strange that
Lily coloured under it: she disliked being ridiculed, and her
father seemed to see something ridiculous in the request. Perhaps
he thought it foolish that she should trouble him about such a
trifle.





``Twelve dollars---twelve dollars a day for flowers? Oh, certainly,
my dear---give him an order for twelve hundred.'' He continued to
laugh.





Mrs.\ Bart gave him a quick glance.





``You needn't wait, Poleworth---I will ring for you,''\ she said to
the butler.





The butler withdrew with an air of silent disapproval, leaving
the remains of the CHAUFROIX on the sideboard.





``What is the matter, Hudson? Are you ill?''\ said Mrs.\ Bart
severely.





She had no tolerance for scenes which were not of her own making,
and it was odious to her that her husband should make a show of
himself before the servants.





``Are you ill?''\ she repeated.





``Ill?----No, I'm ruined,''\ he said.





Lily made a frightened sound, and Mrs.\ Bart rose to her
feet.





``Ruined----?''\ she cried; but controlling herself instantly, she
turned a calm face to Lily.





``Shut the pantry door,''\ she said.





Lily obeyed, and when she turned back into the room her father
was sitting with both elbows on the table, the plate of salmon
between them, and his head bowed on his hands.





Mrs.\ Bart stood over him with a white face which made her hair
unnaturally yellow. She looked at Lily as the latter approached: 
her look was terrible, but her voice was modulated to a ghastly
cheerfulness.





``Your father is not well---he doesn't know what he is saying. It
is nothing---but you had better go upstairs; and don't talk to the
servants,''\ she added.





Lily obeyed; she always obeyed when her mother spoke in that
voice. She had not been deceived by Mrs.\ Bart's words: she knew
at once that they were ruined. In the dark hours which followed,
that awful fact overshadowed even her father's slow and difficult
dying. To his wife he no longer counted: he had become extinct
when he ceased to fulfil his purpose, and she sat at his side
with the provisional air of a traveller who waits for a belated
train to start. Lily's feelings were softer: she pitied him in a
frightened ineffectual way. But the fact that he was for the most
part unconscious, and that his attention, when she stole into the
room, drifted away from her after a moment, made him even more of
a stranger than in the nursery days when he had never come home
till after dark. She seemed always to have seen him through a
blur---first of sleepiness, then of distance and indifference---
and now the fog had thickened till he was almost
indistinguishable. If she could have performed any little
services for him, or have exchanged with him a few of those
affecting words which an extensive perusal of fiction had
led her to connect with such occasions, the filial instinct might
have stirred in her; but her pity, finding no active expression,
remained in a state of spectatorship, overshadowed by her
mother's grim unflagging resentment. Every look and act of Mrs.
Bart's seemed to say: ``You are sorry for him now---but you will
feel differently when you see what he has done to us.''





It was a relief to Lily when her father died.





Then a long winter set in. There was a little money left, but to
Mrs.\ Bart it seemed worse than nothing---the mere mockery of what
she was entitled to. What was the use of living if one had to
live like a pig? She sank into a kind of furious apathy, a state
of inert anger against fate. Her faculty for ``managing''\ deserted
her, or she no longer took sufficient pride in it to exert it. It
was well enough to ``manage''\ when by so doing one could keep one's
own carriage; but when one's best contrivance did not conceal the
fact that one had to go on foot, the effort was no longer worth
making.





Lily and her mother wandered from place to place, now paying long
visits to relations whose house-keeping Mrs.\ Bart criticized, and
who deplored the fact that she let Lily breakfast in bed when the
girl had no prospects before her, and now vegetating in cheap
continental refuges, where Mrs.\ Bart held herself fiercely aloof
from the frugal tea-tables of her companions in misfortune. She
was especially careful to avoid her old friends and the scenes of
her former successes. To be poor seemed to her such a confession
of failure that it amounted to disgrace; and she detected a note
of condescension in the friendliest advances.





Only one thought consoled her, and that was the contemplation of
Lily's beauty. She studied it with a kind of passion, as though
it were some weapon she had slowly fashioned for her vengeance. 
It was the last asset in their fortunes, the nucleus around which
their life was to be rebuilt. She watched it jealously, as though
it were her own property and Lily its mere custodian; and she
tried to instil into the latter a sense of the responsibility
that such a charge involved. She followed in imagination the
career of other beauties, pointing out to her daughter what might
be achieved through such a gift, and dwelling on the awful
warning of those who, in spite of it, had failed to get what they
wanted: to Mrs.\ Bart, only stupidity could explain the lamentable
denouement of some of her examples. She was not above the
inconsistency of charging fate, rather than herself, with her own
misfortunes; but she inveighed so acrimoniously against
love-matches that Lily would have fancied her own marriage had
been of that nature, had not Mrs.\ Bart frequently assured her
that she had been ``talked into it''---by whom, she never made
clear.





Lily was duly impressed by the magnitude of her opportunities. 
The dinginess of her present life threw into enchanting relief
the existence to which she felt herself entitled. To a less
illuminated intelligence Mrs.\ Bart's counsels might have been
dangerous; but Lily understood that beauty is only the raw
material of conquest, and that to convert it into success other
arts are required. She knew that to betray any sense of
superiority was a subtler form of the stupidity her mother
denounced, and it did not take her long to learn that a beauty
needs more tact than the possessor of an average set of features.





Her ambitions were not as crude as Mrs.\ Bart's. It had been among
that lady's grievances that her husband---in the early days,
before he was too tired---had wasted his evenings in what she
vaguely described as ``reading poetry''; and among the effects
packed off to auction after his death were a score or two of
dingy volumes which had struggled for existence among the boots
and medicine bottles of his dressing-room shelves. There was in
Lily a vein of sentiment, perhaps transmitted from this source,
which gave an idealizing touch to her most prosaic purposes. She
liked to think of her beauty as a power for good, as giving her
the opportunity to attain a position where she should make her
influence felt in the vague diffusion of refinement and good
taste. She was fond of pictures and flowers, and of sentimental
fiction, and she could not help thinking that the possession of
such tastes ennobled her desire for worldly advantages. She would
not indeed have cared to marry a man who was merely rich: she
was secretly ashamed of her mother's crude passion for money. 
Lily's preference would have been for an English nobleman with
political ambitions and vast estates; or, for second
choice, an Italian prince with a castle in the Apennines and an
hereditary office in the Vatican. Lost causes had a romantic
charm for her, and she liked to picture herself as standing aloof
from the vulgar press of the Quirinal, and sacrificing her
pleasure to the claims of an immemorial tradition.\ .\ .\ .





How long ago and how far off it all seemed! Those ambitions were
hardly more futile and childish than the earlier ones which had
centred about the possession of a French jointed doll with real
hair. Was it only ten years since she had wavered in imagination
between the English earl and the Italian prince? Relentlessly her
mind travelled on over the dreary interval.\ .\ .\ .





After two years of hungry roaming Mrs.\ Bart had died----died of a
deep disgust. She had hated dinginess, and it was her fate to be
dingy. Her visions of a brilliant marriage for Lily had faded
after the first year.





``People can't marry you if they don't see you---and how can they
see you in these holes where we're stuck?''\ That was the burden of
her lament; and her last adjuration to her daughter was to escape
from dinginess if she could.





``Don't let it creep up on you and drag you down. Fight your way
out of it somehow---you're young and can do it,''\ she insisted.





She had died during one of their brief visits to New York, and
there Lily at once became the centre of a family council composed
of the wealthy relatives whom she had been taught to despise for
living like pigs. It may be that they had an inkling of the
sentiments in which she had been brought up, for none of them
manifested a very lively desire for her company; indeed, the
question threatened to remain unsolved till Mrs.\ Peniston with a
sigh announced: ``I'll try her for a year.''





Every one was surprised, but one and all concealed their
surprise, lest Mrs.\ Peniston should be alarmed by it into
reconsidering her decision.





Mrs.\ Peniston was Mr.\ Bart's widowed sister, and if she was by no
means the richest of the family group, its other members
nevertheless abounded in reasons why she was clearly destined by
Providence to assume the charge of Lily. In the first place she
was alone, and it would be charming for her to have a
young companion. Then she sometimes travelled, and Lily's
familiarity with foreign customs---deplored as a misfortune by her
more conservative relatives---would at least enable her to act as
a kind of courier. But as a matter of fact Mrs.\ Peniston had not
been affected by these considerations. She had taken the girl
simply because no one else would have her, and because she had
the kind of moral MAUVAISE HONTE which makes the public display
of selfishness difficult, though it does not interfere with its
private indulgence. It would have been impossible for Mrs.
Peniston to be heroic on a desert island, but with the eyes of
her little world upon her she took a certain pleasure in her act.





She reaped the reward to which disinterestedness is entitled, and
found an agreeable companion in her niece. She had expected to
find Lily headstrong, critical and ``foreign''---for even Mrs.
Peniston, though she occasionally went abroad, had the family
dread of foreignness---but the girl showed a pliancy, which, to a
more penetrating mind than her aunt's, might have been less
reassuring than the open selfishness of youth. Misfortune had
made Lily supple instead of hardening her, and a pliable
substance is less easy to break than a stiff one.





Mrs.\ Peniston, however, did not suffer from her niece's
adaptability. Lily had no intention of taking advantage of her
aunt's good nature. She was in truth grateful for the refuge
offered her: Mrs.\ Peniston's opulent interior was at least not
externally dingy. But dinginess is a quality which assumes all
manner of disguises; and Lily soon found that it was as latent in
the expensive routine of her aunt's life as in the makeshift
existence of a continental pension.





Mrs.\ Peniston was one of the episodical persons who form the
padding of life. It was impossible to believe that she had
herself ever been a focus of activities. The most vivid thing
about her was the fact that her grandmother had been a Van
Alstyne. This connection with the well-fed and industrious stock
of early New York revealed itself in the glacial neatness of Mrs.
Peniston's drawing-room and in the excellence of her cuisine. She
belonged to the class of old New Yorkers who have always lived
well, dressed expensively, and done little else; and to these
inherited obligations Mrs.\ Peniston faitfully conformed. 
She had always been a looker-on at life, and her mind resembled
one of those little mirrors which her Dutch ancestors were
accustomed to affix to their upper windows, so that from the
depths of an impenetrable domesticity they might see what was
happening in the street.





Mrs.\ Peniston was the owner of a country-place in New Jersey, but
she had never lived there since her husband's death---a remote
event, which appeared to dwell in her memory chiefly as a
dividing point in the personal reminiscences that formed the
staple of her conversation. She was a woman who remembered dates
with intensity, and could tell at a moment's notice whether the
drawing-room curtains had been renewed before or after Mr.
Peniston's last illness.





Mrs.\ Peniston thought the country lonely and trees damp, and
cherished a vague fear of meeting a bull. To guard against such
contingencies she frequented the more populous watering-places,
where she installed herself impersonally in a hired house and
looked on at life through the matting screen of her verandah. In
the care of such a guardian, it soon became clear to Lily that
she was to enjoy only the material advantages of good food and
expensive clothing; and, though far from underrating these, she
would gladly have exchanged them for what Mrs.\ Bart had taught
her to regard as opportunities. She sighed to think what her
mother's fierce energies would have accomplished, had they been
coupled with Mrs.\ Peniston's resources. Lily had abundant energy
of her own, but it was restricted by the necessity of adapting
herself to her aunt's habits. She saw that at all costs she must
keep Mrs.\ Peniston's favour till, as Mrs.\ Bart would have phrased
it, she could stand on her own legs. Lily had no mind for the
vagabond life of the poor relation, and to adapt herself to Mrs.
Peniston she had, to some degree, to assume that lady's passive
attitude. She had fancied at first that it would be easy to draw
her aunt into the whirl of her own activities, but there was a
static force in Mrs.\ Peniston against which her niece's efforts
spent themselves in vain. To attempt to bring her into active
relation with life was like tugging at a piece of furniture which
has been screwed to the floor. She did not, indeed, expect Lily
to remain equally immovable: she had all the American guardian's
indulgence for the volatility of youth.





She had indulgence also for certain other habits of her niece's. 
It seemed to her natural that Lily should spend all her money on
dress, and she supplemented the girl's scanty income by
occasional ``handsome presents''\ meant to be applied to the same
purpose. Lily, who was intensely practical, would have preferred
a fixed allowance; but Mrs.\ Peniston liked the periodical
recurrence of gratitude evoked by unexpected cheques, and was
perhaps shrewd enough to perceive that such a method of giving
kept alive in her niece a salutary sense of dependence.





Beyond this, Mrs.\ Peniston had not felt called upon to do
anything for her charge: she had simply stood aside and let her
take the field. Lily had taken it, at first with the confidence
of assured possessorship, then with gradually narrowing demands,
till now she found herself actually struggling for a foothold on
the broad space which had once seemed her own for the asking. How
it happened she did not yet know. Sometimes she thought it was
because Mrs.\ Peniston had been too passive, and again she feared
it was because she herself had not been passive enough. Had she
shown an undue eagerness for victory? Had she lacked patience,
pliancy and dissimulation? Whether she charged herself with these
faults or absolved herself from them, made no difference in the
sum-total of her failure. Younger and plainer girls had been
married off by dozens, and she was nine-and-twenty, and still
Miss Bart.





She was beginning to have fits of angry rebellion against fate,
when she longed to drop out of the race and make an independent
life for herself. But what manner of life would it be? She had
barely enough money to pay her dress-makers'\ bills and her
gambling debts; and none of the desultory interests which she
dignified with the name of tastes was pronounced enough to enable
her to live contentedly in obscurity. Ah, no---she was too
intelligent not to be honest with herself. She knew that she
hated dinginess as much as her mother had hated it, and to her
last breath she meant to fight against it, dragging herself up
again and again above its flood till she gained the bright
pinnacles of success which presented such a slippery surface to
her clutch.





\chapter*{\raggedright Chapter 4}

\addcontentsline{toc}{chapter}{Chapter 4}

\markboth{House of Mirth}{Chapter 4}





The next morning, on her breakfast tray, Miss Bart found a note
from her hostess.





``Dearest Lily,''\ it ran, ``if it is not too much of a bore to be
down by ten, will you come to my sitting-room to help me with
some tiresome things?''





Lily tossed aside the note and subsided on her pillows with a
sigh. It \textit{was} a bore to be down by ten---an hour regarded at
Bellomont as vaguely synchronous with sunrise---and she knew too
well the nature of the tiresome things in question. Miss Pragg,
the secretary, had been called away, and there would be notes and
dinner-cards to write, lost addresses to hunt up, and other
social drudgery to perform. It was understood that Miss Bart
should fill the gap in such emergencies, and she usually
recognized the obligation without a murmur.





Today, however, it renewed the sense of servitude which the
previous night's review of her cheque-book had produced. 
Everything in her surroundings ministered to feelings of ease and
amenity. The windows stood open to the sparkling freshness of the
September morning, and between the yellow boughs she caught a
perspective of hedges and parterres leading by degrees of
lessening formality to the free undulations of the park. Her maid
had kindled a little fire on the hearth, and it contended
cheerfully with the sunlight which slanted across the moss-green
carpet and caressed the curved sides of an old marquetry desk. 
Near the bed stood a table holding her breakfast tray, with its
harmonious porcelain and silver, a handful of violets in a
slender glass, and the morning paper folded beneath her letters. 
There was nothing new to Lily in these tokens of a studied
luxury; but, though they formed a part of her atmosphere, she
never lost her sensitiveness to their charm. Mere display left
her with a sense of superior distinction; but she felt an
affinity to all the subtler manifestations of wealth.





Mrs.\ Trenor's summons, however, suddenly recalled her state of
dependence, and she rose and dressed in a mood of irritability
that she was usually too prudent to indulge. She knew that such
emotions leave lines on the face as well as in the
character, and she had meant to take warning by the little
creases which her midnight survey had revealed.





The matter-of-course tone of Mrs.\ Trenor's greeting deepened her
irritation. If one did drag one's self out of bed at such an
hour, and come down fresh and radiant to the monotony of
note-writing, some special recognition of the sacrifice seemed
fitting. But Mrs.\ Trenor's tone showed no consciousness of the
fact.





``Oh, Lily, that's nice of you,''\ she merely sighed across the
chaos of letters, bills and other domestic documents which gave
an incongruously commercial touch to the slender elegance of her
writing-table.





``There are such lots of horrors this morning,''\ she added,
clearing a space in the centre of the confusion and rising to
yield her seat to Miss Bart.





Mrs.\ Trenor was a tall fair woman, whose height just saved her
from redundancy. Her rosy blondness had survived some forty years
of futile activity without showing much trace of ill-usage except
in a diminished play of feature. It was difficult to define her
beyond saying that she seemed to exist only as a hostess, not so
much from any exaggerated instinct of hospitality as because she
could not sustain life except in a crowd. The collective nature
of her interests exempted her from the ordinary rivalries of her
sex, and she knew no more personal emotion than that of hatred
for the woman who presumed to give bigger dinners or have more
amusing house-parties than herself. As her social talents, backed
by Mr.\ Trenor's bank-account, almost always assured her ultimate
triumph in such competitions, success had developed in her an
unscrupulous good nature toward the rest of her sex, and in Miss
Bart's utilitarian classification of her friends, Mrs.\ Trenor
ranked as the woman who was least likely to ``go back''\ on her.





``It was simply inhuman of Pragg to go off now,''\ Mrs.\ Trenor
declared, as her friend seated herself at the desk. ``She says her
sister is going to have a baby---as if that were anything to
having a house-party! I'm sure I shall get most horribly mixed up
and there will be some awful rows. When I was down at Tuxedo I
asked a lot of people for next week, and I've mislaid the list
and can't remember who is coming. And this week is going to be a
horrid failure too---and Gwen Van Osburgh will go back and
tell her mother how bored people were. I did mean to ask the
Wetheralls---that was a blunder of Gus's. They disapprove of Carry
Fisher, you know. As if one could help having Carry Fisher! It
\textit{was} foolish of her to get that second divorce---Carry always
overdoes things---but she said the only way to get a penny out of
Fisher was to divorce him and make him pay alimony. And poor
Carry has to consider every dollar. It's really absurd of Alice
Wetherall to make such a fuss about meeting her, when one thinks
of what society is coming to. Some one said the other day that
there was a divorce and a case of appendicitis in every family
one knows. Besides, Carry is the only person who can keep Gus in
a good humour when we have bores in the house. Have you noticed
that \textit{all} the husbands like her? All, I mean, except her own. It's
rather clever of her to have made a specialty of devoting herself
to dull people---the field is such a large one, and she has it
practically to herself. She finds compensations, no doubt---I know
she borrows money of Gus---but then I'd \textit{pay} her to keep him in a
good humour, so I can't complain, after all.''





Mrs.\ Trenor paused to enjoy the spectacle of Miss Bart's efforts
to unravel her tangled correspondence.





``But it is only the Wetheralls and Carry,''\ she resumed, with a
fresh note of lament. ``The truth is, I'm awfully disappointed in
Lady Cressida Raith.''





``Disappointed? Had you known her before?''





``Mercy, no---never saw her till yesterday. Lady Skiddaw sent her
over with letters to the Van Osburghs, and I heard that Maria Van
Osburgh was asking a big party to meet her this week, so I
thought it would be fun to get her away, and Jack Stepney, who
knew her in India, managed it for me. Maria was furious, and
actually had the impudence to make Gwen invite herself here, so
that they shouldn't be \textit{quite} out of it---if I'd known what Lady
Cressida was like, they could have had her and welcome! But I
thought any friend of the Skiddaws'\ was sure to be amusing. You
remember what fun Lady Skiddaw was? There were times when I
simply had to send the girls out of the room. Besides, Lady
Cressida is the Duchess of Beltshire's sister, and I naturally
supposed she was the same sort; but you never can tell in those
English families. They are so big that there's room for
all kinds, and it turns out that Lady Cressida is the moral
one---married a clergy-man and does missionary work in the East
End. Think of my taking such a lot of trouble about a clergyman's
wife, who wears Indian jewelry and botanizes! She made Gus take
her all through the glass-houses yesterday, and bothered him to
death by asking him the names of the plants. Fancy treating Gus
as if he were the gardener!''





Mrs.\ Trenor brought this out in a \textit{Crescendo} of indignation.





``Oh, well, perhaps Lady Cressida will reconcile the Wetheralls to
meeting Carry Fisher,''\ said Miss Bart pacifically.





``I'm sure I hope so! But she is boring all the men horribly, and
if she takes to distributing tracts, as I hear she does, it will
be too depressing. The worst of it is that she would have been so
useful at the right time. You know we have to have the Bishop
once a year, and she would have given just the right tone to
things. I always have horrid luck about the Bishop's visits,''
added Mrs.\ Trenor, whose present misery was being fed by a
rapidly rising tide of reminiscence; ``last year, when he came,
Gus forgot all about his being here, and brought home the Ned
Wintons and the Farleys---five divorces and six sets of children
between them!''





``When is Lady Cressida going?''\ Lily enquired.





Mrs.\ Trenor cast up her eyes in despair. ``My dear, if one only
knew! I was in such a hurry to get her away from Maria that I
actually forgot to name a date, and Gus says she told some one
she meant to stop here all winter.''





``To stop here? In this house?''





``Don't be silly---in America. But if no one else asks her---you
know they \textit{never} go to hotels.''





``Perhaps Gus only said it to frighten you.''





``No---I heard her tell Bertha Dorset that she had six months to
put in while her husband was taking the cure in the Engadine. You
should have seen Bertha look vacant! But it's no joke, you
know---if she stays here all the autumn she'll spoil everything,
and Maria Van Osburgh will simply exult.''





At this affecting vision Mrs.\ Trenor's voice trembled with
self-pity.





``Oh, Judy---as if any one were ever bored at Bellomont!''
Miss Bart tactfully protested. ``You know perfectly well that,
if Mrs.\ Van Osburgh were to get all the right people and leave you
with all the wrong ones, you'd manage to make things go off,
and she wouldn't.''





Such an assurance would usually have restored Mrs.\ Trenor's
complacency; but on this occasion it did not chase the cloud from
her brow.





``It isn't only Lady Cressida,''\ she lamented. ``Everything has gone
wrong this week. I can see that Bertha Dorset is furious with
me.''





``Furious with you? Why?''





``Because I told her that Lawrence Selden was coming; but he
wouldn't, after all, and she's quite unreasonable enough to think
it's my fault.''





Miss Bart put down her pen and sat absently gazing at the note
she had begun.





``I thought that was all over,''\ she said.





``So it is, on his side. And of course Bertha has been idle since. 
But I fancy she's out of a job just at present---and some one gave
me a hint that I had better ask Lawrence. Well, I \textit{did} ask
him---but I couldn't make him come; and now I suppose she'll take
it out of me by being perfectly nasty to every one else.''





``Oh, she may take it out of \textit{him} by being perfectly charming---to
some one else.''





Mrs.\ Trenor shook her head dolefully. ``She knows he wouldn't
mind. And who else is there? Alice Wetherall won't let Lucius out
of her sight. Ned Silverton can't take his eyes off Carry
Fisher---poor boy! Gus is bored by Bertha, Jack Stepney knows her
too well---and---well, to be sure, there's Percy Gryce!''





She sat up smiling at the thought.





Miss Bart's countenance did not reflect the smile.





``Oh, she and Mr.\ Gryce would not be likely to hit it off.''





``You mean that she'd shock him and he'd bore her? Well, that's
not such a bad beginning, you know. But I hope she won't take it
into her head to be nice to him, for I asked him here on purpose
for you.''





Lily laughed. ``MERCI \textit{Du} \textit{compliment}! I should certainly have no
show against Bertha.''





``Do you think I am uncomplimentary? I'm not really, you know. 
Every one knows you're a thousand times handsomer and cleverer
than Bertha; but then you're not nasty. And for always getting
what she wants in the long run, commend me to a nasty woman.''





Miss Bart stared in affected reproval. ``I thought you were so
fond of Bertha.''





``Oh, I am---it's much safer to be fond of dangerous people. But
she \textit{is} dangerous---and if I ever saw her up to mischief it's now. 
I can tell by poor George's manner. That man is a perfect
barometer---he always knows when Bertha is going to----''





``To fall?''\ Miss Bart suggested.





``Don't be shocking! You know he believes in her still. And of
course I don't say there's any real harm in Bertha. Only she
delights in making people miserable, and especially poor George.''





``Well, he seems cut out for the part---I don't wonder she likes
more cheerful companionship.''





``Oh, George is not as dismal as you think. If Bertha did worry
him he would be quite different. Or if she'd leave him alone, and
let him arrange his life as he pleases. But she doesn't dare lose
her hold of him on account of the money, and so when \textit{he} isn't
jealous she pretends to be.''





Miss Bart went on writing in silence, and her hostess sat
following her train of thought with frowning intensity.





``Do you know,''\ she exclaimed after a long pause, ``I believe I'll
call up Lawrence on the telephone and tell him he simply \textit{must}
come?''





``Oh, don't,''\ said Lily, with a quick suffusion of colour. The
blush surprised her almost as much as it did her hostess, who,
though not commonly observant of facial changes, sat staring at
her with puzzled eyes.





``Good gracious, Lily, how handsome you are! Why? Do you dislike
him so much?''





``Not at all; I like him. But if you are actuated by the
benevolent intention of protecting me from Bertha---I don't think
I need your protection.''





Mrs.\ Trenor sat up with an exclamation. ``Lily!----\textit{Percy}? Do you
mean to say you've actually done it?''





Miss Bart smiled. ``I only mean to say that Mr.\ Gryce and I are
getting to be very good friends.''





``H'm---I see.'' Mrs.\ Trenor fixed a rapt eye upon her. ``You know
they say he has eight hundred thousand a year---and spends
nothing, except on some rubbishy old books. And his mother has
heart-disease and will leave him a lot more. \textit{Oh}, \textit{lily}, \textit{do} \textit{go}
\textit{slowly},''\ her friend adjured her.





Miss Bart continued to smile without annoyance. ``I shouldn't, for
instance,''\ she remarked, ``be in any haste to tell him that he had
a lot of rubbishy old books.''





``No, of course not; I know you're wonderful about getting up
people's subjects. But he's horribly shy, and easily shocked,
and---and----''





``Why don't you say it, Judy? I have the reputation of being on
the hunt for a rich husband?''





``Oh, I don't mean that; he wouldn't believe it of you---at first,''
said Mrs.\ Trenor, with candid shrewdness. ``But you know things
are rather lively here at times---I must give Jack and Gus a
hint---and if he thought you were what his mother would call
fast---oh, well, you know what I mean. Don't wear your scarlet
\textit{crepe}-\textit{de}-\textit{Chine} for dinner, and don't smoke if you can help it,
Lily dear!''





Lily pushed aside her finished work with a dry smile. ``You're very
kind, Judy: I'll lock up my cigarettes and wear that last year's
dress you sent me this morning. And if you are really interested
in my career, perhaps you'll be kind enough not to ask me to play
bridge again this evening.''





``Bridge? Does he mind bridge, too? Oh, Lily, what an awful life
you'll lead! But of course I won't---why didn't you give me a hint
last night? There's nothing I wouldn't do, you poor duck, to see
you happy!''





And Mrs.\ Trenor, glowing with her sex's eagerness to smooth the
course of true love, enveloped Lily in a long embrace.





``You're quite sure,''\ she added solicitously, as the latter
extricated herself, ``that you wouldn't like me to telephone for
Lawrence Selden?''





``Quite sure,''\ said Lily.







The next three days demonstrated to her own complete satisfaction


Miss Bart's ability to manage her affairs without extraneous aid.





As she sat, on the Saturday afternoon, on the terrace at
Bellomont, she smiled at Mrs.\ Trenor's fear that she might go too
fast. If such a warning had ever been needful, the years had
taught her a salutary lesson, and she flattered herself that she
now knew how to adapt her pace to the object of pursuit. In the
case of Mr.\ Gryce she had found it well to flutter ahead, losing
herself elusively and luring him on from depth to depth of
unconscious intimacy. The surrounding atmosphere was propitious
to this scheme of courtship. Mrs.\ Trenor, true to her word, had
shown no signs of expecting Lily at the bridge-table, and had
even hinted to the other card-players that they were to betray no
surprise at her unwonted defection. In consequence of this hint,
Lily found herself the centre of that feminine solicitude which
envelops a young woman in the mating season. A solitude was
tacitly created for her in the crowded existence of Bellomont,
and her friends could not have shown a greater readiness for
self-effacement had her wooing been adorned with all the
attributes of romance. In Lily's set this conduct implied a
sympathetic comprehension of her motives, and Mr.\ Gryce rose in
her esteem as she saw the consideration he inspired.





The terrace at Bellomont on a September afternoon was a spot
propitious to sentimental musings, and as Miss Bart stood leaning
against the balustrade above the sunken garden, at a little
distance from the animated group about the tea-table, she might
have been lost in the mazes of an inarticulate happiness. In
reality, her thoughts were finding definite utterance in the
tranquil recapitulation of the blessings in store for her. From
where she stood she could see them embodied in the form of Mr.
Gryce, who, in a light overcoat and muffler, sat somewhat
nervously on the edge of his chair, while Carry Fisher, with all
the energy of eye and gesture with which nature and art had
combined to endow her, pressed on him the duty of taking part in
the task of municipal reform.





Mrs.\ Fisher's latest hobby was municipal reform. It had been
preceded by an equal zeal for socialism, which had in turn
replaced an energetic advocacy of Christian Science. Mrs.\ Fisher
was small, fiery and dramatic; and her hands and eyes were
admirable instruments in the service of whatever causes he
happened to espouse. She had, however, the fault common to
enthusiasts of ignoring any slackness of response on the part of
her hearers, and Lily was amused by her unconsciousness of the
resistance displayed in every angle of Mr.\ Gryce's attitude. Lily
herself knew that his mind was divided between the dread of
catching cold if he remained out of doors too long at that hour,
and the fear that, if he retreated to the house, Mrs.\ Fisher
might follow him up with a paper to be signed. Mr.\ Gryce had a
constitutional dislike to what he called ``committing himself,''
and tenderly as he cherished his health, he evidently concluded
that it was safer to stay out of reach of pen and ink till chance
released him from Mrs.\ Fisher's toils. Meanwhile he cast agonized
glances in the direction of Miss Bart, whose only response was to
sink into an attitude of more graceful abstraction. She had
learned the value of contrast in throwing her charms into relief,
and was fully aware of the extent to which Mrs.\ Fisher's
volubility was enhancing her own repose.





She was roused from her musings by the approach of her cousin
Jack Stepney who, at Gwen Van Osburgh's side, was returning
across the garden from the tennis court.





The couple in question were engaged in the same kind of romance
in which Lily figured, and the latter felt a certain annoyance in
contemplating what seemed to her a caricature of her own
situation. Miss Van Osburgh was a large girl with flat surfaces
and no high lights: Jack Stepney had once said of her that she
was as reliable as roast mutton. His own taste was in the line of
less solid and more highly-seasoned diet; but hunger makes any
fare palatable, and there had been times when Mr.\ Stepney had
been reduced to a crust.





Lily considered with interest the expression of their faces: the
girl's turned toward her companion's like an empty plate held up
to be filled, while the man lounging at her side already betrayed
the encroaching boredom which would presently crack the thin
veneer of his smile.





``How impatient men are!''\ Lily reflected. ``All Jack has to do to
get everything he wants is to keep quiet and let that girl marry
him; whereas I have to calculate and contrive, and retreat and
advance, as if I were going through an intricate dance,
where one misstep would throw me hopelessly out of time.''





As they drew nearer she was whimsically struck by a kind of
family likeness between Miss Van Osburgh and Percy Gryce. There
was no resemblance of feature. Gryce was handsome in a didactic
way---he looked like a clever pupil's drawing from a
plaster-cast---while Gwen's countenance had no more modelling than
a face painted on a toy balloon. But the deeper affinity was
unmistakable: the two had the same prejudices and ideals, and the
same quality of making other standards non-existent by ignoring
them. This attribute was common to most of Lily's set: they had a
force of negation which eliminated everything beyond their own
range of perception. Gryce and Miss Van Osburgh were, in short,
made for each other by every law of moral and physical
correspondence----``Yet they wouldn't look at each other,''\ Lily
mused, ``they never do. Each of them wants a creature of a
different race, of Jack's race and mine, with all sorts of
intuitions, sensations and perceptions that they don't even guess
the existence of. And they always get what they want.''





She stood talking with her cousin and Miss Van Osburgh, till a
slight cloud on the latter's brow advised her that even cousinly
amenities were subject to suspicion, and Miss Bart, mindful of
the necessity of not exciting enmities at this crucial point of
her career, dropped aside while the happy couple proceeded toward
the tea-table.





Seating herself on the upper step of the terrace, Lily leaned her
head against the honeysuckles wreathing the balustrade. The
fragrance of the late blossoms seemed an emanation of the
tranquil scene, a landscape tutored to the last degree of rural
elegance. In the foreground glowed the warm tints of the gardens. 
Beyond the lawn, with its pyramidal pale-gold maples and velvety
firs, sloped pastures dotted with cattle; and through a long
glade the river widened like a lake under the silver light of
September. Lily did not want to join the circle about the
tea-table. They represented the future she had chosen, and she
was content with it, but in no haste to anticipate its joys. The
certainty that she could marry Percy Gryce when she pleased had
lifted a heavy load from her mind, and her money troubles were
too recent for their removal not to leave a sense of
relief which a less discerning intelligence might have taken for
happiness. Her vulgar cares were at an end. She would be able to
arrange her life as she pleased, to soar into that empyrean of
security where creditors cannot penetrate. She would have smarter
gowns than Judy Trenor, and far, far more jewels than Bertha
Dorset. She would be free forever from the shifts, the
expedients, the humiliations of the relatively poor. Instead of
having to flatter, she would be flattered; instead of being
grateful, she would receive thanks. There were old scores she
could pay off as well as old benefits she could return. And she
had no doubts as to the extent of her power. She knew that Mr.
Gryce was of the small chary type most inaccessible to impulses
and emotions. He had the kind of character in which prudence is a
vice, and good advice the most dangerous nourishment. But Lily
had known the species before: she was aware that such a guarded
nature must find one huge outlet of egoism, and she determined to
be to him what his Americana had hitherto been: the one
possession in which he took sufficient pride to spend money on
it. She knew that this generosity to self is one of the forms of
meanness, and she resolved so to identify herself with her
husband's vanity that to gratify her wishes would be to him the
most exquisite form of self-indulgence. The system might at first
necessitate a resort to some of the very shifts and expedients
from which she intended it should free her; but she felt sure
that in a short time she would be able to play the game in her
own way. How should she have distrusted her powers? Her beauty
itself was not the mere ephemeral possession it might have been
in the hands of inexperience: her skill in enhancing it, the care
she took of it, the use she made of it, seemed to give it a kind
of permanence. She felt she could trust it to carry her through
to the end.





And the end, on the whole, was worthwhile. Life was not the
mockery she had thought it three days ago. There was room for
her, after all, in this crowded selfish world of pleasure whence,
so short a time since, her poverty had seemed to exclude her. 
These people whom she had ridiculed and yet envied were glad to
make a place for her in the charmed circle about which all her
desires revolved. They were not as brutal and self-engrossed as
she had fancied---or rather, since it would no longer be
necessary to flatter and humour them, that side of their nature
became less conspicuous. Society is a revolving body which is apt
to be judged according to its place in each man's heaven; and at
present it was turning its illuminated face to Lily.





In the rosy glow it diffused her companions seemed full of
amiable qualities. She liked their elegance, their lightness,
their lack of emphasis: even the self-assurance which at times
was so like obtuseness now seemed the natural sign of social
ascendency. They were lords of the only world she cared for, and
they were ready to admit her to their ranks and let her lord it
with them. Already she felt within her a stealing allegiance to
their standards, an acceptance of their limitations, a disbelief
in the things they did not believe in, a contemptuous pity for
the people who were not able to live as they lived.





The early sunset was slanting across the park. Through the boughs
of the long avenue beyond the gardens she caught the flash of
wheels, and divined that more visitors were approaching. There
was a movement behind her, a scattering of steps and voices: it
was evident that the party about the tea-table was breaking up. 
Presently she heard a tread behind her on the terrace. She
supposed that Mr.\ Gryce had at last found means to escape from
his predicament, and she smiled at the significance of his coming
to join her instead of beating an instant retreat to the
fire-side.





She turned to give him the welcome which such gallantry deserved;
but her greeting wavered into a blush of wonder, for the man who
had approached her was Lawrence Selden.





``You see I came after all,''\ he said; but before she had time to
answer, Mrs.\ Dorset, breaking away from a lifeless colloquy with
her host, had stepped between them with a little gesture of
appropriation.





\chapter*{\raggedright Chapter 5}

\addcontentsline{toc}{chapter}{Chapter 5}

\markboth{House of Mirth}{Chapter 5}





The observance of Sunday at Bellomont was chiefly marked by the
punctual appearance of the smart omnibus destined to convey the
household to the little church at the gates. Whether any one got
into the omnibus or not was a matter of secondary importance,
since by standing there it not only bore witness to the orthodox
intentions of the family, but made Mrs.\ Trenor feel, when she
finally heard it drive away, that she had somehow vicariously
made use of it.





It was Mrs.\ Trenor's theory that her daughters actually did go to
church every Sunday; but their French governess's convictions
calling her to the rival fane, and the fatigues of the week
keeping their mother in her room till luncheon, there was seldom
any one present to verify the fact. Now and then, in a spasmodic
burst of virtue---when the house had been too uproarious over
night---Gus Trenor forced his genial bulk into a tight frock-coat
and routed his daughters from their slumbers; but habitually, as
Lily explained to Mr.\ Gryce, this parental duty was forgotten
till the church bells were ringing across the park, and the
omnibus had driven away empty.





Lily had hinted to Mr.\ Gryce that this neglect of religious
observances was repugnant to her early traditions, and that
during her visits to Bellomont she regularly accompanied Muriel
and Hilda to church. This tallied with the assurance, also
confidentially imparted, that, never having played bridge
before, she had been ``dragged into it''\ on the night of her
arrival, and had lost an appalling amount of money in
consequence of her ignorance of the game and of the rules of
betting. Mr.\ Gryce was undoubtedly enjoying Bellomont. He liked
the ease and glitter of the life, and the lustre conferred on him
by being a member of this group of rich and conspicuous people. 
But he thought it a very materialistic society; there were times
when he was frightened by the talk of the men and the looks of
the ladies, and he was glad to find that Miss Bart, for all her
ease and self-possession, was not at home in so ambiguous an
atmosphere. For this reason he had been especially pleased to
learn that she would, as usual, attend the young Trenors
to church on Sunday morning; and as he paced the gravel sweep
before the door, his light overcoat on his arm and his
prayer-book in one carefully-gloved hand, he reflected agreeably
on the strength of character which kept her true to her early
training in surroundings so subversive to religious principles.





For a long time Mr.\ Gryce and the omnibus had the gravel sweep to
themselves; but, far from regretting this deplorable indifference
on the part of the other guests, he found himself nourishing the
hope that Miss Bart might be unaccompanied. The precious minutes
were flying, however; the big chestnuts pawed the ground and
flecked their impatient sides with foam; the coachman seemed to
be slowly petrifying on the box, and the groom on the doorstep;
and still the lady did not come. Suddenly, however, there was a
sound of voices and a rustle of skirts in the doorway, and Mr.
Gryce, restoring his watch to his pocket, turned with a nervous
start; but it was only to find himself handing Mrs.\ Wetherall
into the carriage.





The Wetheralls always went to church. They belonged to the vast
group of human automata who go through life without neglecting to
perform a single one of the gestures executed by the surrounding
puppets. It is true that the Bellomont puppets did not go to
church; but others equally important did---and Mr.\ and Mrs.
Wetherall's circle was so large that God was included in their
visiting-list. They appeared, therefore, punctual and resigned,
with the air of people bound for a dull ``At Home,''\ and after them
Hilda and Muriel straggled, yawning and pinning each other's
veils and ribbons as they came. They had promised Lily to go to
church with her, they declared, and Lily was such a dear old duck
that they didn't mind doing it to please her, though they
couldn't fancy what had put the idea in her head, and though for
their own part they would much rather have played lawn tennis
with Jack and Gwen, if she hadn't told them she was coming. The
Misses Trenor were followed by Lady Cressida Raith, a
weather-beaten person in Liberty silk and ethnological trinkets,
who, on seeing the omnibus, expressed her surprise that they were
not to walk across the park; but at Mrs.\ Wetherall's horrified
protest that the church was a mile away, her ladyship,
after a glance at the height of the other's heels, acquiesced in
the necessity of driving, and poor Mr.\ Gryce found himself
rolling off between four ladies for whose spiritual welfare he
felt not the least concern.





It might have afforded him some consolation could he have known
that Miss Bart had really meant to go to church. She had even
risen earlier than usual in the execution of her purpose. She had
an idea that the sight of her in a grey gown of devotional cut,
with her famous lashes drooped above a prayer-book, would put the
finishing touch to Mr.\ Gryce's subjugation, and render inevitable
a certain incident which she had resolved should form a part of
the walk they were to take together after luncheon. Her
intentions in short had never been more definite; but poor Lily,
for all the hard glaze of her exterior, was inwardly as malleable
as wax. Her faculty for adapting herself, for entering into other
people's feelings, if it served her now and then in small
contingencies, hampered her in the decisive moments of life. She
was like a water-plant in the flux of the tides, and today the
whole current of her mood was carrying her toward Lawrence
Selden. Why had he come? Was it to see herself or Bertha Dorset? 
It was the last question which, at that moment, should have
engaged her. She might better have contented herself with
thinking that he had simply responded to the despairing summons
of his hostess, anxious to interpose him between herself and the
ill-humour of Mrs.\ Dorset. But Lily had not rested till she
learned from Mrs.\ Trenor that Selden had come of his own accord. 
``He didn't even wire me---he just happened to find the trap at the
station. Perhaps it's not over with Bertha after all,''\ Mrs.
Trenor musingly concluded; and went away to arrange her
dinner-cards accordingly.





Perhaps it was not, Lily reflected; but it should be soon, unless
she had lost her cunning. If Selden had come at Mrs.\ Dorset's
call, it was at her own that he would stay. So much the previous
evening had told her. Mrs.\ Trenor, true to her simple principle
of making her married friends happy, had placed Selden and Mrs.
Dorset next to each other at dinner; but, in obedience to the
time-honoured traditions of the match-maker, she had separated
Lily and Mr.\ Gryce, sending in the former with George
Dorset, while Mr.\ Gryce was coupled with Gwen Van Osburgh.





George Dorset's talk did not interfere with the range of his
neighbour's thoughts. He was a mournful dyspeptic, intent on
finding out the deleterious ingredients of every dish and
diverted from this care only by the sound of his wife's voice. On
this occasion, however, Mrs.\ Dorset took no part in the general
conversation. She sat talking in low murmurs with Selden, and
turning a contemptuous and denuded shoulder toward her host, who,
far from resenting his exclusion, plunged into the excesses of
the \textit{menu} with the joyous irresponsibility of a free man. To Mr.
Dorset, however, his wife's attitude was a subject of such
evident concern that, when he was not scraping the sauce from his
fish, or scooping the moist bread-crumbs from the interior of his
roll, he sat straining his thin neck for a glimpse of her between
the lights.





Mrs.\ Trenor, as it chanced, had placed the husband and wife on
opposite sides of the table, and Lily was therefore able to
observe Mrs.\ Dorset also, and by carrying her glance a few feet
farther, to set up a rapid comparison between Lawrence Selden and
Mr.\ Gryce. It was that comparison which was her undoing. Why else
had she suddenly grown interested in Selden? She had known him
for eight years or more: ever since her return to America he had
formed a part of her background. She had always been glad to sit
next to him at dinner, had found him more agreeable than most
men, and had vaguely wished that he possessed the other qualities
needful to fix her attention; but till now she had been too busy
with her own affairs to regard him as more than one of the
pleasant accessories of life. Miss Bart was a keen reader of her
own heart, and she saw that her sudden preoccupation with Selden
was due to the fact that his presence shed a new light on her
surroundings. Not that he was notably brilliant or exceptional;
in his own profession he was surpassed by more than one man who
had bored Lily through many a weary dinner. It was rather that he
had preserved a certain social detachment, a happy air of viewing
the show objectively, of having points of contact outside the
great gilt cage in which they were all huddled for the mob to
gape at. How alluring the world outside the cage appeared
to Lily, as she heard its door clang on her! In reality, as she
knew, the door never clanged: it stood always open; but most of
the captives were like flies in a bottle, and having once flown
in, could never regain their freedom. It was Selden's distinction
that he had never forgotten the way out.





That was the secret of his way of readjusting her vision. Lily,
turning her eyes from him, found herself scanning her little
world through his retina: it was as though the pink lamps had
been shut off and the dusty daylight let in. She looked down the
long table, studying its occupants one by one, from Gus Trenor,
with his heavy carnivorous head sunk between his shoulders, as he
preyed on a jellied plover, to his wife, at the opposite end of
the long bank of orchids, suggestive, with her glaring
good-looks, of a jeweller's window lit by electricity. And
between the two, what a long stretch of vacuity! How dreary and
trivial these people were! Lily reviewed them with a scornful
impatience: Carry Fisher, with her shoulders, her eyes, her
divorces, her general air of embodying a ``spicy paragraph''; young
Silverton, who had meant to live on proof-reading and write an
epic, and who now lived on his friends and had become critical of
truffles; Alice Wetherall, an animated visiting-list, whose most
fervid convictions turned on the wording of invitations and the
engraving of dinner-cards; Wetherall, with his perpetual nervous
nod of acquiescence, his air of agreeing with people before he
knew what they were saying; Jack Stepney, with his confident
smile and anxious eyes, half way between the sheriff and an
heiress; Gwen Van Osburgh, with all the guileless confidence of a
young girl who has always been told that there is no one richer
than her father.





Lily smiled at her classification of her friends. How different
they had seemed to her a few hours ago! Then they had symbolized
what she was gaining, now they stood for what she was giving up. 
That very afternoon they had seemed full of brilliant qualities;
now she saw that they were merely dull in a loud way. Under the
glitter of their opportunities she saw the poverty of their
achievement. It was not that she wanted them to be more
disinterested; but she would have liked them to be more
picturesque. And she had a shamed recollection of the way
in which, a few hours since, she had felt the centripetal force
of their standards. She closed her eyes an instant, and the
vacuous routine of the life she had chosen stretched before her
like a long white road without dip or turning: it was true she
was to roll over it in a carriage instead of trudging it on foot,
but sometimes the pedestrian enjoys the diversion of a short cut
which is denied to those on wheels.





She was roused by a chuckle which Mr.\ Dorset seemed to eject from
the depths of his lean throat.





``I say, do look at her,''\ he exclaimed, turning to Miss Bart with
lugubrious merriment---``I beg your pardon, but do just look at my
wife making a fool of that poor devil over there! One would
really suppose she was gone on him---and it's all the other way
round, I assure you.''





Thus adjured, Lily turned her eyes on the spectacle which was
affording Mr.\ Dorset such legitimate mirth. It certainly
appeared, as he said, that Mrs.\ Dorset was the more active
participant in the scene: her neighbour seemed to receive her
advances with a temperate zest which did not distract him from
his dinner. The sight restored Lily's good humour, and knowing
the peculiar disguise which Mr.\ Dorset's marital fears assumed,
she asked gaily: ``Aren't you horribly jealous of her?''





Dorset greeted the sally with delight. ``Oh, abominably---you've
just hit it---keeps me awake at night. The doctors tell me that's
what has knocked my digestion out---being so infernally jealous of
her.---I can't eat a mouthful of this stuff, you know,''\ he added
suddenly, pushing back his plate with a clouded countenance; and
Lily, unfailingly adaptable, accorded her radiant attention to
his prolonged denunciation of other people's cooks, with a
supplementary tirade on the toxic qualities of melted butter.





It was not often that he found so ready an ear; and, being a man
as well as a dyspeptic, it may be that as he poured his
grievances into it he was not insensible to its rosy symmetry. At
any rate he engaged Lily so long that the sweets were being
handed when she caught a phrase on her other side, where Miss
Corby, the comic woman of the company, was bantering Jack Stepney
on his approaching engagement. Miss Corby's r\^{o}le was
jocularity: she always entered the conversation with a
handspring.





``And of course you'll have Sim Rosedale as best man!''\ Lily heard
her fling out as the climax of her prognostications; and Stepney
responded, as if struck: ``Jove, that's an idea. What a thumping
present I'd get out of him!''





\textit{Sim} \textit{Rosedale}! The name, made more odious by its diminutive,
obtruded itself on Lily's thoughts like a leer. It stood for one
of the many hated possibilities hovering on the edge of life. If
she did not marry Percy Gryce, the day might come when she would
have to be civil to such men as Rosedale. \textit{If} \textit{she} \textit{did} \textit{not} \textit{marry}
\textit{him}? But she meant to marry him---she was sure of him and sure of
herself. She drew back with a shiver from the pleasant paths in
which her thoughts had been straying, and set her feet once more
in the middle of the long white road.... When she went upstairs
that night she found that the late post had brought her a fresh
batch of bills. Mrs.\ Peniston, who was a conscientious woman, had
forwarded them all to Bellomont.





Miss Bart, accordingly, rose the next morning with the most
earnest conviction that it was her duty to go to church. She tore
herself betimes from the lingering enjoyment of her
breakfast-tray, rang to have her grey gown laid out, and
despatched her maid to borrow a prayer-book from Mrs.\ Trenor.





But her course was too purely reasonable not to contain the germs
of rebellion. No sooner were her preparations made than they
roused a smothered sense of resistance. A small spark was enough
to kindle Lily's imagination, and the sight of the grey dress and
the borrowed prayer-book flashed a long light down the years. She
would have to go to church with Percy Gryce every Sunday. They
would have a front pew in the most expensive church in New York,
and his name would figure handsomely in the list of parish
charities. In a few years, when he grew stouter, he would be made
a warden. Once in the winter the rector would come to dine, and
her husband would beg her to go over the list and see that no
\textit{divorcees} were included, except those who had showed signs of
penitence by being re-married to the very wealthy. There was
nothing especially arduous in this round of relgious
obligations; but it stood for a fraction of that great bulk of
boredom which loomed across her path. And who could consent to be
bored on such a morning? Lily had slept well, and her bath had
filled her with a pleasant glow, which was becomingly reflected
in the clear curve of her cheek. No lines were visible this
morning, or else the glass was at a happier angle.





And the day was the accomplice of her mood: it was a day for
impulse and truancy. The light air seemed full of powdered gold;
below the dewy bloom of the lawns the woodlands blushed and
smouldered, and the hills across the river swam in molten blue. 
Every drop of blood in Lily's veins invited her to happiness.





The sound of wheels roused her from these musings, and leaning
behind her shutters she saw the omnibus take up its freight. She
was too late, then---but the fact did not alarm her. A glimpse of
Mr.\ Gryce's crestfallen face even suggested that she had done
wisely in absenting herself, since the disappointment he so
candidly betrayed would surely whet his appetite for the
afternoon walk. That walk she did not mean to miss; one glance at
the bills on her writing-table was enough to recall its
necessity. But meanwhile she had the morning to herself, and
could muse pleasantly on the disposal of its hours. She was
familiar enough with the habits of Bellomont to know that she was
likely to have a free field till luncheon. She had seen the
Wetheralls, the Trenor girls and Lady Cressida packed safely into
the omnibus; Judy Trenor was sure to be having her hair
shampooed; Carry Fisher had doubtless carried off her host for a
drive; Ned Silverton was probably smoking the cigarette of young
despair in his bedroom; and Kate Corby was certain to be playing
tennis with Jack Stepney and Miss Van Osburgh. Of the ladies,
this left only Mrs.\ Dorset unaccounted for, and Mrs.\ Dorset never
came down till luncheon: her doctors, she averred, had forbidden
her to expose herself to the crude air of the morning.





To the remaining members of the party Lily gave no special
thought; wherever they were, they were not likely to interfere
with her plans. These, for the moment, took the shape of assuming
a dress somewhat more rustic and summerlike in style than the
garment she had first selected, and rustling downstairs,
sunshade in hand, with the disengaged air of a lady in quest of
exercise. The great hall was empty but for the knot of dogs by
the fire, who, taking in at a glance the outdoor aspect of Miss
Bart, were upon her at once with lavish offers of companionship. 
She put aside the ramming paws which conveyed these offers, and
assuring the joyous volunteers that she might presently have a
use for their company, sauntered on through the empty
drawing-room to the library at the end of the house. The library
was almost the only surviving portion of the old manor-house of
Bellomont: a long spacious room, revealing the traditions of the
mother-country in its classically-cased doors, the Dutch tiles of
the chimney, and the elaborate hob-grate with its shining brass
urns. A few family portraits of lantern-jawed gentlemen in
tie-wigs, and ladies with large head-dresses and small bodies,
hung between the shelves lined with pleasantly-shabby books: 
books mostly contemporaneous with the ancestors in question, and
to which the subsequent Trenors had made no perceptible
additions. The library at Bellomont was in fact never used for
reading, though it had a certain popularity as a smoking-room or
a quiet retreat for flirtation. It had occurred to Lily, however,
that it might on this occasion have been resorted to by the only
member of the party in the least likely to put it to its original
use. She advanced noiselessly over the dense old rug scattered
with easy-chairs, and before she reached the middle of the room
she saw that she had not been mistaken. Lawrence Selden was in
fact seated at its farther end; but though a book lay on his
knee, his attention was not engaged with it, but directed to a
lady whose lace-clad figure, as she leaned back in an adjoining
chair, detached itself with exaggerated slimness against the
dusky leather upholstery.





Lily paused as she caught sight of the group; for a moment she
seemed about to withdraw, but thinking better of this, she
announced her approach by a slight shake of her skirts which made
the couple raise their heads, Mrs.\ Dorset with a look of frank
displeasure, and Selden with his usual quiet smile. The sight of
his composure had a disturbing effect on Lily; but to be
disturbed was in her case to make a more brilliant effort at
self-possession.





``Dear me, am I late?''\ she asked, putting a hand in his as he
advanced to greet her.





``Late for what?''\ enquired Mrs.\ Dorset tartly. ``Not for luncheon,
certainly---but perhaps you had an earlier engagement?''





``Yes, I had,''\ said Lily confidingly.





``Really? Perhaps I am in the way, then? But Mr.\ Selden is
entirely at your disposal.'' Mrs.\ Dorset was pale with temper, and
her antagonist felt a certain pleasure in prolonging her
distress.





``Oh, dear, no---do stay,''\ she said good-humouredly. ``I don't in
the least want to drive you away.''





``You're awfully good, dear, but I never interfere with Mr.
Selden's engagements.''





The remark was uttered with a little air of proprietorship not
lost on its object, who concealed a faint blush of annoyance by
stooping to pick up the book he had dropped at Lily's approach. 
The latter's eyes widened charmingly and she broke into a light
laugh.





``But I have no engagement with Mr.\ Selden! My engagement was to
go to church; and I'm afraid the omnibus has started without me. 
\textit{Has} it started, do you know?''





She turned to Selden, who replied that he had heard it drive away
some time since.





``Ah, then I shall have to walk; I promised Hilda and Muriel to go
to church with them. It's too late to walk there, you say? Well,
I shall have the credit of trying, at any rate---and the advantage
of escaping part of the service. I'm not so sorry for myself,
after all!''





And with a bright nod to the couple on whom she had intruded,
Miss Bart strolled through the glass doors and carried her
rustling grace down the long perspective of the garden walk.





She was taking her way churchward, but at no very quick pace; a
fact not lost on one of her observers, who stood in the doorway
looking after her with an air of puzzled amusement. The truth is
that she was conscious of a somewhat keen shock of
disappointment. All her plans for the day had been built on the
assumption that it was to see her that Selden had come to
Bellomont. She had expected, when she came downstairs, to
find him on the watch for her; and she had found him, instead, in
a situation which might well denote that he had been on the watch
for another lady. Was it possible, after all, that he had come
for Bertha Dorset? The latter had acted on the assumption to the
extent of appearing at an hour when she never showed herself to
ordinary mortals, and Lily, for the moment, saw no way of putting
her in the wrong. It did not occur to her that Selden might have
been actuated merely by the desire to spend a Sunday out of town: 
women never learn to dispense with the sentimental motive in
their judgments of men. But Lily was not easily disconcerted;
competition put her on her mettle, and she reflected that
Selden's coming, if it did not declare him to be still in Mrs.
Dorset's toils, showed him to be so completely free from them
that he was not afraid of her proximity.





These thoughts so engaged her that she fell into a gait hardly
likely to carry her to church before the sermon, and at length,
having passed from the gardens to the wood-path beyond, so far
forgot her intention as to sink into a rustic seat at a bend of
the walk. The spot was charming, and Lily was not insensible to
the charm, or to the fact that her presence enhanced it; but she
was not accustomed to taste the joys of solitude except in
company, and the combination of a handsome girl and a romantic
scene struck her as too good to be wasted. No one, however,
appeared to profit by the opportunity; and after a half hour of
fruitless waiting she rose and wandered on. She felt a stealing
sense of fatigue as she walked; the sparkle had died out of her,
and the taste of life was stale on her lips. She hardly knew what
she had been seeking, or why the failure to find it had so
blotted the light from her sky: she was only aware of a vague
sense of failure, of an inner isolation deeper than the
loneliness about her.





Her footsteps flagged, and she stood gazing listlessly ahead,
digging the ferny edge of the path with the tip of her sunshade. 
As she did so a step sounded behind her, and she saw Selden at
her side.





``How fast you walk!''\ he remarked. ``I thought I should never catch
up with you.''





She answered gaily: ``You must be quite breathless! I've been
sitting under that tree for an hour.''





``Waiting for me, I hope?''\ he rejoined; and she said with a vague
laugh:\  



``Well---waiting to see if you would come.''





``I seize the distinction, but I don't mind it, since doing the
one involved doing the other. But weren't you sure that I should
come?''





``If I waited long enough---but you see I had only a limited time
to give to the experiment.''





``Why limited? Limited by luncheon?''





``No; by my other engagement.''





``Your engagement to go to church with Muriel and Hilda?''





``No; but to come home from church with another person.''





``Ah, I see; I might have known you were fully provided with
alternatives. And is the other person coming home this way?''





Lily laughed again. ``That's just what I don't know; and to find
out, it is my business to get to church before the service is
over.''





``Exactly; and it is my business to prevent your doing so; in
which case the other person, piqued by your absence, will form
the desperate resolve of driving back in the omnibus.''





Lily received this with fresh appreciation; his nonsense was like
the bubbling of her inner mood. ``Is that what you would do in
such an emergency?''\ she enquired.





Selden looked at her with solemnity. ``I am here to prove to you,''
he cried, ``what I am capable of doing in an emergency!''





``Walking a mile in an hour---you must own that the omnibus would
be quicker!''





``Ah---but will he find you in the end? That's the only test of
success.''





They looked at each other with the same luxury of enjoyment that
they had felt in exchanging absurdities over his tea-table; but
suddenly Lily's face changed, and she said: ``Well, if it is, he
has succeeded.''





Selden, following her glance, perceived a party of people
advancing toward them from the farther bend of the path. Lady
Cressida had evidently insisted on walking home, and the rest of
the church-goers had thought it their duty to accompany
her. Lily's companion looked rapidly from one to the other of the
two men of the party; Wetherall walking respectfully at Lady
Cressida's side with his little sidelong look of nervous
attention, and Percy Gryce bringing up the rear with Mrs.
Wetherall and the Trenors.





``Ah---now I see why you were getting up your Americana!''\ Selden
exclaimed with a note of the freest admiration but the blush with
which the sally was received checked whatever amplifications he
had meant to give it.





That Lily Bart should object to being bantered about her suitors,
or even about her means of attracting them, was so new to Selden
that he had a momentary flash of surprise, which lit up a number
of possibilities; but she rose gallantly to the defence of her
confusion, by saying, as its object approached: ``That was why I
was waiting for you---to thank you for having given me so many
points!''





``Ah, you can hardly do justice to the subject in such a short
time,''\ said Selden, as the Trenor girls caught sight of Miss
Bart; and while she signalled a response to their boisterous
greeting, he added quickly: ``Won't you devote your afternoon to
it? You know I must be off tomorrow morning. We'll take a walk,
and you can thank me at your leisure.''





\chapter*{\raggedright Chapter 6}

\addcontentsline{toc}{chapter}{Chapter 6}

\markboth{House of Mirth}{Chapter 6}





The afternoon was perfect. A deeper stillness possessed the air,
and the glitter of the American autumn was tempered by a haze
which diffused the brightness without dulling it.





In the woody hollows of the park there was already a faint chill;
but as the ground rose the air grew lighter, and ascending the
long slopes beyond the high-road, Lily and her companion reached
a zone of lingering summer. The path wound across a meadow with
scattered trees; then it dipped into a lane plumed with asters
and purpling sprays of bramble, whence, through the light quiver
of ash-leaves, the country unrolled itself in pastoral distances.





Higher up, the lane showed thickening tufts of fern and of the
creeping glossy verdure of shaded slopes; trees began to overhang
it, and the shade deepened to the checkered dusk of a
beech-grove. The boles of the trees stood well apart, with only a
light feathering of undergrowth; the path wound along the edge of
the wood, now and then looking out on a sunlit pasture or on an
orchard spangled with fruit.





Lily had no real intimacy with nature, but she had a passion for
the appropriate and could be keenly sensitive to a scene which
was the fitting background of her own sensations. The landscape
outspread below her seemed an enlargement of her present mood,
and she found something of herself in its calmness, its breadth,
its long free reaches. On the nearer slopes the sugar-maples
wavered like pyres of light; lower down was a massing of grey
orchards, and here and there the lingering green of an oak-grove. 
Two or three red farm-houses dozed under the apple-trees, and the
white wooden spire of a village church showed beyond the shoulder
of the hill; while far below, in a haze of dust, the high-road
ran between the fields.





``Let us sit here,''\ Selden suggested, as they reached an open
ledge of rock above which the beeches rose steeply between mossy
boulders.





Lily dropped down on the rock, glowing with her long climb. She
sat quiet, her lips parted by the stress of the ascent,
her eyes wandering peacefully over the broken ranges of the
landscape. Selden stretched himself on the grass at her feet,
tilting his hat against the level sun-rays, and clasping his
hands behind his head, which rested against the side of the rock. 
He had no wish to make her talk; her quick-breathing silence
seemed a part of the general hush and harmony of things. In his
own mind there was only a lazy sense of pleasure, veiling the
sharp edges of sensation as the September haze veiled the scene
at their feet. But Lily, though her attitude was as calm as his,
was throbbing inwardly with a rush of thoughts. There were in her
at the moment two beings, one drawing deep breaths of freedom and
exhilaration, the other gasping for air in a little black
prison-house of fears. But gradually the captive's gasps grew
fainter, or the other paid less heed to them: the horizon
expanded, the air grew stronger, and the free spirit quivered for
flight.





She could not herself have explained the sense of buoyancy which
seemed to lift and swing her above the sun-suffused world at her
feet. Was it love, she wondered, or a mere fortuitous combination
of happy thoughts and sensations? How much of it was owing to the
spell of the perfect afternoon, the scent of the fading woods,
the thought of the dulness she had fled from? Lily had no
definite experience by which to test the quality of her feelings. 
She had several times been in love with fortunes or careers, but
only once with a man. That was years ago, when she first came
out, and had been smitten with a romantic passion for a young
gentleman named Herbert Melson, who had blue eyes and a little
wave in his hair. Mr.\ Melson, who was possessed of no other
negotiable securities, had hastened to employ these in capturing
the eldest Miss Van Osburgh: since then he had grown stout and
wheezy, and was given to telling anecdotes about his children. If
Lily recalled this early emotion it was not to compare it with
that which now possessed her; the only point of comparison was
the sense of lightness, of emancipation, which she remembered
feeling, in the whirl of a waltz or the seclusion of a
conservatory, during the brief course of her youthful romance. 
She had not known again till today that lightness, that glow of
freedom; but now it was something more than a blind groping of
the blood. The peculiar charm of her feeling for Selden
was that she understood it; she could put her finger on every
link of the chain that was drawing them together. Though his
popularity was of the quiet kind, felt rather than actively
expressed among his friends, she had never mistaken his
inconspicuousness for obscurity. His reputed cultivation was
generally regarded as a slight obstacle to easy intercourse, but
Lily, who prided herself on her broad-minded recognition of
literature, and always carried an Omar Khayam in her
travelling-bag, was attracted by this attribute, which she felt
would have had its distinction in an older society. It was,
moreover, one of his gifts to look his part; to have a height
which lifted his head above the crowd, and the keenly-modelled
dark features which, in a land of amorphous types, gave him the
air of belonging to a more specialized race, of carrying the
impress of a concentrated past. Expansive persons found him a
little dry, and very young girls thought him sarcastic; but this
air of friendly aloofness, as far removed as possible from any
assertion of personal advantage, was the quality which piqued
Lily's interest. Everything about him accorded with the
fastidious element in her taste, even to the light irony with
which he surveyed what seemed to her most sacred. She admired him
most of all, perhaps, for being able to convey as distinct a
sense of superiority as the richest man she had ever met.





It was the unconscious prolongation of this thought which led her
to say presently, with a laugh: ``I have broken two engagements
for you today. How many have you broken for me?''





``None,''\ said Selden calmly. ``My only engagement at Bellomont was
with you.''





She glanced down at him, faintly smiling.





``Did you really come to Bellomont to see me?''





``Of course I did.''





Her look deepened meditatively. ``Why?''\ she murmured, with an
accent which took all tinge of coquetry from the question.





``Because you're such a wonderful spectacle: I always like to see
what you are doing.''





``How do you know what I should be doing if you were not here?''





Selden smiled. ``I don't flatter myself that my coming has
deflected your course of action by a hair's breadth.''





``That's absurd---since, if you were not here, I could obviously
not be taking a walk with you.''





``No; but your taking a walk with me is only another way of making
use of your material. You are an artist and I happen to be the
bit of colour you are using today. It's a part of your cleverness
to be able to produce premeditated effects extemporaneously.''





Lily smiled also: his words were too acute not to strike her
sense of humour. It was true that she meant to use the accident
of his presence as part of a very definite effect; or that, at
least, was the secret pretext she had found for breaking her
promise to walk with Mr.\ Gryce. She had sometimes been accused of
being too eager---even Judy Trenor had warned her to go slowly. 
Well, she would not be too eager in this case; she would give her
suitor a longer taste of suspense. Where duty and inclination
jumped together, it was not in Lily's nature to hold them
asunder. She had excused herself from the walk on the plea of a
headache: the horrid headache which, in the morning, had
prevented her venturing to church. Her appearance at luncheon
justified the excuse. She looked languid, full of a suffering
sweetness; she carried a scent-bottle in her hand. Mr.\ Gryce was
new to such manifestations; he wondered rather nervously if she
were delicate, having far-reaching fears about the future of his
progeny. But sympathy won the day, and he besought her not to
expose herself: he always connected the outer air with ideas of
exposure.





Lily had received his sympathy with languid gratitude, urging
him, since she should be such poor company, to join the rest of
the party who, after luncheon, were starting in automobiles on a
visit to the Van Osburghs at Peekskill. Mr.\ Gryce was touched by
her disinterestedness, and, to escape from the threatened vacuity
of the afternoon, had taken her advice and departed mournfully,
in a dust-hood and goggles: as the motor-car plunged down the
avenue she smiled at his resemblance to a baffled beetle. Selden
had watched her manoeuvres with lazy amusement. She had made no
reply to his suggestion that they should spend the
afternoon together, but as her plan unfolded itself he felt
fairly confident of being included in it. The house was empty
when at length he heard her step on the stair and strolled out of
the billiard-room to join her.





She had on a hat and walking-dress, and the dogs were bounding at
her feet.





``I thought, after all, the air might do me good,''\ she explained;
and he agreed that so simple a remedy was worth trying.





The excursionists would be gone at least four hours; Lily and
Selden had the whole afternoon before them, and the sense of
leisure and safety gave the last touch of lightness to her
spirit. With so much time to talk, and no definite object to be
led up to, she could taste the rare joys of mental vagrancy.





She felt so free from ulterior motives that she took up his
charge with a touch of resentment.





``I don't know,''\ she said, ``why you are always accusing me of
premeditation.''





``I thought you confessed to it: you told me the other day that
you had to follow a certain line---and if one does a thing at all
it is a merit to do it thoroughly.''





``If you mean that a girl who has no one to think for her is
obliged to think for herself, I am quite willing to accept the
imputation. But you must find me a dismal kind of person if you
suppose that I never yield to an impulse.''





``Ah, but I don't suppose that: haven't I told you that your
genius lies in converting impulses into intentions?''





``My genius?''\ she echoed with a sudden note of weariness. ``Is
there any final test of genius but success? And I certainly
haven't succeeded.''





Selden pushed his hat back and took a side-glance at her. 
``Success---what is success? I shall be interested to have your
definition.''





``Success?''\ She hesitated. ``Why, to get as much as one can out of
life, I suppose. It's a relative quality, after all. Isn't that
your idea of it?''





``My idea of it? God forbid!''\ He sat up with sudden energy,
resting his elbows on his knees and staring out upon the mellow
fields. ``My idea of success,''\ he said, ``is personal freedom.''





``Freedom? Freedom from worries?''





``From everything---from money, from poverty, from ease and
anxiety, from all the material accidents. To keep a kind of
republic of the spirit---that's what I call success.''





She leaned forward with a responsive flash. ``I know---I know---it's
strange; but that's just what I've been feeling today.''





He met her eyes with the latent sweetness of his. ``Is the feeling
so rare with you?''\ he said.





She blushed a little under his gaze. ``You think me horribly
sordid, don't you? But perhaps it's rather that I never had any
choice. There was no one, I mean, to tell me about the republic
of the spirit.''





``There never is---it's a country one has to find the way to one's
self.''





``But I should never have found my way there if you hadn't told
me.''





``Ah, there are sign-posts---but one has to know how to read them.''





``Well, I have known, I have known!''\ she cried with a glow of
eagerness. ``Whenever I see you, I find myself spelling out a
letter of the sign---and yesterday---last evening at dinner---I
suddenly saw a little way into your republic.''





Selden was still looking at her, but with a changed eye. Hitherto
he had found, in her presence and her talk, the aesthetic
amusement which a reflective man is apt to seek in desultory
intercourse with pretty women. His attitude had been one of
admiring spectatorship, and he would have been almost sorry to
detect in her any emotional weakness which should interfere with
the fulfilment of her aims. But now the hint of this weakness had
become the most interesting thing about her. He had come on her
that morning in a moment of disarray; her face had been pale and
altered, and the diminution of her beauty had lent her a poignant
charm. \textit{That} \textit{is} \textit{how} \textit{she} \textit{looks} \textit{when} \textit{she} \textit{is} \textit{alone}!\ had been his
first thought; and the second was to note in her the change which
his coming produced. It was the danger-point of their intercourse
that he could not doubt the spontaneity of her liking. From
whatever angle he viewed their dawning intimacy, he could not see
it as part of her scheme of life; and to be the unforeseen
element in a career so accurately planned was stimulating
even to a man who had renounced sentimental experiments.





``Well,''\ he said, ``did it make you want to see more? Are you going
to become one of us?''





He had drawn out his cigarettes as he spoke, and she reached her
hand toward the case.





``Oh, do give me one---I haven't smoked for days!''





``Why such unnatural abstinence? Everybody smokes at Bellomont.''





``Yes---but it is not considered becoming in a \textit{Jeune} \textit{fille} A
\textit{Marier}; and at the present moment I am a \textit{Jeune} \textit{fille} A \textit{Marier}.''





``Ah, then I'm afraid we can't let you into the republic.''





``Why not? Is it a celibate order?''





``Not in the least, though I'm bound to say there are not many
married people in it. But you will marry some one very rich, and
it's as hard for rich people to get into as the kingdom of
heaven.''





``That's unjust, I think, because, as I understand it, one of the
conditions of citizenship is not to think too much about money,
and the only way not to think about money is to have a great deal
of it.''





``You might as well say that the only way not to think about air
is to have enough to breathe. That is true enough in a sense; but
your lungs are thinking about the air, if you are not. And so it
is with your rich people---they may not be thinking of money, but
they're breathing it all the while; take them into another
element and see how they squirm and gasp!''





Lily sat gazing absently through the blue rings of her
cigarette-smoke.





``It seems to me,''\ she said at length, ``that you spend a good deal
of your time in the element you disapprove of.''





Selden received this thrust without discomposure. ``Yes; but I
have tried to remain amphibious: it's all right as long as one's
lungs can work in another air. The real alchemy consists in being
able to turn gold back again into something else; and that's the
secret that most of your friends have lost.''





Lily mused. ``Don't you think,''\ she rejoined after a moment, ``that
the people who find fault with society are too apt to regard it
as an end and not a means, just as the people who despise
money speak as if its only use were to be kept in bags and
gloated over? Isn't it fairer to look at them both as
opportunities, which may be used either stupidly or
intelligently, according to the capacity of the user?''





``That is certainly the sane view; but the queer thing about
society is that the people who regard it as an end are those who
are in it, and not the critics on the fence. It's just the other
way with most shows---the audience may be under the illusion, but
the actors know that real life is on the other side of the
footlights. The people who take society as an escape from work
are putting it to its proper use; but when it becomes the thing
worked for it distorts all the relations of life.'' Selden raised
himself on his elbow. ``Good heavens!''\ he went on, ``I don't
underrate the decorative side of life. It seems to me the sense
of splendour has justified itself by what it has produced. The
worst of it is that so much human nature is used up in the
process. If we're all the raw stuff of the cosmic effects, one
would rather be the fire that tempers a sword than the fish that
dyes a purple cloak. And a society like ours wastes such good
material in producing its little patch of purple! Look at a boy
like Ned Silverton---he's really too good to be used to refurbish
anybody's social shabbiness. There's a lad just setting out to
discover the universe: isn't it a pity he should end by finding
it in Mrs.\ Fisher's drawing-room?''





``Ned is a dear boy, and I hope he will keep his illusions long
enough to write some nice poetry about them; but do you think it
is only in society that he is likely to lose them?''





Selden answered her with a shrug. ``Why do we call all our
generous ideas illusions, and the mean ones truths? Isn't it a
sufficient condemnation of society to find one's self accepting
such phraseology? I very nearly acquired the jargon at
Silverton's age, and I know how names can alter the colour of
beliefs.''





She had never heard him speak with such energy of affirmation. 
His habitual touch was that of the eclectic, who lightly turns
over and compares; and she was moved by this sudden glimpse into
the laboratory where his faiths were formed.





``Ah, you are as bad as the other sectarians,''\ she exclaimed; ``why
do you call your republic a republic? It is a closed corporation,
and you create arbitrary objections in order to keep people out.''





``It is not \textit{my} republic; if it were, I should have a \textit{Coup} \textit{d'etat}
and seat you on the throne.''





``Whereas, in reality, you think I can never even get my foot
across the threshold? Oh, I understand what you mean. You despise
my ambitions---you think them unworthy of me!''





Selden smiled, but not ironically. ``Well, isn't that a tribute? I
think them quite worthy of most of the people who live by them.''





She had turned to gaze on him gravely. ``But isn't it possible
that, if I had the opportunities of these people, I might make a
better use of them? Money stands for all kinds of things---its
purchasing quality isn't limited to diamonds and motor-cars.''






``Not in the least: you might expiate your enjoyment of them by
founding a hospital.''





``But if you think they are what I should really enjoy, you must
think my ambitions are good enough for me.''





Selden met this appeal with a laugh. ``Ah, my dear Miss Bart, I am
not divine Providence, to guarantee your enjoying the things you
are trying to get!''





``Then the best you can say for me is, that after struggling to
get them I probably shan't like them?''\ She drew a deep breath. 
``What a miserable future you foresee for me!''





``Well---have you never foreseen it for yourself?''\ The slow colour
rose to her cheek, not a blush of excitement but drawn from the
deep wells of feeling; it was as if the effort of her spirit had
produced it.





``Often and often,''\ she said. ``But it looks so much darker when
you show it to me!''





He made no answer to this exclamation, and for a while they sat
silent, while something throbbed between them in the wide quiet
of the air.





But suddenly she turned on him with a kind of vehemence. ``Why do
you do this to me?''\ she cried. ``Why do you make the things I have
chosen seem hateful to me, if you have nothing to give me
instead?''





The words roused Selden from the musing fit into which he had
fallen. He himself did not know why he had led their talk along
such lines; it was the last use he would have imagined himself
making of an afternoon's solitude with Miss Bart. But it was one
of those moments when neither seemed to speak deliberately, when
an indwelling voice in each called to the other across unsounded
depths of feeling.





``No, I have nothing to give you instead,''\ he said, sitting up and
turning so that he faced her. ``If I had, it should be yours, you
know.''





She received this abrupt declaration in a way even stranger than
the manner of its making: she dropped her face on her hands and
he saw that for a moment she wept.





It was for a moment only, however; for when he leaned nearer and
drew down her hands with a gesture less passionate than grave,
she turned on him a face softened but not disfigured by emotion,
and he said to himself, somewhat cruelly, that even her weeping
was an art.





The reflection steadied his voice as he asked, between pity and
irony: ``Isn't it natural that I should try to belittle all the
things I can't offer you?''





Her face brightened at this, but she drew her hand away, not with
a gesture of coquetry, but as though renouncing something to
which she had no claim.





``But you belittle \textit{me}, don't you,''\ she returned gently, ``in being
so sure they are the only things I care for?''





Selden felt an inner start; but it was only the last quiver of
his egoism. Almost at once he answered quite simply: ``But you do
care for them, don't you? And no wishing of mine can alter that.''





He had so completely ceased to consider how far this might carry
him, that he had a distinct sense of disappointment when she
turned on him a face sparkling with derision.





``Ah,''\ she cried, ``for all your fine phrases you're really as
great a coward as I am, for you wouldn't have made one of them if
you hadn't been so sure of my answer.''





The shock of this retort had the effect of crystallizing Selden's
wavering intentions.





``I am not so sure of your answer,''\ he said quietly. ``And I do you
the justice to believe that you are not either.''





It was her turn to look at him with surprise; and after a
moment---``Do you want to marry me?''\ she asked.





He broke into a laugh. ``No, I don't want to---but perhaps I should
if you did!''





``That's what I told you---you're so sure of me that you can amuse
yourself with experiments.'' She drew back the hand he had
regained, and sat looking down on him sadly.





``I am not making experiments,''\ he returned. ``Or if I am, it is
not on you but on myself. I don't know what effect they are going
to have on me---but if marrying you is one of them, I will take
the risk.''





She smiled faintly. ``It would be a great risk, certainly---I have
never concealed from you how great.''





``Ah, it's you who are the coward!''\ he exclaimed.





She had risen, and he stood facing her with his eyes on hers. The
soft isolation of the falling day enveloped them: they seemed
lifted into a finer air. All the exquisite influences of the hour
trembled in their veins, and drew them to each other as the
loosened leaves were drawn to the earth.





``It's you who are the coward,''\ he repeated, catching her hands in
his.





She leaned on him for a moment, as if with a drop of tired wings: 
he felt as though her heart were beating rather with the stress
of a long flight than the thrill of new distances. Then, drawing
back with a little smile of warning---``I shall look hideous in
dowdy clothes; but I can trim my own hats,''\ she declared.





They stood silent for a while after this, smiling at each other
like adventurous children who have climbed to a forbidden height
from which they discover a new world. The actual world at their
feet was veiling itself in dimness, and across the valley a clear
moon rose in the denser blue.





Suddenly they heard a remote sound, like the hum of a giant
insect, and following the high-road, which wound whiter through
the surrounding twilight, a black object rushed across their
vision.





Lily started from her attitude of absorption; her smile faded and
she began to move toward the lane.





``I had no idea it was so late! We shall not be back till after
dark,''\ she said, almost impatiently.





Selden was looking at her with surprise: it took him a moment to
regain his usual view of her; then he said, with an
uncontrollable note of dryness: ``That was not one of our party;
the motor was going the other way.''





``I know---I know----''\ She paused, and he saw her redden through the
twilight. ``But I told them I was not well---that I should not go
out. Let us go down!''\ she murmured.





Selden continued to look at her; then he drew his cigarette-case
from his pocket and slowly lit a cigarette. It seemed to him
necessary, at that moment, to proclaim, by some habitual gesture
of this sort, his recovered hold on the actual: he had an almost
puerile wish to let his companion see that, their flight over, he
had landed on his feet.





She waited while the spark flickered under his curved palm; then
he held out the cigarettes to her.





She took one with an unsteady hand, and putting it to her lips,
leaned forward to draw her light from his. In the indistinctness
the little red gleam lit up the lower part of her face, and he
saw her mouth tremble into a smile.





``Were you serious?''\ she asked, with an odd thrill of gaiety which
she might have caught up, in haste, from a heap of stock
inflections, without having time to select the just note. 
Selden's voice was under better control. ``Why not?''\ he returned. 
``You see I took no risks in being so.'' And as she continued to
stand before him, a little pale under the retort, he added
quickly: ``Let us go down.''





\chapter*{\raggedright Chapter 7}

\addcontentsline{toc}{chapter}{Chapter 7}

\markboth{House of Mirth}{Chapter 7}





It spoke much for the depth of Mrs.\ Trenor's friendship that her
voice, in admonishing Miss Bart, took the same note of personal
despair as if she had been lamenting the collapse of a
house-party.





``All I can say is, Lily, that I can't make you out!''\ She leaned
back, sighing, in the morning abandon of lace and muslin, turning
an indifferent shoulder to the heaped-up importunities of her
desk, while she considered, with the eye of a physician who has
given up the case, the erect exterior of the patient confronting
her.





``If you hadn't told me you were going in for him seriously---but
I'm sure you made that plain enough from the beginning! Why else
did you ask me to let you off bridge, and to keep away Carry and
Kate Corby? I don't suppose you did it because he amused you; we
could none of us imagine your putting up with him for a moment
unless you meant to marry him. And I'm sure everybody played
fair! They all wanted to help it along. Even Bertha kept her
hands off---I will say that---till Lawrence came down and you
dragged him away from her. After that she had a right to
retaliate---why on earth did you interfere with her? You've known
Lawrence Selden for years---why did you behave as if you had just
discovered him? If you had a grudge against Bertha it was a
stupid time to show it---you could have paid her back just as well
after you were married! I told you Bertha was dangerous. She was
in an odious mood when she came here, but Lawrence's turning up
put her in a good humour, and if you'd only let her think he came
for \textit{her} it would have never occurred to her to play you this
trick. Oh, Lily, you'll never do anything if you're not serious!''





Miss Bart accepted this exhortation in a spirit of the purest
impartiality. Why should she have been angry? It was the voice of
her own conscience which spoke to her through Mrs.\ Trenor's
reproachful accents. But even to her own conscience she must
trump up a semblance of defence. ``I only took a day off---I
thought he meant to stay on all this week, and I knew Mr.\ Selden
was leaving this morning.''





Mrs.\ Trenor brushed aside the plea with a gesture which laid bare
its weakness.





``He did mean to stay---that's the worst of it. It shows that he's
run away from you; that Bertha's done her work and poisoned him
thoroughly.''





Lily gave a slight laugh. ``Oh, if he's running I'll overtake
him!''





Her friend threw out an arresting hand. ``Whatever you do, Lily,
do nothing!''





Miss Bart received the warning with a smile. ``I don't mean,
literally, to take the next train. There are ways----''\ But she did
not go on to specify them.





Mrs.\ Trenor sharply corrected the tense. ``There \textit{were} ways---plenty
of them! I didn't suppose you needed to have them pointed out. 
But don't deceive yourself---he's thoroughly frightened. He has
run straight home to his mother, and she'll protect him!''





``Oh, to the death,''\ Lily agreed, dimpling at the vision.





``How you can \textit{laugh}----''\ her friend rebuked her; and she dropped
back to a soberer perception of things with the question: ``What
was it Bertha really told him?''





``Don't ask me---horrors! She seemed to have raked up everything. 
Oh, you know what I mean---of course there isn't anything, \textit{really};
but I suppose she brought in Prince Varigliano---and Lord
Hubert---and there was some story of your having borrowed money of
old Ned Van Alstyne: did you ever?''





``He is my father's cousin,''\ Miss Bart interposed.





``Well, of course she left \textit{that} out. It seems Ned told Carry
Fisher; and she told Bertha, naturally. They're all alike, you
know: they hold their tongues for years, and you think you're
safe, but when their opportunity comes they remember everything.''





Lily had grown pale: her voice had a harsh note in it. ``It was
some money I lost at bridge at the Van Osburghs'. I repaid it, of
course.''





``Ah, well, they wouldn't remember that; besides, it was the idea
of the gambling debt that frightened Percy. Oh, Bertha knew her
man---she knew just what to tell him!''





In this strain Mrs.\ Trenor continued for nearly an hour to
admonish her friend. Miss Bart listened with admirable
equanimity. Her naturally good temper had been disciplined by
years of enforced compliance, since she had almost always had to
attain her ends by the circuitous path of other people's; and,
being naturally inclined to face unpleasant facts as soon as they
presented themselves, she was not sorry to hear an impartial
statement of what her folly was likely to cost, the more so as
her own thoughts were still insisting on the other side of the
case. Presented in the light of Mrs.\ Trenor's vigorous comments,
the reckoning was certainly a formidable one, and Lily, as she
listened, found herself gradually reverting to her friend's view
of the situation. Mrs.\ Trenor's words were moreover emphasized
for her hearer by anxieties which she herself could scarcely
guess. Affluence, unless stimulated by a keen imagination, forms
but the vaguest notion of the practical strain of poverty. Judy
knew it must be ``horrid''\ for poor Lily to have to stop to
consider whether she could afford real lace on her petticoats,
and not to have a motor-car and a steam-yacht at her orders; but
the daily friction of unpaid bills, the daily nibble of small
temptations to expenditure, were trials as far out of her
experience as the domestic problems of the char-woman. Mrs.
Trenor's unconsciousness of the real stress of the situation had
the effect of making it more galling to Lily. While her friend
reproached her for missing the opportunity to eclipse her rivals,
she was once more battling in imagination with the mounting tide
of indebtedness from which she had so nearly escaped. What wind
of folly had driven her out again on those dark seas?





If anything was needed to put the last touch to her
self-abasement it was the sense of the way her old life was
opening its ruts again to receive her. Yesterday her fancy had
fluttered free pinions above a choice of occupations; now she had
to drop to the level of the familiar routine, in which moments of
seeming brilliancy and freedom alternated with long hours of
subjection.





She laid a deprecating hand on her friend's. ``Dear Judy! I'm
sorry to have been such a bore, and you are very good to me. But
you must have some letters for me to answer---let me at least be
useful.''





She settled herself at the desk, and Mrs.\ Trenor accepted her
resumption of the morning's task with a sigh which implied that,
after all, she had proved herself unfit for higher uses.





The luncheon table showed a depleted circle. All the men but Jack
Stepney and Dorset had returned to town (it seemed to Lily a last
touch of irony that Selden and Percy Gryce should have gone in
the same train), and Lady Cressida and the attendant Wetheralls
had been despatched by motor to lunch at a distant country-house. 
At such moments of diminished interest it was usual for Mrs.
Dorset to keep her room till the afternoon; but on this occasion
she drifted in when luncheon was half over, hollowed-eyed and
drooping, but with an edge of malice under her indifference.





She raised her eyebrows as she looked about the table. ``How few
of us are left! I do so enjoy the quiet---don't you, Lily? I wish
the men would always stop away---it's really much nicer without
them. Oh, you don't count, George: one doesn't have to talk to
one's husband. But I thought Mr.\ Gryce was to stay for the rest
of the week?''\ she added enquiringly. ``Didn't he intend to, Judy? 
He's such a nice boy---I wonder what drove him away? He is rather
shy, and I'm afraid we may have shocked him: he has been brought
up in such an old-fashioned way. Do you know, Lily, he told me he
had never seen a girl play cards for money till he saw you doing
it the other night? And he lives on the interest of his income,
and always has a lot left over to invest!''





Mrs.\ Fisher leaned forward eagerly. ``I do believe it is some
one's duty to educate that young man. It is shocking that he has
never been made to realize his duties as a citizen. Every wealthy
man should be compelled to study the laws of his country.''





Mrs.\ Dorset glanced at her quietly. ``I think he \textit{has} studied the
divorce laws. He told me he had promised the Bishop to sign some
kind of a petition against divorce.''





Mrs.\ Fisher reddened under her powder, and Stepney said with a
laughing glance at Miss Bart: ``I suppose he is thinking of
marriage, and wants to tinker up the old ship before he goes
aboard.''





His betrothed looked shocked at the metaphor, and George Dorset
exclaimed with a sardonic growl: ``Poor devil! It isn't the ship
that will do for him, it's the crew.''





``Or the stowaways,''\ said Miss Corby brightly. ``If I contemplated
a voyage with him I should try to start with a friend in the
hold.''





Miss Van Osburgh's vague feeling of pique was struggling for
appropriate expression. ``I'm sure I don't see why you laugh at
him; I think he's very nice,''\ she exclaimed; ``and, at any rate, a
girl who married him would always have enough to be comfortable.''





She looked puzzled at the redoubled laughter which hailed her
words, but it might have consoled her to know how deeply they had
sunk into the breast of one of her hearers.





Comfortable! At that moment the word was more eloquent to Lily
Bart than any other in the language. She could not even pause to
smile over the heiress's view of a colossal fortune as a mere
shelter against want: her mind was filled with the vision of what
that shelter might have been to her. Mrs.\ Dorset's pin-pricks did
not smart, for her own irony cut deeper: no one could hurt her as
much as she was hurting herself, for no one else---not even Judy
Trenor---knew the full magnitude of her folly.





She was roused from these unprofitable considerations by a
whispered request from her hostess, who drew her apart as they
left the luncheon-table.





``Lily, dear, if you've nothing special to do, may I tell Carry
Fisher that you intend to drive to the station and fetch Gus? He
will be back at four, and I know she has it in her mind to meet
him. Of course I'm very glad to have him amused, but I happen to
know that she has bled him rather severely since she's been here,
and she is so keen about going to fetch him that I fancy she must
have got a lot more bills this morning. It seems to me,''\ Mrs.
Trenor feelingly concluded, ``that most of her alimony is paid by
other women's husbands!''





Miss Bart, on her way to the station, had leisure to muse over
her friend's words, and their peculiar application to herself. 
Why should she have to suffer for having once, for a few hours,
borrowed money of an elderly cousin, when a woman like Carry
Fisher could make a living unrebuked from the good-nature of her
men friends and the tolerance of their wives? It all turned on
the tiresome distinction between what a married woman
might, and a girl might not, do. Of course it was shocking for a
married woman to borrow money---and Lily was expertly aware of the
implication involved---but still, it was the mere MALUM PROHIBITUM
which the world decries but condones, and which, though it may be
punished by private vengeance, does not provoke the collective
disapprobation of society. To Miss Bart, in short, no such
opportunities were possible. She could of course borrow from her
women friends---a hundred here or there, at the utmost---but they
were more ready to give a gown or a trinket, and looked a little
askance when she hinted her preference for a cheque. Women are
not generous lenders, and those among whom her lot was cast were
either in the same case as herself, or else too far removed from
it to understand its necessities. The result of her meditations
was the decision to join her aunt at Richfield. She could not
remain at Bellomont without playing bridge, and being involved in
other expenses; and to continue her usual series of autumn visits
would merely prolong the same difficulties. She had reached a
point where abrupt retrenchment was necessary, and the only cheap
life was a dull life. She would start the next morning for
Richfield.





At the station she thought Gus Trenor seemed surprised, and not
wholly unrelieved, to see her. She yielded up the reins of the
light runabout in which she had driven over, and as he climbed
heavily to her side, crushing her into a scant third of the seat,
he said: ``Halloo! It isn't often you honour me. You must have
been uncommonly hard up for something to do.''





The afternoon was warm, and propinquity made her more than
usually conscious that he was red and massive, and that beads of
moisture had caused the dust of the train to adhere unpleasantly
to the broad expanse of cheek and neck which he turned to her;
but she was aware also, from the look in his small dull eyes,
that the contact with her freshness and slenderness was as
agreeable to him as the sight of a cooling beverage.





The perception of this fact helped her to answer gaily: ``It's not
often I have the chance. There are too many ladies to dispute the
privilege with me.''





``The privilege of driving me home? Well, I'm glad you won
the race, anyhow. But I know what really happened---my wife sent
you. Now didn't she?''





He had the dull man's unexpected flashes of astuteness, and Lily
could not help joining in the laugh with which he had pounced on
the truth.





``You see, Judy thinks I'm the safest person for you to be with;
and she's quite right,''\ she rejoined.





``Oh, is she, though? If she is, it's because you wouldn't waste
your time on an old hulk like me. We married men have to put up
with what we can get: all the prizes are for the clever chaps
who've kept a free foot. Let me light a cigar, will you? I've had
a beastly day of it.''





He drew up in the shade of the village street, and passed the
reins to her while he held a match to his cigar. The little flame
under his hand cast a deeper crimson on his puffing face, and
Lily averted her eyes with a momentary feeling of repugnance. And
yet some women thought him handsome!





As she handed back the reins, she said sympathetically: ``Did you
have such a lot of tiresome things to do?''





``I should say so---rather!''\ Trenor, who was seldom listened to,
either by his wife or her friends, settled down into the rare
enjoyment of a confidential talk. ``You don't know how a fellow
has to hustle to keep this kind of thing going.'' He waved his
whip in the direction of the Bellomont acres, which lay outspread
before them in opulent undulations. ``Judy has no idea of what she
spends---not that there isn't plenty to keep the thing going,''\ he
interrupted himself, ``but a man has got to keep his eyes open and
pick up all the tips he can. My father and mother used to live
like fighting-cocks on their income, and put by a good bit of it
too---luckily for me---but at the pace we go now, I don't know
where I should be if it weren't for taking a flyer now and then. 
The women all think---I mean Judy thinks---I've nothing to do but
to go down town once a month and cut off coupons, but the truth
is it takes a devilish lot of hard work to keep the machinery
running. Not that I ought to complain to-day, though,''\ he went on
after a moment, ``for I did a very neat stroke of business, thanks
to Stepney's friend Rosedale: by the way, Miss Lily, I wish you'd
try to persuade Judy to be decently civil to that chap. He's
going to be rich enough to buy us all out one of these
days, and if she'd only ask him to dine now and then I could get
almost anything out of him. The man is mad to know the people who
don't want to know him, and when a fellow's in that state there
is nothing he won't do for the first woman who takes him up.''





Lily hesitated a moment. The first part of her companion's
discourse had started an interesting train of thought, which was
rudely interrupted by the mention of Mr.\ Rosedale's name. She
uttered a faint protest.





``But you know Jack did try to take him about, and he was
impossible.''





``Oh, hang it---because he's fat and shiny, and has a sloppy
manner! Well, all I can say is that the people who are clever
enough to be civil to him now will make a mighty good thing of
it. A few years from now he'll be in it whether we want him or
not, and then he won't be giving away a half-a-million tip for a
dinner.''





Lily's mind had reverted from the intrusive personality of Mr.
Rosedale to the train of thought set in motion by Trenor's first
words. This vast mysterious Wall Street world of ``tips''\ and
``deals''---might she not find in it the means of escape from her
dreary predicament? She had often heard of women making money in
this way through their friends: she had no more notion than most
of her sex of the exact nature of the transaction, and its
vagueness seemed to diminish its indelicacy. She could not,
indeed, imagine herself, in any extremity, stooping to extract a
``tip''\ from Mr.\ Rosedale; but at her side was a man in possession
of that precious commodity, and who, as the husband of her
dearest friend, stood to her in a relation of almost fraternal
intimacy.





In her inmost heart Lily knew it was not by appealing to the
fraternal instinct that she was likely to move Gus Trenor; but
this way of explaining the situation helped to drape its crudity,
and she was always scrupulous about keeping up appearances to
herself. Her personal fastidiousness had a moral equivalent, and
when she made a tour of inspection in her own mind there were
certain closed doors she did not open.





As they reached the gates of Bellomont she turned to Trenor with
a smile. ``The afternoon is so perfect---don't you want to drive me
a little farther? I've been rather out of spirits all day,
and it's so restful to be away from people, with some one who
won't mind if I'm a little dull.''





She looked so plaintively lovely as she proffered the request, so
trustfully sure of his sympathy and understanding, that Trenor
felt himself wishing that his wife could see how other women
treated him---not battered wire-pullers like Mrs.\ Fisher, but a
girl that most men would have given their boots to get such a
look from.





``Out of spirits? Why on earth should you ever be out of spirits? 
Is your last box of Doucet dresses a failure, or did Judy rook
you out of everything at bridge last night?''





Lily shook her head with a sigh. ``I have had to give up Doucet;
and bridge too---I can't afford it. In fact I can't afford any of
the things my friends do, and I am afraid Judy often thinks me a
bore because I don't play cards any longer, and because I am not
as smartly dressed as the other women. But you will think me a
bore too if I talk to you about my worries, and I only mention
them because I want you to do me a favour---the very greatest of
favours.''





Her eyes sought his once more, and she smiled inwardly at the
tinge of apprehension that she read in them.





``Why, of course---if it's anything I can manage----''\ He broke off,
and she guessed that his enjoyment was disturbed by the
remembrance of Mrs.\ Fisher's methods.





``The greatest of favours,''\ she rejoined gently. ``The fact is,
Judy is angry with me, and I want you to make my peace.''





``Angry with you? Oh, come, nonsense----''\ his relief broke through
in a laugh. ``Why, you know she's devoted to you.''





``She is the best friend I have, and that is why I mind having to
vex her. But I daresay you know what she has wanted me to do. She
has set her heart---poor dear---on my marrying---marrying a great
deal of money.''





She paused with a slight falter of embarrassment, and Trenor,
turning abruptly, fixed on her a look of growing intelligence.





``A great deal of money? Oh, by Jove---you don't mean Gryce? 
What---you do? Oh, no, of course I won't mention it---you can trust
me to keep my mouth shut---but Gryce---good Lord, \textit{Gryce}! Did
Judy really think you could bring yourself to marry that
portentous little ass? But you couldn't, eh? And so you gave him
the sack, and that's the reason why he lit out by the first train
this morning?''\ He leaned back, spreading himself farther across
the seat, as if dilated by the joyful sense of his own
discernment. ``How on earth could Judy think you would do such a
thing? I could have told her you'd never put up with such a
little milksop!''





Lily sighed more deeply. ``I sometimes think,''\ she murmured, ``that
men understand a woman's motives better than other women do.''





``Some men---I'm certain of it! I could have \textit{told} Judy,''\ he
repeated, exulting in the implied superiority over his wife.





``I thought you would understand; that's why I wanted to speak to
you,''\ Miss Bart rejoined. ``I can't make that kind of marriage;
it's impossible. But neither can I go on living as all the women
in my set do. I am almost entirely dependent on my aunt, and
though she is very kind to me she makes me no regular allowance,
and lately I've lost money at cards, and I don't dare tell her
about it. I have paid my card debts, of course, but there is
hardly anything left for my other expenses, and if I go on with
my present life I shall be in horrible difficulties. I have a
tiny income of my own, but I'm afraid it's badly invested, for it
seems to bring in less every year, and I am so ignorant of money
matters that I don't know if my aunt's agent, who looks after it,
is a good adviser.'' She paused a moment, and added in a lighter
tone: ``I didn't mean to bore you with all this, but I want your
help in making Judy understand that I can't, at present, go on
living as one must live among you all. I am going away tomorrow
to join my aunt at Richfield, and I shall stay there for the rest
of the autumn, and dismiss my maid and learn how to mend my own
clothes.''





At this picture of loveliness in distress, the pathos of which
was heightened by the light touch with which it was drawn, a
murmur of indignant sympathy broke from Trenor. Twenty-four hours
earlier, if his wife had consulted him on the subject of Miss
Bart's future, he would have said that a girl with extravagant
tastes and no money had better marry the first rich man she could
get; but with the subject of discussion at his side,
turning to him for sympathy, making him feel that he understood
her better than her dearest friends, and confirming the assurance
by the appeal of her exquisite nearness, he was ready to swear
that such a marriage was a desecration, and that, as a man of
honour, he was bound to do all he could to protect her from the
results of her disinterestedness. This impulse was reinforced by
the reflection that if she had married Gryce she would have been
surrounded by flattery and approval, whereas, having refused to
sacrifice herself to expediency, she was left to bear the whole
cost of her resistance. Hang it, if he could find a way out of
such difficulties for a professional sponge like Carry Fisher,
who was simply a mental habit corresponding to the physical
titillations of the cigarette or the cock-tail, he could surely
do as much for a girl who appealed to his highest sympathies, and
who brought her troubles to him with the trustfulness of a child.





Trenor and Miss Bart prolonged their drive till long after
sunset; and before it was over he had tried, with some show of
success, to prove to her that, if she would only trust him, he
could make a handsome sum of money for her without endangering
the small amount she possessed. She was too genuinely ignorant of
the manipulations of the stock-market to understand his technical
explanations, or even perhaps to perceive that certain points in
them were slurred; the haziness enveloping the transaction served
as a veil for her embarrassment, and through the general blur her
hopes dilated like lamps in a fog. She understood only that her
modest investments were to be mysteriously multiplied without
risk to herself; and the assurance that this miracle would take
place within a short time, that there would be no tedious
interval for suspense and reaction, relieved her of her lingering
scruples.





Again she felt the lightening of her load, and with it the
release of repressed activities. Her immediate worries conjured,
it was easy to resolve that she would never again find herself in
such straits, and as the need of economy and self-denial receded
from her foreground she felt herself ready to meet any other
demand which life might make. Even the immediate one of letting
Trenor, as they drove homeward, lean a little nearer and
rest his hand reassuringly on hers, cost her only a momentary
shiver of reluctance. It was part of the game to make him feel
that her appeal had been an uncalculated impulse, provoked by the
liking he inspired; and the renewed sense of power in handling
men, while it consoled her wounded vanity, helped also to obscure
the thought of the claim at which his manner hinted. He was a
coarse dull man who, under all his show of authority, was a mere
supernumerary in the costly show for which his money paid: 
surely, to a clever girl, it would be easy to hold him by his
vanity, and so keep the obligation on his side.





\chapter*{\raggedright Chapter 8}

\addcontentsline{toc}{chapter}{Chapter 8}

\markboth{House of Mirth}{Chapter 8}





The first thousand dollar cheque which Lily received with a
blotted scrawl from Gus Trenor strengthened her self-confidence
in the exact degree to which it effaced her debts.





The transaction had justified itself by its results: she saw now
how absurd it would have been to let any primitive scruple
deprive her of this easy means of appeasing her creditors. Lily
felt really virtuous as she dispensed the sum in sops to her
tradesmen, and the fact that a fresh order accompanied each
payment did not lessen her sense of disinterestedness. How many
women, in her place, would have given the orders without making
the payment!





She had found it reassuringly easy to keep Trenor in a good
humour. To listen to his stories, to receive his confidences and
laugh at his jokes, seemed for the moment all that was required
of her, and the complacency with which her hostess regarded these
attentions freed them of the least hint of ambiguity. Mrs.\ Trenor
evidently assumed that Lily's growing intimacy with her husband
was simply an indirect way of returning her own kindness.





``I'm so glad you and Gus have become such good friends,''\ she said
approvingly. ``It's too delightful of you to be so nice to him,
and put up with all his tiresome stories. I know what they are,
because I had to listen to them when we were engaged---I'm sure he
is telling the same ones still. And now I shan't always have to
be asking Carry Fisher here to keep him in a good-humour. She's a
perfect vulture, you know; and she hasn't the least moral sense. 
She is always getting Gus to speculate for her, and I'm sure she
never pays when she loses.''





Miss Bart could shudder at this state of things without the
embarrassment of a personal application. Her own position was
surely quite different. There could be no question of her not
paying when she lost, since Trenor had assured her that she was
certain not to lose. In sending her the cheque he had explained
that he had made five thousand for her out of Rosedale's ``tip,''
and had put four thousand back in the same venture, as
there was the promise of another ``big rise''; she understood
therefore that he was now speculating with her own money, and
that she consequently owed him no more than the gratitude which
such a trifling service demanded. She vaguely supposed that, to
raise the first sum, he had borrowed on her securities; but this
was a point over which her curiosity did not linger. It was
concentrated, for the moment, on the probable date of the next
``big rise.''





The news of this event was received by her some weeks later, on
the occasion of Jack Stepney's marriage to Miss Van Osburgh. As a
cousin of the bridegroom, Miss Bart had been asked to act as
bridesmaid; but she had declined on the plea that, since she was
much taller than the other attendant virgins, her presence might
mar the symmetry of the group. The truth was, she had attended
too many brides to the altar: when next seen there she meant to
be the chief figure in the ceremony. She knew the pleasantries
made at the expense of young girls who have been too long before
the public, and she was resolved to avoid such assumptions of
youthfulness as might lead people to think her older than she
really was.





The Van Osburgh marriage was celebrated in the village church
near the paternal estate on the Hudson. It was the ``simple
country wedding''\ to which guests are convoyed in special trains,
and from which the hordes of the uninvited have to be fended off
by the intervention of the police. While these sylvan rites were
taking place, in a church packed with fashion and festooned with
orchids, the representatives of the press were threading their
way, note-book in hand, through the labyrinth of wedding
presents, and the agent of a cinematograph syndicate was setting
up his apparatus at the church door. It was the kind of scene in
which Lily had often pictured herself as taking the principal
part, and on this occasion the fact that she was once more merely
a casual spectator, instead of the mystically veiled figure
occupying the centre of attention, strengthened her resolve to
assume the latter part before the year was over. The fact that
her immediate anxieties were relieved did not blind her to a
possibility of their recurrence; it merely gave her enough
buoyancy to rise once more above her doubts and feel a renewed
faith in her beauty, her power, and her general fitness to
attract a brilliant destiny. It could not be that one
conscious of such aptitudes for mastery and enjoyment was doomed
to a perpetuity of failure; and her mistakes looked easily
reparable in the light of her restored self-confidence.





A special appositeness was given to these reflections by the
discovery, in a neighbouring pew, of the serious profile and
neatly-trimmed beard of Mr.\ Percy Gryce. There was something
almost bridal in his own aspect: his large white gardenia had a
symbolic air that struck Lily as a good omen. After all, seen in
an assemblage of his kind he was not ridiculous-looking: a
friendly critic might have called his heaviness weighty, and he
was at his best in the attitude of vacant passivity which brings
out the oddities of the restless. She fancied he was the kind of
man whose sentimental associations would be stirred by the
conventional imagery of a wedding, and she pictured herself, in
the seclusion of the Van Osburgh conservatories, playing
skillfully upon sensibilities thus prepared for her touch. In
fact, when she looked at the other women about her, and recalled
the image she had brought away from her own glass, it did not
seem as though any special skill would be needed to repair her
blunder and bring him once more to her feet.





The sight of Selden's dark head, in a pew almost facing her,
disturbed for a moment the balance of her complacency. The rise
of her blood as their eyes met was succeeded by a contrary
motion, a wave of resistance and withdrawal. She did not wish to
see him again, not because she feared his influence, but because
his presence always had the effect of cheapening her aspirations,
of throwing her whole world out of focus. Besides, he was a
living reminder of the worst mistake in her career, and the fact
that he had been its cause did not soften her feelings toward
him. She could still imagine an ideal state of existence in
which, all else being superadded, intercourse with Selden might
be the last touch of luxury; but in the world as it was, such a
privilege was likely to cost more than it was worth.





``Lily, dear, I never saw you look so lovely! You look as if
something delightful had just happened to you!''





The young lady who thus formulated her admiration of her
brilliant friend did not, in her own person, suggest such
happy possibilities. Miss Gertrude Farish, in fact, typified the
mediocre and the ineffectual. If there were compensating
qualities in her wide frank glance and the freshness of her
smile, these were qualities which only the sympathetic observer
would perceive before noticing that her eyes were of a workaday
grey and her lips without haunting curves. Lily's own view of her
wavered between pity for her limitations and impatience at her
cheerful acceptance of them. To Miss Bart, as to her mother,
acquiescence in dinginess was evidence of stupidity; and there
were moments when, in the consciousness of her own power to look
and to be so exactly what the occasion required, she almost felt
that other girls were plain and inferior from choice. Certainly
no one need have confessed such acquiescence in her lot as was
revealed in the ``useful''\ colour of Gerty Farish's gown and the
subdued lines of her hat: it is almost as stupid to let your
clothes betray that you know you are ugly as to have them
proclaim that you think you are beautiful.





Of course, being fatally poor and dingy, it was wise of Gerty to
have taken up philanthropy and symphony concerts; but there was
something irritating in her assumption that existence yielded no
higher pleasures, and that one might get as much interest and
excitement out of life in a cramped flat as in the splendours of
the Van Osburgh establishment. Today, however, her chirping
enthusiasms did not irritate Lily. They seemed only to throw her
own exceptionalness into becoming relief, and give a soaring
vastness to her scheme of life.





``Do let us go and take a peep at the presents before everyone
else leaves the dining-room!''\ suggested Miss Farish, linking her
arm in her friend's. It was characteristic of her to take a
sentimental and unenvious interest in all the details of a
wedding: she was the kind of person who always kept her
handkerchief out during the service, and departed clutching a box
of wedding-cake.





``Isn't everything beautifully done?''\ she pursued, as they entered
the distant drawing-room assigned to the display of Miss Van
Osburgh's bridal spoils. ``I always say no one does things better
than cousin Grace! Did you ever taste anything more delicious
than that \textit{mousse} of lobster with champagne sauce? I made up my
mind weeks ago that I wouldn't miss this wedding, and just
fancy how delightfully it all came about. When Lawrence Selden
heard I was coming, he insisted on fetching me himself and
driving me to the station, and when we go back this evening I am
to dine with him at Sherry's. I really feel as excited as if I
were getting married myself!''





Lily smiled: she knew that Selden had always been kind to his
dull cousin, and she had sometimes wondered why he wasted so much
time in such an unremunerative manner; but now the thought gave
her a vague pleasure.





``Do you see him often?''\ she asked.





``Yes; he is very good about dropping in on Sundays. And now and
then we do a play together; but lately I haven't seen much of
him. He doesn't look well, and he seems nervous and unsettled. 
The dear fellow! I do wish he would marry some nice girl. I told
him so today, but he said he didn't care for the really nice
ones, and the other kind didn't care for him---but that was just
his joke, of course. He could never marry a girl who \textit{wasn't} nice. 
Oh, my dear, did you ever see such pearls?''





They had paused before the table on which the bride's jewels were
displayed, and Lily's heart gave an envious throb as she caught
the refraction of light from their surfaces---the milky gleam of
perfectly matched pearls, the flash of rubies relieved against
contrasting velvet, the intense blue rays of sapphires kindled
into light by surrounding diamonds: all these precious tints
enhanced and deepened by the varied art of their setting. The
glow of the stones warmed Lily's veins like wine. More completely
than any other expression of wealth they symbolized the life she
longed to lead, the life of fastidious aloofness and refinement
in which every detail should have the finish of a jewel, and the
whole form a harmonious setting to her own jewel-like rareness.





``Oh, Lily, do look at this diamond pendant---it's as big as a
dinner-plate! Who can have given it?''\ Miss Farish bent
short-sightedly over the accompanying card. ``\textit{Mr}.\ \textit{Simon} \textit{Rosedale}. 
What, that horrid man? Oh, yes---I remember he's a friend of
Jack's, and I suppose cousin Grace had to ask him here today; but
she must rather hate having to let Gwen accept such a present
from him.''





Lily smiled. She doubted Mrs.\ Van Osburgh's reluctance, but was
aware of Miss Farish's habit of ascribing her own delicacies of
feeling to the persons least likely to be encumbered by them.





``Well, if Gwen doesn't care to be seen wearing it she can always
exchange it for something else,''\ she remarked.





``Ah, here is something so much prettier,''\ Miss Farish continued. 
``Do look at this exquisite white sapphire. I'm sure the person
who chose it must have taken particular pains. What is the name? 
Percy Gryce? Ah, then I'm not surprised!''\ She smiled
significantly as she replaced the card. ``Of course you've heard
that he's perfectly devoted to Evie Van Osburgh? Cousin Grace is
so pleased about it---it's quite a romance! He met her first at
the George Dorsets', only about six weeks ago, and it's just the
nicest possible marriage for dear Evie. Oh, I don't mean the
money---of course she has plenty of her own---but she's such a
quiet stay-at-home kind of girl, and it seems he has just the
same tastes; so they are exactly suited to each other.''





Lily stood staring vacantly at the white sapphire on its velvet
bed. Evie Van Osburgh and Percy Gryce? The names rang derisively
through her brain. \textit{Evie} \textit{Van} \textit{Osburgh}? The youngest, dumpiest,
dullest of the four dull and dumpy daughters whom Mrs.\ Van
Osburgh, with unsurpassed astuteness, had ``placed''\ one by one in
enviable niches of existence! Ah, lucky girls who grow up in the
shelter of a mother's love---a mother who knows how to contrive
opportunities without conceding favours, how to take advantage of
propinquity without allowing appetite to be dulled by habit! The
cleverest girl may miscalculate where her own interests are
concerned, may yield too much at one moment and withdraw too far
at the next: it takes a mother's unerring vigilance and foresight
to land her daughters safely in the arms of wealth and
suitability.





Lily's passing light-heartedness sank beneath a renewed sense of
failure. Life was too stupid, too blundering! Why should Percy
Gryce's millions be joined to another great fortune, why should
this clumsy girl be put in possession of powers she would never
know how to use?





She was roused from these speculations by a familiar touch
on her arm, and turning saw Gus Trenor beside her. She felt a
thrill of vexation: what right had he to touch her? Luckily Gerty
Farish had wandered off to the next table, and they were alone.





Trenor, looking stouter than ever in his tight frock-coat, and
unbecomingly flushed by the bridal libations, gazed at her with
undisguised approval.





``By Jove, Lily, you do look a stunner!''\ He had slipped insensibly
into the use of her Christian name, and she had never found the
right moment to correct him. Besides, in her set all the men and
women called each other by their Christian names; it was only on
Trenor's lips that the familiar address had an unpleasant
significance.





``Well,''\ he continued, still jovially impervious to her annoyance,
``have you made up your mind which of these little trinkets you
mean to duplicate at Tiffany's tomorrow? I've got a cheque for
you in my pocket that will go a long way in that line!''





Lily gave him a startled look: his voice was louder than usual,
and the room was beginning to fill with people. But as her glance
assured her that they were still beyond ear-shot a sense of
pleasure replaced her apprehension.





``Another dividend?''\ she asked, smiling and drawing near him in
the desire not to be overheard.





``Well, not exactly: I sold out on the rise and I've pulled off
four thou'\ for you. Not so bad for a beginner, eh? I suppose
you'll begin to think you're a pretty knowing speculator. And
perhaps you won't think poor old Gus such an awful ass as some
people do.''





``I think you the kindest of friends; but I can't thank you
properly now.''





She let her eyes shine into his with a look that made up for the
hand-clasp he would have claimed if they had been alone---and how
glad she was that they were not! The news filled her with the
glow produced by a sudden cessation of physical pain. The world
was not so stupid and blundering after all: now and then a stroke
of luck came to the unluckiest. At the thought her spirits began
to rise: it was characteristic of her that one trifling piece of
good fortune should give wings to all her hopes. Instantly came
the reflection that Percy Gryce was not irretrievably
lost; and she smiled to think of the excitement of recapturing
him from Evie Van Osburgh. What chance could such a simpleton
have against her if she chose to exert herself? She glanced
about, hoping to catch a glimpse of Gryce; but her eyes lit
instead on the glossy countenance of Mr.\ Rosedale, who was
slipping through the crowd with an air half obsequious, half
obtrusive, as though, the moment his presence was recognized, it
would swell to the dimensions of the room.





Not wishing to be the means of effecting this enlargement, Lily
quickly transferred her glance to Trenor, to whom the expression
of her gratitude seemed not to have brought the complete
gratification she had meant it to give.





``Hang thanking me---I don't want to be thanked, but I \textit{should} like
the chance to say two words to you now and then,''\ he grumbled. ``I
thought you were going to spend the whole autumn with us, and
I've hardly laid eyes on you for the last month. Why can't you
come back to Bellomont this evening? We're all alone, and Judy is
as cross as two sticks. Do come and cheer a fellow up. If you say
yes I'll run you over in the motor, and you can telephone your
maid to bring your traps from town by the next train.''





Lily shook her head with a charming semblance of regret. ``I wish
I could---but it's quite impossible. My aunt has come back to
town, and I must be with her for the next few days.''





``Well, I've seen a good deal less of you since we've got to be
such pals than I used to when you were Judy's friend,''\ he
continued with unconscious penetration.





``When I was Judy's friend? Am I not her friend still? Really, you
say the most absurd things! If I were always at Bellomont you
would tire of me much sooner than Judy---but come and see me at my
aunt's the next afternoon you are in town; then we can have a
nice quiet talk, and you can tell me how I had better invest my
fortune.''





It was true that, during the last three or four weeks, she had
absented herself from Bellomont on the pretext of having other
visits to pay; but she now began to feel that the reckoning she
had thus contrived to evade had rolled up interest in the
interval.





The prospect of the nice quiet talk did not appear as all-sufficing
to Trenor as she had hoped, and his brows continued to lower
as he said: ``Oh, I don't know that I can promise you a fresh tip
every day. But there's one thing you might do for me;
and that is, just to be a little civil to Rosedale. Judy has
promised to ask him to dine when we get to town, but I can't
induce her to have him at Bellomont, and if you would let me
bring him up now it would make a lot of difference. I don't
believe two women have spoken to him this afternoon, and I can
tell you he's a chap it pays to be decent to.''





Miss Bart made an impatient movement, but suppressed the words
which seemed about to accompany it. After all, this was an
unexpectedly easy way of acquitting her debt; and had she not
reasons of her own for wishing to be civil to Mr.\ Rosedale?





``Oh, bring him by all means,''\ she said smiling; ``perhaps I can
get a tip out of him on my own account.''





Trenor paused abruptly, and his eyes fixed themselves on hers
with a look which made her change colour.





``I say, you know---you'll please remember he's a blooming
bounder,''\ he said; and with a slight laugh she turned toward the
open window near which they had been standing.





The throng in the room had increased, and she felt a desire for
space and fresh air. Both of these she found on the terrace,
where only a few men were lingering over cigarettes and liqueur,
while scattered couples strolled across the lawn to the
autumn-tinted borders of the flower-garden.





As she emerged, a man moved toward her from the knot of smokers,
and she found herself face to face with Selden. The stir of the
pulses which his nearness always caused was increased by a slight
sense of constraint. They had not met since their Sunday
afternoon walk at Bellomont, and that episode was still so vivid
to her that she could hardly believe him to be less conscious of
it. But his greeting expressed no more than the satisfaction
which every pretty woman expects to see reflected in masculine
eyes; and the discovery, if distasteful to her vanity, was
reassuring to her nerves. Between the relief of her escape from
Trenor, and the vague apprehension of her meeting with Rosedale,
it was pleasant to rest a moment on the sense of complete
understanding which Lawrence Selden's manner always conveyed.





``This is luck,''\ he said smiling. ``I was wondering if I should be
able to have a word with you before the special snatches us away. 
I came with Gerty Farish, and promised not to let her miss the
train, but I am sure she is still extracting sentimental solace
from the wedding presents. She appears to regard their number and
value as evidence of the disinterested affection of the
contracting parties.''





There was not the least trace of embarrassment in his voice, and
as he spoke, leaning slightly against the jamb of the window, and
letting his eyes rest on her in the frank enjoyment of her grace,
she felt with a faint chill of regret that he had gone back
without an effort to the footing on which they had stood before
their last talk together. Her vanity was stung by the sight of
his unscathed smile. She longed to be to him something more than
a piece of sentient prettiness, a passing diversion to his eye
and brain; and the longing betrayed itself in her reply.





``Ah,''\ she said, ``I envy Gerty that power she has of dressing up
with romance all our ugly and prosaic arrangements! I have never
recovered my self-respect since you showed me how poor and
unimportant my ambitions were.''





The words were hardly spoken when she realized their infelicity. 
It seemed to be her fate to appear at her worst to Selden.





``I thought, on the contrary,''\ he returned lightly, ``that I had
been the means of proving they were more important to you than
anything else.''





It was as if the eager current of her being had been checked by a
sudden obstacle which drove it back upon itself. She looked at
him helplessly, like a hurt or frightened child: this real self
of hers, which he had the faculty of drawing out of the depths,
was so little accustomed to go alone!





The appeal of her helplessness touched in him, as it always did,
a latent chord of inclination. It would have meant nothing to him
to discover that his nearness made her more brilliant, but this
glimpse of a twilight mood to which he alone had the clue seemed
once more to set him in a world apart with her.





``At least you can't think worse things of me than you say!''\ she
exclaimed with a trembling laugh; but before he could
answer, the flow of comprehension between them was abruptly
stayed by the reappearance of Gus Trenor, who advanced with Mr.
Rosedale in his wake.





``Hang it, Lily, I thought you'd given me the slip: Rosedale and I
have been hunting all over for you!''





His voice had a note of conjugal familiarity: Miss Bart fancied
she detected in Rosedale's eye a twinkling perception of the
fact, and the idea turned her dislike of him to repugnance.





She returned his profound bow with a slight nod, made more
disdainful by the sense of Selden's surprise that she should
number Rosedale among her acquaintances. Trenor had turned away,
and his companion continued to stand before Miss Bart, alert and
expectant, his lips parted in a smile at whatever she might be
about to say, and his very back conscious of the privilege of
being seen with her.





It was the moment for tact; for the quick bridging over of gaps;
but Selden still leaned against the window, a detached observer
of the scene, and under the spell of his observation Lily felt
herself powerless to exert her usual arts. The dread of Selden's
suspecting that there was any need for her to propitiate such a
man as Rosedale checked the trivial phrases of politeness. 
Rosedale still stood before her in an expectant attitude, and she
continued to face him in silence, her glance just level with his
polished baldness. The look put the finishing touch to what her
silence implied.





He reddened slowly, shifting from one foot to the other, fingered
the plump black pearl in his tie, and gave a nervous twist to his
moustache; then, running his eye over her, he drew back, and
said, with a side-glance at Selden: ``Upon my soul, I never saw a
more ripping get-up. Is that the last creation of the dress-maker
you go to see at the Benedick? If so, I wonder all the other
women don't go to her too!''





The words were projected sharply against Lily's silence, and she
saw in a flash that her own act had given them their emphasis. In
ordinary talk they might have passed unheeded; but following on
her prolonged pause they acquired a special meaning. She felt,
without looking, that Selden had immediately seized it, and would
inevitably connect the allusion with her visit to himself. The
consciousness increased her irritation against Rosedale, but also
her feeling that now, if ever, was the moment to
propitiate him, hateful as it was to do so in Selden's presence.





``How do you know the other women don't go to my dress-maker?''\ she
returned. ``You see I'm not afraid to give her address to my
friends!''





Her glance and accent so plainly included Rosedale in this
privileged circle that his small eyes puckered with
gratification, and a knowing smile drew up his moustache.





``By Jove, you needn't be!''\ he declared. ``You could give 'em the
whole outfit and win at a canter!''





``Ah, that's nice of you; and it would be nicer still if you would
carry me off to a quiet corner, and get me a glass of lemonade or
some innocent drink before we all have to rush for the train.''





She turned away as she spoke, letting him strut at her side
through the gathering groups on the terrace, while every nerve in
her throbbed with the consciousness of what Selden must have
thought of the scene.





But under her angry sense of the perverseness of things, and the
light surface of her talk with Rosedale, a third idea persisted: 
she did not mean to leave without an attempt to discover the
truth about Percy Gryce. Chance, or perhaps his own resolve, had
kept them apart since his hasty withdrawal from Bellomont; but
Miss Bart was an expert in making the most of the unexpected, and
the distasteful incidents of the last few minutes---the revelation
to Selden of precisely that part of her life which she most
wished him to ignore---increased her longing for shelter, for
escape from such humiliating contingencies. Any definite
situation would be more tolerable than this buffeting of chances,
which kept her in an attitude of uneasy alertness toward every
possibility of life.





Indoors there was a general sense of dispersal in the air, as of
an audience gathering itself up for departure after the principal
actors had left the stage; but among the remaining groups, Lily
could discover neither Gryce nor the youngest Miss Van Osburgh. 
That both should be missing struck her with foreboding; and she
charmed Mr.\ Rosedale by proposing that they should make their way
to the conservatories at the farther end of the house. There were
just enough people left in the long suite of rooms to make their
progress conspicuous, and Lily was aware of being followed
by looks of amusement and interrogation, which glanced off as
harmlessly from her indifference as from her companion's
self-satisfaction. She cared very little at that moment about
being seen with Rosedale: all her thoughts were centred on the
object of her search. The latter, however, was not discoverable
in the conservatories, and Lily, oppressed by a sudden conviction
of failure, was casting about for a way to rid herself of her now
superfluous companion, when they came upon Mrs.\ Van Osburgh,
flushed and exhausted, but beaming with the consciousness of duty
performed.





She glanced at them a moment with the benign but vacant eye of
the tired hostess, to whom her guests have become mere whirling
spots in a kaleidoscope of fatigue; then her attention became
suddenly fixed, and she seized on Miss Bart with a confidential
gesture. ``My dear Lily, I haven't had time for a word with you,
and now I suppose you are just off. Have you seen Evie? She's
been looking everywhere for you: she wanted to tell you her
little secret; but I daresay you have guessed it already. The
engagement is not to be announced till next week---but you are
such a friend of Mr.\ Gryce's that they both wished you to be the
first to know of their happiness.''





\chapter*{\raggedright Chapter 9}

\addcontentsline{toc}{chapter}{Chapter 9}

\markboth{House of Mirth}{Chapter 9}





In Mrs.\ Peniston's youth, fashion had returned to town in
October; therefore on the tenth day of the month the blinds of
her Fifth Avenue residence were drawn up, and the eyes of the
Dying Gladiator in bronze who occupied the drawing-room window
resumed their survey of that deserted thoroughfare.





The first two weeks after her return represented to Mrs.\ Peniston
the domestic equivalent of a religious retreat. She ``went
through''\ the linen and blankets in the precise spirit of the
penitent exploring the inner folds of conscience; she sought for
moths as the stricken soul seeks for lurking infirmities. The
topmost shelf of every closet was made to yield up its secret,
cellar and coal-bin were probed to their darkest depths and, as a
final stage in the lustral rites, the entire house was swathed in
penitential white and deluged with expiatory soapsuds.





It was on this phase of the proceedings that Miss Bart entered on
the afternoon of her return from the Van Osburgh wedding. The
journey back to town had not been calculated to soothe her
nerves. Though Evie Van Osburgh's engagement was still officially
a secret, it was one of which the innumerable intimate friends of
the family were already possessed; and the trainful of returning
guests buzzed with allusions and anticipations. Lily was acutely
aware of her own part in this drama of innuendo: she knew the
exact quality of the amusement the situation evoked. The crude
forms in which her friends took their pleasure included a loud
enjoyment of such complications: the zest of surprising destiny
in the act of playing a practical joke. Lily knew well enough how
to bear herself in difficult situations. She had, to a shade, the
exact manner between victory and defeat: every insinuation was
shed without an effort by the bright indifference of her manner. 
But she was beginning to feel the strain of the attitude; the
reaction was more rapid, and she lapsed to a deeper self-disgust.






As was always the case with her, this moral repulsion found a
physical outlet in a quickened distaste for her surroundings. 
She revolted from the complacent ugliness of Mrs.\ Peniston's
black walnut, from the slippery gloss of the vestibule tiles,
and the mingled odour of sapolio and furniture-polish that
met her at the door.





The stairs were still carpetless, and on the way up to her room
she was arrested on the landing by an encroaching tide of
soapsuds. Gathering up her skirts, she drew aside with an
impatient gesture; and as she did so she had the odd sensation of
having already found herself in the same situation but in
different surroundings. It seemed to her that she was again
descending the staircase from Selden's rooms; and looking down to
remonstrate with the dispenser of the soapy flood, she found
herself met by a lifted stare which had once before confronted
her under similar circumstances. It was the char-woman of the
Benedick who, resting on crimson elbows, examined her with the
same unflinching curiosity, the same apparent reluctance to let
her pass. On this occasion, however, Miss Bart was on her own
ground.





``Don't you see that I wish to go by? Please move your pail,''\ she
said sharply.





The woman at first seemed not to hear; then, without a word of
excuse, she pushed back her pail and dragged a wet floor-cloth
across the landing, keeping her eyes fixed on Lily while the
latter swept by. It was insufferable that Mrs.\ Peniston should
have such creatures about the house; and Lily entered her room
resolved that the woman should be dismissed that evening.





Mrs.\ Peniston, however, was at the moment inaccessible to
remonstrance: since early morning she had been shut up with her
maid, going over her furs, a process which formed the culminating
episode in the drama of household renovation. In the evening also
Lily found herself alone, for her aunt, who rarely dined out, had
responded to the summons of a Van Alstyne cousin who was passing
through town. The house, in its state of unnatural immaculateness
and order, was as dreary as a tomb, and as Lily, turning from her
brief repast between shrouded sideboards, wandered into the
newly-uncovered glare of the drawing-room she felt as though she
were buried alive in the stifling limits of Mrs.\ Peniston's
existence.





She usually contrived to avoid being at home during the season of
domestic renewal. On the present occasion, however, a variety of
reasons had combined to bring her to town; and foremost among
them was the fact that she had fewer invitations than usual for
the autumn. She had so long been accustomed to pass from one
country-house to another, till the close of the holidays brought
her friends to town, that the unfilled gaps of time confronting
her produced a sharp sense of waning popularity. It was as she
had said to Selden---people were tired of her. They would welcome
her in a new character, but as Miss Bart they knew her by heart. 
She knew herself by heart too, and was sick of the old story. 
There were moments when she longed blindly for anything
different, anything strange, remote and untried; but the utmost
reach of her imagination did not go beyond picturing her usual
life in a new setting. She could not figure herself as anywhere
but in a drawing-room, diffusing elegance as a flower sheds
perfume.





Meanwhile, as October advanced she had to face the alternative of
returning to the Trenors or joining her aunt in town. Even the
desolating dulness of New York in October, and the soapy
discomforts of Mrs.\ Peniston's interior, seemed preferable to
what might await her at Bellomont; and with an air of heroic
devotion she announced her intention of remaining with her
aunt till the holidays.





Sacrifices of this nature are sometimes received with feelings as
mixed as those which actuate them; and Mrs.\ Peniston remarked to
her confidential maid that, if any of the family were to be with
her at such a crisis (though for forty years she had been thought
competent to see to the hanging of her own curtains), she would
certainly have preferred Miss Grace to Miss Lily. Grace Stepney
was an obscure cousin, of adaptable manners and vicarious
interests, who ``ran in''\ to sit with Mrs.\ Peniston when Lily dined
out too continuously; who played bezique, picked up dropped
stitches, read out the deaths from the Times, and sincerely
admired the purple satin drawing-room curtains, the Dying
Gladiator in the window, and the seven-by-five painting of
Niagara which represented the one artistic excess of Mr.
Peniston's temperate career.





Mrs.\ Peniston, under ordinary circumstances, was as much
bored by her excellent cousin as the recipient of such services
usually is by the person who performs them. She greatly preferred
the brilliant and unreliable Lily, who did not know one end of a
crochet-needle from the other, and had frequently wounded her
susceptibilities by suggesting that the drawing-room should be
``done over.'' But when it came to hunting for missing napkins, or
helping to decide whether the backstairs needed re-carpeting,
Grace's judgment was certainly sounder than Lily's: not to
mention the fact that the latter resented the smell of beeswax
and brown soap, and behaved as though she thought a house ought
to keep clean of itself, without extraneous assistance.





Seated under the cheerless blaze of the drawing-room
chandelier---Mrs.\ Peniston never lit the lamps unless there was
``company''---Lily seemed to watch her own figure retreating down
vistas of neutral-tinted dulness to a middle age like Grace
Stepney's. When she ceased to amuse Judy Trenor and her friends
she would have to fall back on amusing Mrs.\ Peniston; whichever
way she looked she saw only a future of servitude to the whims of
others, never the possibility of asserting her own eager
individuality.





A ring at the door-bell, sounding emphatically through the empty
house, roused her suddenly to the extent of her boredom. It was
as though all the weariness of the past months had culminated in
the vacuity of that interminable evening. If only the ring meant
a summons from the outer world---a token that she was still
remembered and wanted!





After some delay a parlour-maid presented herself with the
announcement that there was a person outside who was asking to
see Miss Bart; and on Lily's pressing for a more specific
description, she added:\  



``It's Mrs.\ Haffen, Miss; she won't say what she wants.''





Lily, to whom the name conveyed nothing, opened the door upon a
woman in a battered bonnet, who stood firmly planted under the
hall-light. The glare of the unshaded gas shone familiarly on her
pock-marked face and the reddish baldness visible through thin
strands of straw-coloured hair. Lily looked at the char-woman in
surprise.





``Do you wish to see me?''\ she asked.





``I should like to say a word to you, Miss.'' The tone was
neither aggressive nor conciliatory: it revealed nothing of the
speaker's errand. Nevertheless, some precautionary instinct
warned Lily to withdraw beyond ear-shot of the hovering
parlour-maid.





She signed to Mrs.\ Haffen to follow her into the drawing-room,
and closed the door when they had entered.





``What is it that you wish?''\ she enquired.





The char-woman, after the manner of her kind, stood with her arms
folded in her shawl. Unwinding the latter, she produced a small
parcel wrapped in dirty newspaper.





``I have something here that you might like to see, Miss Bart.'' 
She spoke the name with an unpleasant emphasis, as though her
knowing it made a part of her reason for being there. To Lily the
intonation sounded like a threat.





``You have found something belonging to me?''\ she asked, extending
her hand.





Mrs.\ Haffen drew back. ``Well, if it comes to that, I guess it's
mine as much as anybody's,''\ she returned.





Lily looked at her perplexedly. She was sure, now, that her
visitor's manner conveyed a threat; but, expert as she was in
certain directions, there was nothing in her experience to
prepare her for the exact significance of the present scene. She
felt, however, that it must be ended as promptly as possible.





``I don't understand; if this parcel is not mine, why have you
asked for me?''





The woman was unabashed by the question. She was evidently
prepared to answer it, but like all her class she had to go a
long way back to make a beginning, and it was only after a pause
that she replied: ``My husband was janitor to the Benedick till
the first of the month; since then he can't get nothing to do.''





Lily remained silent and she continued: ``It wasn't no fault of
our own, neither: the agent had another man he wanted the place
for, and we was put out, bag and baggage, just to suit his fancy. 
I had a long sickness last winter, and an operation that ate up
all we'd put by; and it's hard for me and the children, Haffen
being so long out of a job.''





After all, then, she had come only to ask Miss Bart to find a
place for her husband; or, more probably, to seek the young
lady's intervention with Mrs.\ Peniston. Lily had such an air
of always getting what she wanted that she was used to being
appealed to as an intermediary, and, relieved of her vague
apprehension, she took refuge in the conventional formula.





``I am sorry you have been in trouble,''\ she said.





``Oh, that we have, Miss, and it's on'y just beginning. If on'y
we'd 'a got another situation---but the agent, he's dead against
us. It ain't no fault of ours, neither, but----''





At this point Lily's impatience overcame her. ``If you have
anything to say to me----''\ she interposed.





The woman's resentment of the rebuff seemed to spur her lagging ideas.





``Yes, Miss; I'm coming to that,''\ she said. She paused again, with
her eyes on Lily, and then continued, in a tone of diffuse
narrative: ``When we was at the Benedick I had charge of some of
the gentlemen's rooms; leastways, I swep'\ 'em out on Saturdays. 
Some of the gentlemen got the greatest sight of letters: I never
saw the like of it. Their waste-paper baskets 'd be fairly
brimming, and papers falling over on the floor. Maybe havin'\ so
many is how they get so careless. Some of 'em is worse than
others. Mr.\ Selden, Mr.\ Lawrence Selden, he was always one of the
carefullest: burnt his letters in winter, and tore 'em in little
bits in summer. But sometimes he'd have so many he'd just bunch
'em together, the way the others did, and tear the lot through
once---like this.''





While she spoke she had loosened the string from the parcel in
her hand, and now she drew forth a letter which she laid on the
table between Miss Bart and herself. As she had said, the letter
was torn in two; but with a rapid gesture she laid the torn edges
together and smoothed out the page.





A wave of indignation swept over Lily. She felt herself in the
presence of something vile, as yet but dimly conjectured---the
kind of vileness of which people whispered, but which she had
never thought of as touching her own life. She drew back with a
motion of disgust, but her withdrawal was checked by a sudden
discovery: under the glare of Mrs.\ Peniston's chandelier she had
recognized the hand-writing of the letter. It was a large
disjointed hand, with a flourish of masculinity which but
slightly disguised its rambling weakness, and the words, scrawled
in heavy ink on pale-tinted notepaper, smote on Lily's
ear as though she had heard them spoken.





At first she did not grasp the full import of the situation. She
understood only that before her lay a letter written by Bertha
Dorset, and addressed, presumably, to Lawrence Selden. There was
no date, but the blackness of the ink proved the writing to be
comparatively recent. The packet in Mrs.\ Haffen's hand doubtless
contained more letters of the same kind---a dozen, Lily
conjectured from its thickness. The letter before her was short,
but its few words, which had leapt into her brain before she was
conscious of reading them, told a long history---a history over
which, for the last four years, the friends of the writer had
smiled and shrugged, viewing it merely as one among the countless
``good situations''\ of the mundane comedy. Now the other side
presented itself to Lily, the volcanic nether side of the surface
over which conjecture and innuendo glide so lightly till the
first fissure turns their whisper to a shriek. Lily knew that
there is nothing society resents so much as having given its
protection to those who have not known how to profit by it: it is
for having betrayed its connivance that the body social punishes
the offender who is found out. And in this case there was no
doubt of the issue. The code of Lily's world decreed that a
woman's husband should be the only judge of her conduct: she was
technically above suspicion while she had the shelter of his
approval, or even of his indifference. But with a man of George
Dorset's temper there could be no thought of condonation---the
possessor of his wife's letters could overthrow with a touch the
whole structure of her existence. And into what hands Bertha
Dorset's secret had been delivered! For a moment the irony of the
coincidence tinged Lily's disgust with a confused sense of
triumph. But the disgust prevailed---all her instinctive
resistances, of taste, of training, of blind inherited scruples,
rose against the other feeling. Her strongest sense was one of
personal contamination.





She moved away, as though to put as much distance as possible
between herself and her visitor. ``I know nothing of these
letters,''\ she said; ``I have no idea why you have brought them
here.''





Mrs.\ Haffen faced her steadily. ``I'll tell you why, Miss. 
I brought 'em to you to sell, because I ain't got no other way
of raising money, and if we don't pay our rent by tomorrow night
we'll be put out. I never done anythin'\ of the kind before, and
if you'd speak to Mr.\ Selden or to Mr.\ Rosedale about getting
Haffen taken on again at the Benedick---I seen you talking to Mr.
Rosedale on the steps that day you come out of Mr.\ Selden's
rooms----''





The blood rushed to Lily's forehead. She understood now---Mrs.
Haffen supposed her to be the writer of the letters. In the first
leap of her anger she was about to ring and order the woman out;
but an obscure impulse restrained her. The mention of Selden's
name had started a new train of thought. Bertha Dorset's letters
were nothing to her---they might go where the current of chance
carried them! But Selden was inextricably involved in their fate. 
Men do not, at worst, suffer much from such exposure; and in this
instance the flash of divination which had carried the meaning of
the letters to Lily's brain had revealed also that they were
appeals---repeated and therefore probably unanswered---for the
renewal of a tie which time had evidently relaxed. Nevertheless,
the fact that the correspondence had been allowed to fall into
strange hands would convict Selden of negligence in a matter
where the world holds it least pardonable; and there were graver
risks to consider where a man of Dorset's ticklish balance was
concerned.





If she weighed all these things it was unconsciously: she was
aware only of feeling that Selden would wish the letters rescued,
and that therefore she must obtain possession of them. Beyond
that her mind did not travel. She had, indeed, a quick vision of
returning the packet to Bertha Dorset, and of the opportunities
the restitution offered; but this thought lit up abysses from
which she shrank back ashamed.





Meanwhile Mrs.\ Haffen, prompt to perceive her hesitation, had
already opened the packet and ranged its contents on the table. 
All the letters had been pieced together with strips of thin
paper. Some were in small fragments, the others merely torn in
half. Though there were not many, thus spread out they nearly
covered the table. Lily's glance fell on a word here and
there---then she said in a low voice: ``What do you wish me to pay
you?''





Mrs.\ Haffen's face reddened with satisfaction. It was clear that
the young lady was badly frightened, and Mrs.\ Haffen was the
woman to make the most of such fears. Anticipating an easier
victory than she had foreseen, she named an exorbitant sum.





But Miss Bart showed herself a less ready prey than might have
been expected from her imprudent opening. She refused to pay the
price named, and after a moment's hesitation, met it by a
counter-offer of half the amount.





Mrs.\ Haffen immediately stiffened. Her hand travelled toward the
outspread letters, and folding them slowly, she made as though to
restore them to their wrapping.





``I guess they're worth more to you than to me, Miss, but the poor
has got to live as well as the rich,''\ she observed sententiously.






Lily was throbbing with fear, but the insinuation fortified her
resistance.





``You are mistaken,''\ she said indifferently. ``I have offered all I
am willing to give for the letters; but there may be other ways
of getting them.''





Mrs.\ Haffen raised a suspicious glance: she was too experienced
not to know that the traffic she was engaged in had perils as
great as its rewards, and she had a vision of the elaborate
machinery of revenge which a word of this commanding young lady's
might set in motion.





She applied the corner of her shawl to her eyes, and murmured
through it that no good came of bearing too hard on the poor, but
that for her part she had never been mixed up in such a business
before, and that on her honour as a Christian all she and Haffen
had thought of was that the letters mustn't go any farther.





Lily stood motionless, keeping between herself and the char-woman
the greatest distance compatible with the need of speaking in low
tones. The idea of bargaining for the letters was intolerable to
her, but she knew that, if she appeared to weaken, Mrs.\ Haffen
would at once increase her original demand.





She could never afterward recall how long the duel lasted, or
what was the decisive stroke which finally, after a lapse of time
recorded in minutes by the clock, in hours by the precipitate
beat of her pulses, put her in possession of the letters; she
knew only that the door had finally closed, and that she stood
alone with the packet in her hand.





She had no idea of reading the letters; even to unfold Mrs.
Haffen's dirty newspaper would have seemed degrading. But what
did she intend to do with its contents? The recipient of the
letters had meant to destroy them, and it was her duty to carry
out his intention. She had no right to keep them---to do so was to
lessen whatever merit lay in having secured their possession. But
how destroy them so effectually that there should be no second
risk of their falling in such hands? Mrs.\ Peniston's icy
drawing-room grate shone with a forbidding lustre: the fire, like
the lamps, was never lit except when there was company.





Miss Bart was turning to carry the letters upstairs when she
heard the opening of the outer door, and her aunt entered the
drawing-room. Mrs.\ Peniston was a small plump woman, with a
colourless skin lined with trivial wrinkles. Her grey hair was
arranged with precision, and her clothes looked excessively new
and yet slightly old-fashioned. They were always black and
tightly fitting, with an expensive glitter: she was the kind of
woman who wore jet at breakfast. Lily had never seen her when she
was not cuirassed in shining black, with small tight boots, and
an air of being packed and ready to start; yet she never started.





She looked about the drawing-room with an expression of minute
scrutiny. ``I saw a streak of light under one of the blinds as I
drove up: it's extraordinary that I can never teach that woman to
draw them down evenly.''





Having corrected the irregularity, she seated herself on one of
the glossy purple arm-chairs; Mrs.\ Peniston always sat on a
chair, never in it.





Then she turned her glance to Miss Bart. ``My dear, you look
tired; I suppose it's the excitement of the wedding. Cornelia Van
Alstyne was full of it: Molly was there, and Gerty Farish ran in
for a minute to tell us about it. I think it was odd, their
serving melons before the CONSOMME: a wedding breakfast should
always begin with CONSOMME.\ Molly didn't care for the
bridesmaids'\ dresses. She had it straight from Julia Melson that
they cost three hundred dollars apiece at Celeste's, but she says
they didn't look it. I'm glad you decided not to be a
bridesmaid; that shade of salmon-pink wouldn't have suited you.'' 
Mrs.\ Peniston delighted in discussing the minutest details of
festivities in which she had not taken part. Nothing would have
induced her to undergo the exertion and fatigue of attending the
Van Osburgh wedding, but so great was her interest in the event
that, having heard two versions of it, she now prepared to
extract a third from her niece. Lily, however, had been
deplorably careless in noting the particulars of the
entertainment. She had failed to observe the colour of Mrs.\ Van
Osburgh's gown, and could not even say whether the old Van
Osburgh S\`{e}vres had been used at the bride's table: Mrs.\ Peniston,
in short, found that she was of more service as a listener than
as a narrator.





``Really, Lily, I don't see why you took the trouble to go to the
wedding, if you don't remember what happened or whom you saw
there. When I was a girl I used to keep the \textit{menu} of every dinner
I went to, and write the names of the people on the back; and I
never threw away my cotillion favours till after your uncle's
death, when it seemed unsuitable to have so many coloured things
about the house. I had a whole closet-full, I remember; and I can
tell to this day what balls I got them at. Molly Van Alstyne
reminds me of what I was at that age; it's wonderful how she
notices. She was able to tell her mother exactly how the
wedding-dress was cut, and we knew at once, from the fold in the
back, that it must have come from Paquin.''





Mrs.\ Peniston rose abruptly, and, advancing to the ormolu clock
surmounted by a helmeted Minerva, which throned on the
chimney-piece between two malachite vases, passed her lace
handkerchief between the helmet and its visor.





``I knew it---the parlour-maid never dusts there!''\ she exclaimed,
triumphantly displaying a minute spot on the handkerchief; then,
reseating herself, she went on: ``Molly thought Mrs.\ Dorset the
best-dressed woman at the wedding. I've no doubt her dress \textit{did}
cost more than any one else's, but I can't quite like the idea---a
combination of sable and \textit{point} \textit{de} \textit{Milan}. It seems she goes to a
new man in Paris, who won't take an order till his client has
spent a day with him at his villa at Neuilly. He says he must
study his subject's home life---a most peculiar
arrangement, I should say! But Mrs.\ Dorset told Molly about it
herself: she said the villa was full of the most exquisite things
and she was really sorry to leave. Molly said she never saw her
looking better; she was in tremendous spirits, and said she had
made a match between Evie Van Osburgh and Percy Gryce. She really
seems to have a very good influence on young men. I hear she is
interesting herself now in that silly Silverton boy, who has had
his head turned by Carry Fisher, and has been gambling so
dreadfully. Well, as I was saying, Evie is really engaged: Mrs.
Dorset had her to stay with Percy Gryce, and managed it all, and
Grace Van Osburgh is in the seventh heaven---she had almost
despaired of marrying Evie.''





Mrs.\ Peniston again paused, but this time her scrutiny addressed
itself, not to the furniture, but to her niece.





``Cornelia Van Alstyne was so surprised: she had heard that you
were to marry young Gryce. She saw the Wetheralls just after they
had stopped with you at Bellomont, and Alice Wetherall was quite
sure there was an engagement. She said that when Mr.\ Gryce left
unexpectedly one morning, they all thought he had rushed to town
for the ring.''





Lily rose and moved toward the door.





``I believe I \textit{am} tired: I think I will go to bed,''\ she said; and
Mrs.\ Peniston, suddenly distracted by the discovery that the
easel sustaining the late Mr.\ Peniston's crayon-portrait was not
exactly in line with the sofa in front of it, presented an
absent-minded brow to her kiss.





In her own room Lily turned up the gas-jet and glanced toward the
grate. It was as brilliantly polished as the one below, but here
at least she could burn a few papers with less risk of incurring
her aunt's disapproval. She made no immediate motion to do so,
however, but dropping into a chair looked wearily about her. Her
room was large and comfortably-furnished---it was the envy and
admiration of poor Grace Stepney, who boarded; but, contrasted
with the light tints and luxurious appointments of the
guest-rooms where so many weeks of Lily's existence were spent,
it seemed as dreary as a prison. The monumental wardrobe and
bedstead of black walnut had migrated from Mr.\ Peniston's
bedroom, and the magenta ``flock''\ wall-paper, of a pattern dear to
the early 'sixties, was hung with large steel engravings
of an anecdotic character. Lily had tried to mitigate this
charmless background by a few frivolous touches, in the shape of
a lace-decked toilet table and a little painted desk surmounted
by photographs; but the futility of the attempt struck her as she
looked about the room. What a contrast to the subtle elegance of
the setting she had pictured for herself---an apartment which
should surpass the complicated luxury of her friends'
surroundings by the whole extent of that artistic sensibility
which made her feel herself their superior; in which every tint
and line should combine to enhance her beauty and give
distinction to her leisure! Once more the haunting sense of
physical ugliness was intensified by her mental depression, so
that each piece of the offending furniture seemed to thrust forth
its most aggressive angle.





Her aunt's words had told her nothing new; but they had revived
the vision of Bertha Dorset, smiling, flattered, victorious,
holding her up to ridicule by insinuations intelligible to every
member of their little group. The thought of the ridicule struck
deeper than any other sensation: Lily knew every turn of the
allusive jargon which could flay its victims without the shedding
of blood. Her cheek burned at the recollection, and she rose and
caught up the letters. She no longer meant to destroy them: that
intention had been effaced by the quick corrosion of Mrs.
Peniston's words.





Instead, she approached her desk, and lighting a taper, tied and
sealed the packet; then she opened the wardrobe, drew out a
despatch-box, and deposited the letters within it. As she did so,
it struck her with a flash of irony that she was indebted to Gus
Trenor for the means of buying them.





\chapter*{\raggedright Chapter 10}

\addcontentsline{toc}{chapter}{Chapter 10}

\markboth{House of Mirth}{Chapter 10}





The autumn dragged on monotonously. Miss Bart had received one
or two notes from Judy Trenor, reproaching her for not returning
to Bellomont; but she replied evasively, alleging the obligation
to remain with her aunt. In truth, however, she was fast wearying
of her solitary existence with Mrs.\ Peniston, and only the
excitement of spending her newly-acquired money lightened the
dulness of the days.





All her life Lily had seen money go out as quickly as it came in,
and whatever theories she cultivated as to the prudence of
setting aside a part of her gains, she had unhappily no saving
vision of the risks of the opposite course. It was a keen
satisfaction to feel that, for a few months at least, she would
be independent of her friends'\ bounty, that she could show
herself abroad without wondering whether some penetrating eye
would detect in her dress the traces of Judy Trenor's refurbished
splendour. The fact that the money freed her temporarily from all
minor obligations obscured her sense of the greater one it
represented, and having never before known what it was to command
so large a sum, she lingered delectably over the amusement of
spending it.





It was on one of these occasions that, leaving a shop where she
had spent an hour of deliberation over a dressing-case of the
most complicated elegance, she ran across Miss Farish, who had
entered the same establishment with the modest object of having
her watch repaired. Lily was feeling unusually virtuous. She had
decided to defer the purchase of the dressing-case till she
should receive the bill for her new opera-cloak, and the resolve
made her feel much richer than when she had entered the shop. In
this mood of self-approval she had a sympathetic eye for others,
and she was struck by her friend's air of dejection.





Miss Farish, it appeared, had just left the committee-meeting of
a struggling charity in which she was interested. The object of
the association was to provide comfortable lodgings, with a
reading-room and other modest distractions, where young women of
the class employed in down town offices might find a home when
out of work, or in need of rest, and the first year's
financial report showed so deplorably small a balance that Miss
Farish, who was convinced of the urgency of the work, felt
proportionately discouraged by the small amount of interest it
aroused. The other-regarding sentiments had not been cultivated
in Lily, and she was often bored by the relation of her friend's
philanthropic efforts, but today her quick dramatizing fancy
seized on the contrast between her own situation and that
represented by some of Gerty's ``cases.'' These were young girls,
like herself; some perhaps pretty, some not without a trace of
her finer sensibilities. She pictured herself leading such a life
as theirs---a life in which achievement seemed as squalid as
failure---and the vision made her shudder sympathetically. The
price of the dressing-case was still in her pocket; and drawing
out her little gold purse she slipped a liberal fraction of the
amount into Miss Farish's hand.





The satisfaction derived from this act was all that the most
ardent moralist could have desired. Lily felt a new interest in
herself as a person of charitable instincts: she had never before
thought of doing good with the wealth she had so often dreamed of
possessing, but now her horizon was enlarged by the vision of a
prodigal philanthropy. Moreover, by some obscure process of
logic, she felt that her momentary burst of generosity had
justified all previous extravagances, and excused any in which
she might subsequently indulge. Miss Farish's surprise and
gratitude confirmed this feeling, and Lily parted from her with a
sense of self-esteem which she naturally mistook for the fruits
of altruism.





About this time she was farther cheered by an invitation to spend
the Thanksgiving week at a camp in the Adirondacks. The
invitation was one which, a year earlier, would have provoked a
less ready response, for the party, though organized by Mrs.
Fisher, was ostensibly given by a lady of obscure origin and
indomitable social ambitions, whose acquaintance Lily had
hitherto avoided. Now, however, she was disposed to coincide with
Mrs.\ Fisher's view, that it didn't matter who gave the party, as
long as things were well done; and doing things well (under
competent direction)\ was Mrs.\ Wellington Bry's strong point. The
lady (whose consort was known as ``Welly''\ Bry on the Stock
Exchange and in sporting circles)\ had already sacrificed
one husband, and sundry minor considerations, to her
determination to get on; and, having obtained a hold on Carry
Fisher, she was astute enough to perceive the wisdom of
committing herself entirely to that lady's guidance. Everything,
accordingly, was well done, for there was no limit to Mrs.
Fisher's prodigality when she was not spending her own money, and
as she remarked to her pupil, a good cook was the best
introduction to society. If the company was not as select as the
\textit{cuisine}, the Welly Brys at least had the satisfaction of
figuring for the first time in the society columns in company
with one or two noticeable names; and foremost among these was of
course Miss Bart's. The young lady was treated by her hosts with
corresponding deference; and she was in the mood when such
attentions are acceptable, whatever their source. Mrs.\ Bry's
admiration was a mirror in which Lily's self-complacency
recovered its lost outline. No insect hangs its nest on threads
as frail as those which will sustain the weight of human vanity;
and the sense of being of importance among the insignificant was
enough to restore to Miss Bart the gratifying consciousness of
power. If these people paid court to her it proved that she was
still conspicuous in the world to which they aspired; and she was
not above a certain enjoyment in dazzling them by her fineness,
in developing their puzzled perception of her superiorities.





Perhaps, however, her enjoyment proceeded more than she was aware
from the physical stimulus of the excursion, the challenge of
crisp cold and hard exercise, the responsive thrill of her body
to the influences of the winter woods. She returned to town in a
glow of rejuvenation, conscious of a clearer colour in her
cheeks,
a fresh elasticity in her muscles. The future seemed full of a
vague promise, and all her apprehensions were swept out of sight
on the buoyant current of her mood.





A few days after her return to town she had the unpleasant
surprise of a visit from Mr.\ Rosedale. He came late, at the
confidential hour when the tea-table still lingers by the fire in
friendly expectancy; and his manner showed a readiness to adapt
itself to the intimacy of the occasion.





Lily, who had a vague sense of his being somehow connected
with her lucky speculations, tried to give him the welcome he
expected; but there was something in the quality of his geniality
which chilled her own, and she was conscious of marking each step
in their acquaintance by a fresh blunder.





Mr.\ Rosedale---making himself promptly at home in an adjoining
easy-chair, and sipping his tea critically, with the comment: 
``You ought to go to my man for something really good''---appeared
totally unconscious of the repugnance which kept her in frozen
erectness behind the urn. It was perhaps her very manner of
holding herself aloof that appealed to his collector's passion
for the rare and unattainable. He gave, at any rate, no sign of
resenting it and seemed prepared to supply in his own manner all
the ease that was lacking in hers.





His object in calling was to ask her to go to the opera in his
box on the opening night, and seeing her hesitate he said
persuasively: ``Mrs.\ Fisher is coming, and I've secured a
tremendous admirer of yours, who'll never forgive me if you don't
accept.''





As Lily's silence left him with this allusion on his hands, he
added with a confidential smile: ``Gus Trenor has promised to come
to town on purpose. I fancy he'd go a good deal farther for the
pleasure of seeing you.''





Miss Bart felt an inward motion of annoyance: it was distasteful
enough to hear her name coupled with Trenor's, and on Rosedale's
lips the allusion was peculiarly unpleasant.





``The Trenors are my best friends---I think we should all go a long
way to see each other,''\ she said, absorbing herself in the
preparation of fresh tea.





Her visitor's smile grew increasingly intimate. ``Well, I wasn't
thinking of Mrs.\ Trenor at the moment---they say Gus doesn't
always, you know.'' Then, dimly conscious that he had not struck
the right note, he added, with a well-meant effort at diversion: 
``How's your luck been going in Wall Street, by the way? I hear
Gus pulled off a nice little pile for you last month.''





Lily put down the tea-caddy with an abrupt gesture. She felt that
her hands were trembling, and clasped them on her knee to steady
them; but her lip trembled too, and for a moment she was afraid
the tremor might communicate itself to her voice. When
she spoke, however, it was in a tone of perfect lightness.





``Ah, yes---I had a little bit of money to invest, and Mr.\ Trenor,
who helps me about such matters, advised my putting it in stocks
instead of a mortgage, as my aunt's agent wanted me to do; and as
it happened, I made a lucky 'turn'---is that what you call it? For
you make a great many yourself, I believe.''





She was smiling back at him now, relaxing the tension of her
attitude, and admitting him, by imperceptible gradations of
glance and manner, a step farther toward intimacy. The protective
instinct always nerved her to successful dissimulation, and it
was not the first time she had used her beauty to divert
attention from an inconvenient topic.





When Mr.\ Rosedale took leave, he carried with him, not only her
acceptance of his invitation, but a general sense of having
comported himself in a way calculated to advance his cause. He
had always believed he had a light touch and a knowing way with
women, and the prompt manner in which Miss Bart (as he would have
phrased it)\ had ``come into line,''\ confirmed his confidence in his
powers of handling this skittish sex. Her way of glossing over
the transaction with Trenor he regarded at once as a tribute to
his own acuteness, and a confirmation of his suspicions. The girl
was evidently nervous, and Mr.\ Rosedale, if he saw no other means
of advancing his acquaintance with her, was not above taking
advantage of her nervousness.





He left Lily to a passion of disgust and fear. It seemed
incredible that Gus Trenor should have spoken of her to Rosedale. 
With all his faults, Trenor had the safeguard of his traditions,
and was the less likely to overstep them because they were so
purely instinctive. But Lily recalled with a pang that there were
convivial moments when, as Judy had confided to her, Gus ``talked
foolishly'': in one of these, no doubt, the fatal word had slipped
from him. As for Rosedale, she did not, after the first shock,
greatly care what conclusions he had drawn. Though usually adroit
enough where her own interests were concerned, she made the
mistake, not uncommon to persons in whom the social habits are
instinctive, of supposing that the inability to acquire them
quickly implies a general dulness. Because a blue-bottle
bangs irrationally against a window-pane, the drawing-room
naturalist may forget that under less artificial conditions it is
capable of measuring distances and drawing conclusions with all
the accuracy needful to its welfare; and the fact that Mr.
Rosedale's drawing-room manner lacked perspective made Lily class
him with Trenor and the other dull men she knew, and assume that
a little flattery, and the occasional acceptance of his
hospitality, would suffice to render him innocuous. However,
there could be no doubt of the expediency of showing herself in
his box on the opening night of the opera; and after all, since
Judy Trenor had promised to take him up that winter, it was as
well to reap the advantage of being first in the field.





For a day or two after Rosedale's visit, Lily's thoughts were
dogged by the consciousness of Trenor's shadowy claim, and she
wished she had a clearer notion of the exact nature of the
transaction which seemed to have put her in his power; but her
mind shrank from any unusual application, and she was always
helplessly puzzled by figures. Moreover she had not seen Trenor
since the day of the Van Osburgh wedding, and in his continued
absence the trace of Rosedale's words was soon effaced by other
impressions.





When the opening night of the opera came, her apprehensions had
so completely vanished that the sight of Trenor's ruddy
countenance in the back of Mr.\ Rosedale's box filled her with a
sense of pleasant reassurance. Lily had not quite reconciled
herself to the necessity of appearing as Rosedale's guest on so
conspicuous an occasion, and it was a relief to find herself
supported by any one of her own set---for Mrs.\ Fisher's social
habits were too promiscuous for her presence to justify Miss
Bart's.





To Lily, always inspirited by the prospect of showing her beauty
in public, and conscious tonight of all the added enhancements of
dress, the insistency of Trenor's gaze merged itself in the
general stream of admiring looks of which she felt herself the
centre. Ah, it was good to be young, to be radiant, to glow with
the sense of slenderness, strength and elasticity, of well-poised
lines and happy tints, to feel one's self lifted to a height
apart by that incommunicable grace which is the bodily
counterpart of genius!





All means seemed justifiable to attain such an end, or rather, by
a happy shifting of lights with which practice had familiarized
Miss Bart, the cause shrank to a pin-point in the general
brightness of the effect. But brilliant young ladies, a little
blinded by their own effulgence, are apt to forget that the
modest satellite drowned in their light is still performing its
own revolutions and generating heat at its own rate. If Lily's
poetic enjoyment of the moment was undisturbed by the base
thought that her gown and opera cloak had been indirectly paid
for by Gus Trenor, the latter had not sufficient poetry in his
composition to lose sight of these prosaic facts. He knew only
that he had never seen Lily look smarter in her life, that there
wasn't a woman in the house who showed off good clothes as she
did, and that hitherto he, to whom she owed the opportunity of
making this display, had reaped no return beyond that of gazing
at her in company with several hundred other pairs of eyes.





It came to Lily therefore as a disagreeable surprise when, in the
back of the box, where they found themselves alone between two
acts, Trenor said, without preamble, and in a tone of sulky
authority: ``Look here, Lily, how is a fellow ever to see anything
of you? I'm in town three or four days in the week, and you know
a line to the club will always find me, but you don't seem to
remember my existence nowadays unless you want to get a tip out
of me.''





The fact that the remark was in distinctly bad taste did not make
it any easier to answer, for Lily was vividly aware that it was
not the moment for that drawing up of her slim figure and
surprised lifting of the brows by which she usually quelled
incipient signs of familiarity.





``I'm very much flattered by your wanting to see me,''\ she
returned, essaying lightness instead, ``but, unless you have
mislaid my address, it would have been easy to find me any
afternoon at my aunt's---in fact, I rather expected you to look me
up there.''





If she hoped to mollify him by this last concession the attempt
was a failure, for he only replied, with the familiar lowering of
the brows that made him look his dullest when he was angry: ``Hang
going to your aunt's, and wasting the afternoon listening to a
lot of other chaps talking to you! You know I'm not the
kind to sit in a crowd and jaw---I'd always rather clear out when
that sort of circus is going on. But why can't we go off
somewhere on a little lark together---a nice quiet little
expedition like that drive at Bellomont, the day you met me at
the station?''





He leaned unpleasantly close in order to convey this suggestion,
and she fancied she caught a significant aroma which explained
the dark flush on his face and the glistening dampness of his
forehead.





The idea that any rash answer might provoke an unpleasant
outburst tempered her disgust with caution, and she answered with
a laugh: ``I don't see how one can very well take country drives
in town, but I am not always surrounded by an admiring throng,
and if you will let me know what afternoon you are coming I will
arrange things so that we can have a nice quiet talk.''





``Hang talking! That's what you always say,''\ returned Trenor,
whose expletives lacked variety. ``You put me off with that at the
Van Osburgh wedding---but the plain English of it is that, now
you've got what you wanted out of me, you'd rather have any other
fellow about.''





His voice had risen sharply with the last words, and Lily flushed
with annoyance, but she kept command of the situation and laid a
persuasive hand on his arm.





``Don't be foolish, Gus; I can't let you talk to me in that
ridiculous way. If you really want to see me, why shouldn't we
take a walk in the Park some afternoon? I agree with you that
it's amusing to be rustic in town, and if you like I'll meet you
there, and we'll go and feed the squirrels, and you shall take me
out on the lake in the steam-gondola.''





She smiled as she spoke, letting her eyes rest on his in a way
that took the edge from her banter and made him suddenly
malleable to her will.





``All right, then: that's a go. Will you come tomorrow? Tomorrow
at three o'clock, at the end of the Mall. I'll be there sharp,
remember; you won't go back on me, Lily?''





But to Miss Bart's relief the repetition of her promise was cut
short by the opening of the box door to admit George Dorset.





Trenor sulkily yielded his place, and Lily turned a brilliant
smile on the newcomer. She had not talked with Dorset since
their visit at Bellomont, but something in his look and manner
told her that he recalled the friendly footing on which they had
last met. He was not a man to whom the expression of admiration
came easily: his long sallow face and distrustful eyes seemed
always barricaded against the expansive emotions. But, where her
own influence was concerned, Lily's intuitions sent out
thread-like feelers, and as she made room for him on the narrow
sofa she was sure he found a dumb pleasure in being near her. Few
women took the trouble to make themselves agreeable to Dorset,
and Lily had been kind to him at Bellomont, and was now smiling
on him with a divine renewal of kindness.





``Well, here we are, in for another six months of caterwauling,''
he began complainingly. ``Not a shade of difference between this
year and last, except that the women have got new clothes and the
singers haven't got new voices. My wife's musical, you know---puts
me through a course of this every winter. It isn't so bad on
Italian nights---then she comes late, and there's time to digest. 
But when they give Wagner we have to rush dinner, and I pay up
for it. And the draughts are damnable---asphyxia in front and
pleurisy in the back. There's Trenor leaving the box without
drawing the curtain! With a hide like that draughts don't make
any difference. Did you ever watch Trenor eat? If you did, you'd
wonder why he's alive; I suppose he's leather inside too.---But I
came to say that my wife wants you to come down to our place next
Sunday. Do for heaven's sake say yes. She's got a lot of bores
coming---intellectual ones, I mean; that's her new line, you
know, and I'm not sure it ain't worse than the music. Some of 'em
have long hair, and they start an argument with the soup, and
don't notice when things are handed to them. The consequence is
the dinner gets cold, and I have dyspepsia. That silly ass
Silverton brings them to the house---he writes poetry, you know,
and Bertha and he are getting tremendously thick. She could write
better than any of 'em if she chose, and I don't blame her for
wanting clever fellows about; all I say is: 'Don't let me see 'em
eat!'''





The gist of this strange communication gave Lily a distinct
thrill of pleasure. Under ordinary circumstances, there would
have been nothing surprising in an invitation from Bertha Dorset;
but since the Bellomont episode an unavowed hostility had
kept the two women apart. Now, with a start of inner wonder,
Lily felt that her thirst for retaliation had died out. \textit{If} \textit{you} \textit{would}
\textit{forgive} \textit{your} \textit{enemy}, says the Malay proverb, \textit{first} \textit{inflict} A \textit{hurt}
\textit{on} \textit{him}; and Lily was experiencing the truth of the apothegm. 
If she had destroyed Mrs.\ Dorset's letters, she might have
continued to hate her; but the fact that they remained in her
possession had fed her resentment to satiety.





She uttered a smiling acceptance, hailing in the renewal of the
tie an escape from Trenor's importunities.





\chapter*{\raggedright Chapter 11}

\addcontentsline{toc}{chapter}{Chapter 11}

\markboth{House of Mirth}{Chapter 11}





Meanwhile the holidays had gone by and the season was beginning. 
Fifth Avenue had become a nightly torrent of carriages surging
upward to the fashionable quarters about the Park, where
illuminated windows and outspread awnings betokened the usual
routine of hospitality. Other tributary currents crossed the
mainstream, bearing their freight to the theatres, restaurants or
opera; and Mrs.\ Peniston, from the secluded watch-tower of her
upper window, could tell to a nicety just when the chronic volume
of sound was increased by the sudden influx setting toward a Van
Osburgh ball, or when the multiplication of wheels meant merely
that the opera was over, or that there was a big supper at
Sherry's.





Mrs.\ Peniston followed the rise and culmination of the season as
keenly as the most active sharer in its gaieties; and, as a
looker-on, she enjoyed opportunities of comparison and
generalization such as those who take part must proverbially
forego. No one could have kept a more accurate record of social
fluctuations, or have put a more unerring finger on the
distinguishing features of each season: its dulness, its
extravagance, its lack of balls or excess of divorces. She had a
special memory for the vicissitudes of the ``new people''\ who rose
to the surface with each recurring tide, and were either
submerged beneath its rush or landed triumphantly beyond the
reach of envious breakers; and she was apt to display a
remarkable retrospective insight into their ultimate fate, so
that, when they had fulfilled their destiny, she was almost
always able to say to Grace Stepney---the recipient of her
prophecies---that she had known exactly what would happen.





This particular season Mrs.\ Peniston would have characterized as
that in which everybody ``felt poor''\ except the Welly Brys and Mr.
Simon Rosedale. It had been a bad autumn in Wall Street, where
prices fell in accordance with that peculiar law which proves
railway stocks and bales of cotton to be more sensitive to the
allotment of executive power than many estimable citizens trained
to all the advantages of self-government. Even fortunes supposed
to be independent of the market either betrayed a secret
dependence on it, or suffered from a sympathetic affection: 
fashion sulked in its country houses, or came to town incognito,
general entertainments were discountenanced, and informality and
short dinners became the fashion.





But society, amused for a while at playing Cinderella, soon
wearied of the hearthside r\^{o}le, and welcomed the Fairy Godmother
in the shape of any magician powerful enough to turn the shrunken
pumpkin back again into the golden coach. The mere fact of
growing richer at a time when most people's investments are
shrinking, is calculated to attract envious attention; and
according to Wall Street rumours, Welly Bry and Rosedale had
found the secret of performing this miracle.





Rosedale, in particular, was said to have doubled his fortune,
and there was talk of his buying the newly-finished house of one
of the victims of the crash, who, in the space of twelve short
months, had made the same number of millions, built a house in
Fifth Avenue, filled a picture-gallery with old masters,
entertained all New York in it, and been smuggled out of the
country between a trained nurse and a doctor, while his creditors
mounted guard over the old masters, and his guests explained to
each other that they had dined with him only because they wanted
to see the pictures. Mr.\ Rosedale meant to have a less meteoric
career. He knew he should have to go slowly, and the instincts of
his race fitted him to suffer rebuffs and put up with delays. But
he was prompt to perceive that the general dulness of the season
afforded him an unusual opportunity to shine, and he set about
with patient industry to form a background for his growing glory. 
Mrs.\ Fisher was of immense service to him at this period. She had
set off so many newcomers on the social stage that she was like
one of those pieces of stock scenery which tell the experienced
spectator exactly what is going to take place. But Mr.\ Rosedale
wanted, in the long run, a more individual environment. He was
sensitive to shades of difference which Miss Bart would never
have credited him with perceiving, because he had no
corresponding variations of manner; and it was becoming more and
more clear to him that Miss Bart herself possessed precisely the
complementary qualities needed to round off his social
personality.





Such details did not fall within the range of Mrs.\ Peniston's
vision. Like many minds of panoramic sweep, hers was apt to
overlook the \textit{minutiae} of the foreground, and she was much more
likely to know where Carry Fisher had found the Welly Brys'\ \textit{chef}
for them, than what was happening to her own niece. She was not,
however, without purveyors of information ready to supplement her
deficiencies. Grace Stepney's mind was like a kind of moral
fly-paper, to which the buzzing items of gossip were drawn by a
fatal attraction, and where they hung fast in the toils of an
inexorable memory. Lily would have been surprised to know how
many trivial facts concerning herself were lodged in Miss
Stepney's head. She was quite aware that she was of interest to
dingy people, but she assumed that there is only one form of
dinginess, and that admiration for brilliancy is the natural
expression of its inferior state. She knew that Gerty Farish
admired her blindly, and therefore supposed that she inspired the
same sentiments in Grace Stepney, whom she classified as a Gerty
Farish without the saving traits of youth and enthusiasm.





In reality, the two differed from each other as much as they
differed from the object of their mutual contemplation. Miss
Farish's heart was a fountain of tender illusions, Miss Stepney's
a precise register of facts as manifested in their relation to
herself. She had sensibilities which, to Lily, would have seemed
comic in a person with a freckled nose and red eyelids, who lived
in a boarding-house and admired Mrs.\ Peniston's drawing-room; but
poor Grace's limitations gave them a more concentrated inner
life, as poor soil starves certain plants into intenser
efflorescence. She had in truth no abstract propensity to malice: 
she did not dislike Lily because the latter was brilliant and
predominant, but because she thought that Lily disliked her. It
is less mortifying to believe one's self unpopular than
insignificant, and vanity prefers to assume that indifference is
a latent form of unfriendliness. Even such scant civilities as
Lily accorded to Mr.\ Rosedale would have made Miss Stepney her
friend for life; but how could she foresee that such a friend was
worth cultivating? How, moreover, can a young woman who has never
been ignored measure the pang which this injury inflicts? And,
lastly, how could Lily, accustomed to choose between a
pressure of engagements, guess that she had mortally offended
Miss Stepney by causing her to be excluded from one of Mrs.
Peniston's infrequent dinner-parties?





Mrs.\ Peniston disliked giving dinners, but she had a high sense
of family obligation, and on the Jack Stepneys'\ return from their
honeymoon she felt it incumbent upon her to light the
drawing-room lamps and extract her best silver from the Safe
Deposit vaults. Mrs.\ Peniston's rare entertainments were preceded
by days of heart-rending vacillation as to every detail of the
feast, from the seating of the guests to the pattern of the
table-cloth, and in the course of one of these preliminary
discussions she had imprudently suggested to her cousin Grace
that, as the dinner was a family affair, she might be included in
it. For a week the prospect had lighted up Miss Stepney's
colourless existence; then she had been given to understand that
it would be more convenient to have her another day. Miss Stepney
knew exactly what had happened. Lily, to whom family reunions
were occasions of unalloyed dulness, had persuaded her aunt that
a dinner of ``smart''\ people would be much more to the taste of the
young couple, and Mrs.\ Peniston, who leaned helplessly on her
niece in social matters, had been prevailed upon to pronounce
Grace's exile. After all, Grace could come any other day; why
should she mind being put off?





It was precisely because Miss Stepney could come any other
day---and because she knew her relations were in the secret of her
unoccupied evenings---that this incident loomed gigantically on
her horizon. She was aware that she had Lily to thank for it; and
dull resentment was turned to active animosity.





Mrs.\ Peniston, on whom she had looked in a day or two after the
dinner, laid down her crochet-work and turned abruptly from her
oblique survey of Fifth Avenue.





``Gus Trenor?---Lily and Gus Trenor?''\ she said, growing so suddenly
pale that her visitor was almost alarmed.





``Oh, cousin Julia .\ .\ .\ of course I don't mean .\ .\ .''





``I don't know what you \textit{do} mean,''\ said Mrs.\ Peniston, with a
frightened quiver in her small fretful voice. ``Such things were
never heard of in my day. And my own niece! I'm not sure I
understand you. Do people say he's in love with her?''





Mrs.\ Peniston's horror was genuine. Though she boasted an
unequalled familiarity with the secret chronicles of society, she
had the innocence of the school-girl who regards wickedness as a
part of ``history,''\ and to whom it never occurs that the scandals
she reads of in lesson-hours may be repeating themselves in the
next street. Mrs.\ Peniston had kept her imagination shrouded,
like the drawing-room furniture. She knew, of course, that
society was ``very much changed,''\ and that many women her mother
would have thought ``peculiar''\ were now in a position to be
critical about their visiting-lists; she had discussed the perils
of divorce with her rector, and had felt thankful at times that
Lily was still unmarried; but the idea that any scandal could
attach to a young girl's name, above all that it could be lightly
coupled with that of a married man, was so new to her that she
was as much aghast as if she had been accused of leaving her
carpets down all summer, or of violating any of the other
cardinal laws of housekeeping.





Miss Stepney, when her first fright had subsided, began to feel
the superiority that greater breadth of mind confers. It was
really pitiable to be as ignorant of the world as Mrs.\ Peniston! 
She smiled at the latter's question. ``People always say
unpleasant things---and certainly they're a great deal together. A
friend of mine met them the other afternoon in the Park-quite
late, after the lamps were lit. It's a pity Lily makes herself so
conspicuous.''





``\textit{Conspicuous}!''\ gasped Mrs.\ Peniston. She bent forward, lowering
her voice to mitigate the horror. ``What sort of things do they
say? That he means to get a divorce and marry her?''





Grace Stepney laughed outright. ``Dear me, no! He would hardly do
that. It---it's a flirtation---nothing more.''





``A flirtation? Between my niece and a married man? Do you mean to
tell me that, with Lily's looks and advantages, she could find no
better use for her time than to waste it on a fat stupid man
almost old enough to be her father?''\ This argument had such a
convincing ring that it gave Mrs.\ Peniston sufficient reassurance
to pick up her work, while she waited for Grace Stepney to rally
her scattered forces.





But Miss Stepney was on the spot in an instant. ``That's the worst
of it---people say she isn't wasting her time! Every one knows, as
you say, that Lily is too handsome and-and charming---to devote
herself to a man like Gus Trenor unless---''





``Unless?''\ echoed Mrs.\ Peniston. Her visitor drew breath
nervously. It was agreeable to shock Mrs.\ Peniston, but not to
shock her to the verge of anger. Miss Stepney was not
sufficiently familiar with the classic drama to have recalled in
advance how bearers of bad tidings are proverbially received, but
she now had a rapid vision of forfeited dinners and a reduced
wardrobe as the possible consequence of her disinterestedness. To
the honour of her sex, however, hatred of Lily prevailed over
more personal considerations. Mrs.\ Peniston had chosen the wrong
moment to boast of her niece's charms.





``Unless,''\ said Grace, leaning forward to speak with low-toned
emphasis, ``unless there are material advantages to be gained by
making herself agreeable to him.''





She felt that the moment was tremendous, and remembered suddenly
that Mrs.\ Peniston's black brocade, with the cut jet fringe,
would have been hers at the end of the season.





Mrs.\ Peniston put down her work again. Another aspect of the same
idea had presented itself to her, and she felt that it was
beneath her dignity to have her nerves racked by a dependent
relative who wore her old clothes.





``If you take pleasure in annoying me by mysterious insinuations,''
she said coldly, ``you might at least have chosen a more suitable
time than just as I am recovering from the strain of giving a
large dinner.''





The mention of the dinner dispelled Miss Stepney's last scruples. 
``I don't know why I should be accused of taking pleasure in
telling you about Lily. I was sure I shouldn't get any thanks for
it,''\ she returned with a flare of temper. ``But I have some family
feeling left, and as you are the only person who has any
authority over Lily, I thought you ought to know what is being
said of her.''





``Well,''\ said Mrs.\ Peniston, ``what I complain of is that you
haven't told me yet what \textit{is} being said.''





``I didn't suppose I should have to put it so plainly. People say
that Gus Trenor pays her bills.''





``Pays her bills---her bills?''\ Mrs.\ Peniston broke into a laugh. ``I
can't imagine where you can have picked up such rubbish. Lily has
her own income---and I provide for her very handsomely---''





``Oh, we all know that,''\ interposed Miss Stepney drily. ``But Lily
wears a great many smart gowns---''





``I like her to be well-dressed---it's only suitable!''





``Certainly; but then there are her gambling debts besides.''





Miss Stepney, in the beginning, had not meant to bring up this
point; but Mrs.\ Peniston had only her own incredulity to blame. 
She was like the stiff-necked unbelievers of Scripture, who must
be annihilated to be convinced.





``Gambling debts? Lily?''\ Mrs.\ Peniston's voice shook with anger
and bewilderment. She wondered whether Grace Stepney had gone out
of her mind. ``What do you mean by her gambling debts?''





``Simply that if one plays bridge for money in Lily's set one is
liable to lose a great deal---and I don't suppose Lily always
wins.''





``Who told you that my niece played cards for money?''





``Mercy, cousin Julia, don't look at me as if I were trying to
turn you against Lily! Everybody knows she is crazy about bridge. 
Mrs.\ Gryce told me herself that it was her gambling that
frightened Percy Gryce---it seems he was really taken with her at
first. But, of course, among Lily's friends it's quite the custom
for girls to play for money. In fact, people are inclined to
excuse her on that account----''





``To excuse her for what?''





``For being hard up---and accepting attentions from men like Gus
Trenor---and George Dorset----''





Mrs.\ Peniston gave another cry. ``George Dorset? Is there any one
else? I should like to know the worst, if you please.''





``Don't put it in that way, cousin Julia. Lately Lily has been a
good deal with the Dorsets, and he seems to admire her---but of
course that's only natural. And I'm sure there is no truth in the
horrid things people say; but she \textit{has} been spending a great deal
of money this winter. Evie Van Osburgh was at Celeste's
ordering her trousseau the other day---yes, the marriage takes
place next month---and she told me that Celeste showed her the
most exquisite things she was just sending home to Lily. And
people say that Judy Trenor has quarrelled with her on account of
Gus; but I'm sure I'm sorry I spoke, though I only meant it as a
kindness.''





Mrs.\ Peniston's genuine incredulity enabled her to dismiss Miss
Stepney with a disdain which boded ill for that lady's prospect
of succeeding to the black brocade; but minds impenetrable to
reason have generally some crack through which suspicion filters,
and her visitor's insinuations did not glide off as easily as she
had expected. Mrs.\ Peniston disliked scenes, and her
determination to avoid them had always led her to hold herself
aloof from the details of Lily's life. In her youth, girls had
not been supposed to require close supervision. They were
generally assumed to be taken up with the legitimate business of
courtship and marriage, and interference in such affairs on the
part of their natural guardians was considered as unwarrantable
as a spectator's suddenly joining in a game. There had of course
been ``fast''\ girls even in Mrs.\ Peniston's early experience; but
their fastness, at worst, was understood to be a mere excess of
animal spirits, against which there could be no graver charge
than that of being ``unladylike.'' The modern fastness appeared
synonymous with immorality, and the mere idea of immorality was
as offensive to Mrs.\ Peniston as a smell of cooking in the
drawing-room: it was one of the conceptions her mind refused to
admit.





She had no immediate intention of repeating to Lily what she had
heard, or even of trying to ascertain its truth by means of
discreet interrogation. To do so might be to provoke a scene; and
a scene, in the shaken state of Mrs.\ Peniston's nerves, with the
effects of her dinner not worn off, and her mind still tremulous
with new impressions, was a risk she deemed it her duty to avoid. 
But there remained in her thoughts a settled deposit of
resentment against her niece, all the denser because it was not
to be cleared by explanation or discussion. It was horrible of a
young girl to let herself be talked about; however unfounded the
charges against her, she must be to blame for their
having been made. Mrs.\ Peniston felt as if there had been a
contagious illness in the house, and she was doomed to sit
shivering among her contaminated furniture.





\chapter*{\raggedright Chapter 12}

\addcontentsline{toc}{chapter}{Chapter 12}

\markboth{House of Mirth}{Chapter 12}





Miss Bart had in fact been treading a devious way, and none of
her critics could have been more alive to the fact than herself;
but she had a fatalistic sense of being drawn from one wrong
turning to another, without ever perceiving the right road till
it was too late to take it.





Lily, who considered herself above narrow prejudices, had not
imagined that the fact of letting Gus Trenor make a little money
for her would ever disturb her self-complacency. And the fact in
itself still seemed harmless enough; only it was a fertile source
of harmful complications. As she exhausted the amusement of
spending the money these complications became more pressing, and
Lily, whose mind could be severely logical in tracing the causes
of her ill-luck to others, justified herself by the thought that
she owed all her troubles to the enmity of Bertha Dorset. This
enmity, however, had apparently expired in a renewal of
friendliness between the two women. Lily's visit to the Dorsets
had resulted, for both, in the discovery that they could be of
use to each other; and the civilized instinct finds a subtler
pleasure in making use of its antagonist than in confounding him. 
Mrs.\ Dorset was, in fact, engaged in a new sentimental
experiment, of which Mrs.\ Fisher's late property, Ned Silverton,
was the rosy victim; and at such moments, as Judy Trenor had once
remarked, she felt a peculiar need of distracting her husband's
attention. Dorset was as difficult to amuse as a savage; but even
his self-engrossment was not proof against Lily's arts, or rather
these were especially adapted to soothe an uneasy egoism. Her
experience with Percy Gryce stood her in good stead in
ministering to Dorset's humours, and if the incentive to please
was less urgent, the difficulties of her situation were teaching
her to make much of minor opportunities.





Intimacy with the Dorsets was not likely to lessen such
difficulties on the material side. Mrs.\ Dorset had none of Judy
Trenor's lavish impulses, and Dorset's admiration was not likely
to express itself in financial ``tips,''\ even had Lily cared to
renew her experiences in that line. What she required, for the
moment, of the Dorsets'\ friendship, was simply its social
sanction. She knew that people were beginning to talk of her; but
this fact did not alarm her as it had alarmed Mrs.\ Peniston. In
her set such gossip was not unusual, and a handsome girl who
flirted with a married man was merely assumed to be pressing to
the limit of her opportunities. It was Trenor himself who
frightened her. Their walk in the Park had not been a success. 
Trenor had married young, and since his marriage his intercourse
with women had not taken the form of the sentimental small-talk
which doubles upon itself like the paths in a maze. He was first
puzzled and then irritated to find himself always led back to the
same starting-point, and Lily felt that she was gradually losing
control of the situation. Trenor was in truth in an unmanageable
mood. In spite of his understanding with Rosedale he had been
somewhat heavily ``touched''\ by the fall in stocks; his household
expenses weighed on him, and he seemed to be meeting, on all
sides, a sullen opposition to his wishes, instead of the easy
good luck he had hitherto encountered.





Mrs.\ Trenor was still at Bellomont, keeping the town-house open,
and descending on it now and then for a taste of the world, but
preferring the recurrent excitement of week-end parties to the
restrictions of a dull season. Since the holidays she had not
urged Lily to return to Bellomont, and the first time they met in
town Lily fancied there was a shade of coldness in her manner. 
Was it merely the expression of her displeasure at Miss Bart's
neglect, or had disquieting rumours reached her? The latter
contingency seemed improbable, yet Lily was not without a sense
of uneasiness. If her roaming sympathies had struck root
anywhere, it was in her friendship with Judy Trenor. She believed
in the sincerity of her friend's affection, though it sometimes
showed itself in self-interested ways, and she shrank with
peculiar reluctance from any risk of estranging it. But, aside
from this, she was keenly conscious of the way in which such an
estrangement would react on herself. The fact that Gus Trenor was
Judy's husband was at times Lily's strongest reason for disliking
him, and for resenting the obligation under which he had placed
her. To set her doubts at rest, Miss Bart, soon after the New
Year, ``proposed''\ herself for a week-end at Bellomont. She had
learned in advance that the presence of a large party
would protect her from too great assiduity on Trenor's part, and
his wife's telegraphic ``come by all means''\ seemed to assure her
of her usual welcome.





Judy received her amicably. The cares of a large party always
prevailed over personal feelings, and Lily saw no change in her
hostess's manner. Nevertheless, she was soon aware that the
experiment of coming to Bellomont was destined not to be
successful. The party was made up of what Mrs.\ Trenor called
``poky people''---her generic name for persons who did not play
bridge---and, it being her habit to group all such obstructionists
in one class, she usually invited them together, regardless of
their other characteristics. The result was apt to be an
irreducible combination of persons having no other quality in
common than their abstinence from bridge, and the antagonisms
developed in a group lacking the one taste which might have
amalgamated them, were in this case aggravated by bad weather,
and by the ill-concealed boredom of their host and hostess. In
such emergencies, Judy would usually have turned to Lily to fuse
the discordant elements; and Miss Bart, assuming that such a
service was expected of her, threw herself into it with her
accustomed zeal. But at the outset she perceived a subtle
resistance to her efforts. If Mrs.\ Trenor's manner toward her was
unchanged, there was certainly a faint coldness in that of the
other ladies. An occasional caustic allusion to ``your friends the
Wellington Brys,''\ or to ``the little Jew who has bought the
Greiner house---some one told us you knew him, Miss Bart,''---showed
Lily that she was in disfavour with that portion of society
which, while contributing least to its amusement, has assumed the
right to decide what forms that amusement shall take. The
indication was a slight one, and a year ago Lily would have
smiled at it, trusting to the charm of her personality to dispel
any prejudice against her. But now she had grown more sensitive
to criticism and less confident in her power of disarming it. She
knew, moreover, that if the ladies at Bellomont permitted
themselves to criticize her friends openly, it was a proof that
they were not afraid of subjecting her to the same treatment
behind her back. The nervous dread lest anything in Trenor's
manner should seem to justify their disapproval made her seek
every pretext for avoiding him, and she left Bellomont conscious
of having failed in every purpose which had taken her
there.





In town she returned to preoccupations which, for the moment, had
the happy effect of banishing troublesome thoughts. The Welly
Brys, after much debate, and anxious counsel with their newly
acquired friends, had decided on the bold move of giving a
general entertainment. To attack society collectively, when one's
means of approach are limited to a few acquaintances, is like
advancing into a strange country with an insufficient number of
scouts; but such rash tactics have sometimes led to brilliant
victories, and the Brys had determined to put their fate to the
touch. Mrs.\ Fisher, to whom they had entrusted the conduct of the
affair, had decided that \textit{tableaux} VIVANTS and expensive music
were the two baits most likely to attract the desired prey, and
after prolonged negotiations, and the kind of wire-pulling in
which she was known to excel, she had induced a dozen fashionable
women to exhibit themselves in a series of pictures which, by a
farther miracle of persuasion, the distinguished portrait
painter, Paul Morpeth, had been prevailed upon to organize.





Lily was in her element on such occasions. Under Morpeth's
guidance her vivid plastic sense, hitherto nurtured on no higher
food than dress-making and upholstery, found eager expression in
the disposal of draperies, the study of attitudes, the shifting
of lights and shadows. Her dramatic instinct was roused by the
choice of subjects, and the gorgeous reproductions of historic
dress stirred an imagination which only visual impressions could
reach. But keenest of all was the exhilaration of displaying her
own beauty under a new aspect: of showing that her loveliness was
no mere fixed quality, but an element shaping all emotions to
fresh forms of grace.





Mrs.\ Fisher's measures had been well-taken, and society,
surprised in a dull moment, succumbed to the temptation of Mrs.
Bry's hospitality. The protesting minority were forgotten in the
throng which abjured and came; and the audience was almost as
brilliant as the show.





Lawrence Selden was among those who had yielded to the proffered
inducements. If he did not often act on the accepted social axiom
that a man may go where he pleases, it was because he had
long since learned that his pleasures were mainly to be found in
a small group of the like-minded. But he enjoyed spectacular
effects, and was not insensible to the part money plays in their
production: all he asked was that the very rich should live up to
their calling as stage-managers, and not spend their money in a
dull way. This the Brys could certainly not be charged with
doing. Their recently built house, whatever it might lack as a
frame for domesticity, was almost as well-designed for the
display of a festal assemblage as one of those airy
pleasure-halls which the Italian architects improvised to set off
the hospitality of princes. The air of improvisation was in fact
strikingly present: so recent, so rapidly-evoked was the whole
\textit{mise}-\textit{en}-\textit{scene} that one had to touch the marble columns to learn
they were not of cardboard, to seat one's self in one of the
damask-and-gold arm-chairs to be sure it was not painted against
the wall.





Selden, who had put one of these seats to the test, found
himself, from an angle of the ball-room, surveying the scene with
frank enjoyment. The company, in obedience to the decorative
instinct which calls for fine clothes in fine surroundings, had
dressed rather with an eye to Mrs.\ Bry's background than to
herself. The seated throng, filling the immense room without
undue crowding, presented a surface of rich tissues and jewelled
shoulders in harmony with the festooned and gilded walls, and the
flushed splendours of the Venetian ceiling. At the farther end of
the room a stage had been constructed behind a proscenium arch
curtained with folds of old damask; but in the pause before the
parting of the folds there was little thought of what they might
reveal, for every woman who had accepted Mrs.\ Bry's invitation
was engaged in trying to find out how many of her friends had
done the same.





Gerty Farish, seated next to Selden, was lost in that
indiscriminate and uncritical enjoyment so irritating to Miss
Bart's finer perceptions. It may be that Selden's nearness had
something to do with the quality of his cousin's pleasure; but
Miss Farish was so little accustomed to refer her enjoyment of
such scenes to her own share in them, that she was merely
conscious of a deeper sense of contentment.





``Wasn't it dear of Lily to get me an invitation? Of course
it would never have occurred to Carry Fisher to put me on
the list, and I should have been so sorry to miss seeing it
all---and especially Lily herself. Some one told me the ceiling was
by Veronese---you would know, of course, Lawrence. I suppose it's
very beautiful, but his women are so dreadfully fat. Goddesses? 
Well, I can only say that if they'd been mortals and had to wear
corsets, it would have been better for them. I think our women
are much handsomer. And this room is wonderfully becoming---every
one looks so well! Did you ever see such jewels? Do look at Mrs.
George Dorset's pearls---I suppose the smallest of them would pay
the rent of our Girls'\ Club for a year. Not that I ought to
complain about the club; every one has been so wonderfully kind. 
Did I tell you that Lily had given us three hundred dollars? 
Wasn't it splendid of her? And then she collected a lot of money
from her friends---Mrs.\ Bry gave us five hundred, and Mr.\ Rosedale
a thousand. I wish Lily were not so nice to Mr.\ Rosedale, but she
says it's no use being rude to him, because he doesn't see the
difference. She really can't bear to hurt people's feelings---it
makes me so angry when I hear her called cold and conceited! The
girls at the club don't call her that. Do you know she has been
there with me twice?---yes, Lily! And you should have seen their
eyes! One of them said it was as good as a day in the country
just to look at her. And she sat there, and laughed and talked
with them---not a bit as if she were being \textit{charitable}, you know,
but as if she liked it as much as they did. They've been asking
ever since when she's coming back; and she's promised me----oh!''





Miss Farish's confidences were cut short by the parting of the
curtain on the first \textit{tableau}---a group of nymphs dancing across
flower-strewn sward in the rhythmic postures of Botticelli's
Spring. \textit{Tableaux} VIVANTS depend for their effect not only on the
happy disposal of lights and the delusive-interposition of layers
of gauze, but on a corresponding adjustment of the mental vision. 
To unfurnished minds they remain, in spite of every enhancement
of art, only a superior kind of wax-works; but to the responsive
fancy they may give magic glimpses of the boundary world between
fact and imagination. Selden's mind was of this order: he could
yield to vision-making influences as completely as a child to the
spell of a fairy-tale. Mrs.\ Bry's \textit{tableaux} wanted none of
the qualities which go to the producing of such illusions, and
under Morpeth's organizing hand the pictures succeeded each other
with the rhythmic march of some splendid frieze, in which the
fugitive curves of living flesh and the wandering light of young
eyes have been subdued to plastic harmony without losing the
charm of life.





The scenes were taken from old pictures, and the participators
had been cleverly fitted with characters suited to their types. 
No one, for instance, could have made a more typical Goya than
Carry Fisher, with her short dark-skinned face, the exaggerated
glow of her eyes, the provocation of her frankly-painted smile. A
brilliant Miss Smedden from Brooklyn showed to perfection the
sumptuous curves of Titian's Daughter, lifting her gold salver
laden with grapes above the harmonizing gold of rippled hair and
rich brocade, and a young Mrs.\ Van Alstyne, who showed the
frailer Dutch type, with high blue-veined forehead and pale eyes
and lashes, made a characteristic Vandyck, in black satin,
against a curtained archway. Then there were Kauffmann nymphs
garlanding the altar of Love; a Veronese supper, all sheeny
textures, pearl-woven heads and marble architecture; and a
Watteau group of lute-playing comedians, lounging by a fountain
in a sunlit glade.





Each evanescent picture touched the vision-building faculty in
Selden, leading him so far down the vistas of fancy that even
Gerty Farish's running commentary---``Oh, how lovely Lulu Melson
looks!''\ or: ``That must be Kate Corby, to the right there, in
purple''---did not break the spell of the illusion. Indeed, so
skilfully had the personality of the actors been subdued to the
scenes they figured in that even the least imaginative of the
audience must have felt a thrill of contrast when the curtain
suddenly parted on a picture which was simply and undisguisedly
the portrait of Miss Bart.





Here there could be no mistaking the predominance of
personality---the unanimous ``Oh!''\ of the spectators was a tribute,
not to the brush-work of Reynolds's ``Mrs.\ Lloyd''\ but to the flesh
and blood loveliness of Lily Bart. She had shown her artistic
intelligence in selecting a type so like her own that she could
embody the person represented without ceasing to be
herself. It was as though she had stepped, not out of, but into,
Reynolds's canvas, banishing the phantom of his dead beauty by
the beams of her living grace. The impulse to show herself in a
splendid setting---she had thought for a moment of representing
Tiepolo's Cleopatra---had yielded to the truer instinct of
trusting to her unassisted beauty, and she had purposely chosen a
picture without distracting accessories of dress or surroundings. 
Her pale draperies, and the background of foliage against which
she stood, served only to relieve the long dryad-like curves that
swept upward from her poised foot to her lifted arm. The noble
buoyancy of her attitude, its suggestion of soaring grace,
revealed the touch of poetry in her beauty that Selden always
felt in her presence, yet lost the sense of when he was not with
her. Its expression was now so vivid that for the first time he
seemed to see before him the real Lily Bart, divested of the
trivialities of her little world, and catching for a moment a
note of that eternal harmony of which her beauty was a part.





``Deuced bold thing to show herself in that get-up; but, gad,
there isn't a break in the lines anywhere, and I suppose she
wanted us to know it!''





These words, uttered by that experienced connoisseur, Mr.\ Ned Van
Alstyne, whose scented white moustache had brushed Selden's
shoulder whenever the parting of the curtains presented any
exceptional opportunity for the study of the female outline,
affected their hearer in an unexpected way. It was not the first
time that Selden had heard Lily's beauty lightly remarked on, and
hitherto the tone of the comments had imperceptibly coloured his
view of her. But now it woke only a motion of indignant contempt. 
This was the world she lived in, these were the standards by
which she was fated to be measured! Does one go to Caliban for a
judgment on Miranda?





In the long moment before the curtain fell, he had time to feel
the whole tragedy of her life. It was as though her beauty, thus
detached from all that cheapened and vulgarized it, had held out
suppliant hands to him from the world in which he and she had
once met for a moment, and where he felt an overmastering longing
to be with her again.





He was roused by the pressure of ecstatic fingers. ``Wasn't she
too beautiful, Lawrence? Don't you like her best in that simple
dress? It makes her look like the real Lily---the Lily I know.''





He met Gerty Farish's brimming gaze. ``The Lily we know,''\ he
corrected; and his cousin, beaming at the implied understanding,
exclaimed joyfully: ``I'll tell her that! She always says you
dislike her.''





The performance over, Selden's first impulse was to seek Miss
Bart. During the interlude of music which succeeded the \textit{tableaux},
the actors had seated themselves here and there in the audience,
diversifying its conventional appearance by the varied
picturesqueness of their dress. Lily, however, was not among
them, and her absence served to protract the effect she had
produced on Selden: it would have broken the spell to see her too
soon in the surroundings from which accident had so happily
detached her. They had not met since the day of the Van Osburgh
wedding, and on his side the avoidance had been intentional. 
Tonight, however, he knew that, sooner or later, he should find
himself at her side; and though he let the dispersing crowd drift
him whither it would, without making an immediate effort to reach
her, his procrastination was not due to any lingering resistance,
but to the desire to luxuriate a moment in the sense of complete
surrender.





Lily had not an instant's doubt as to the meaning of the murmur
greeting her appearance. No other tableau had been received with
that precise note of approval: it had obviously been called forth
by herself, and not by the picture she impersonated. She had
feared at the last moment that she was risking too much in
dispensing with the advantages of a more sumptuous setting, and
the completeness of her triumph gave her an intoxicating sense of
recovered power. Not caring to diminish the impression she had
produced, she held herself aloof from the audience till the
movement of dispersal before supper, and thus had a second
opportunity of showing herself to advantage, as the throng poured
slowly into the empty drawing-room where she was standing.





She was soon the centre of a group which increased and renewed
itself as the circulation became general, and the individual
comments on her success were a delightful prolongation of
the collective applause. At such moments she lost something of
her natural fastidiousness, and cared less for the quality of the
admiration received than for its quantity. Differences of
personality were merged in a warm atmosphere of praise, in which
her beauty expanded like a flower in sunlight; and if Selden had
approached a moment or two sooner he would have seen her turning
on Ned Van Alstyne and George Dorset the look he had dreamed of
capturing for himself.





Fortune willed, however, that the hurried approach of Mrs.
Fisher, as whose \textit{Aide}-\textit{de}-\textit{camp} Van Alstyne was acting, should
break up the group before Selden reached the threshold of the
room. One or two of the men wandered off in search of their
partners for supper, and the others, noticing Selden's approach,
gave way to him in accordance with the tacit freemasonry of the
ball-room. Lily was therefore standing alone when he reached her;
and finding the expected look in her eye, he had the satisfaction
of supposing he had kindled it. The look did indeed deepen as it
rested on him, for even in that moment of self-intoxication Lily
felt the quicker beat of life that his nearness always produced. 
She read, too, in his answering gaze the delicious confirmation
of her triumph, and for the moment it seemed to her that it was
for him only she cared to be beautiful.





Selden had given her his arm without speaking. She took it in
silence, and they moved away, not toward the supper-room, but
against the tide which was setting thither. The faces about her
flowed by like the streaming images of sleep: she hardly noticed
where Selden was leading her, till they passed through a glass
doorway at the end of the long suite of rooms and stood suddenly
in the fragrant hush of a garden. Gravel grated beneath their
feet, and about them was the transparent dimness of a midsummer
night. Hanging lights made emerald caverns in the depths of
foliage, and whitened the spray of a fountain falling among
lilies. The magic place was deserted: there was no sound but the
splash of the water on the lily-pads, and a distant drift of
music that might have been blown across a sleeping lake.





Selden and Lily stood still, accepting the unreality of the scene
as a part of their own dream-like sensations. It would not have
surprised them to feel a summer breeze on their faces, or
to see the lights among the boughs reduplicated in the arch of a
starry sky. The strange solitude about them was no stranger than
the sweetness of being alone in it together. At length Lily
withdrew her hand, and moved away a step, so that her white-robed
slimness was outlined against the dusk of the branches. Selden
followed her, and still without speaking they seated themselves
on a bench beside the fountain.





Suddenly she raised her eyes with the beseeching earnestness of a
child. ``You never speak to me---you think hard things of me,''\ she
murmured.





``I think of you at any rate, God knows!''\ he said.





``Then why do we never see each other? Why can't we be friends? 
You promised once to help me,''\ she continued in the same tone, as
though the words were drawn from her unwillingly.





``The only way I can help you is by loving you,''\ Selden said in a
low voice.





She made no reply, but her face turned to him with the soft
motion of a flower. His own met it slowly, and their lips
touched. She drew back and rose from her seat. Selden rose too,
and they stood facing each other. Suddenly she caught his hand
and pressed it a moment against her cheek.





``Ah, love me, love me---but don't tell me so!''\ she sighed with her
eyes in his; and before he could speak she had turned and slipped
through the arch of boughs, disappearing in the brightness of the
room beyond.





Selden stood where she had left him. He knew too well the
transiency of exquisite moments to attempt to follow her; but
presently he reentered the house and made his way through the
deserted rooms to the door. A few sumptuously-cloaked ladies were
already gathered in the marble vestibule, and in the coat-room he
found Van Alstyne and Gus Trenor.





The former, at Selden's approach, paused in the careful selection
of a cigar from one of the silver boxes invitingly set out near
the door.





``Hallo, Selden, going too? You're an Epicurean like myself, I
see: you don't want to see all those goddesses gobbling terrapin. 
Gad, what a show of good-looking women; but not one of
'em could touch that little cousin of mine. Talk of
jewels---what's a woman want with jewels when she's got herself to
show? The trouble is that all these \textit{Fal}-BALS they wear cover up
their figures when they've got 'em. I never knew till tonight
what an outline Lily has.''





``It's not her fault if everybody don't know it now,''\ growled
Trenor, flushed with the struggle of getting into his fur-lined
coat. ``Damned bad taste, I call it---no, no cigar for me. You
can't tell what you're smoking in one of these new houses---likely
as not the \textit{chef} buys the cigars. Stay for supper? Not if I know
it! When people crowd their rooms so that you can't get near any
one you want to speak to, I'd as soon sup in the elevated at the
rush hour. My wife was dead right to stay away: she says life's
too short to spend it in breaking in new people.''





\chapter*{\raggedright Chapter 13}

\addcontentsline{toc}{chapter}{Chapter 13}

\markboth{House of Mirth}{Chapter 13}





Lily woke from happy dreams to find two notes at her bedside.





One was from Mrs.\ Trenor, who announced that she was coming to
town that afternoon for a flying visit, and hoped Miss Bart would
be able to dine with her. The other was from Selden. He wrote
briefly that an important case called him to Albany, whence he
would be unable to return till the evening, and asked Lily to let
him know at what hour on the following day she would see him.





Lily, leaning back among her pillows, gazed musingly at his
letter. The scene in the Brys'\ conservatory had been like a part
of her dreams; she had not expected to wake to such evidence of
its reality. Her first movement was one of annoyance: this
unforeseen act of Selden's added another complication to life. It
was so unlike him to yield to such an irrational impulse! Did he
really mean to ask her to marry him? She had once shown him the
impossibility of such a hope, and his subsequent behaviour seemed
to prove that he had accepted the situation with a reasonableness
somewhat mortifying to her vanity. It was all the more agreeable
to find that this reasonableness was maintained only at the cost
of not seeing her; but, though nothing in life was as sweet as
the sense of her power over him, she saw the danger of allowing
the episode of the previous night to have a sequel. Since she
could not marry him, it would be kinder to him, as well as easier
for herself, to write a line amicably evading his request to see
her: he was not the man to mistake such a hint, and when next
they met it would be on their usual friendly footing.





Lily sprang out of bed, and went straight to her desk. She wanted
to write at once, while she could trust to the strength of her
resolve. She was still languid from her brief sleep and the
exhilaration of the evening, and the sight of Selden's writing
brought back the culminating moment of her triumph: the moment
when she had read in his eyes that no philosophy was proof
against her power. It would be pleasant to have that sensation
again .\ .\ .\ no one else could give it to her in its fulness; and
she could not bear to mar her mood of luxurious retrospection
by an act of definite refusal. She took up her pen and wrote hastily: 
``\textit{Tomorrow} \textit{at} \textit{four};''\ murmuring to herself, as she slipped the sheet
into its envelope: ``I can easily put him off when tomorrow comes.''







Judy Trenor's summons was very welcome to Lily. It was the first
time she had received a direct communication from Bellomont since
the close of her last visit there, and she was still visited by
the dread of having incurred Judy's displeasure. But this
characteristic command seemed to reestablish their former
relations; and Lily smiled at the thought that her friend had
probably summoned her in order to hear about the Brys'
entertainment. Mrs.\ Trenor had absented herself from the feast,
perhaps for the reason so frankly enunciated by her husband,
perhaps because, as Mrs.\ Fisher somewhat differently put it, she
``couldn't bear new people when she hadn't discovered them
herself.'' At any rate, though she remained haughtily at
Bellomont, Lily suspected in her a devouring eagerness to hear of
what she had missed, and to learn exactly in what measure Mrs.
Wellington Bry had surpassed all previous competitors for social
recognition. Lily was quite ready to gratify this curiosity, but
it happened that she was dining out. She determined, however, to
see Mrs.\ Trenor for a few moments, and ringing for her maid she
despatched a telegram to say that she would be with her friend
that evening at ten.





She was dining with Mrs.\ Fisher, who had gathered at an informal
feast a few of the performers of the previous evening. There was
to be plantation music in the studio after dinner---for Mrs.\ Fisher,
despairing of the republic, had taken up modelling, and annexed to
her small crowded house a spacious apartment, which, whatever its
uses in her hours of plastic inspiration, served at other times
for the exercise of an indefatigable hospitality. Lily was
reluctant to leave, for the dinner was amusing, and she would have
liked to lounge over a cigarette and hear a few songs; but she
could not break her engagement with Judy, and shortly after ten
she asked her hostess to ring for a hansom, and drove up Fifth
Avenue to the Trenors'.





She waited long enough on the doorstep to wonder that
Judy's presence in town was not signalized by a greater
promptness in admitting her; and her surprise was increased when,
instead of the expected footman, pushing his shoulders into a
tardy coat, a shabby care-taking person in calico let her into
the shrouded hall. Trenor, however, appeared at once on the
threshold of the drawing-room, welcoming her with unusual
volubility while he relieved her of her cloak and drew her into
the room.





``Come along to the den; it's the only comfortable place in the
house. Doesn't this room look as if it was waiting for the body
to be brought down? Can't see why Judy keeps the house wrapped up
in this awful slippery white stuff---it's enough to give a fellow
pneumonia to walk through these rooms on a cold day. You look a
little pinched yourself, by the way: it's rather a sharp night
out. I noticed it walking up from the club. Come along, and I'll
give you a nip of brandy, and you can toast yourself over the
fire and try some of my new Egyptians---that little Turkish chap
at the Embassy put me on to a brand that I want you to try, and
if you like 'em I'll get out a lot for you: they don't have 'em
here yet, but I'll cable.''





He led her through the house to the large room at the back, where
Mrs.\ Trenor usually sat, and where, even in her absence, there
was an air of occupancy. Here, as usual, were flowers,
newspapers, a littered writing-table, and a general aspect of
lamp-lit familiarity, so that it was a surprise not to see Judy's
energetic figure start up from the arm-chair near the fire.





It was apparently Trenor himself who had been occupying the seat
in question, for it was overhung by a cloud of cigar smoke, and
near it stood one of those intricate folding tables which British
ingenuity has devised to facilitate the circulation of tobacco
and spirits. The sight of such appliances in a drawing-room was
not unusual in Lily's set, where smoking and drinking were
unrestricted by considerations of time and place, and her first
movement was to help herself to one of the cigarettes recommended
by Trenor, while she checked his loquacity by asking, with a
surprised glance: ``Where's Judy?''





Trenor, a little heated by his unusual flow of words, and
perhaps by prolonged propinquity with the decanters, was bending
over the latter to decipher their silver labels.





``Here, now, Lily, just a drop of cognac in a little fizzy
water---you do look pinched, you know: I swear the end of your
nose is red. I'll take another glass to keep you
company---Judy?---Why, you see, Judy's got a devil of a 
ache---quite knocked out with it, poor thing---she asked me to
explain---make it all right, you know---Do come up to the fire,
though; you look dead-beat, really. Now do let me make you
comfortable, there's a good girl.''





He had taken her hand, half-banteringly, and was drawing her
toward a low seat by the hearth; but she stopped and freed
herself quietly.





``Do you mean to say that Judy's not well enough to see me? 
Doesn't she want me to go upstairs?''





Trenor drained the glass he had filled for himself, and paused to
set it down before he answered.





``Why, no---the fact is, she's not up to seeing anybody. It came on
suddenly, you know, and she asked me to tell you how awfully
sorry she was---if she'd known where you were dining she'd have
sent you word.''





``She did know where I was dining; I mentioned it in my telegram. 
But it doesn't matter, of course. I suppose if she's so poorly
she won't go back to Bellomont in the morning, and I can come and
see her then.''





``Yes: exactly---that's capital. I'll tell her you'll pop in 
tomorrow morning. And now do sit down a minute, there's a dear, and
let's have a nice quiet jaw together. You won't take a drop, just
for sociability? Tell me what you think of that cigarette. Why,
don't you like it? What are you chucking it away for?''





``I am chucking it away because I must go, if you'll have the
goodness to call a cab for me,''\ Lily returned with a smile.





She did not like Trenor's unusual excitability, with its too
evident explanation, and the thought of being alone with him,
with her friend out of reach upstairs, at the other end of the
great empty house, did not conduce to a desire to prolong their
\textit{tete}-A-\textit{tete}.





But Trenor, with a promptness which did not escape her, had moved
between herself and the door.





``Why must you go, I should like to know? If Judy'd been here
you'd have sat gossiping till all hours---and you can't even give
me five minutes! It's always the same story. Last night I
couldn't get near you---I went to that damned vulgar party just to
see you, and there was everybody talking about you, and asking me
if I'd ever seen anything so stunning, and when I tried to come
up and say a word, you never took any notice, but just went on
laughing and joking with a lot of asses who only wanted to be
able to swagger about afterward, and look knowing when you were
mentioned.''





He paused, flushed by his diatribe, and fixing on her a look in
which resentment was the ingredient she least disliked. But she
had regained her presence of mind, and stood composedly in the
middle of the room, while her slight smile seemed to put an ever
increasing distance between herself and Trenor.





Across it she said: ``Don't be absurd, Gus. It's past eleven, and
I must really ask you to ring for a cab.''





He remained immovable, with the lowering forehead she had grown
to detest.





``And supposing I won't ring for one---what'll you do then?''





``I shall go upstairs to Judy if you force me to disturb her.''





Trenor drew a step nearer and laid his hand on her arm. ``Look
here, Lily: won't you give me five minutes of your own accord?''





``Not tonight, Gus: you----''





``Very good, then: I'll take 'em. And as many more as I want.'' He
had squared himself on the threshold, his hands thrust deep in
his pockets. He nodded toward the chair on the hearth.





``Go and sit down there, please: I've got a word to say to you.''





Lily's quick temper was getting the better of her fears. She drew
herself up and moved toward the door.





``If you have anything to say to me, you must say it another time. 
I shall go up to Judy unless you call a cab for me at once.''





He burst into a laugh. ``Go upstairs and welcome, my dear; but you
won't find Judy. She ain't there.''





Lily cast a startled look upon him. ``Do you mean that Judy is not
in the house---not in town?''\ she exclaimed.





``That's just what I do mean,''\ returned Trenor, his bluster
sinking to sullenness under her look.





``Nonsense---I don't believe you. I am going upstairs,''\ she said
impatiently.





He drew unexpectedly aside, letting her reach the threshold
unimpeded.





``Go up and welcome; but my wife is at Bellomont.''





But Lily had a flash of reassurance. ``If she hadn't come she
would have sent me word----''





``She did; she telephoned me this afternoon to let you know.''





``I received no message.''





``I didn't send any.''





The two measured each other for a moment, but Lily still saw her
opponent through a blur of scorn that made all other
considerations indistinct.





``I can't imagine your object in playing such a stupid trick on
me; but if you have fully gratified your peculiar sense of humour
I must again ask you to send for a cab.''





It was the wrong note, and she knew it as she spoke. To be stung
by irony it is not necessary to understand it, and the angry
streaks on Trenor's face might have been raised by an actual
lash.





``Look here, Lily, don't take that high and mighty tone with me.'' 
He had again moved toward the door, and in her instinctive
shrinking from him she let him regain command of the threshold. 
``I \textit{did} play a trick on you; I own up to it; but if you think I'm
ashamed you're mistaken. Lord knows I've been patient
enough---I've hung round and looked like an ass. And all the while
you were letting a lot of other fellows make up to you .\ .\ .
letting 'em make fun of me, I daresay .\ .\ . I'm not sharp, and
can't dress my friends up to look funny, as you do .\ .\ .\ but I
can tell when it's being done to me .\ .\ . I can tell fast enough
when I'm made a fool of .\ .\ .''





``Ah, I shouldn't have thought that!''\ flashed from Lily; but her
laugh dropped to silence under his look.





``No; you wouldn't have thought it; but you'll know better
now. That's what you're here for tonight. I've been waiting for a
quiet time to talk things over, and now I've got it I mean to
make you hear me out.''





His first rush of inarticulate resentment had been followed by a
steadiness and concentration of tone more disconcerting to Lily
than the excitement preceding it. For a moment her presence of
mind forsook her. She had more than once been in situations where
a quick sword-play of wit had been needful to cover her retreat;
but her frightened heart-throbs told her that here such skill
would not avail.





To gain time she repeated: ``I don't understand what you want.''





Trenor had pushed a chair between herself and the door. He threw
himself in it, and leaned back, looking up at her.





``I'll tell you what I want: I want to know just where you and I
stand. Hang it, the man who pays for the dinner is generally
allowed to have a seat at table.''





She flamed with anger and abasement, and the sickening need of
having to conciliate where she longed to humble.





``I don't know what you mean---but you must see, Gus, that I can't
stay here talking to you at this hour----''





``Gad, you go to men's houses fast enough in broad 
daylight---strikes me you're not always so deuced careful of
appearances.''





The brutality of the thrust gave her the sense of dizziness that
follows on a physical blow. Rosedale had spoken then---this was
the way men talked of her---She felt suddenly weak and
defenceless: there was a throb of self-pity in her throat. But
all the while another self was sharpening her to vigilance,
whispering the terrified warning that every word and gesture must
be measured.





``If you have brought me here to say insulting things----''\ she
began.





Trenor laughed. ``Don't talk stage-rot. I don't want to insult
you. But a man's got his feelings---and you've played with mine
too long. I didn't begin this business---kept out of the way, and
left the track clear for the other chaps, till you rummaged me
out and set to work to make an ass of me---and an easy job you had
of it, too. That's the trouble---it was too easy for
you---you got reckless---thought you could turn me inside out, and
chuck me in the gutter like an empty purse. But, by gad, that
ain't playing fair: that's dodging the rules of the game. Of
course I know now what you wanted---it wasn't my beautiful eyes
you were after---but I tell you what, Miss Lily, you've got to pay
up for making me think so----''





He rose, squaring his shoulders aggressively, and stepped toward
her with a reddening brow; but she held her footing, though every
nerve tore at her to retreat as he advanced.





``Pay up?''\ she faltered. ``Do you mean that I owe you money?''





He laughed again. ``Oh, I'm not asking for payment in kind. But
there's such a thing as fair play---and interest on one's
money---and hang me if I've had as much as a look from you----''





``Your money? What have I to do with your money? You advised me
how to invest mine .\ .\ .\ you must have seen I knew nothing of
business .\ .\ .\ you told me it was all right----''





``It \textit{was} all right---it is, Lily: you're welcome to all of it, and
ten times more. I'm only asking for a word of thanks from you.'' 
He was closer still, with a hand that grew formidable; and the
frightened self in her was dragging the other down.





``I \textit{have} thanked you; I've shown I was grateful. What more have
you done than any friend might do, or any one accept from a
friend?''





Trenor caught her up with a sneer. ``I don't doubt you've accepted
as much before---and chucked the other chaps as you'd like to
chuck me. I don't care how you settled your score with them---if
you fooled 'em I'm that much to the good. Don't stare at me like
that---I know I'm not talking the way a man is supposed to talk to
a girl---but, hang it, if you don't like it you can stop me quick
enough---you know I'm mad about you---damn the money, there's
plenty more of it---if \textit{that} bothers you .\ .\ . I was a brute,
Lily---Lily!---just look at me----''





Over and over her the sea of humiliation broke---wave crashing on
wave so close that the moral shame was one with the physical
dread. It seemed to her that self-esteem would have made
her invulnerable---that it was her own dishonour which put a
fearful solitude about her.





His touch was a shock to her drowning consciousness. She drew
back from him with a desperate assumption of scorn.





``I've told you I don't understand---but if I owe you money you
shall be paid----''





Trenor's face darkened to rage: her recoil of abhorrence had
called out the primitive man.





``Ah---you'll borrow from Selden or Rosedale---and take your chances
of fooling them as you've fooled me! Unless---unless you've
settled your other scores already---and I'm the only one left out
in the cold!''





She stood silent, frozen to her place. The words---the words were
worse than the touch! Her heart was beating all over her body---in
her throat, her limbs, her helpless useless hands. Her eyes
travelled despairingly about the room---they lit on the bell, and
she remembered that help was in call. Yes, but scandal with it---a
hideous mustering of tongues. No, she must fight her way out
alone. It was enough that the servants knew her to be in the
house with Trenor---there must be nothing to excite conjecture in
her way of leaving it.





She raised her head, and achieved a last clear look at him.





``I am here alone with you,''\ she said. ``What more have you to
say?''





To her surprise, Trenor answered the look with a speechless
stare. With his last gust of words the flame had died out,
leaving him chill and humbled. It was as though a cold air had
dispersed the fumes of his libations, and the situation loomed
before him black and naked as the ruins of a fire. Old habits,
old restraints, the hand of inherited order, plucked back the
bewildered mind which passion had jolted from its ruts. Trenor's
eye had the haggard look of the sleep-walker waked on a deathly
ledge.





``Go home! Go away from here''----he stammered, and turning his back
on her walked toward the hearth.





The sharp release from her fears restored Lily to immediate
lucidity. The collapse of Trenor's will left her in control, and
she heard herself, in a voice that was her own yet outside
herself, bidding him ring for the servant, bidding him give the
order for a hansom, directing him to put her in it when
it came. Whence the strength came to her she knew not; but an
insistent voice warned her that she must leave the house openly,
and nerved her, in the hall before the hovering care taker, to
exchange light words with Trenor, and charge him with the usual
messages for Judy, while all the while she shook with inward
loathing. On the doorstep, with the street before her, she felt a
mad throb of liberation, intoxicating as the prisoner's first
draught of free air; but the clearness of brain continued, and
she noted the mute aspect of Fifth Avenue, guessed at the
lateness of the hour, and even observed a man's figure---was there
something half-familiar in its outline?---which, as she entered
the hansom, turned from the opposite corner and vanished in the
obscurity of the side street.





But with the turn of the wheels reaction came, and shuddering
darkness closed on her. ``I can't think---I can't think,''\ she
moaned, and leaned her head against the rattling side of the cab. 
She seemed a stranger to herself, or rather there were two selves
in her, the one she had always known, and a new abhorrent being
to which it found itself chained. She had once picked up, in a
house where she was staying, a translation of the \textit{Eumenides}, and
her imagination had been seized by the high terror of the scene
where Orestes, in the cave of the oracle, finds his implacable
huntresses asleep, and snatches an hour's repose. Yes, the Furies
might sometimes sleep, but they were there, always there in the
dark corners, and now they were awake and the iron clang of their
wings was in her brain .\ .\ . She opened her eyes and saw the
streets passing---the familiar alien streets. All she looked on
was the same and yet changed. There was a great gulf fixed
between today and yesterday. Everything in the past seemed
simple, natural, full of daylight---and she was alone in a place
of darkness and pollution.---Alone! It was the loneliness that
frightened her. Her eyes fell on an illuminated clock at a street
corner, and she saw that the hands marked the half hour after
eleven. Only half-past eleven---there were hours and hours left of
the night! And she must spend them alone, shuddering sleepless on
her bed. Her soft nature recoiled from this ordeal, which had
none of the stimulus of conflict to goad her through it. Oh, the
slow cold drip of the minutes on her head! She had a
vision of herself lying on the black walnut bed---and the darkness
would frighten her, and if she left the light burning the dreary
details of the room would brand themselves forever on her brain. 
She had always hated her room at Mrs.\ Peniston's---its ugliness,
its impersonality, the fact that nothing in it was really hers. 
To a torn heart uncomforted by human nearness a room may open
almost human arms, and the being to whom no four walls mean more
than any others, is, at such hours, expatriate everywhere.





Lily had no heart to lean on. Her relation with her aunt was as
superficial as that of chance lodgers who pass on the stairs. But
even had the two been in closer contact, it was impossible to
think of Mrs.\ Peniston's mind as offering shelter or
comprehension to such misery as Lily's. As the pain that can be
told is but half a pain, so the pity that questions has little
healing in its touch. What Lily craved was the darkness made by
enfolding arms, the silence which is not solitude, but compassion
holding its breath.





She started up and looked forth on the passing streets. 
Gerty!---they were nearing Gerty's corner. If only she could reach
there before this labouring anguish burst from her breast to her
lips---if only she could feel the hold of Gerty's arms while she
shook in the ague-fit of fear that was coming upon her! She
pushed up the door in the roof and called the address to the
driver. It was not so late---Gerty might still be waking. And even
if she were not, the sound of the bell would penetrate every
recess of her tiny apartment, and rouse her to answer her
friend's call.





\chapter*{\raggedright Chapter 14}

\addcontentsline{toc}{chapter}{Chapter 14}

\markboth{House of Mirth}{Chapter 14}





Gerty Farish, the morning after the Wellington Brys'
entertainment, woke from dreams as happy as Lily's. If they were
less vivid in hue, more subdued to the half-tints of her
personality and her experience, they were for that very reason
better suited to her mental vision. Such flashes of joy as Lily
moved in would have blinded Miss Farish, who was accustomed, in
the way of happiness, to such scant light as shone through the
cracks of other people's lives.





Now she was the centre of a little illumination of her own: a
mild but unmistakable beam, compounded of Lawrence Selden's
growing kindness to herself and the discovery that he extended
his liking to Lily Bart. If these two factors seem incompatible
to the student of feminine psychology, it must be remembered that
Gerty had always been a parasite in the moral order, living on
the crumbs of other tables, and content to look through the
window at the banquet spread for her friends. Now that she was
enjoying a little private feast of her own, it would have seemed
incredibly selfish not to lay a plate for a friend; and there was
no one with whom she would rather have shared her enjoyment than
Miss Bart.





As to the nature of Selden's growing kindness, Gerty would no
more have dared to define it than she would have tried to learn a
butterfly's colours by knocking the dust from its wings. To seize
on the wonder would be to brush off its bloom, and perhaps see it
fade and stiffen in her hand: better the sense of beauty
palpitating out of reach, while she held her breath and watched
where it would alight. Yet Selden's manner at the Brys'\ had
brought the flutter of wings so close that they seemed to be
beating in her own heart. She had never seen him so alert, so
responsive, so attentive to what she had to say. His habitual
manner had an absent-minded kindliness which she accepted, and
was grateful for, as the liveliest sentiment her presence was
likely to inspire; but she was quick to feel in him a change
implying that for once she could give pleasure as well as receive
it.





And it was so delightful that this higher degree of sympathy
should be reached through their interest in Lily Bart!





Gerty's affection for her friend---a sentiment that had learned to
keep itself alive on the scantiest diet---had grown to active
adoration since Lily's restless curiosity had drawn her into the
circle of Miss Farish's work. Lily's taste of beneficence had
wakened in her a momentary appetite for well-doing. Her visit to
the Girls'\ Club had first brought her in contact with the
dramatic contrasts of life. She had always accepted with
philosophic calm the fact that such existences as hers were
pedestalled on foundations of obscure humanity. The dreary limbo
of dinginess lay all around and beneath that little illuminated
circle in which life reached its finest efflorescence, as the mud
and sleet of a winter night enclose a hot-house filled with
tropical flowers. All this was in the natural order of things,
and the orchid basking in its artificially created atmosphere
could round the delicate curves of its petals undisturbed by the
ice on the panes.





But it is one thing to live comfortably with the abstract
conception of poverty, another to be brought in contact with its
human embodiments. Lily had never conceived of these victims of
fate otherwise than in the mass. That the mass was composed of
individual lives, innumerable separate centres of sensation, with
her own eager reachings for pleasure, her own fierce revulsions
from pain---that some of these bundles of feeling were clothed in
shapes not so unlike her own, with eyes meant to look on
gladness, and young lips shaped for love---this discovery gave
Lily one of those sudden shocks of pity that sometimes
decentralize a life. Lily's nature was incapable of such renewal: 
she could feel other demands only through her own, and no pain
was long vivid which did not press on an answering nerve. But for
the moment she was drawn out of herself by the interest of her
direct relation with a world so unlike her own. She had
supplemented her first gift by personal assistance to one or two
of Miss Farish's most appealing subjects, and the admiration and
interest her presence excited among the tired workers at the club
ministered in a new form to her insatiable desire to please.





Gerty Farish was not a close enough reader of character to
disentangle the mixed threads of which Lily's philanthropy was
woven. She supposed her beautiful friend to be actuated by the
same motive as herself---that sharpening of the moral
vision which makes all human suffering so near and insistent that
the other aspects of life fade into remoteness. Gerty lived by
such simple formulas that she did not hesitate to class her
friend's state with the emotional ``change of heart''\ to which her
dealings with the poor had accustomed her; and she rejoiced in
the thought that she had been the humble instrument of this
renewal. Now she had an answer to all criticisms of Lily's
conduct: as she had said, she knew ``the real Lily,''\ and the
discovery that Selden shared her knowledge raised her placid
acceptance of life to a dazzled sense of its possibilities---a
sense farther enlarged, in the course of the afternoon, by the
receipt of a telegram from Selden asking if he might dine with
her that evening.





While Gerty was lost in the happy bustle which this announcement
produced in her small household, Selden was at one with her in
thinking with intensity of Lily Bart. The case which had called
him to Albany was not complicated enough to absorb all his
attention, and he had the professional faculty of keeping a part
of his mind free when its services were not needed. This
part---which at the moment seemed dangerously like the whole---was
filled to the brim with the sensations of the previous evening. 
Selden understood the symptoms: he recognized the fact that he
was paying up, as there had always been a chance of his having to
pay up, for the voluntary exclusions of his past. He had meant to
keep free from permanent ties, not from any poverty of feeling,
but because, in a different way, he was, as much as Lily, the
victim of his environment. There had been a germ of truth in his
declaration to Gerty Farish that he had never wanted to marry a
``nice''\ girl: the adjective connoting, in his cousin's vocabulary,
certain utilitarian qualities which are apt to preclude the
luxury of charm. Now it had been Selden's fate to have a charming
mother: her graceful portrait, all smiles and Cashmere, still
emitted a faded scent of the undefinable quality. His father was
the kind of man who delights in a charming woman: who quotes her,
stimulates her, and keeps her perennially charming. Neither one
of the couple cared for money, but their disdain of it took the
form of always spending a little more than was prudent. If their
house was shabby, it was exquisitely kept; if there were good
books on the shelves there were also good dishes on the
table. Selden senior had an eye for a picture, his wife an
understanding of old lace; and both were so conscious of
restraint and discrimination in buying that they never quite knew
how it was that the bills mounted up.





Though many of Selden's friends would have called his parents
poor, he had grown up in an atmosphere where restricted means
were felt only as a check on aimless profusion: where the few
possessions were so good that their rarity gave them a merited
relief, and abstinence was combined with elegance in a way
exemplified by Mrs.\ Selden's knack of wearing her old velvet as
if it were new. A man has the advantage of being delivered early
from the home point of view, and before Selden left college he
had learned that there are as many different ways of going
without money as of spending it. Unfortunately, he found no way
as agreeable as that practised at home; and his views of
womankind in especial were tinged by the remembrance of the one
woman who had given him his sense of ``values.'' It was from her
that he inherited his detachment from the sumptuary side of life: 
the stoic's carelessness of material things, combined with the
Epicurean's pleasure in them. Life shorn of either feeling
appeared to him a diminished thing; and nowhere was the blending
of the two ingredients so essential as in the character of a
pretty woman.





It had always seemed to Selden that experience offered a great
deal besides the sentimental adventure, yet he could vividly
conceive of a love which should broaden and deepen till it became
the central fact of life. What he could not accept, in his own
case, was the makeshift alternative of a relation that should be
less than this: that should leave some portions of his nature
unsatisfied, while it put an undue strain on others. He would
not, in other words, yield to the growth of an affection which
might appeal to pity yet leave the understanding untouched: 
sympathy should no more delude him than a trick of the eyes, the
grace of helplessness than a curve of the cheek.





But now---that little \textit{but} passed like a sponge over all his vows. 
His reasoned-out resistances seemed for the moment so much less
important than the question as to when Lily would receive his
note! He yielded himself to the charm of trivial
preoccupations, wondering at what hour her reply would be sent,
with what words it would begin. As to its import he had no
doubt---he was as sure of her surrender as of his own. And so he
had leisure to muse on all its exquisite details, as a hard
worker, on a holiday morning, might lie still and watch the beam
of light travel gradually across his room. But if the new light
dazzled, it did not blind him. He could still discern the outline
of facts, though his own relation to them had changed. He was no
less conscious than before of what was said of Lily Bart, but he
could separate the woman he knew from the vulgar estimate of her. 
His mind turned to Gerty Farish's words, and the wisdom of the
world seemed a groping thing beside the insight of innocence. 
\textit{Blessed} \textit{are} \textit{the} \textit{pure} \textit{in} \textit{heart}, \textit{for} \textit{they} \textit{shall} \textit{see} \textit{god}---even the
hidden god in their neighbour's breast! Selden was in the state
of impassioned self-absorption that the first surrender to love
produces. His craving was for the companionship of one whose
point of view should justify his own, who should confirm, by
deliberate observation, the truth to which his intuitions had
leaped. He could not wait for the midday recess, but seized a
moment's leisure in court to scribble his telegram to Gerty
Farish.





Reaching town, he was driven direct to his club, where he hoped a
note from Miss Bart might await him. But his box contained only a
line of rapturous assent from Gerty, and he was turning away
disappointed when he was hailed by a voice from the smoking room.





``Hallo, Lawrence! Dining here? Take a bite with me---I've ordered a
canvas-back.''





He discovered Trenor, in his day clothes, sitting, with a tall
glass at his elbow, behind the folds of a sporting journal.





Selden thanked him, but pleaded an engagement.





``Hang it, I believe every man in town has an engagement tonight. 
I shall have the club to myself. You know how I'm living this
winter, rattling round in that empty house. My wife meant to come
to town today, but she's put it off again, and how is a fellow to
dine alone in a room with the looking-glasses covered, and
nothing but a bottle of Harvey sauce on the side-board? I say,
Lawrence, chuck your engagement and take pity on me---it gives me
the blue devils to dine alone, and there's nobody but
that canting ass Wetherall in the club.''





``Sorry, Gus---I can't do it.''





As Selden turned away, he noticed the dark flush on Trenor's
face, the unpleasant moisture of his intensely white forehead,
the way his jewelled rings were wedged in the creases of his fat
red fingers. Certainly the beast was predominating---the beast at
the bottom of the glass. And he had heard this man's name coupled
with Lily's! Bah---the thought sickened him; all the way back to
his rooms he was haunted by the sight of Trenor's fat creased
hands----





On his table lay the note: Lily had sent it to his rooms. He knew
what was in it before he broke the seal---a grey seal with \textit{beyond}!
beneath a flying ship. Ah, he would take her beyond---beyond the
ugliness, the pettiness, the attrition and corrosion of the
soul----







Gerty's little sitting-room sparkled with welcome when Selden
entered it. Its modest ``effects,''\ compact of enamel paint and
ingenuity, spoke to him in the language just then sweetest to his
ear. It is surprising how little narrow walls and a low ceiling
matter, when the roof of the soul has suddenly been raised. Gerty
sparkled too; or at least shone with a tempered radiance. He had
never before noticed that she had ``points''---really, some good
fellow might do worse .\ .\ . Over the little dinner (and here,
again, the effects were wonderful)\ he told her she ought to
marry---he was in a mood to pair off the whole world. She had made
the caramel custard with her own hands? It was sinful to keep
such gifts to herself. He reflected with a throb of pride that
Lily could trim her own hats---she had told him so the day of
their walk at Bellomont.





He did not speak of Lily till after dinner. During the little
repast he kept the talk on his hostess, who, fluttered at being
the centre of observation, shone as rosy as the candle-shades she
had manufactured for the occasion. Selden evinced an
extraordinary interest in her household arrangements: 
complimented her on the ingenuity with which she had utilized
every inch of her small quarters, asked how her servant managed
about afternoons out, learned that one may improvise
delicious dinners in a chafing-dish, and uttered thoughtful
generalizations on the burden of a large establishment.





When they were in the sitting-room again, where they fitted as
snugly as bits in a puzzle, and she had brewed the coffee, and
poured it into her grandmother's egg-shell cups, his eye, as he
leaned back, basking in the warm fragrance, lighted on a recent
photograph of Miss Bart, and the desired transition was effected
without an effort. The photograph was well enough---but to catch
her as she had looked last night! Gerty agreed with him---never
had she been so radiant. But could photography capture that
light? There had been a new look in her face---something
different; yes, Selden agreed there had been something different. 
The coffee was so exquisite that he asked for a second cup: such
a contrast to the watery stuff at the club! Ah, your poor
bachelor with his impersonal club fare, alternating with the
equally impersonal \textit{cuisine} of the dinner-party! A man who lived
in lodgings missed the best part of life---he pictured the
flavourless solitude of Trenor's repast, and felt a moment's
compassion for the man .\ .\ . But to return to Lily---and again and
again he returned, questioning, conjecturing, leading Gerty on,
draining her inmost thoughts of their stored tenderness for her
friend.





At first she poured herself out unstintingly, happy in this
perfect communion of their sympathies. His understanding of Lily
helped to confirm her own belief in her friend. They dwelt
together on the fact that Lily had had no chance. Gerty instanced
her generous impulses---her restlessness and discontent. The fact
that her life had never satisfied her proved that she was made
for better things. She might have married more than once---the
conventional rich marriage which she had been taught to consider
the sole end of existence---but when the opportunity came she had
always shrunk from it. Percy Gryce, for instance, had been in
love with her---every one at Bellomont had supposed them to be
engaged, and her dismissal of him was thought inexplicable. This
view of the Gryce incident chimed too well with Selden's mood not
to be instantly adopted by him, with a flash of retrospective
contempt for what had once seemed the obvious solution. If
rejection there had been---and he wondered now that he had
ever doubted it!---then he held the key to the secret, and the
hillsides of Bellomont were lit up, not with sunset, but with
dawn. It was he who had wavered and disowned the face of
opportunity---and the joy now warming his breast might have been a
familiar inmate if he had captured it in its first flight.





It was at this point, perhaps, that a joy just trying its wings
in Gerty's heart dropped to earth and lay still. She sat facing
Selden, repeating mechanically: ``No, she has never been
understood----''\ and all the while she herself seemed to be sitting
in the centre of a great glare of comprehension. The little
confidential room, where a moment ago their thoughts had touched
elbows like their chairs, grew to unfriendly vastness, separating
her from Selden by all the length of her new vision of the
future---and that future stretched out interminably, with her
lonely figure toiling down it, a mere speck on the solitude.





``She is herself with a few people only; and you are one of them,''
she heard Selden saying. And again: ``Be good to her, Gerty, won't
you?''\ and: ``She has it in her to become whatever she is believed
to be---you'll help her by believing the best of her?''





The words beat on Gerty's brain like the sound of a language
which has seemed familiar at a distance, but on approaching is
found to be unintelligible. He had come to talk to her of
Lily---that was all! There had been a third at the feast she had
spread for him, and that third had taken her own place. She tried
to follow what he was saying, to cling to her own part in the
talk---but it was all as meaningless as the boom of waves in a
drowning head, and she felt, as the drowning may feel, that to
sink would be nothing beside the pain of struggling to keep up.





Selden rose, and she drew a deep breath, feeling that soon she
could yield to the blessed waves.





``Mrs.\ Fisher's? You say she was dining there? There's music
afterward; I believe I had a card from her.'' He glanced at the
foolish pink-faced clock that was drumming out this hideous
hour. ``A quarter past ten? I might look in there now; the Fisher
evenings are amusing. I haven't kept you up too late, Gerty? You
look tired---I've rambled on and bored you.'' And in the
unwonted overflow of his feelings, he left a cousinly kiss upon
her cheek.







At Mrs.\ Fisher's, through the cigar-smoke of the studio, a dozen
voices greeted Selden. A song was pending as he entered, and he
dropped into a seat near his hostess, his eyes roaming in search
of Miss Bart. But she was not there, and the discovery gave him a
pang out of all proportion to its seriousness; since the note in
his breast-pocket assured him that at four the next day they
would meet. To his impatience it seemed immeasurably long to
wait, and half-ashamed of the impulse, he leaned to Mrs.\ Fisher
to ask, as the music ceased, if Miss Bart had not dined with her.





``Lily? She's just gone. She had to run off, I forget where. 
Wasn't she wonderful last night?''





``Who's that? Lily?''\ asked Jack Stepney, from the depths of a
neighbouring arm-chair. ``Really, you know, I'm no prude, but when
it comes to a girl standing there as if she was up at auction---I
thought seriously of speaking to cousin Julia.''





``You didn't know Jack had become our social censor?''\ Mrs.\ Fisher
said to Selden with a laugh; and Stepney spluttered, amid the
general derision: ``But she's a cousin, hang it, and when a man's
married---\textit{town} \textit{talk} was full of her this morning.''





``Yes: lively reading that was,''\ said Mr.\ Ned Van Alstyne,
stroking his moustache to hide the smile behind it. ``Buy the
dirty sheet? No, of course not; some fellow showed it to me---but
I'd heard the stories before. When a girl's as good-looking as
that she'd better marry; then no questions are asked. In our
imperfectly organized society there is no provision as yet for
the young woman who claims the privileges of marriage without
assuming its obligations.''





``Well, I understand Lily is about to assume them in the shape of
Mr.\ Rosedale,''\ Mrs.\ Fisher said with a laugh.





``Rosedale---good heavens!''\ exclaimed Van Alstyne, dropping his
eye-glass. ``Stepney, that's your fault for foisting the brute on
us.''





``Oh, confound it, you know, we don't \textit{marry} Rosedale in our
family,''\ Stepney languidly protested; but his wife, who
sat in oppressive bridal finery at the other side of the room,
quelled him with the judicial reflection: ``In Lily's
circumstances it's a mistake to have too high a standard.''





``I hear even Rosedale has been scared by the talk lately,''\ Mrs.
Fisher rejoined; ``but the sight of her last night sent him off
his head. What do you think he said to me after her \textit{tableau}? 
'My God, Mrs.\ Fisher, if I could get Paul Morpeth to paint her
like that, the picture'd appreciate a hundred per cent in ten
years.'''





``By Jove,---but isn't she about somewhere?''\ exclaimed Van Alstyne,
restoring his glass with an uneasy glance.





``No; she ran off while you were all mixing the punch down stairs. 
Where was she going, by the way? What's on tonight? I hadn't
heard of anything.''





``Oh, not a party, I think,''\ said an inexperienced young Farish
who had arrived late. ``I put her in her cab as I was coming in,
and she gave the driver the Trenors'\ address.''





``The Trenors'?''\ exclaimed Mrs.\ Jack Stepney. ``Why, the house is
closed---Judy telephoned me from Bellomont this evening.''





``Did she? That's queer. I'm sure I'm not mistaken. Well, come
now, Trenor's there, anyhow---I---oh, well---the fact is, I've no
head for numbers,''\ he broke off, admonished by the nudge of an
adjoining foot, and the smile that circled the room.





In its unpleasant light Selden had risen and was shaking hands
with his hostess. The air of the place stifled him, and he
wondered why he had stayed in it so long.





On the doorstep he stood still, remembering a phrase of Lily's: 
``It seems to me you spend a good deal of time in the element you
disapprove of.''





Well---what had brought him there but the quest of her? It was her
element, not his. But he would lift her out of it, take her
beyond! That \textit{beyond}!\ on her letter was like a cry for rescue. He
knew that Perseus's task is not done when he has loosed
Andromeda's chains, for her limbs are numb with bondage, and she
cannot rise and walk, but clings to him with dragging arms as he
beats back to land with his burden. Well, he had strength for
both---it was her weakness which had put the strength in him. It
was not, alas, a clean rush of waves they had to win
through, but a clogging morass of old associations and habits,
and for the moment its vapours were in his throat. But he would
see clearer, breathe freer in her presence: she was at once the
dead weight at his breast and the spar which should float them to
safety. He smiled at the whirl of metaphor with which he was
trying to build up a defence against the influences of the last
hour. It was pitiable that he, who knew the mixed motives on
which social judgments depend, should still feel himself so
swayed by them. How could he lift Lily to a freer vision of life,
if his own view of her was to be coloured by any mind in which he
saw her reflected?





The moral oppression had produced a physical craving for air, and
he strode on, opening his lungs to the reverberating coldness of
the night. At the corner of Fifth Avenue Van Alstyne hailed him
with an offer of company.





``Walking? A good thing to blow the smoke out of one's head. Now
that women have taken to tobacco we live in a bath of nicotine. 
It would be a curious thing to study the effect of cigarettes on
the relation of the sexes. Smoke is almost as great a solvent as
divorce: both tend to obscure the moral issue.''





Nothing could have been less consonant with Selden's mood than
Van Alstyne's after-dinner aphorisms, but as long as the latter
confined himself to generalities his listener's nerves were in
control. Happily Van Alstyne prided himself on his summing up of
social aspects, and with Selden for audience was eager to show
the sureness of his touch. Mrs.\ Fisher lived in an East side
street near the Park, and as the two men walked down Fifth Avenue
the new architectural developments of that versatile thoroughfare
invited Van Alstyne's comment.





``That Greiner house, now---a typical rung in the social ladder! 
The man who built it came from a \textit{milieu} where all the dishes are
put on the table at once. His fa\c{c}ade is a complete architectural
meal; if he had omitted a style his friends might have thought
the money had given out. Not a bad purchase for Rosedale, though: 
attracts attention, and awes the Western sight-seer. By and bye
he'll get out of that phase, and want something that the crowd
will pass and the few pause before. Especially if he marries my
clever cousin----''





Selden dashed in with the query: ``And the Wellington Brys'? 
Rather clever of its kind, don't you think?''





They were just beneath the wide white fa\c{c}ade, with its rich
restraint of line, which suggested the clever corseting of a
redundant figure.





``That's the next stage: the desire to imply that one has been to
Europe, and has a standard. I'm sure Mrs.\ Bry thinks her house a
copy of the \textit{Trianon}; in America every marble house with gilt
furniture is thought to be a copy of the \textit{Trianon}. What a clever
chap that architect is, though---how he takes his client's
measure! He has put the whole of Mrs.\ Bry in his use of the
composite order. Now for the Trenors, you remember, he chose the
Corinthian: exuberant, but based on the best precedent. The
Trenor house is one of his best things---doesn't look like a
banqueting-hall turned inside out. I hear Mrs.\ Trenor wants to
build out a new ball-room, and that divergence from Gus on that
point keeps her at Bellomont. The dimensions of the Brys'
ball-room must rankle: you may be sure she knows 'em as well as
if she'd been there last night with a yard-measure. Who said she
was in town, by the way? That Farish boy? She isn't, I know; Mrs.
Stepney was right; the house is dark, you see: I suppose Gus
lives in the back.''





He had halted opposite the Trenors'\ corner, and Selden perforce
stayed his steps also. The house loomed obscure and uninhabited;
only an oblong gleam above the door spoke of provisional
occupancy.





``They've bought the house at the back: it gives them a hundred
and fifty feet in the side street. There's where the ball-room's
to be, with a gallery connecting it: billiard-room and so on
above. I suggested changing the entrance, and carrying the
drawing-room across the whole Fifth Avenue front; you see the
front door corresponds with the windows----''





The walking-stick which Van Alstyne swung in demonstration
dropped to a startled ``Hallo!''\ as the door opened and two figures
were seen silhouetted against the hall-light. At the same moment
a hansom halted at the curb-stone, and one of the figures floated
down to it in a haze of evening draperies; while the other, black
and bulky, remained persistently projected against the light.





For an immeasurable second the two spectators of the incident
were silent; then the house-door closed, the hansom rolled off,
and the whole scene slipped by as if with the turn of a
stereopticon.





Van Alstyne dropped his eye-glass with a low whistle.





``A---hem---nothing of this, eh, Selden? As one of the family, I
know I may count on you---appearances are deceptive---and Fifth
Avenue is so imperfectly lighted----''





``Goodnight,''\ said Selden, turning sharply down the side street
without seeing the other's extended hand.







Alone with her cousin's kiss, Gerty stared upon her thoughts. He
had kissed her before---but not with another woman on his lips. If
he had spared her that she could have drowned quietly, welcoming
the dark flood as it submerged her. But now the flood was shot
through with glory, and it was harder to drown at sunrise than in
darkness. Gerty hid her face from the light, but it pierced to
the crannies of her soul. She had been so contented, life had
seemed so simple and sufficient---why had he come to trouble her
with new hopes? And Lily---Lily, her best friend! Woman-like, she
accused the woman. Perhaps, had it not been for Lily, her fond
imagining might have become truth. Selden had always liked
her---had understood and sympathized with the modest independence
of her life. He, who had the reputation of weighing all things in
the nice balance of fastidious perceptions, had been uncritical
and simple in his view of her: his cleverness had never overawed
her because she had felt at home in his heart. And now she was
thrust out, and the door barred against her by Lily's hand! Lily,
for whose admission there she herself had pleaded! The situation
was lighted up by a dreary flash of irony. She knew Selden---she
saw how the force of her faith in Lily must have helped to dispel
his hesitations. She remembered, too, how Lily had talked of
him---she saw herself bringing the two together, making them known
to each other. On Selden's part, no doubt, the wound inflicted
was inconscient; he had never guessed her foolish secret; but
Lily---Lily must have known! When, in such matters, are a woman's
perceptions at fault? And if she knew, then she had deliberately
despoiled her friend, and in mere wantonness of power,
since, even to Gerty's suddenly flaming jealousy, it seemed
incredible that Lily should wish to be Selden's wife. Lily might
be incapable of marrying for money, but she was equally incapable
of living without it, and Selden's eager investigations into the
small economies of house-keeping made him appear to Gerty as
tragically duped as herself.





She remained long in her sitting-room, where the embers were
crumbling to cold grey, and the lamp paled under its gay shade. 
Just beneath it stood the photograph of Lily Bart, looking out
imperially on the cheap gimcracks, the cramped furniture of the
little room. Could Selden picture her in such an interior? Gerty
felt the poverty, the insignificance of her surroundings: she
beheld her life as it must appear to Lily. And the cruelty of
Lily's judgments smote upon her memory. She saw that she had
dressed her idol with attributes of her own making. When had Lily
ever really felt, or pitied, or understood? All she wanted was
the taste of new experiences: she seemed like some cruel creature
experimenting in a laboratory.





The pink-faced clock drummed out another hour, and Gerty rose
with a start. She had an appointment early the next morning with
a district visitor on the East side. She put out her lamp,
covered the fire, and went into her bedroom to undress. In the
little glass above her dressing-table she saw her face reflected
against the shadows of the room, and tears blotted the
reflection. What right had she to dream the dreams of loveliness? 
A dull face invited a dull fate. She cried quietly as she
undressed, laying aside her clothes with her habitual precision,
setting everything in order for the next day, when the old life
must be taken up as though there had been no break in its
routine. Her servant did not come till eight o'clock, and she
prepared her own tea-tray and placed it beside the bed. Then she
locked the door of the flat, extinguished her light and lay down. 
But on her bed sleep would not come, and she lay face to face
with the fact that she hated Lily Bart. It closed with her in the
darkness like some formless evil to be blindly grappled with. 
Reason, judgment, renunciation, all the sane daylight forces,
were beaten back in the sharp struggle for self-preservation. She
wanted happiness---wanted it as fiercely and unscrupulously as Lily
did, but without Lily's power of obtaining it. And in her conscious
impotence she lay shivering, and hated her friend----





A ring at the door-bell caught her to her feet. She struck a
light and stood startled, listening. For a moment her heart beat
incoherently, then she felt the sobering touch of fact, and
remembered that such calls were not unknown in her charitable
work. She flung on her dressing-gown to answer the summons, and
unlocking her door, confronted the shining vision of Lily Bart.





Gerty's first movement was one of revulsion. She shrank back as
though Lily's presence flashed too sudden a light upon her
misery. Then she heard her name in a cry, had a glimpse of her
friend's face, and felt herself caught and clung to.





``Lily---what is it?''\ she exclaimed.





Miss Bart released her, and stood breathing brokenly, like one
who has gained shelter after a long flight.





``I was so cold---I couldn't go home. Have you a fire?''





Gerty's compassionate instincts, responding to the swift call of
habit, swept aside all her reluctances. Lily was simply some one
who needed help---for what reason, there was no time to pause and
conjecture: disciplined sympathy checked the wonder on Gerty's
lips, and made her draw her friend silently into the sitting-room
and seat her by the darkened hearth.





``There is kindling wood here: the fire will burn in a minute.''





She knelt down, and the flame leapt under her rapid hands. It
flashed strangely through the tears which still blurred her eyes,
and smote on the white ruin of Lily's face. The girls looked at
each other in silence; then Lily repeated: ``I couldn't go home.''





``No---no---you came here, dear! You're cold and tired---sit quiet,
and I'll make you some tea.''





Gerty had unconsciously adopted the soothing note of her trade: 
all personal feeling was merged in the sense of ministry, and
experience had taught her that the bleeding must be stayed before
the wound is probed.





Lily sat quiet, leaning to the fire: the clatter of cups behind
her soothed her as familiar noises hush a child whom silence has
kept wakeful. But when Gerty stood at her side with the tea she
pushed it away, and turned an estranged eye on the familiar room.





``I came here because I couldn't bear to be alone,''\ she said.





Gerty set down the cup and knelt beside her.





``Lily! Something has happened---can't you tell me?''





``I couldn't bear to lie awake in my room till morning. I hate my
room at Aunt Julia's---so I came here----''





She stirred suddenly, broke from her apathy, and clung to Gerty in
a fresh burst of fear.





``Oh, Gerty, the furies .\ .\ .\ you know the noise of their
wings---alone, at night, in the dark? But you don't know---there is
nothing to make the dark dreadful to you----''





The words, flashing back on Gerty's last hours, struck from her a
faint derisive murmur; but Lily, in the blaze of her own misery,
was blinded to everything outside it.





``You'll let me stay? I shan't mind when daylight comes---Is it
late? Is the night nearly over? It must be awful to be
sleepless---everything stands by the bed and stares----''





Miss Farish caught her straying hands. ``Lily, look at me! 
Something has happened---an accident? You have been
frightened---what has frightened you? Tell me if you can---a word
or two---so that I can help you.''





Lily shook her head.





``I am not frightened: that's not the word. Can you imagine
looking into your glass some morning and seeing a
disfigurement---some hideous change that has come to you while you
slept? Well, I seem to myself like that---I can't bear to see
myself in my own thoughts---I hate ugliness, you know---I've always
turned from it---but I can't explain to you---you wouldn't
understand.''





She lifted her head and her eyes fell on the clock.





``How long the night is! And I know I shan't sleep tomorrow. Some
one told me my father used to lie sleepless and think of horrors. 
And he was not wicked, only unfortunate---and I see now how he
must have suffered, lying alone with his thoughts! But I am
bad---a bad girl---all my thoughts are bad---I have always had bad
people about me. Is that any excuse? I thought I could
manage my own life---I was proud---proud!\ but now I'm on their
level----''





Sobs shook her, and she bowed to them like a tree in a dry storm.





Gerty knelt beside her, waiting, with the patience born of
experience, till this gust of misery should loosen fresh speech. 
She had first imagined some physical shock, some peril of the
crowded streets, since Lily was presumably on her way home from
Carry Fisher's; but she now saw that other nerve-centres were
smitten, and her mind trembled back from conjecture.





Lily's sobs ceased, and she lifted her head.





``There are bad girls in your slums. Tell me---do they ever pick
themselves up? Ever forget, and feel as they did before?''





``Lily!\ you mustn't speak so---you're dreaming.''





``Don't they always go from bad to worse? There's no turning
back---your old self rejects you, and shuts you out.''





She rose, stretching her arms as if in utter physical weariness. 
``Go to bed, dear! You work hard and get up early. I'll watch here
by the fire, and you'll leave the light, and your door open. All
I want is to feel that you are near me.'' She laid both hands on
Gerty's shoulders, with a smile that was like sunrise on a sea
strewn with wreckage.





``I can't leave you, Lily. Come and lie on my bed. Your hands are
frozen---you must undress and be made warm.'' Gerty paused with
sudden compunction. ``But Mrs.\ Peniston---it's past midnight! What
will she think?''





``She goes to bed. I have a latch-key. It doesn't matter---I can't
go back there.''





``There's no need to: you shall stay here. But you must tell me
where you have been. Listen, Lily---it will help you to speak!''
She regained Miss Bart's hands, and pressed them against her. 
``Try to tell me---it will clear your poor head. Listen---you were
dining at Carry Fisher's.'' Gerty paused and added with a flash of
heroism: ``Lawrence Selden went from here to find you.''





At the word, Lily's face melted from locked anguish to the open
misery of a child. Her lips trembled and her gaze widened with
tears.





``He went to find me? And I missed him! Oh, Gerty, he
tried to help me. He told me---he warned me long ago---he foresaw
that I should grow hateful to myself!''





The name, as Gerty saw with a clutch at the heart, had loosened
the springs of self-pity in her friend's dry breast, and tear by
tear Lily poured out the measure of her anguish. She had dropped
sideways in Gerty's big arm-chair, her head buried where lately
Selden's had leaned, in a beauty of abandonment that drove home
to Gerty's aching senses the inevitableness of her own defeat. 
Ah, it needed no deliberate purpose on Lily's part to rob her of
her dream! To look on that prone loveliness was to see in it a
natural force, to recognize that love and power belong to such as
Lily, as renunciation and service are the lot of those they
despoil. But if Selden's infatuation seemed a fatal necessity,
the effect that his name produced shook Gerty's steadfastness
with a last pang. Men pass through such superhuman loves and
outlive them: they are the probation subduing the heart to human
joys. How gladly Gerty would have welcomed the ministry of
healing: how willingly have soothed the sufferer back to
tolerance of life! But Lily's self-betrayal took this last hope
from her. The mortal maid on the shore is helpless against the
siren who loves her prey: such victims are floated back dead
from their adventure.





Lily sprang up and caught her with strong hands. ``Gerty, you know
him---you understand him---tell me; if I went to him, if I told
him everything---if I said: 'I am bad through and through---I want
admiration, I want excitement, I want money---'\ yes, \textit{money}! 
That's my shame, Gerty---and it's known, it's said of me---it's
what men think of me---If I said it all to him---told him the
whole story---said plainly: 'I've sunk lower than the lowest, for
I've taken what they take, and not paid as they pay'---oh, Gerty,
you know him, you can speak for him: if I told him everything
would he loathe me? Or would he pity me, and understand me, and
save me from loathing myself?''





Gerty stood cold and passive. She knew the hour of her probation
had come, and her poor heart beat wildly against its destiny. As
a dark river sweeps by under a lightning flash, she saw her
chance of happiness surge past under a flash of temptation. What
prevented her from saying: ``He is like other men?''\ She
was not so sure of him, after all! But to do so would have been
like blaspheming her love. She could not put him before herself
in any light but the noblest: she must trust him to the height of
her own passion.





``Yes: I know him; he will help you,''\ she said; and in a moment
Lily's passion was weeping itself out against her breast.





There was but one bed in the little flat, and the two girls lay
down on it side by side when Gerty had unlaced Lily's dress and
persuaded her to put her lips to the warm tea. The light
extinguished, they lay still in the darkness, Gerty shrinking to
the outer edge of the narrow couch to avoid contact with her
bed-fellow. Knowing that Lily disliked to be caressed, she had
long ago learned to check her demonstrative impulses toward her
friend. But tonight every fibre in her body shrank from Lily's
nearness: it was torture to listen to her breathing, and feel the
sheet stir with it. As Lily turned, and settled to completer
rest, a strand of her hair swept Gerty's cheek with its
fragrance. Everything about her was warm and soft and scented: 
even the stains of her grief became her as rain-drops do the
beaten rose. But as Gerty lay with arms drawn down her side, in
the motionless narrowness of an effigy, she felt a stir of sobs
from the breathing warmth beside her, and Lily flung out her
hand, groped for her friend's, and held it fast.





``Hold me, Gerty, hold me, or I shall think of things,''\ she
moaned; and Gerty silently slipped an arm under her, pillowing
her head in its hollow as a mother makes a nest for a tossing
child. In the warm hollow Lily lay still and her breathing grew
low and regular. Her hand still clung to Gerty's as if to ward off
evil dreams, but the hold of her fingers relaxed, her head sank
deeper into its shelter, and Gerty felt that she slept.





\chapter*{\raggedright Chapter 15}

\addcontentsline{toc}{chapter}{Chapter 15}

\markboth{House of Mirth}{Chapter 15}





When lily woke she had the bed to herself, and the winter light
was in the room.





She sat up, bewildered by the strangeness of her surroundings;
then memory returned, and she looked about her with a shiver. In
the cold slant of light reflected from the back wall of a
neighbouring building, she saw her evening dress and opera cloak
lying in a tawdry heap on a chair. Finery laid off is as
unappetizing as the remains of a feast, and it occurred to Lily
that, at home, her maid's vigilance had always spared her the
sight of such incongruities. Her body ached with fatigue, and
with the constriction of her attitude in Gerty's bed. All through
her troubled sleep she had been conscious of having no space to
toss in, and the long effort to remain motionless made her feel
as if she had spent her night in a train.





This sense of physical discomfort was the first to assert itself;
then she perceived, beneath it, a corresponding mental
prostration, a languor of horror more insufferable than the first
rush of her disgust. The thought of having to wake every morning
with this weight on her breast roused her tired mind to fresh
effort. She must find some way out of the slough into which she
had stumbled: it was not so much compunction as the dread of her
morning thoughts that pressed on her the need of action. But she
was unutterably tired; it was weariness to think connectedly. She
lay back, looking about the poor slit of a room with a renewal of
physical distaste. The outer air, penned between high buildings,
brought no freshness through the window; steam-heat was beginning
to sing in a coil of dingy pipes, and a smell of cooking
penetrated the crack of the door.





The door opened, and Gerty, dressed and hatted, entered with a
cup of tea. Her face looked sallow and swollen in the dreary
light, and her dull hair shaded imperceptibly into the tones of
her skin.





She glanced shyly at Lily, asking in an embarrassed tone how she
felt; Lily answered with the same constraint, and raised herself
up to drink the tea.





``I must have been over-tired last night; I think I had a
nervous attack in the carriage,''\ she said, as the drink brought
clearness to her sluggish thoughts.





``You were not well; I am so glad you came here,''\ Gerty returned.





``But how am I to get home? And Aunt Julia---?''





``She knows; I telephoned early, and your maid has brought your
things. But won't you eat something? I scrambled the eggs
myself.''





Lily could not eat; but the tea strengthened her to rise and
dress under her maid's searching gaze. It was a relief to her
that Gerty was obliged to hasten away: the two kissed silently,
but without a trace of the previous night's emotion.





Lily found Mrs.\ Peniston in a state of agitation. She had sent
for Grace Stepney and was taking digitalis. Lily breasted the
storm of enquiries as best she could, explaining that she had had
an attack of faintness on her way back from Carry Fisher's; that,
fearing she would not have strength to reach home, she had gone
to Miss Farish's instead; but that a quiet night had restored
her, and that she had no need of a doctor.





This was a relief to Mrs.\ Peniston, who could give herself up to
her own symptoms, and Lily was advised to go and lie down, her
aunt's panacea for all physical and moral disorders. In the
solitude of her own room she was brought back to a sharp
contemplation of facts. Her daylight view of them necessarily
differed from the cloudy vision of the night. The winged furies
were now prowling gossips who dropped in on each other for tea. 
But her fears seemed the uglier, thus shorn of their vagueness;
and besides, she had to act, not rave. For the first time she
forced herself to reckon up the exact amount of her debt to
Trenor; and the result of this hateful computation was the
discovery that she had, in all, received nine thousand dollars
from him. The flimsy pretext on which it had been given and
received shrivelled up in the blaze of her shame: she knew that
not a penny of it was her own, and that to restore her
self-respect she must at once repay the whole amount. The
inability thus to solace her outraged feelings gave her a
paralyzing sense of insignificance. She was realizing for the
first time that a woman's dignity may cost more to keep up than
her carriage; and that the maintenance of a moral
attribute should be dependent on dollars and cents, made the
world appear a more sordid place than she had conceived it.





After luncheon, when Grace Stepney's prying eyes had been
removed, Lily asked for a word with her aunt. The two ladies went
upstairs to the sitting-room, where Mrs.\ Peniston seated herself
in her black satin arm-chair tufted with yellow buttons, beside a
bead-work table bearing a bronze box with a miniature of Beatrice
Cenci in the lid. Lily felt for these objects the same distaste
which the prisoner may entertain for the fittings of the
court-room. It was here that her aunt received her rare
confidences, and the pink-eyed smirk of the turbaned Beatrice was
associated in her mind with the gradual fading of the smile from
Mrs.\ Peniston's lips. That lady's dread of a scene gave her an
inexorableness which the greatest strength of character could not
have produced, since it was independent of all considerations of
right or wrong; and knowing this, Lily seldom ventured to assail
it. She had never felt less like making the attempt than on the
present occasion; but she had sought in vain for any other means
of escape from an intolerable situation.





Mrs.\ Peniston examined her critically. ``You're a bad colour,
Lily: this incessant rushing about is beginning to tell on you,''
she said.





Miss Bart saw an opening. ``I don't think it's that, Aunt Julia;
I've had worries,''\ she replied.





``Ah,''\ said Mrs.\ Peniston, shutting her lips with the snap of a
purse closing against a beggar.





``I'm sorry to bother you with them,''\ Lily continued, ``but I
really believe my faintness last night was brought on partly by
anxious thoughts---''





``I should have said Carry Fisher's cook was enough to account for
it. She has a woman who was with Maria Melson in 1891---the spring
of the year we went to Aix---and I remember dining there two days
before we sailed, and feeling \textit{sure} the coppers hadn't been
scoured.''





``I don't think I ate much; I can't eat or sleep.'' Lily paused,
and then said abruptly: ``The fact is, Aunt Julia, I owe some
money.''





Mrs.\ Peniston's face clouded perceptibly, but did not
express the astonishment her niece had expected. She was silent,
and Lily was forced to continue: ``I have been foolish----''





``No doubt you have: extremely foolish,''\ Mrs.\ Peniston interposed. 
``I fail to see how any one with your income, and no expenses---not
to mention the handsome presents I've always given you----''





``Oh, you've been most generous, Aunt Julia; I shall never forget
your kindness. But perhaps you don't quite realize the expense a
girl is put to nowadays----''





``I don't realize that \textit{you} are put to any expense except for your
clothes and your railway fares. I expect you to be handsomely
dressed; but I paid Celeste's bill for you last October.''





Lily hesitated: her aunt's implacable memory had never been more
inconvenient. ``You were as kind as possible; but I have had to
get a few things since----''





``What kind of things? Clothes? How much have you spent? Let me
see the bill---I daresay the woman is swindling you.''





``Oh, no, I think not: clothes have grown so frightfully
expensive; and one needs so many different kinds, with country
visits, and golf and skating, and Aiken and Tuxedo----''





``Let me see the bill,''\ Mrs.\ Peniston repeated.





Lily hesitated again. In the first place, \textit{Mme}.\ Celeste had not
yet sent in her account, and secondly, the amount it represented
was only a fraction of the sum that Lily needed.





``She hasn't sent in the bill for my winter things, but I \textit{know}
it's large; and there are one or two other things; I've been
careless and imprudent---I'm frightened to think of what I owe----''





She raised the troubled loveliness of her face to Mrs.\ Peniston,
vainly hoping that a sight so moving to the other sex might not
be without effect upon her own. But the effect produced was that
of making Mrs.\ Peniston shrink back apprehensively.





``Really, Lily, you are old enough to manage your own affairs, and
after frightening me to death by your performance of last night
you might at least choose a better time to worry me with such
matters.'' Mrs.\ Peniston glanced at the clock, and swallowed a
tablet of digitalis. ``If you owe Celeste another
thousand, she may send me her account,''\ she added, as though to
end the discussion at any cost.





``I am very sorry, Aunt Julia; I hate to trouble you at such a
time; but I have really no choice---I ought to have spoken
sooner---I owe a great deal more than a thousand dollars.''





``A great deal more? Do you owe two? She must have robbed you!''





``I told you it was not only Celeste. I---there are other
bills---more pressing---that must be settled.''





``What on earth have you been buying? Jewelry? You must have gone
off your head,''\ said Mrs.\ Peniston with asperity. ``But if you
have run into debt, you must suffer the consequences, and put
aside your monthly income till your bills are paid. If you stay
quietly here until next spring, instead of racing about all over
the country, you will have no expenses at all, and surely in four
or five months you can settle the rest of your bills if I pay the
dress-maker now.''





Lily was again silent. She knew she could not hope to extract
even a thousand dollars from Mrs.\ Peniston on the mere plea of
paying Celeste's bill: Mrs.\ Peniston would expect to go over the
dress-maker's account, and would make out the cheque to her and
not to Lily. And yet the money must be obtained before the day
was over!





``The debts I speak of are---different---not like tradesmen's
bills,''\ she began confusedly; but Mrs.\ Peniston's look made her
almost afraid to continue. Could it be that her aunt suspected
anything? The idea precipitated Lily's avowal.





``The fact is, I've played cards a good deal---bridge; the women
all do it; girls too---it's expected. Sometimes I've won---won a
good deal---but lately I've been unlucky---and of course such debts
can't be paid off gradually----''





She paused: Mrs.\ Peniston's face seemed to be petrifying as she
listened.





``Cards---you've played cards for money? It's true, then: when I
was told so I wouldn't believe it. I won't ask if the other
horrors I was told were true too; I've heard enough for the state
of my nerves. When I think of the example you've had in this
house! But I suppose it's your foreign bringing-up---no one knew
where your mother picked up her friends. And her Sundays were a
scandal---that I know.''





Mrs.\ Peniston wheeled round suddenly. ``You play cards on Sunday?''





Lily flushed with the recollection of certain rainy Sundays at
Bellomont and with the Dorsets.





``You're hard on me, Aunt Julia: I have never really cared for
cards, but a girl hates to be thought priggish and superior, and
one drifts into doing what the others do. I've had a dreadful
lesson, and if you'll help me out this time I promise you---''





Mrs.\ Peniston raised her hand warningly. ``You needn't make any
promises: it's unnecessary. When I offered you a home I didn't
undertake to pay your gambling debts.''





``Aunt Julia! You don't mean that you won't help me?''





``I shall certainly not do anything to give the impression that I
countenance your behaviour. If you really owe your dress-maker, I
will settle with her---beyond that I recognize no obligation to
assume your debts.''





Lily had risen, and stood pale and quivering before her aunt. 
Pride stormed in her, but humiliation forced the cry from her
lips: ``Aunt Julia, I shall be disgraced---I---''\ But she could go no
farther. If her aunt turned such a stony ear to the fiction of
the gambling debts, in what spirit would she receive the terrible
avowal of the truth?





``I consider that you \textit{are} disgraced, Lily: disgraced by your
conduct far more than by its results. You say your friends have
persuaded you to play cards with them; well, they may as well
learn a lesson too. They can probably afford to lose a little
money---and at any rate, I am not going to waste any of mine in
paying them. And now I must ask you to leave me---this scene has
been extremely painful, and I have my own health to consider. 
Draw down the blinds, please; and tell Jennings I will see no one
this afternoon but Grace Stepney.''





Lily went up to her own room and bolted the door. She was
trembling with fear and anger---the rush of the furies'\ wings was
in her ears. She walked up and down the room with blind irregular
steps. The last door of escape was closed---she felt herself shut
in with her dishonour.





Suddenly her wild pacing brought her before the clock on the
chimney-piece. Its hands stood at half-past three, and she
remembered that Selden was to come to her at four. She had
meant to put him off with a word---but now her heart leaped at the
thought of seeing him. Was there not a promise of rescue in his
love? As she had lain at Gerty's side the night before, she had
thought of his coming, and of the sweetness of weeping out her
pain upon his breast. Of course she had meant to clear herself of
its consequences before she met him---she had never really doubted
that Mrs.\ Peniston would come to her aid. And she had felt, even
in the full storm of her misery, that Selden's love could not be
her ultimate refuge; only it would be so sweet to take a moment's
shelter there, while she gathered fresh strength to go on.





But now his love was her only hope, and as she sat alone with her
wretchedness the thought of confiding in him became as seductive
as the river's flow to the suicide. The first plunge would be
terrible---but afterward, what blessedness might come! She
remembered Gerty's words: ``I know him---he will help you''; and her
mind clung to them as a sick person might cling to a healing
relic. Oh, if he really understood---if he would help her to
gather up her broken life, and put it together in some new
semblance in which no trace of the past should remain! He had
always made her feel that she was worthy of better things, and
she had never been in greater need of such solace. Once and again
she shrank at the thought of imperilling his love by her
confession: for love was what she needed---it would take the glow
of passion to weld together the shattered fragments of her
self-esteem. But she recurred to Gerty's words and held fast to
them. She was sure that Gerty knew Selden's feeling for her, and
it had never dawned upon her blindness that Gerty's own judgment
of him was coloured by emotions far more ardent than her own.





Four o'clock found her in the drawing-room: she was sure that
Selden would be punctual. But the hour came and passed---it moved
on feverishly, measured by her impatient heart-beats. She had
time to take a fresh survey of her wretchedness, and to fluctuate
anew between the impulse to confide in Selden and the dread of
destroying his illusions. But as the minutes passed the need of
throwing herself on his comprehension became more urgent: she
could not bear the weight of her misery alone. There would be a
perilous moment, perhaps: but could she not trust to her
beauty to bridge it over, to land her safe in the shelter of his
devotion?





But the hour sped on and Selden did not come. Doubtless he had
been detained, or had misread her hurriedly scrawled note, taking
the four for a five. The ringing of the door-bell a few minutes
after five confirmed this supposition, and made Lily hastily
resolve to write more legibly in future. The sound of steps in
the hall, and of the butler's voice preceding them, poured fresh
energy into her veins. She felt herself once more the alert and
competent moulder of emergencies, and the remembrance of her
power over Selden flushed her with sudden confidence. But when
the drawing-room door opened it was Rosedale who came in.





The reaction caused her a sharp pang, but after a passing
movement of irritation at the clumsiness of fate, and at her own
carelessness in not denying the door to all but Selden, she
controlled herself and greeted Rosedale amicably. It was annoying
that Selden, when he came, should find that particular visitor in
possession, but Lily was mistress of the art of ridding herself
of superfluous company, and to her present mood Rosedale seemed
distinctly negligible.





His own view of the situation forced itself upon her after a few
moments'\ conversation. She had caught at the Brys'\ entertainment
as an easy impersonal subject, likely to tide them over the
interval till Selden appeared, but Mr.\ Rosedale, tenaciously
planted beside the tea-table, his hands in his pockets, his legs
a little too freely extended, at once gave the topic a personal
turn.





``Pretty well done---well, yes, I suppose it was: Welly Bry's got
his back up and don't mean to let go till he's got the hang of
the thing. Of course, there were things here and there---things
Mrs.\ Fisher couldn't be expected to see to---the champagne wasn't
cold, and the coats got mixed in the coat-room. I would have
spent more money on the music. But that's my character: if I want
a thing I'm willing to pay: I don't go up to the counter, and
then wonder if the article's worth the price. I wouldn't be
satisfied to entertain like the Welly Brys; I'd want something
that would look more easy and natural, more as if I took it in my
stride. And it takes just two things to do that, Miss
Bart: money, and the right woman to spend it.''





He paused, and examined her attentively while she affected to
rearrange the tea-cups.





``I've got the money,''\ he continued, clearing his throat, ``and
what I want is the woman---and I mean to have her too.''





He leaned forward a little, resting his hands on the head of his
walking-stick. He had seen men of Ned Van Alstyne's type bring
their hats and sticks into a drawing-room, and he thought it
added a touch of elegant familiarity to their appearance.





Lily was silent, smiling faintly, with her eyes absently resting
on his face. She was in reality reflecting that a declaration
would take some time to make, and that Selden must surely appear
before the moment of refusal had been reached. Her brooding look,
as of a mind withdrawn yet not averted, seemed to Mr.\ Rosedale
full of a subtle encouragement. He would not have liked any
evidence of eagerness.





``I mean to have her too,''\ he repeated, with a laugh intended to
strengthen his self-assurance. ``I generally \textit{have} got what I
wanted in life, Miss Bart. I wanted money, and I've got more than
I know how to invest; and now the money doesn't seem to be of any
account unless I can spend it on the right woman. That's what I
want to do with it: I want my wife to make all the other women
feel small. I'd never grudge a dollar that was spent on that. But
it isn't every woman can do it, no matter how much you spend on
her. There was a girl in some history book who wanted gold
shields, or something, and the fellows threw 'em at her, and she
was crushed under 'em: they killed her. Well, that's true enough: 
some women looked buried under their jewelry. What I want is a
woman who'll hold her head higher the more diamonds I put on it. 
And when I looked at you the other night at the Brys', in that
plain white dress, looking as if you had a crown on, I said to
myself: 'By gad, if she had one she'd wear it as if it grew on
her.'''





Still Lily did not speak, and he continued, warming with his
theme: ``Tell you what it is, though, that kind of woman costs
more than all the rest of 'em put together. If a woman's
going to ignore her pearls, they want to be better than anybody
else's---and so it is with everything else. You know what I
mean---you know it's only the showy things that are cheap. Well, I
should want my wife to be able to take the earth for granted if
she wanted to. I know there's one thing vulgar about money, and
that's the thinking about it; and my wife would never have to
demean herself in that way.'' He paused, and then added, with an
unfortunate lapse to an earlier manner: ``I guess you know the
lady I've got in view, Miss Bart.''





Lily raised her head, brightening a little under the challenge. 
Even through the dark tumult of her thoughts, the clink of Mr.
Rosedale's millions had a faintly seductive note. Oh, for enough
of them to cancel her one miserable debt! But the man behind them
grew increasingly repugnant in the light of Selden's expected
coming. The contrast was too grotesque: she could scarcely
suppress the smile it provoked. She decided that directness would
be best.





``If you mean me, Mr.\ Rosedale, I am very grateful---very much
flattered; but I don't know what I have ever done to make you
think---''





``Oh, if you mean you're not dead in love with me, I've got sense
enough left to see that. And I ain't talking to you as if you
were---I presume I know the kind of talk that's expected under
those circumstances. I'm confoundedly gone on you---that's about
the size of it---and I'm just giving you a plain business
statement of the consequences. You're not very fond of
me---\textit{yet}---but you're fond of luxury, and style, and amusement, and
of not having to worry about cash. You like to have a good time,
and not have to settle for it; and what I propose to do is to
provide for the good time and do the settling.''





He paused, and she returned with a chilling smile: ``You are
mistaken in one point, Mr.\ Rosedale: whatever I enjoy I am
prepared to settle for.''





She spoke with the intention of making him see that, if his words
implied a tentative allusion to her private affairs, she was
prepared to meet and repudiate it. But if he recognized her
meaning it failed to abash him, and he went on in the same tone: 
``I didn't mean to give offence; excuse me if I've spoken too
plainly. But why ain't you straight with me---why do you
put up that kind of bluff? You know there've been times when you
were bothered---damned bothered---and as a girl gets older, and
things keep moving along, why, before she knows it, the things
she wants are liable to move past her and not come back. I don't
say it's anywhere near that with you yet; but you've had a taste
of bothers that a girl like yourself ought never to have known
about, and what I'm offering you is the chance to turn your back
on them once for all.''





The colour burned in Lily's face as he ended; there was no
mistaking the point he meant to make, and to permit it to pass
unheeded was a fatal confession of weakness, while to resent it
too openly was to risk offending him at a perilous moment. 
Indignation quivered on her lip; but it was quelled by the secret
voice which warned her that she must not quarrel with him. He
knew too much about her, and even at the moment when it was
essential that he should show himself at his best, he did not
scruple to let her see how much he knew. How then would he use
his power when her expression of contempt had dispelled his one
motive for restraint? Her whole future might hinge on her way
of answering him: she had to stop and consider that, in the
stress of her other anxieties, as a breathless fugitive may have
to pause at the cross-roads and try to decide coolly which turn
to take.





``You are quite right, Mr.\ Rosedale. I \textit{have} had bothers; and
I am grateful to you for wanting to relieve me of them. It is
not always easy to be quite independent and self-respecting
when one is poor and lives among rich people; I have been
careless about money, and have worried about my bills. But I
should be selfish and ungrateful if I made that a reason for
accepting all you offer, with no better return to make than
the desire to be free from my anxieties. You must give me
time---time to think of your kindness---and of what I could
give you in return for it----''





She held out her hand with a charming gesture in which
dismissal was shorn of its rigour. Its hint of future leniency
made Rosedale rise in obedience to it, a little flushed with his
unhoped-for success, and disciplined by the tradition of his
blood to accept what was conceded, without undue haste to
press for more. Something in his prompt acquiescence frightened
her; she felt behind it the stored force of a patience that
might subdue the strongest will. But at least they had parted
amicably, and he was out of the house without meeting
Selden---Selden, whose continued absence now smote her with
a new alarm. Rosedale had remained over an hour, and she
understood that it was now too late to hope for Selden. He
would write explaining his absence, of course; there would be
a note from him by the late post. But her confession would
have to be postponed; and the chill of the delay settled heavily
on her fagged spirit.





It lay heavier when the postman's last ring brought no note
for her, and she had to go upstairs to a lonely night---a night
as grim and sleepless as her tortured fancy had pictured it to
Gerty. She had never learned to live with her own thoughts,
and to be confronted with them through such hours of lucid
misery made the confused wretchedness of her previous vigil
seem easily bearable.





Daylight disbanded the phantom crew, and made it clear
to her that she would hear from Selden before noon; but the
day passed without his writing or coming. Lily remained at
home, lunching and dining alone with her aunt, who complained of
flutterings of the heart, and talked icily on general
topics. Mrs.\ Peniston went to bed early, and when she had
gone Lily sat down and wrote a note to Selden. She was
about to ring for a messenger to despatch it when her eye fell
on a paragraph in the evening paper which lay at her elbow: 
``Mr.\ Lawrence Selden was among the passengers sailing this
afternoon for Havana and the West Indies on the Windward
Liner Antilles.''





She laid down the paper and sat motionless, staring at her
note. She understood now that he was never coming---that
he had gone away because he was afraid that he might come. 
She rose, and walking across the floor stood gazing at herself
for a long time in the brightly-lit mirror above the mantel-piece. The lines in her face came out terribly---she looked
old; and when a girl looks old to herself, how does she look
to other people? She moved away, and began to wander
aimlessly about the room, fitting her steps with mechanical
precision between the monstrous roses of Mrs.\ Peniston's
Axminster. Suddenly she noticed that the pen with which she
had written to Selden still rested against the uncovered
inkstand. She seated herself again, and taking out an envelope,
addressed it rapidly to Rosedale. Then she laid out a sheet of
paper, and sat over it with suspended pen. It had been easy
enough to write the date, and ``Dear Mr.\ Rosedale''---but after that
her inspiration flagged. She meant to tell him to come
to her, but the words refused to shape themselves. At length
she began: ``I have been thinking----''\ then she laid the pen
down, and sat with her elbows on the table and her face hidden in
her hands.





Suddenly she started up at the sound of the door-bell. It
was not late---barely ten o'clock---and there might still be a
note from Selden, or a message---or he might be there himself, on
the other side of the door! The announcement of his
sailing might have been a mistake---it might be another Lawrence
Selden who had gone to Havana---all these possibilities
had time to flash through her mind, and build up the conviction
that she was after all to see or hear from him, before the
drawing-room door opened to admit a servant carrying a
telegram.





Lily tore it open with shaking hands, and read Bertha Dorset's
name below the message: ``Sailing unexpectedly tomorrow. Will you
join us on a cruise in Mediterranean?''






\chapter*{\raggedright BOOK TWO}

\addcontentsline{toc}{chapter}{BOOK TWO}

\markboth{HOUSE OF MIRTH}{BOOK TWO}




\section*{\raggedright Chapter 1}



It came vividly to Selden on the Casino steps that Monte Carlo
had, more than any other place he knew, the gift of accommodating
itself to each man's humour. His own, at the moment, lent it a
festive readiness of welcome that might well, in a disenchanted
eye, have turned to paint and facility. So frank an appeal for
participation---so outspoken a recognition of the holiday vein in
human nature---struck refreshingly on a mind jaded by prolonged
hard work in surroundings made for the discipline of the senses. 
As he surveyed the white square set in an exotic coquetry of
architecture, the studied tropicality of the gardens, the groups
loitering in the foreground against mauve mountains which
suggested a sublime stage-setting forgotten in a hurried shifting
of scenes---as he took in the whole outspread effect of light and
leisure, he felt a movement of revulsion from the last few months
of his life.





The New York winter had presented an interminable perspective of
snow-burdened days, reaching toward a spring of raw sunshine and
furious air, when the ugliness of things rasped the eye as the
gritty wind ground into the skin. Selden, immersed in his work,
had told himself that external conditions did not matter to a man
in his state, and that cold and ugliness were a good tonic for
relaxed sensibilities. When an urgent case summoned him abroad to
confer with a client in Paris, he broke reluctantly with the
routine of the office; and it was only now that, having
despatched his business, and slipped away for a week in the
south, he began to feel the renewed zest of spectatorship that is
the solace of those who take an objective interest in life.





The multiplicity of its appeals---the perpetual surprise of its
contrasts and resemblances! All these tricks and turns of the
show were upon him with a spring as he descended the Casino steps
and paused on the pavement at its doors. He had not been abroad
for seven years---and what changes the renewed contact produced! 
If the central depths were untouched, hardly a pin-point of
surface remained the same. And this was the very place to
bring out the completeness of the renewal. The sublimities, the
perpetuities, might have left him as he was: but this tent
pitched for a day's revelry spread a roof of oblivion between
himself and his fixed sky.





It was mid-April, and one felt that the revelry had reached its
climax and that the desultory groups in the square and gardens
would soon dissolve and re-form in other scenes. Meanwhile the
last moments of the performance seemed to gain an added
brightness from the hovering threat of the curtain. The quality
of the air, the exuberance of the flowers, the blue intensity of
sea and sky, produced the effect of a closing \textit{tableau}, when all
the lights are turned on at once. This impression was presently
heightened by the way in which a consciously conspicuous group of
people advanced to the middle front, and stood before Selden with
the air of the chief performers gathered together by the
exigencies of the final effect. Their appearance confirmed the
impression that the show had been staged regardless of expense,
and emphasized its resemblance to one of those ``costume-plays''\ in
which the protagonists walk through the passions without
displacing a drapery. The ladies stood in unrelated attitudes
calculated to isolate their effects, and the men hung about them
as irrelevantly as stage heroes whose tailors are named in the
programme. It was Selden himself who unwittingly fused the group
by arresting the attention of one of its members.





``Why, Mr.\ Selden!''\ Mrs.\ Fisher exclaimed in surprise; and with a
gesture toward Mrs.\ Jack Stepney and Mrs.\ Wellington Bry, she
added plaintively: ``We're starving to death because we can't
decide where to lunch.''





Welcomed into their group, and made the confidant of their
difficulty, Selden learned with amusement that there were several
places where one might miss something by not lunching, or forfeit
something by lunching; so that eating actually became a minor
consideration on the very spot consecrated to its rites.





``Of course one gets the best things at the TERRASSE---but that
looks as if one hadn't any other reason for being there: the
Americans who don't know any one always rush for the best food. 
And the Duchess of Beltshire has taken up Becassin's lately,''
Mrs.\ Bry earnestly summed up.





Mrs.\ Bry, to Mrs.\ Fisher's despair, had not progressed beyond the
point of weighing her social alternatives in public. She could
not acquire the air of doing things because she wanted to, and
making her choice the final seal of their fitness.





Mr.\ Bry, a short pale man, with a business face and leisure
clothes, met the dilemma hilariously.





``I guess the Duchess goes where it's cheapest, unless she can get
her meal paid for. If you offered to blow her off at the TERRASSE
she'd turn up fast enough.''





But Mrs.\ Jack Stepney interposed. ``The Grand Dukes go to that
little place at the Condamine. Lord Hubert says it's the only
restaurant in Europe where they can cook peas.''





Lord Hubert Dacey, a slender shabby-looking man, with a charming
worn smile, and the air of having spent his best years in
piloting the wealthy to the right restaurant, assented with
gentle emphasis: ``It's quite that.''





``\textit{Peas}?''\ said Mr.\ Bry contemptuously. ``Can they cook terrapin? It
just shows,''\ he continued, ``what these European markets are, when
a fellow can make a reputation cooking peas!''





Jack Stepney intervened with authority. ``I don't know that I
quite agree with Dacey: there's a little hole in Paris, off the
Quai Voltaire---but in any case, I can't advise the Condamine
GARGOTE; at least not with ladies.''





Stepney, since his marriage, had thickened and grown prudish, as
the Van Osburgh husbands were apt to do; but his wife, to his
surprise and discomfiture, had developed an earth-shaking
fastness of gait which left him trailing breathlessly in her
wake.





``That's where we'll go then!''\ she declared, with a heavy toss of
her plumage. ``I'm so tired of the TERRASSE: it's as dull as one
of mother's dinners. And Lord Hubert has promised to tell us who
all the awful people are at the other place---hasn't he, Carry? 
Now, Jack, don't look so solemn!''





``Well,''\ said Mrs.\ Bry, ``all I want to know is who their
dress-makers are.''





``No doubt Dacey can tell you that too,''\ remarked Stepney, with an
ironic intention which the other received with the light murmur,
``I can at least \textit{find} \textit{out}, my dear fellow''; and Mrs.\ Bry
having declared that she couldn't walk another step, the party
hailed two or three of the light phaetons which hover attentively
on the confines of the gardens, and rattled off in procession
toward the Condamine.





Their destination was one of the little restaurants overhanging
the boulevard which dips steeply down from Monte Carlo to the low
intermediate quarter along the quay. From the window in which
they presently found themselves installed, they overlooked the
intense blue curve of the harbour, set between the verdure of
twin promontories: to the right, the cliff of Monaco, topped by
the mediaeval silhouette of its church and castle, to the left
the terraces and pinnacles of the gambling-house. Between the
two, the waters of the bay were furrowed by a light coming and
going of pleasure-craft, through which, just at the culminating
moment of luncheon, the majestic advance of a great steam-yacht
drew the company's attention from the peas.





``By Jove, I believe that's the Dorsets back!''\ Stepney exclaimed;
and Lord Hubert, dropping his single eye-glass, corroborated: 
``It's the Sabrina---yes.''





``So soon? They were to spend a month in Sicily,''\ Mrs.\ Fisher
observed.





``I guess they feel as if they had: there's only one up-to-date
hotel in the whole place,''\ said Mr.\ Bry disparagingly.





``It was Ned Silverton's idea---but poor Dorset and Lily Bart must
have been horribly bored.'' Mrs.\ Fisher added in an undertone to
Selden: ``I do hope there hasn't been a row.''





``It's most awfully jolly having Miss Bart back,''\ said Lord
Hubert, in his mild deliberate voice; and Mrs.\ Bry added
ingenuously: ``I daresay the Duchess will dine with us, now that
Lily's here.''





``The Duchess admires her immensely: I'm sure she'd be charmed to
have it arranged,''\ Lord Hubert agreed, with the professional
promptness of the man accustomed to draw his profit from
facilitating social contacts: Selden was struck by the
businesslike change in his manner.





``Lily has been a tremendous success here,''\ Mrs.\ Fisher continued,
still addressing herself confidentially to Selden. ``She looks ten
years younger---I never saw her so handsome. Lady Skiddaw took her
everywhere in Cannes, and the Crown Princess of Macedonia
had her to stop for a week at Cimiez. People say that was one
reason why Bertha whisked the yacht off to Sicily: the Crown
Princess didn't take much notice of her, and she couldn't bear to
look on at Lily's triumph.''





Selden made no reply. He was vaguely aware that Miss Bart was
cruising in the Mediterranean with the Dorsets, but it had not
occurred to him that there was any chance of running across her
on the Riviera, where the season was virtually at an end. As he
leaned back, silently contemplating his filigree cup of Turkish
coffee, he was trying to put some order in his thoughts, to tell
himself how the news of her nearness was really affecting him. He
had a personal detachment enabling him, even in moments of
emotional high-pressure, to get a fairly clear view of his
feelings, and he was sincerely surprised by the disturbance which
the sight of the Sabrina had produced in him. He had reason to
think that his three months of engrossing professional work,
following on the sharp shock of his disillusionment, had cleared
his mind of its sentimental vapours. The feeling he had nourished
and given prominence to was one of thankfulness for his escape: 
he was like a traveller so grateful for rescue from a dangerous
accident that at first he is hardly conscious of his bruises. Now
he suddenly felt the latent ache, and realized that after all he
had not come off unhurt.





An hour later, at Mrs.\ Fisher's side in the Casino gardens, he
was trying to find fresh reasons for forgetting the injury
received in the contemplation of the peril avoided. The party had
dispersed with the loitering indecision characteristic of social
movements at Monte Carlo, where the whole place, and the long
gilded hours of the day, seem to offer an infinity of ways of
being idle. Lord Hubert Dacey had finally gone off in quest of
the Duchess of Beltshire, charged by Mrs.\ Bry with the delicate
negotiation of securing that lady's presence at dinner, the
Stepneys had left for Nice in their motor-car, and Mr.\ Bry had
departed to take his place in the pigeon shooting match which was
at the moment engaging his highest faculties.





Mrs.\ Bry, who had a tendency to grow red and stertorous after
luncheon, had been judiciously prevailed upon by Carry
Fisher to withdraw to her hotel for an hour's repose; and Selden
and his companion were thus left to a stroll propitious to
confidences. The stroll soon resolved itself into a tranquil
session on a bench overhung with laurel and Banksian roses, from
which they caught a dazzle of blue sea between marble balusters,
and the fiery shafts of cactus-blossoms shooting meteor-like from
the rock. The soft shade of their niche, and the adjacent glitter
of the air, were conducive to an easy lounging mood, and to the
smoking of many cigarettes; and Selden, yielding to these
influences, suffered Mrs.\ Fisher to unfold to him the history of
her recent experiences. She had come abroad with the Welly Brys
at the moment when fashion flees the inclemency of the New York
spring. The Brys, intoxicated by their first success, already
thirsted for new kingdoms, and Mrs.\ Fisher, viewing the Riviera
as an easy introduction to London society, had guided their
course thither. She had affiliations of her own in every capital,
and a facility for picking them up again after long absences; and
the carefully disseminated rumour of the Brys'\ wealth had at once
gathered about them a group of cosmopolitan pleasure-seekers.





``But things are not going as well as I expected,''\ Mrs.\ Fisher
frankly admitted. ``It's all very well to say that every body with
money can get into society; but it would be truer to say that
\textit{nearly} everybody can. And the London market is so glutted with
new Americans that, to succeed there now, they must be either
very clever or awfully queer. The Brys are neither. \textit{He} would get
on well enough if she'd let him alone; they like his slang and
his brag and his blunders. But Louisa spoils it all by trying to
repress him and put herself forward. If she'd be natural
herself---fat and vulgar and bouncing---it would be all right; but
as soon as she meets anybody smart she tries to be slender and
queenly. She tried it with the Duchess of Beltshire and Lady
Skiddaw, and they fled. I've done my best to make her see her
mistake---I've said to her again and again: 'Just let yourself go,
Louisa'; but she keeps up the humbug even with me---I believe she
keeps on being queenly in her own room, with the door shut.





``The worst of it is,''\ Mrs.\ Fisher went on, ``that she thinks it's
all \textit{my} fault. When the Dorsets turned up here six weeks ago, and
everybody began to make a fuss about Lily Bart, I could
see Louisa thought that if she'd had Lily in tow instead of me
she would have been hob-nobbing with all the royalties by this
time. She doesn't realize that it's Lily's beauty that does it: 
Lord Hubert tells me Lily is thought even handsomer than when he
knew her at Aix ten years ago. It seems she was tremendously
admired there. An Italian Prince, rich and the real thing, wanted
to marry her; but just at the critical moment a good-looking
step-son turned up, and Lily was silly enough to flirt with him
while her marriage-settlements with the step-father were being
drawn up. Some people said the young man did it on purpose. You
can fancy the scandal: there was an awful row between the men,
and people began to look at Lily so queerly that Mrs.\ Peniston
had to pack up and finish her cure elsewhere. Not that \textit{she} ever
understood: to this day she thinks that Aix didn't suit her, and
mentions her having been sent there as proof of the incompetence
of French doctors. That's Lily all over, you know: she works like
a slave preparing the ground and sowing her seed; but the day she
ought to be reaping the harvest she over-sleeps herself or goes
off on a picnic.''





Mrs.\ Fisher paused and looked reflectively at the deep shimmer of
sea between the cactus-flowers. ``Sometimes,''\ she added, ``I think
it's just flightiness---and sometimes I think it's because, at
heart, she despises the things she's trying for. And it's the
difficulty of deciding that makes her such an interesting study.'' 
She glanced tentatively at Selden's motionless profile, and
resumed with a slight sigh: ``Well, all I can say is, I wish she'd
give \textit{me} some of her discarded opportunities. I wish we could
change places now, for instance. She could make a very good thing
out of the Brys if she managed them properly, and I should know
just how to look after George Dorset while Bertha is reading
Verlaine with Neddy Silverton.''





She met Selden's sound of protest with a sharp derisive glance. 
``Well, what's the use of mincing matters? We all know that's what
Bertha brought her abroad for. When Bertha wants to have a good
time she has to provide occupation for George. At first I thought
Lily was going to play her cards well \textit{this} time, but there are
rumours that Bertha is jealous of her success here and at Cannes,
and I shouldn't be surprised if there were a break any
day. Lily's only safeguard is that Bertha needs her badly---oh,
very badly. The Silverton affair is in the acute stage: it's
necessary that George's attention should be pretty continuously
distracted. And I'm bound to say Lily \textit{does} distract it: I believe
he'd marry her tomorrow if he found out there was anything wrong
with Bertha. But you know him---he's as blind as he's jealous; and
of course Lily's present business is to keep him blind. A clever
woman might know just the right moment to tear off the bandage: 
but Lily isn't clever in that way, and when George does open his
eyes she'll probably contrive not to be in his line of vision.''





Selden tossed away his cigarette. ``By Jove---it's time for my
train,''\ he exclaimed, with a glance at his watch; adding, in
reply to Mrs.\ Fisher's surprised comment---``Why, I thought of
course you were at Monte!''---a murmured word to the effect that he
was making Nice his head-quarters.





``The worst of it is, she snubs the Brys now,''\ he heard
irrelevantly flung after him.





Ten minutes later, in the high-perched bedroom of an hotel
overlooking the Casino, he was tossing his effects into a couple
of gaping portmanteaux, while the porter waited outside to
transport them to the cab at the door. It took but a brief plunge
down the steep white road to the station to land him safely in
the afternoon express for Nice; and not till he was installed in
the corner of an empty carriage, did he exclaim to himself, with
a reaction of self-contempt: ``What the deuce am I running away
from?''





The pertinence of the question checked Selden's fugitive impulse
before the train had started. It was ridiculous to be flying like
an emotional coward from an infatuation his reason had conquered. 
He had instructed his bankers to forward some important business
letters to Nice, and at Nice he would quietly await them. He was
already annoyed with himself for having left Monte Carlo, where
he had intended to pass the week which remained to him before
sailing; but it would now be difficult to return on his steps
without an appearance of inconsistency from which his pride
recoiled. In his inmost heart he was not sorry to put himself
beyond the probability of meeting Miss Bart. Completely as he had
detached himself from her, he could not yet regard her merely as a
social instance; and viewed in a more personal way she was not
likely to be a reassuring object of study. Chance encounters, or
even the repeated mention of her name, would send his thoughts
back into grooves from which he had resolutely detached them;
whereas, if she could be entirely excluded from his life, the
pressure of new and varied impressions, with which no thought of
her was connected, would soon complete the work of separation. 
Mrs.\ Fisher's conversation had, indeed, operated to that end; but
the treatment was too painful to be voluntarily chosen while
milder remedies were untried; and Selden thought he could trust
himself to return gradually to a reasonable view of Miss Bart, if
only he did not see her.





Having reached the station early, he had arrived at this point in
his reflections before the increasing throng on the platform
warned him that he could not hope to preserve his privacy; the
next moment there was a hand on the door, and he turned to
confront the very face he was fleeing.





Miss Bart, glowing with the haste of a precipitate descent upon
the train, headed a group composed of the Dorsets, young
Silverton and Lord Hubert Dacey, who had barely time to spring
into the carriage, and envelop Selden in ejaculations of surprise
and welcome, before the whistle of departure sounded. The party,
it appeared, were hastening to Nice in response to a sudden
summons to dine with the Duchess of Beltshire and to see the
water-f\^{e}te in the bay; a plan evidently improvised---in spite of
Lord Hubert's protesting ``Oh, I say, you know,''---for the express
purpose of defeating Mrs.\ Bry's endeavour to capture the Duchess.





During the laughing relation of this manoeuvre, Selden had time
for a rapid impression of Miss Bart, who had seated herself
opposite to him in the golden afternoon light. Scarcely three
months had elapsed since he had parted from her on the threshold
of the Brys'\ conservatory; but a subtle change had passed over
the quality of her beauty. Then it had had a transparency through
which the fluctuations of the spirit were sometimes tragically
visible; now its impenetrable surface suggested a process of
crystallization which had fused her whole being into one hard
brilliant substance. The change had struck Mrs.\ Fisher as
a rejuvenation: to Selden it seemed like that moment of pause and
arrest when the warm fluidity of youth is chilled into its final
shape.





He felt it in the way she smiled on him, and in the readiness and
competence with which, flung unexpectedly into his presence, she
took up the thread of their intercourse as though that thread had
not been snapped with a violence from which he still reeled. Such
facility sickened him---but he told himself that it was with the
pang which precedes recovery. Now he would really get well---would
eject the last drop of poison from his blood. Already he felt
himself calmer in her presence than he had learned to be in the
thought of her. Her assumptions and elisions, her short-cuts and
long \textit{detours}, the skill with which she contrived to meet him at a
point from which no inconvenient glimpses of the past were
visible, suggested what opportunities she had had for practising
such arts since their last meeting. He felt that she had at last
arrived at an understanding with herself: had made a pact with
her rebellious impulses, and achieved a uniform system of
self-government, under which all vagrant tendencies were either
held captive or forced into the service of the state.





And he saw other things too in her manner: saw how it had
adjusted itself to the hidden intricacies of a situation in
which, even after Mrs.\ Fisher's elucidating flashes, he still
felt himself agrope. Surely Mrs.\ Fisher could no longer charge
Miss Bart with neglecting her opportunities! To Selden's
exasperated observation she was only too completely alive to
them. She was ``perfect''\ to every one: subservient to Bertha's
anxious predominance, good-naturedly watchful of Dorset's moods,
brightly companionable to Silverton and Dacey, the latter of whom
met her on an evident footing of old admiration, while young
Silverton, portentously self-absorbed, seemed conscious of her
only as of something vaguely obstructive. And suddenly, as Selden
noted the fine shades of manner by which she harmonized herself
with her surroundings, it flashed on him that, to need such
adroit handling, the situation must indeed be desperate. She was
on the edge of something---that was the impression left with him. 
He seemed to see her poised on the brink of a chasm, with one
graceful foot advanced to assert her unconsciousness that
the ground was failing her.





On the Promenade des Anglais, where Ned Silverton hung on him for
the half hour before dinner, he received a deeper impression of
the general insecurity. Silverton was in a mood of Titanic
pessimism. How any one could come to such a damned hole as the
Riviera---any one with a grain of imagination---with the whole
Mediterranean to choose from: but then, if one's estimate of a
place depended on the way they broiled a spring chicken! Gad!
what a study might be made of the tyranny of the stomach---the way
a sluggish liver or insufficient gastric juices might affect the
whole course of the universe, overshadow everything in
reach---chronic dyspepsia ought to be among the ``statutory
causes''; a woman's life might be ruined by a man's inability to
digest fresh bread. Grotesque? Yes---and tragic---like most
absurdities. There's nothing grimmer than the tragedy that wears
a comic mask.... Where was he? Oh---the reason they chucked Sicily
and rushed back? Well---partly, no doubt, Miss Bart's desire to
get back to bridge and smartness. Dead as a stone to art and
poetry---the light never \textit{was} on sea or land for her! And of course
she persuaded Dorset that the Italian food was bad for him. Oh,
she could make him believe anything---\textit{anything}! Mrs.\ Dorset was
aware of it---oh, perfectly: nothing \textit{she} didn't see! But she could
hold her tongue---she'd had to, often enough. Miss Bart was an
intimate friend---she wouldn't hear a word against her. Only it
hurts a woman's pride---there are some things one doesn't get used
to .\ .\ . All this in confidence, of course? Ah---and there were
the ladies signalling from the balcony of the hotel.... He
plunged across the Promenade, leaving Selden to a meditative
cigar.





The conclusions it led him to were fortified, later in the
evening, by some of those faint corroborative hints that generate
a light of their own in the dusk of a doubting mind. Selden,
stumbling on a chance acquaintance, had dined with him, and
adjourned, still in his company, to the brightly lit Promenade,
where a line of crowded stands commanded the glittering darkness
of the waters. The night was soft and persuasive. Overhead hung
a summer sky furrowed with the rush of rockets; and from
the east a late moon, pushing up beyond the lofty bend of the
coast, sent across the bay a shaft of brightness which paled to
ashes in the red glitter of the illuminated boats. Down the
lantern-hung Promenade, snatches of band-music floated above the
hum of the crowd and the soft tossing of boughs in dusky gardens;
and between these gardens and the backs of the stands there
flowed a stream of people in whom the vociferous carnival mood
seemed tempered by the growing languor of the season.





Selden and his companion, unable to get seats on one of the
stands facing the bay, had wandered for a while with the throng,
and then found a point of vantage on a high garden-parapet above
the Promenade. Thence they caught but a triangular glimpse of
the water, and of the flashing play of boats across its surface;
but the crowd in the street was under their immediate view, and
seemed to Selden, on the whole, of more interest than the show
itself. After a while, however, he wearied of his perch and,
dropping alone to the pavement, pushed his way to the first
corner and turned into the moonlit silence of a side street. Long
garden-walls overhung by trees made a dark boundary to the
pavement; an empty cab trailed along the deserted thoroughfare,
and presently Selden saw two persons emerge from the opposite
shadows, signal to the cab, and drive off in it toward the centre
of the town. The moonlight touched them as they paused to enter
the carriage, and he recognized Mrs.\ Dorset and young Silverton.





Beneath the nearest lamp-post he glanced at his watch and saw
that the time was close on eleven. He took another cross street,
and without breasting the throng on the Promenade, made his way
to the fashionable club which overlooks that thoroughfare. Here,
amid the blaze of crowded baccarat tables, he caught sight of
Lord Hubert Dacey, seated with his habitual worn smile behind a
rapidly dwindling heap of gold. The heap being in due course
wiped out, Lord Hubert rose with a shrug, and joining Selden,
adjourned with him to the deserted terrace of the club. It was
now past midnight, and the throng on the stands was dispersing,
while the long trails of red-lit boats scattered and faded
beneath a sky repossessed by the tranquil splendour of the moon.





Lord Hubert looked at his watch. ``By Jove, I promised to
join the Duchess for supper at the \textit{London} \textit{house}; but it's past
twelve, and I suppose they've all scattered. The fact is, I lost
them in the crowd soon after dinner, and took refuge here, for my
sins. They had seats on one of the stands, but of course they
couldn't stop quiet: the Duchess never can. She and Miss Bart
went off in quest of what they call adventures---gad, it ain't
their fault if they don't have some queer ones!''\ He added
tentatively, after pausing to grope for a cigarette: ``Miss Bart's
an old friend of yours, I believe? So she told me.---Ah, thanks---I
don't seem to have one left.'' He lit Selden's proffered
cigarette, and continued, in his high-pitched drawling tone: 
``None of my business, of course, but I didn't introduce her to
the Duchess. Charming woman, the Duchess, you understand; and a
very good friend of mine; but \textit{rather} a liberal education.''





Selden received this in silence, and after a few puffs Lord
Hubert broke out again: ``Sort of thing one can't communicate to
the young lady---though young ladies nowadays are so competent to
judge for themselves; but in this case---I'm an old friend too,
you know .\ .\ .\ and there seemed no one else to speak to. The
whole situation's a little mixed, as I see it---but there used to
be an aunt somewhere, a diffuse and innocent person, who was
great at bridging over chasms she didn't see .\ .\ . Ah, in New
York, is she? Pity New York's such a long way off!''





\chapter*{\raggedright Chapter 2}

\addcontentsline{toc}{chapter}{Chapter 2}

\markboth{HOUSE OF MIRTH}{CHAPTER 2}





Miss Bart, emerging late the next morning from her cabin, found
herself alone on the deck of the Sabrina. The cushioned chairs,
disposed expectantly under the wide awning, showed no signs of
recent occupancy, and she presently learned from a steward that
Mrs.\ Dorset had not yet appeared, and that the
gentlemen---separately---had gone ashore as soon as they had
breakfasted. Supplied with these facts, Lily leaned awhile over
the side, giving herself up to a leisurely enjoyment of the
spectacle before her. Unclouded sunlight enveloped sea and shore
in a bath of purest radiancy. The purpling waters drew a sharp
white line of foam at the base of the shore; against its
irregular eminences, hotels and villas flashed from the greyish
verdure of olive and eucalyptus; and the background of bare and
finely-pencilled mountains quivered in a pale intensity of light.






How beautiful it was---and how she loved beauty! She had always
felt that her sensibility in this direction made up for certain
obtusenesses of feeling of which she was less proud; and during
the last three months she had indulged it passionately. The
Dorsets'\ invitation to go abroad with them had come as an almost
miraculous release from crushing difficulties; and her faculty
for renewing herself in new scenes, and casting off problems of
conduct as easily as the surroundings in which they had arisen,
made the mere change from one place to another seem, not merely a
postponement, but a solution of her troubles. Moral complications
existed for her only in the environment that had produced them;
she did not mean to slight or ignore them, but they lost their
reality when they changed their background. She could not have
remained in New York without repaying the money she owed to
Trenor; to acquit herself of that odious debt she might even have
faced a marriage with Rosedale; but the accident of placing the
Atlantic between herself and her obligations made them dwindle
out of sight as if they had been milestones and she had travelled
past them.





Her two months on the Sabrina had been especially calculated to
aid this illusion of distance. She had been plunged into
new scenes, and had found in them a renewal of old hopes and
ambitions. The cruise itself charmed her as a romantic adventure. 
She was vaguely touched by the names and scenes amid which she
moved, and had listened to Ned Silverton reading Theocritus by
moonlight, as the yacht rounded the Sicilian promontories, with a
thrill of the nerves that confirmed her belief in her
intellectual superiority. But the weeks at Cannes and Nice had
really given her more pleasure. The gratification of being
welcomed in high company, and of making her own ascendency felt
there, so that she found herself figuring once more as the
``beautiful Miss Bart''\ in the interesting journal devoted to
recording the least movements of her cosmopolitan companions---all
these experiences tended to throw into the extreme background of
memory the prosaic and sordid difficulties from which she had
escaped.





If she was faintly aware of fresh difficulties ahead, she was
sure of her ability to meet them: it was characteristic of her to
feel that the only problems she could not solve were those with
which she was familiar. Meanwhile she could honestly be proud of
the skill with which she had adapted herself to somewhat delicate
conditions. She had reason to think that she had made herself
equally necessary to her host and hostess; and if only she had
seen any perfectly irreproachable means of drawing a financial
profit from the situation, there would have been no cloud on her
horizon. The truth was that her funds, as usual, were
inconveniently low; and to neither Dorset nor his wife could this
vulgar embarrassment be safely hinted. Still, the need was not a
pressing one; she could worry along, as she had so often done
before, with the hope of some happy change of fortune to sustain
her; and meanwhile life was gay and beautiful and easy, and she
was conscious of figuring not unworthily in such a setting.





She was engaged to breakfast that morning with the Duchess of
Beltshire, and at twelve o'clock she asked to be set ashore in
the gig. Before this she had sent her maid to enquire if she
might see Mrs.\ Dorset; but the reply came back that the latter
was tired, and trying to sleep. Lily thought she understood the
reason of the rebuff. Her hostess had not been included in the
Duchess's invitation, though she herself had made the
most loyal efforts in that direction. But her grace was
impervious to hints, and invited or omitted as she chose. It was
not Lily's fault if Mrs.\ Dorset's complicated attitudes did not
fall in with the Duchess's easy gait. The Duchess, who seldom
explained herself, had not formulated her objection beyond
saying: ``She's rather a bore, you know. The only one of your
friends I like is that little Mr.\ Bry---\textit{he's} funny---''\ but Lily
knew enough not to press the point, and was not altogether sorry
to be thus distinguished at her friend's expense. Bertha
certainly \textit{had} grown tiresome since she had taken to poetry and
Ned Silverton.





On the whole, it was a relief to break away now and then from the
Sabrina; and the Duchess's little breakfast, organized by Lord
Hubert with all his usual virtuosity, was the pleasanter to Lily
for not including her travelling-companions. Dorset, of late, had
grown more than usually morose and incalculable, and Ned
Silverton went about with an air that seemed to challenge the
universe. The freedom and lightness of the ducal intercourse made
an agreeable change from these complications, and Lily was
tempted, after luncheon, to adjourn in the wake of her companions
to the hectic atmosphere of the Casino. She did not mean to play;
her diminished pocket-money offered small scope for the
adventure; but it amused her to sit on a divan, under the
doubtful protection of the Duchess's back, while the latter hung
above her stakes at a neighbouring table.





The rooms were packed with the gazing throng which, in the
afternoon hours, trickles heavily between the tables, like the
Sunday crowd in a lion-house. In the stagnant flow of the mass,
identities were hardly distinguishable; but Lily presently saw
Mrs.\ Bry cleaving her determined way through the doors, and, in
the broad wake she left, the light figure of Mrs.\ Fisher bobbing
after her like a row-boat at the stern of a tug. Mrs.\ Bry pressed
on, evidently animated by the resolve to reach a certain point in
the rooms; but Mrs.\ Fisher, as she passed Lily, broke from her
towing-line, and let herself float to the girl's side.





``Lose her?''\ she echoed the latter's query, with an indifferent
glance at Mrs.\ Bry's retreating back. ``I daresay---it doesn't
matter: I \textit{have} lost her already.'' And, as Lily exclaimed,
she added: ``We had an awful row this morning. You know, of
course, that the Duchess chucked her at dinner last night, and
she thinks it was my fault---my want of management. The worst of
it is, the message---just a mere word by telephone---came so late
that the dinner \textit{had} to be paid for; and Becassin \textit{had} run it
up---it had been so drummed into him that the Duchess was coming!''
Mrs.\ Fisher indulged in a faint laugh at the remembrance. ``Paying
for what she doesn't get rankles so dreadfully with Louisa: I
can't make her see that it's one of the preliminary steps to
getting what you haven't paid for---and as I was the nearest thing
to smash, she smashed me to atoms, poor dear!''





Lily murmured her commiseration. Impulses of sympathy came
naturally to her, and it was instinctive to proffer her help to
Mrs.\ Fisher.





``If there's anything I can do---if it's only a question of meeting
the Duchess! I heard her say she thought Mr.\ Bry amusing----''





But Mrs.\ Fisher interposed with a decisive gesture. ``My dear, I
have my pride: the pride of my trade. I couldn't manage the
Duchess, and I can't palm off your arts on Louisa Bry as mine. 
I've taken the final step: I go to Paris tonight with the Sam
Gormers. \textit{They're} still in the elementary stage; an Italian Prince
is a great deal more than a Prince to them, and they're always on
the brink of taking a courier for one. To save them from that is
my present mission.'' She laughed again at the picture. ``But
before I go I want to make my last will and testament---I want to
leave you the Brys.''





``Me?''\ Miss Bart joined in her amusement. ``It's charming of you to
remember me, dear; but really----''





``You're already so well provided for?''\ Mrs.\ Fisher flashed a
sharp glance at her. ``\textit{Are} you, though, Lily---to the point of
rejecting my offer?''





Miss Bart coloured slowly. ``What I really meant was, that the
Brys wouldn't in the least care to be so disposed of.''





Mrs.\ Fisher continued to probe her embarrassment with an
unflinching eye. ``What you really meant was that you've snubbed
the Brys horribly; and you know that they know----''





``Carry!''





``Oh, on certain sides Louisa bristles with perceptions. If you'd
even managed to have them asked once on the Sabrina---especially
when royalties were coming! But it's not too late,''\ she ended
earnestly, ``it's not too late for either of you.''





Lily smiled. ``Stay over, and I'll get the Duchess to dine with
them.''





``I shan't stay over---the Gormers have paid for my \textit{Salon}-\textit{lit},''
said Mrs.\ Fisher with simplicity. ``But get the Duchess to dine
with them all the same.''





Lily's smile again flowed into a slight laugh: her friend's
importunity was beginning to strike her as irrelevant. ``I'm sorry
I have been negligent about the Brys----''\ she began.





``Oh, as to the Brys---it's you I'm thinking of,''\ said Mrs.\ Fisher
abruptly. She paused, and then, bending forward, with a lowered
voice: ``You know we all went on to Nice last night when the
Duchess chucked us. It was Louisa's idea---I told her what I
thought of it.''





Miss Bart assented. ``Yes---I caught sight of you on the way back,
at the station.''





``Well, the man who was in the carriage with you and George
Dorset---that horrid little Dabham who does 'Society Notes from
the Riviera'---had been dining with us at Nice. And he's telling
everybody that you and Dorset came back alone after midnight.''





``Alone---? When he was with us?''\ Lily laughed, but her laugh faded
into gravity under the prolonged implication of Mrs.\ Fisher's
look. ``We \textit{did} come back alone---if that's so very dreadful! But
whose fault was it? The Duchess was spending the night at Cimiez
with the Crown Princess; Bertha got bored with the show, and went
off early, promising to meet us at the station. We turned up on
time, but she didn't---she didn't turn up at all!''





Miss Bart made this announcement in the tone of one who presents,
with careless assurance, a complete vindication; but Mrs.\ Fisher
received it in a manner almost inconsequent. She seemed to have
lost sight of her friend's part in the incident: her inward
vision had taken another slant.





``Bertha never turned up at all? Then how on earth did she get
back?''





``Oh, by the next train, I suppose; there were two extra ones for
the \textit{F\^{e}te}. At any rate, I know she's safe on the yacht, though I
haven't yet seen her; but you see it was not my fault,''\ Lily
summed up.





``Not your fault that Bertha didn't turn up? My poor child, if
only you don't have to pay for it!''\ Mrs.\ Fisher rose---she had
seen Mrs.\ Bry surging back in her direction. ``There's Louisa, and
I must be off---oh, we're on the best of terms externally; we're
lunching together; but at heart it's \textit{me} she's lunching on,''\ she
explained; and with a last hand-clasp and a last look, she added: 
``Remember, I leave her to you; she's hovering now, ready to take
you in.''







Lily carried the impression of Mrs.\ Fisher's leave-taking away
with her from the Casino doors. She had accomplished, before
leaving, the first step toward her reinstatement in Mrs.\ Bry's
good graces. An affable advance---a vague murmur that they must
see more of each other---an allusive glance to a near future that
was felt to include the Duchess as well as the Sabrina---how
easily it was all done, if one possessed the knack of doing it! 
She wondered at herself, as she had so often wondered, that,
possessing the knack, she did not more consistently exercise it. 
But sometimes she was forgetful---and sometimes, could it be that
she was proud? Today, at any rate, she had been vaguely conscious
of a reason for sinking her pride, had in fact even sunk it to
the point of suggesting to Lord Hubert Dacey, whom she ran across
on the Casino steps, that he might really get the Duchess to dine
with the Brys, if \textit{she} undertook to have them asked on the
Sabrina. Lord Hubert had promised his help, with the readiness on
which she could always count: it was his only way of ever
reminding her that he had once been ready to do so much more for
her. Her path, in short, seemed to smooth itself before her as
she advanced; yet the faint stir of uneasiness persisted. Had it
been produced, she wondered, by her chance meeting with Selden? 
She thought not---time and change seemed so completely to have
relegated him to his proper distance. The sudden and exquisite
reaction from her anxieties had had the effect of throwing the
recent past so far back that even Selden, as part of it, retained
a certain air of unreality. And he had made it so clear
that they were not to meet again; that he had merely dropped down
to Nice for a day or two, and had almost his foot on the next
steamer. No---that part of the past had merely surged up for a
moment on the fleeing surface of events; and now that it was
submerged again, the uncertainty, the apprehension persisted.





They grew to sudden acuteness as she caught sight of George
Dorset descending the steps of the Hotel de Paris and making for
her across the square. She had meant to drive down to the quay
and regain the yacht; but she now had the immediate impression
that something more was to happen first.





``Which way are you going? Shall we walk a bit?''\ he began, putting
the second question before the first was answered, and not
waiting for a reply to either before he directed her silently
toward the comparative seclusion of the lower gardens.





She detected in him at once all the signs of extreme nervous
tension. The skin was puffed out under his sunken eyes, and its
sallowness had paled to a leaden white against which his
irregular eyebrows and long reddish moustache were relieved with
a saturnine effect. His appearance, in short, presented an odd
mixture of the bedraggled and the ferocious.





He walked beside her in silence, with quick precipitate steps,
till they reached the embowered slopes to the east of the Casino;
then, pulling up abruptly, he said: ``Have you seen Bertha?''





``No---when I left the yacht she was not yet up.''





He received this with a laugh like the whirring sound in a
disabled clock. ``Not yet up? Had she gone to bed? Do you know at
what time she came on board? This morning at seven!''\ he
exclaimed.





``At seven?''\ Lily started. ``What happened---an accident to the
train?''





He laughed again. ``They missed the train---all the trains---they
had to drive back.''





``Well----?''\ She hesitated, feeling at once how little even this
necessity accounted for the fatal lapse of hours.





``Well, they couldn't get a carriage at once---at that time of
night, you know---''\ the explanatory note made it almost
seem as though he were putting the case for his wife---``and when
they finally did, it was only a one-horse cab, and the horse was
lame!''





``How tiresome! I see,''\ she affirmed, with the more earnestness
because she was so nervously conscious that she did not; and
after a pause she added: ``I'm so sorry---but ought we to have
waited?''





``Waited for the one-horse cab? It would scarcely have carried the
four of us, do you think?''





She took this in what seemed the only possible way, with a laugh
intended to sink the question itself in his humorous treatment of
it. ``Well, it would have been difficult; we should have had to
walk by turns. But it would have been jolly to see the sunrise.''





``Yes: the sunrise \textit{was} jolly,''\ he agreed.





``Was it? You saw it, then?''





``I saw it, yes; from the deck. I waited up for them.''





``Naturally---I suppose you were worried. Why didn't you call on me
to share your vigil?''





He stood still, dragging at his moustache with a lean weak hand. 
``I don't think you would have cared for its \textit{denouement},''\ he said
with sudden grimness.





Again she was disconcerted by the abrupt change in his tone, and
as in one flash she saw the peril of the moment, and the need of
keeping her sense of it out of her eyes.





``\textit{Denouement}---isn't that too big a word for such a small incident? 
The worst of it, after all, is the fatigue which Bertha has
probably slept off by this time.''





She clung to the note bravely, though its futility was now plain
to her in the glare of his miserable eyes.





``Don't---don't----!''\ he broke out, with the hurt cry of a child;
and while she tried to merge her sympathy, and her resolve to
ignore any cause for it, in one ambiguous murmur of deprecation,
he dropped down on the bench near which they had paused, and
poured out the wretchedness of his soul.





It was a dreadful hour---an hour from which she emerged shrinking
and seared, as though her lids had been scorched by its actual
glare. It was not that she had never had premonitory glimpses of
such an outbreak; but rather because, here and there
throughout the three months, the surface of life had shown such
ominous cracks and vapours that her fears had always been on the
alert for an upheaval. There had been moments when the situation
had presented itself under a homelier yet more vivid image---that
of a shaky vehicle, dashed by unbroken steeds over a bumping
road, while she cowered within, aware that the harness wanted
mending, and wondering what would give way first. 
Well---everything had given way now; and the wonder was that the
crazy outfit had held together so long. Her sense of being
involved in the crash, instead of merely witnessing it from the
road, was intensified by the way in which Dorset, through his
furies of denunciation and wild reactions of self-contempt, made
her feel the need he had of her, the place she had taken in his
life. But for her, what ear would have been open to his cries? 
And what hand but hers could drag him up again to a footing of
sanity and self-respect? All through the stress of the struggle
with him, she had been conscious of something faintly maternal in
her efforts to guide and uplift him. But for the present, if he
clung to her, it was not in order to be dragged up, but to feel
some one floundering in the depths with him: he wanted her to
suffer with him, not to help him to suffer less.





Happily for both, there was little physical strength to sustain
his frenzy. It left him, collapsed and breathing heavily, to an
apathy so deep and prolonged that Lily almost feared the
passers-by would think it the result of a seizure, and stop to
offer their aid. But Monte Carlo is, of all places, the one where
the human bond is least close, and odd sights are the least
arresting. If a glance or two lingered on the couple, no
intrusive sympathy disturbed them; and it was Lily herself who
broke the silence by rising from her seat. With the clearing of
her vision the sweep of peril had extended, and she saw that the
post of danger was no longer at Dorset's side.





``If you won't go back, I must---don't make me leave you!''\ she
urged.





But he remained mutely resistant, and she added: ``What are you
going to do? You really can't sit here all night.''





``I can go to an hotel. I can telegraph my lawyers.'' He sat up,
roused by a new thought. ``By Jove, Selden's at Nice---I'll send
for Selden!''





Lily, at this, reseated herself with a cry of alarm. ``No, no, \textit{no}!''
she protested.





He swung round on her distrustfully. ``Why not Selden? He's a
lawyer isn't he? One will do as well as another in a case like
this.''





``As badly as another, you mean. I thought you relied on \textit{me} to
help you.''





``You do---by being so sweet and patient with me. If it hadn't been
for you I'd have ended the thing long ago. But now it's got to
end.'' He rose suddenly, straightening himself with an effort. 
``You can't want to see me ridiculous.''





She looked at him kindly. ``That's just it.'' Then, after a
moment's pondering, almost to her own surprise she broke out with
a flash of inspiration: ``Well, go over and see Mr.\ Selden. You'll
have time to do it before dinner.''





``Oh, \textit{dinner}----''\ he mocked her; but she left him with the smiling
rejoinder: ``Dinner on board, remember; we'll put it off till nine
if you like.''





It was past four already; and when a cab had dropped her at the
quay, and she stood waiting for the gig to put off for her, she
began to wonder what had been happening on the yacht. Of
Silverton's whereabouts there had been no mention. Had he
returned to the Sabrina? Or could Bertha---the dread alternative
sprang on her suddenly---could Bertha, left to herself, have gone
ashore to rejoin him? Lily's heart stood still at the thought. 
All her concern had hitherto been for young Silverton, not only
because, in such affairs, the woman's instinct is to side with
the man, but because his case made a peculiar appeal to her
sympathies. He was so desperately in earnest, poor youth, and his
earnestness was of so different a quality from Bertha's, though
hers too was desperate enough. The difference was that Bertha was
in earnest only about herself, while he was in earnest about her. 
But now, at the actual crisis, this difference seemed to throw
the weight of destitution on Bertha's side, since at least he had
her to suffer for, and she had only herself. At any rate, viewed
less ideally, all the disadvantages of such a situation were for
the woman; and it was to Bertha that Lily's sympathies now went
out. She was not fond of Bertha Dorset, but neither was she
without a sense of obligation, the heavier for having so little
personal liking to sustain it. Bertha had been kind to her, they
had lived together, during the last months, on terms of easy
friendship, and the sense of friction of which Lily had recently
become aware seemed to make it the more urgent that she should
work undividedly in her friend's interest.





It was in Bertha's interest, certainly, that she had despatched
Dorset to consult with Lawrence Selden. Once the grotesqueness of
the situation accepted, she had seen at a glance that it was the
safest in which Dorset could find himself. Who but Selden could
thus miraculously combine the skill to save Bertha with the
obligation of doing so? The consciousness that much skill would
be required made Lily rest thankfully in the greatness of the
obligation. Since he would \textit{have} to pull Bertha through she could
trust him to find a way; and she put the fulness of her trust in
the telegram she managed to send him on her way to the quay.





Thus far, then, Lily felt that she had done well; and the
conviction strengthened her for the task that remained. She and
Bertha had never been on confidential terms, but at such a crisis
the barriers of reserve must surely fall: Dorset's wild allusions
to the scene of the morning made Lily feel that they were down
already, and that any attempt to rebuild them would be beyond
Bertha's strength. She pictured the poor creature shivering
behind her fallen defences and awaiting with suspense the moment
when she could take refuge in the first shelter that offered. If
only that shelter had not already offered itself elsewhere! As
the gig traversed the short distance between the quay and the
yacht, Lily grew more than ever alarmed at the possible
consequences of her long absence. What if the wretched Bertha,
finding in all the long hours no soul to turn to---but by this
time Lily's eager foot was on the side-ladder, and her first step
on the Sabrina showed the worst of her apprehensions to be
unfounded; for there, in the luxurious shade of the after-deck,
the wretched Bertha, in full command of her usual attenuated
elegance, sat dispensing tea to the Duchess of Beltshire and Lord
Hubert.





The sight filled Lily with such surprise that she felt that
Bertha, at least, must read its meaning in her look, and she was
proportionately disconcerted by the blankness of the look
returned. But in an instant she saw that Mrs.\ Dorset had, of
necessity, to look blank before the others, and that, to mitigate
the effect of her own surprise, she must at once produce some
simple reason for it. The long habit of rapid transitions made it
easy for her to exclaim to the Duchess: ``Why, I thought you'd
gone back to the Princess!''\ and this sufficed for the lady she
addressed, if it was hardly enough for Lord Hubert.





At least it opened the way to a lively explanation of how the
Duchess was, in fact, going back the next moment, but had first
rushed out to the yacht for a word with Mrs.\ Dorset on the
subject of tomorrow's dinner---the dinner with the Brys, to which
Lord Hubert had finally insisted on dragging them.





``To save my neck, you know!''\ he explained, with a glance that
appealed to Lily for some recognition of his promptness; and the
Duchess added, with her noble candour: ``Mr.\ Bry has promised him
a tip, and he says if we go he'll pass it onto us.''





This led to some final pleasantries, in which, as it seemed to
Lily, Mrs.\ Dorset bore her part with astounding bravery, and at
the close of which Lord Hubert, from half way down the
side-ladder, called back, with an air of numbering heads: ``And of
course we may count on Dorset too?''





``Oh, count on him,''\ his wife assented gaily. She was keeping up
well to the last---but as she turned back from waving her adieux
over the side, Lily said to herself that the mask must drop and
the soul of fear look out.





Mrs.\ Dorset turned back slowly; perhaps she wanted time to steady
her muscles; at any rate, they were still under perfect control
when, dropping once more into her seat behind the tea-table, she
remarked to Miss Bart with a faint touch of irony: ``I suppose I
ought to say good morning.''





If it was a cue, Lily was ready to take it, though with only the
vaguest sense of what was expected of her in return. There was
something unnerving in the contemplation of Mrs.\ Dorset's
composure, and she had to force the light tone in which she
answered: ``I tried to see you this morning, but you were not yet
up.''





``No---I got to bed late. After we missed you at the station
I thought we ought to wait for you till the last train.'' 
She spoke very gently, but with just the least tinge of reproach.





``You missed us? You waited for us at the station?''\ Now indeed
Lily was too far adrift in bewilderment to measure the other's
words or keep watch on her own. ``But I thought you didn't get to
the station till after the last train had left!''





Mrs.\ Dorset, examining her between lowered lids, met this with
the immediate query: ``Who told you that?''





``George---I saw him just now in the gardens.''





``Ah, is that George's version? Poor George---he was in no state to
remember what I told him. He had one of his worst attacks this
morning, and I packed him off to see the doctor. Do you know if
he found him?''





Lily, still lost in conjecture, made no reply, and Mrs.\ Dorset
settled herself indolently in her seat. ``He'll wait to see him;
he was horribly frightened about himself. It's very bad for him
to be worried, and whenever anything upsetting happens, it always
brings on an attack.''





This time Lily felt sure that a cue was being pressed on her; but
it was put forth with such startling suddenness, and with so
incredible an air of ignoring what it led up to, that she could
only falter out doubtfully: ``Anything upsetting?''





``Yes---such as having you so conspicuously on his hands in the
small hours. You know, my dear, you're rather a big
responsibility in such a scandalous place after midnight.''





At that---at the complete unexpectedness and the inconceivable
audacity of it---Lily could not restrain the tribute of an
astonished laugh.





``Well, really---considering it was you who burdened him with the
responsibility!''





Mrs.\ Dorset took this with an exquisite mildness. ``By not having
the superhuman cleverness to discover you in that frightful rush
for the train? Or the imagination to believe that you'd take it
without us---you and he all alone---instead of waiting quietly in
the station till we \textit{did} manage to meet you?''





Lily's colour rose: it was growing clear to her that Bertha was
pursuing an object, following a line she had marked out for
herself. Only, with such a doom impending, why waste time in
these childish efforts to avert it? The puerility of the
attempt disarmed Lily's indignation: did it not prove how
horribly the poor creature was frightened?





``No; by our simply all keeping together at Nice,''\ she returned.





``Keeping together? When it was you who seized the first
opportunity to rush off with the Duchess and her friends? My dear
Lily, you are not a child to be led by the hand!''





``No---nor to be lectured, Bertha, really; if that's what you are
doing to me now.''





Mrs.\ Dorset smiled on her reproachfully. ``Lecture you---I? Heaven
forbid! I was merely trying to give you a friendly hint. But it's
usually the other way round, isn't it? I'm expected to take
hints, not to give them: I've positively lived on them all these
last months.''





``Hints---from me to you?''\ Lily repeated.





``Oh, negative ones merely---what not to be and to do and to see. 
And I think I've taken them to admiration. Only, my dear, if
you'll let me say so, I didn't understand that one of my negative
duties was \textit{not} to warn you when you carried your imprudence too
far.''





A chill of fear passed over Miss Bart: a sense of remembered
treachery that was like the gleam of a knife in the dusk. But
compassion, in a moment, got the better of her instinctive
recoil. What was this outpouring of senseless bitterness but the
tracked creature's attempt to cloud the medium through which it
was fleeing? It was on Lily's lips to exclaim: ``You poor soul,
don't double and turn---come straight back to me, and we'll find a
way out!''\ But the words died under the impenetrable insolence of
Bertha's smile. Lily sat silent, taking the brunt of it quietly,
letting it spend itself on her to the last drop of its
accumulated falseness; then, without a word, she rose and went
down to her cabin.





\chapter*{\raggedright Chapter 3}

\addcontentsline{toc}{chapter}{Chapter 3}

\markboth{HOUSE OF MIRTH}{CHAPTER 3}





Miss Bart's telegram caught Lawrence Selden at the door of his
hotel; and having read it, he turned back to wait for Dorset. The
message necessarily left large gaps for conjecture; but all that
he had recently heard and seen made these but too easy to fill
in. On the whole he was surprised; for though he had perceived
that the situation contained all the elements of an explosion, he
had often enough, in the range of his personal experience, seen
just such combinations subside into harmlessness. Still, Dorset's
spasmodic temper, and his wife's reckless disregard of
appearances, gave the situation a peculiar insecurity; and it was
less from the sense of any special relation to the case than from
a purely professional zeal, that Selden resolved to guide the
pair to safety. Whether, in the present instance, safety for
either lay in repairing so damaged a tie, it was no business of
his to consider: he had only, on general principles, to think of
averting a scandal, and his desire to avert it was increased by
his fear of its involving Miss Bart. There was nothing specific
in this apprehension; he merely wished to spare her the
embarrassment of being ever so remotely connected with the public
washing of the Dorset linen.





How exhaustive and unpleasant such a process would be, he saw
even more vividly after his two hours'\ talk with poor Dorset. If
anything came out at all, it would be such a vast unpacking of
accumulated moral rags as left him, after his visitor had gone,
with the feeling that he must fling open the windows and have his
room swept out. But nothing should come out; and happily for his
side of the case, the dirty rags, however pieced together, could
not, without considerable difficulty, be turned into a
homogeneous grievance. The torn edges did not always fit---there
were missing bits, there were disparities of size and colour, all
of which it was naturally Selden's business to make the most of
in putting them under his client's eye. But to a man in Dorset's
mood the completest demonstration could not carry conviction, and
Selden saw that for the moment all he could do was to soothe and
temporize, to offer sympathy and to counsel prudence. He let
Dorset depart charged to the brim with the sense that, till
their next meeting, he must maintain a strictly noncommittal
attitude; that, in short, his share in the game consisted for the
present in looking on. Selden knew, however, that he could not
long keep such violences in equilibrium; and he promised to meet
Dorset, the next morning, at an hotel in Monte Carlo. Meanwhile
he counted not a little on the reaction of weakness and
self-distrust that, in such natures, follows on every unwonted
expenditure of moral force; and his telegraphic reply to Miss
Bart consisted simply in the injunction: ``Assume that everything
is as usual.''





On this assumption, in fact, the early part of the following day
was lived through. Dorset, as if in obedience to Lily's
imperative bidding, had actually returned in time for a late
dinner on the yacht. The repast had been the most difficult
moment of the day. Dorset was sunk in one of the abysmal silences
which so commonly followed on what his wife called his ``attacks''
that it was easy, before the servants, to refer it to this cause;
but Bertha herself seemed, perversely enough, little disposed to
make use of this obvious means of protection. She simply left the
brunt of the situation on her husband's hands, as if too absorbed
in a grievance of her own to suspect that she might be the object
of one herself. To Lily this attitude was the most ominous,
because the most perplexing, element in the situation. As she
tried to fan the weak flicker of talk, to build up, again and
again, the crumbling structure of ``appearances,''\ her own
attention was perpetually distracted by the question: ``What on
earth can she be driving at?''\ There was something positively
exasperating in Bertha's attitude of isolated defiance. If only
she would have given her friend a hint they might still have
worked together successfully; but how could Lily be of use, while
she was thus obstinately shut out from participation? To be of
use was what she honestly wanted; and not for her own sake but
for the Dorsets'. She had not thought of her own situation at
all: she was simply engrossed in trying to put a little order in
theirs. But the close of the short dreary evening left her with a
sense of effort hopelessly wasted. She had not tried to see
Dorset alone: she had positively shrunk from a renewal of his
confidences. It was Bertha whose confidence she sought, and who
should as eagerly have invited her own; and Bertha, as if in the
infatuation of self-destruction, was actually pushing away her
rescuing hand.





Lily, going to bed early, had left the couple to themselves; and
it seemed part of the general mystery in which she moved that
more than an hour should elapse before she heard Bertha walk down
the silent passage and regain her room. The morrow, rising on an
apparent continuance of the same conditions, revealed nothing of
what had occurred between the confronted pair. One fact alone
outwardly proclaimed the change they were all conspiring to
ignore; and that was the non-appearance of Ned Silverton. No one
referred to it, and this tacit avoidance of the subject kept it
in the immediate foreground of consciousness. But there was
another change, perceptible only to Lily; and that was that
Dorset now avoided her almost as pointedly as his wife. Perhaps
he was repenting his rash outpourings of the previous day;
perhaps only trying, in his clumsy way, to conform to Selden's
counsel to behave ``as usual.'' Such instructions no more make for
easiness of attitude than the photographer's behest to ``look
natural''; and in a creature as unconscious as poor Dorset of the
appearance he habitually presented, the struggle to maintain a
pose was sure to result in queer contortions.





It resulted, at any rate, in throwing Lily strangely on her own
resources. She had learned, on leaving her room, that Mrs.\ Dorset
was still invisible, and that Dorset had left the yacht early;
and feeling too restless to remain alone, she too had herself
ferried ashore. Straying toward the Casino, she attached herself
to a group of acquaintances from Nice, with whom she lunched, and
in whose company she was returning to the rooms when she
encountered Selden crossing the square. She could not, at the
moment, separate herself definitely from her party, who had
hospitably assumed that she would remain with them till they took
their departure; but she found time for a momentary pause of
enquiry, to which he promptly returned: ``I've seen him
again---he's just left me.''





She waited before him anxiously. ``Well?\ what has happened? What
\textit{will} happen?''





``Nothing as yet---and nothing in the future, I think.''





``It's over, then? It's settled? You're sure?''





He smiled. ``Give me time. I'm not sure---but I'm a good deal
surer.'' And with that she had to content herself, and hasten on
to the expectant group on the steps.





Selden had in fact given her the utmost measure of his sureness,
had even stretched it a shade to meet the anxiety in her eyes. 
And now, as he turned away, strolling down the hill toward the
station, that anxiety remained with him as the visible
justification of his own. It was not, indeed, anything specific
that he feared: there had been a literal truth in his declaration
that he did not think anything would happen. What troubled him
was that, though Dorset's attitude had perceptibly changed, the
change was not clearly to be accounted for. It had certainly not
been produced by Selden's arguments, or by the action of his own
soberer reason. Five minutes'\ talk sufficed to show that some
alien influence had been at work, and that it had not so much
subdued his resentment as weakened his will, so that he moved
under it in a state of apathy, like a dangerous lunatic who has
been drugged. Temporarily, no doubt, however exerted, it worked
for the general safety: the question was how long it would last,
and by what kind of reaction it was likely to be followed. On
these points Selden could gain no light; for he saw that one
effect of the transformation had been to shut him off from free
communion with Dorset. The latter, indeed, was still moved by the
irresistible desire to discuss his wrong; but, though he revolved
about it with the same forlorn tenacity, Selden was aware that
something always restrained him from full expression. His state
was one to produce first weariness and then impatience in his
hearer; and when their talk was over, Selden began to feel that
he had done his utmost, and might justifiably wash his hands of
the sequel.





It was in this mind that he had been making his way back to the
station when Miss Bart crossed his path; but though, after his
brief word with her, he kept mechanically on his course, he was
conscious of a gradual change in his purpose. The change had been
produced by the look in her eyes; and in his eagerness to define
the nature of that look, he dropped into a seat in the gardens,
and sat brooding upon the question. It was natural enough, in all
conscience, that she should appear anxious: a young woman
placed, in the close intimacy of a yachting-cruise, between a
couple on the verge of disaster, could hardly, aside from her
concern for her friends, be insensible to the awkwardness of her
own position. The worst of it was that, in interpreting Miss
Bart's state of mind, so many alternative readings were possible;
and one of these, in Selden's troubled mind, took the ugly form
suggested by Mrs.\ Fisher. If the girl was afraid, was she afraid
for herself or for her friends? And to what degree was her dread
of a catastrophe intensified by the sense of being fatally
involved in it? The burden of offence lying manifestly with Mrs.
Dorset, this conjecture seemed on the face of it gratuitously
unkind; but Selden knew that in the most one-sided matrimonial
quarrel there are generally counter-charges to be brought, and
that they are brought with the greater audacity where the
original grievance is so emphatic. Mrs.\ Fisher had not hesitated
to suggest the likelihood of Dorset's marrying Miss Bart if
``anything happened''; and though Mrs.\ Fisher's conclusions were
notoriously rash, she was shrewd enough in reading the signs from
which they were drawn. Dorset had apparently shown marked
interest in the girl, and this interest might be used to cruel
advantage in his wife's struggle for rehabilitation. Selden knew
that Bertha would fight to the last round of powder: the rashness
of her conduct was illogically combined with a cold determination
to escape its consequences. She could be as unscrupulous in
fighting for herself as she was reckless in courting danger, and
whatever came to her hand at such moments was likely to be used
as a defensive missile. He did not, as yet, see clearly just what
course she was likely to take, but his perplexity increased his
apprehension, and with it the sense that, before leaving, he must
speak again with Miss Bart. Whatever her share in the
situation---and he had always honestly tried to resist judging her
by her surroundings---however free she might be from any personal
connection with it, she would be better out of the way of a
possible crash; and since she had appealed to him for help, it
was clearly his business to tell her so.





This decision at last brought him to his feet, and carried him
back to the gambling rooms, within whose doors he had seen her
disappearing; but a prolonged exploration of the crowd
failed to put him on her traces. He saw instead, to his surprise,
Ned Silverton loitering somewhat ostentatiously about the tables;
and the discovery that this actor in the drama was not only
hovering in the wings, but actually inviting the exposure of the
footlights, though it might have seemed to imply that all peril
was over, served rather to deepen Selden's sense of foreboding. 
Charged with this impression he returned to the square, hoping to
see Miss Bart move across it, as every one in Monte Carlo seemed
inevitably to do at least a dozen times a day; but here again he
waited vainly for a glimpse of her, and the conclusion was slowly
forced on him that she had gone back to the Sabrina. It would be
difficult to follow her there, and still more difficult, should
he do so, to contrive the opportunity for a private word; and he
had almost decided on the unsatisfactory alternative of writing,
when the ceaseless diorama of the square suddenly unrolled before
him the figures of Lord Hubert and Mrs.\ Bry.





Hailing them at once with his question, he learned from Lord
Hubert that Miss Bart had just returned to the Sabrina in
Dorset's company; an announcement so evidently disconcerting to
him that Mrs.\ Bry, after a glance from her companion, which
seemed to act like the pressure on a spring, brought forth the
prompt proposal that he should come and meet his friends at
dinner that evening---``At Becassin's---a little dinner to the
Duchess,''\ she flashed out before Lord Hubert had time to remove
the pressure.





Selden's sense of the privilege of being included in such company
brought him early in the evening to the door of the restaurant,
where he paused to scan the ranks of diners approaching down the
brightly lit terrace. There, while the Brys hovered within over
the last agitating alternatives of the \textit{menu}, he kept watch for
the guests from the Sabrina, who at length rose on the horizon in
company with the Duchess, Lord and Lady Skiddaw and the Stepneys. 
From this group it was easy for him to detach Miss Bart on the
pretext of a moment's glance into one of the brilliant shops
along the terrace, and to say to her, while they lingered
together in the white dazzle of a jeweller's window: ``I stopped
over to see you---to beg of you to leave the yacht.''





The eyes she turned on him showed a quick gleam of her former
fear. ``To leave---? What do you mean? What has happened?''





``Nothing. But if anything should, why be in the way of it?''





The glare from the jeweller's window, deepening the pallour of
her face, gave to its delicate lines the sharpness of a tragic
mask. ``Nothing will, I am sure; but while there's even a doubt
left, how can you think I would leave Bertha?''





The words rang out on a note of contempt---was it possibly of
contempt for himself? Well, he was willing to risk its renewal to
the extent of insisting, with an undeniable throb of added
interest: ``You have yourself to think of, you know---''\ to which,
with a strange fall of sadness in her voice, she answered,
meeting his eyes: ``If you knew how little difference that makes!''





``Oh, well, nothing \textit{will} happen,''\ he said, more for his own
reassurance than for hers; and ``Nothing, nothing, of course!''\ she
valiantly assented, as they turned to overtake their companions.





In the thronged restaurant, taking their places about Mrs.\ Bry's
illuminated board, their confidence seemed to gain support from
the familiarity of their surroundings. Here were Dorset and his
wife once more presenting their customary faces to the world, she
engrossed in establishing her relation with an intensely new
gown, he shrinking with dyspeptic dread from the multiplied
solicitations of the \textit{menu}. The mere fact that they thus showed
themselves together, with the utmost openness the place afforded,
seemed to declare beyond a doubt that their differences were
composed. How this end had been attained was still matter for
wonder, but it was clear that for the moment Miss Bart rested
confidently in the result; and Selden tried to achieve the same
view by telling himself that her opportunities for observation
had been ampler than his own.





Meanwhile, as the dinner advanced through a labyrinth of courses,
in which it became clear that Mrs.\ Bry had occasionally broken
away from Lord Hubert's restraining hand, Selden's general
watchfulness began to lose itself in a particular study of Miss
Bart. It was one of the days when she was so handsome
that to be handsome was enough, and all the rest---her grace, her
quickness, her social felicities---seemed the overflow of a
bounteous nature. But what especially struck him was the way in
which she detached herself, by a hundred undefinable shades, from
the persons who most abounded in her own style. It was in just
such company, the fine flower and complete expression of the
state she aspired to, that the differences came out with special
poignancy, her grace cheapening the other women's smartness as
her finely-discriminated silences made their chatter dull. The
strain of the last hours had restored to her face the deeper
eloquence which Selden had lately missed in it, and the bravery
of her words to him still fluttered in her voice and eyes. Yes,
she was matchless---it was the one word for her; and he could give
his admiration the freer play because so little personal feeling
remained in it. His real detachment from her had taken place, not
at the lurid moment of disenchantment, but now, in the sober
after-light of discrimination, where he saw her definitely
divided from him by the crudeness of a choice which seemed to
deny the very differences he felt in her. It was before him again
in its completeness---the choice in which she was content to rest: 
in the stupid costliness of the food and the showy dulness of the
talk, in the freedom of speech which never arrived at wit and the
freedom of act which never made for romance. The strident setting
of the restaurant, in which their table seemed set apart in a
special glare of publicity, and the presence at it of little
Dabham of the ``Riviera Notes,''\ emphasized the ideals of a world
where conspicuousness passed for distinction, and the society
column had become the roll of fame.





It was as the immortalizer of such occasions that little Dabham,
wedged in modest watchfulness between two brilliant neighbours,
suddenly became the centre of Selden's scrutiny. How much did he
know of what was going on, and how much, for his purpose, was
still worth finding out? His little eyes were like tentacles
thrown out to catch the floating intimations with which, to
Selden, the air at moments seemed thick; then again it cleared to
its normal emptiness, and he could see nothing in it for the
journalist but leisure to note the elegance of the ladies'\ gowns. 
Mrs.\ Dorset's, in particular, challenged all the wealth
of Mr.\ Dabham's vocabulary: it had surprises and subtleties
worthy of what he would have called ``the literary style.'' At
first, as Selden had noticed, it had been almost too preoccupying
to its wearer; but now she was in full command of it, and was
even producing her effects with unwonted freedom. Was she not,
indeed, too free, too fluent, for perfect naturalness? And was
not Dorset, to whom his glance had passed by a natural
transition, too jerkily wavering between the same extremes? 
Dorset indeed was always jerky; but it seemed to Selden that
tonight each vibration swung him farther from his centre.





The dinner, meanwhile, was moving to its triumphant close, to the
evident satisfaction of Mrs.\ Bry, who, throned in apoplectic
majesty between Lord Skiddaw and Lord Hubert, seemed in spirit to
be calling on Mrs.\ Fisher to witness her achievement. Short of
Mrs.\ Fisher her audience might have been called complete; for the
restaurant was crowded with persons mainly gathered there for the
purpose of spectatorship, and accurately posted as to the names
and faces of the celebrities they had come to see. Mrs.\ Bry,
conscious that all her feminine guests came under that heading,
and that each one looked her part to admiration, shone on Lily
with all the pent-up gratitude that Mrs.\ Fisher had failed to
deserve. Selden, catching the glance, wondered what part Miss
Bart had played in organizing the entertainment. She did, at
least, a great deal to adorn it; and as he watched the bright
security with which she bore herself, he smiled to think that he
should have fancied her in need of help. Never had she appeared
more serenely mistress of the situation than when, at the moment
of dispersal, detaching herself a little from the group about the
table, she turned with a smile and a graceful slant of the
shoulders to receive her cloak from Dorset.





The dinner had been protracted over Mr.\ Bry's exceptional cigars
and a bewildering array of liqueurs, and many of the other tables
were empty; but a sufficient number of diners still lingered to
give relief to the leave-taking of Mrs.\ Bry's distinguished
guests. This ceremony was drawn out and complicated by the fact
that it involved, on the part of the Duchess and Lady Skiddaw,
definite farewells, and pledges of speedy reunion in Paris, where
they were to pause and replenish their wardrobes on the way to
England. The quality of Mrs.\ Bry's hospitality, and of the tips
her husband had presumably imparted, lent to the manner of the
English ladies a general effusiveness which shed the rosiest
light over their hostess's future. In its glow Mrs.\ Dorset and
the Stepneys were also visibly included, and the whole scene had
touches of intimacy worth their weight in gold to the watchful
pen of Mr.\ Dabham.





A glance at her watch caused the Duchess to exclaim to her sister
that they had just time to dash for their train, and the flurry
of this departure over, the Stepneys, who had their motor at the
door, offered to convey the Dorsets and Miss Bart to the quay. 
The offer was accepted, and Mrs.\ Dorset moved away with her
husband in attendance. Miss Bart had lingered for a last word
with Lord Hubert, and Stepney, on whom Mr.\ Bry was pressing a
final, and still more expensive, cigar, called out: ``Come on,
Lily, if you're going back to the yacht.''





Lily turned to obey; but as she did so, Mrs.\ Dorset, who had
paused on her way out, moved a few steps back toward the table.





``Miss Bart is not going back to the yacht,''\ she said in a voice
of singular distinctness.





A startled look ran from eye to eye; Mrs.\ Bry crimsoned to the
verge of congestion, Mrs.\ Stepney slipped nervously behind her
husband, and Selden, in the general turmoil of his sensations,
was mainly conscious of a longing to grip Dabham by the collar
and fling him out into the street.





Dorset, meanwhile, had stepped back to his wife's side. His face
was white, and he looked about him with cowed angry eyes. 
``Bertha!---Miss Bart .\ .\ .\ this is some misunderstanding .\ .\ .
some mistake .\ .\ .''





``Miss Bart remains here,''\ his wife rejoined incisively. ``And, I
think, George, we had better not detain Mrs.\ Stepney any longer.''





Miss Bart, during this brief exchange of words, remained in
admirable erectness, slightly isolated from the embarrassed group
about her. She had paled a little under the shock of the insult,
but the discomposure of the surrounding faces was not reflected
in her own. The faint disdain of her smile seemed to lift
her high above her antagonist's reach, and it was not till she
had given Mrs.\ Dorset the full measure of the distance between
them that she turned and extended her hand to her hostess.





``I am joining the Duchess tomorrow,''\ she explained, ``and it
seemed easier for me to remain on shore for the night.''





She held firmly to Mrs.\ Bry's wavering eye while she gave this
explanation, but when it was over Selden saw her send a tentative
glance from one to another of the women's faces. She read their
incredulity in their averted looks, and in the mute wretchedness
of the men behind them, and for a miserable half-second he
thought she quivered on the brink of failure. Then, turning to
him with an easy gesture, and the pale bravery of her recovered
smile---``Dear Mr.\ Selden,''\ she said, ``you promised to see me to my
cab.''







Outside, the sky was gusty and overcast, and as Lily and Selden
moved toward the deserted gardens below the restaurant, spurts of
warm rain blew fitfully against their faces. The fiction of the
cab had been tacitly abandoned; they walked on in silence, her
hand on his arm, till the deeper shade of the gardens received
them, and pausing beside a bench, he said: ``Sit down a moment.''





She dropped to the seat without answering, but the electric lamp
at the bend of the path shed a gleam on the struggling misery of
her face. Selden sat down beside her, waiting for her to speak,
fearful lest any word he chose should touch too roughly on her
wound, and kept also from free utterance by the wretched doubt
which had slowly renewed itself within him. What had brought her
to this pass? What weakness had placed her so abominably at her
enemy's mercy? And why should Bertha Dorset have turned into an
enemy at the very moment when she so obviously needed the support
of her sex? Even while his nerves raged at the subjection of husbands
to their wives, and at the cruelty of women to their kind,
reason obstinately harped on the proverbial relation between
smoke and fire. The memory of Mrs.\ Fisher's hints, and the
corroboration of his own impressions, while they deepened his pity
also increased his constraint, since, whichever way he sought a free
outlet for sympathy, it was blocked by the fear of committing a blunder.





Suddenly it struck him that his silence must seem almost as
accusatory as that of the men he had despised for turning from
her; but before he could find the fitting word she had cut him
short with a question.





``Do you know of a quiet hotel? I can send for my maid in the
morning.''





``An hotel---\textit{here}---that you can go to alone? It's not possible.''





She met this with a pale gleam of her old playfulness. ``What \textit{is},
then? It's too wet to sleep in the gardens.''





``But there must be some one----''





``Some one to whom I can go? Of course---any number---but at \textit{this}
hour? You see my change of plan was rather sudden----''





``Good God---if you'd listened to me!''\ he cried, venting his
helplessness in a burst of anger.





She still held him off with the gentle mockery of her smile. ``But
haven't I?''\ she rejoined. ``You advised me to leave the yacht, and
I'm leaving it.''





He saw then, with a pang of self-reproach, that she meant neither
to explain nor to defend herself; that by his miserable silence
he had forfeited all chance of helping her, and that the decisive
hour was past.





She had risen, and stood before him in a kind of clouded majesty,
like some deposed princess moving tranquilly to exile.





``Lily!''\ he exclaimed, with a note of despairing appeal; but---``Oh,
not now,''\ she gently admonished him; and then, in all the
sweetness of her recovered composure: ``Since I must find shelter
somewhere, and since you're so kindly here to help me----''





He gathered himself up at the challenge. ``You will do as I tell
you? There's but one thing, then; you must go straight to your
cousins, the Stepneys.''





``Oh---''\ broke from her with a movement of instinctive resistance;
but he insisted: ``Come---it's late, and you must appear to have
gone there directly.''





He had drawn her hand into his arm, but she held him back with a
last gesture of protest. ``I can't---I can't---not that---you don't
know Gwen: you mustn't ask me!''





``I \textit{must} ask you---you must obey me,''\ he persisted, though infected
at heart by her own fear.





Her voice sank to a whisper: ``And if she refuses?''---but, ``Oh,
trust me---trust me!''\ he could only insist in return; and yielding
to his touch, she let him lead her back in silence to the edge of
the square.





In the cab they continued to remain silent through the brief
drive which carried them to the illuminated portals of the
Stepneys'\ hotel. Here he left her outside, in the darkness of the
raised hood, while his name was sent up to Stepney, and he paced
the showy hall, awaiting the latter's descent. Ten minutes later
the two men passed out together between the gold-laced custodians
of the threshold; but in the vestibule Stepney drew up with a
last flare of reluctance.





``It's understood, then?''\ he stipulated nervously, with his hand
on Selden's arm. ``She leaves tomorrow by the early train---and my
wife's asleep, and can't be disturbed.''





\chapter*{\raggedright Chapter 4}

\addcontentsline{toc}{chapter}{Chapter 4}

\markboth{HOUSE OF MIRTH}{CHAPTER 4}





The blinds of Mrs.\ Peniston's drawing-room were drawn down
against the oppressive June sun, and in the sultry twilight the
faces of her assembled relatives took on a fitting shadow of
bereavement. They were all there: Van Alstynes, Stepneys and
Melsons---even a stray Peniston or two, indicating, by a greater
latitude in dress and manner, the fact of remoter relationship
and more settled hopes. The Peniston side was, in fact, secure in
the knowledge that the bulk of Mr.\ Peniston's property ``went
back''; while the direct connection hung suspended on the disposal
of his widow's private fortune and on the uncertainty of its
extent. Jack Stepney, in his new character as the richest nephew,
tacitly took the lead, emphasizing his importance by the deeper
gloss of his mourning and the subdued authority of his manner;
while his wife's bored attitude and frivolous gown proclaimed the
heiress's disregard of the insignificant interests at stake. Old
Ned Van Alstyne, seated next to her in a coat that made
affliction dapper, twirled his white moustache to conceal the
eager twitch of his lips; and Grace Stepney, red-nosed and
smelling of crape, whispered emotionally to Mrs.\ Herbert Melson: 
``I couldn't \textit{bear} to see the Niagara anywhere else!''





A rustle of weeds and quick turning of heads hailed the opening
of the door, and Lily Bart appeared, tall and noble in her black
dress, with Gerty Farish at her side. The women's faces, as she
paused interrogatively on the threshold, were a study in
hesitation. One or two made faint motions of recognition, which
might have been subdued either by the solemnity of the scene, or
by the doubt as to how far the others meant to go; Mrs.\ Jack
Stepney gave a careless nod, and Grace Stepney, with a sepulchral
gesture, indicated a seat at her side. But Lily, ignoring the
invitation, as well as Jack Stepney's official attempt to direct
her, moved across the room with her smooth free gait, and seated
herself in a chair which seemed to have been purposely placed
apart from the others.





It was the first time that she had faced her family since her
return from Europe, two weeks earlier; but if she perceived
any uncertainty in their welcome, it served only to add a tinge
of irony to the usual composure of her bearing. The shock of
dismay with which, on the dock, she had heard from Gerty Farish
of Mrs.\ Peniston's sudden death, had been mitigated, almost at
once, by the irrepressible thought that now, at last, she would
be able to pay her debts. She had looked forward with
considerable uneasiness to her first encounter with her aunt. 
Mrs.\ Peniston had vehemently opposed her niece's departure with
the Dorsets, and had marked her continued disapproval by not
writing during Lily's absence. The certainty that she had heard
of the rupture with the Dorsets made the prospect of the meeting
more formidable; and how should Lily have repressed a quick sense
of relief at the thought that, instead of undergoing the
anticipated ordeal, she had only to enter gracefully on a
long-assured inheritance? It had been, in the consecrated phrase,
``always understood''\ that Mrs.\ Peniston was to provide handsomely
for her niece; and in the latter's mind the understanding had
long since crystallized into fact.





``She gets everything, of course---I don't see what we're here
for,''\ Mrs.\ Jack Stepney remarked with careless loudness to Ned
Van Alstyne; and the latter's deprecating murmur---``Julia was
always a just woman''---might have been interpreted as signifying
either acquiescence or doubt.





``Well, it's only about four hundred thousand,''\ Mrs.\ Stepney
rejoined with a yawn; and Grace Stepney, in the silence produced
by the lawyer's preliminary cough, was heard to sob out: ``They
won't find a towel missing---I went over them with her the very
day----''





Lily, oppressed by the close atmosphere, and the stifling odour
of fresh mourning, felt her attention straying as Mrs.\ Peniston's
lawyer, solemnly erect behind the Buhl table at the end of the
room, began to rattle through the preamble of the will.





``It's like being in church,''\ she reflected, wondering vaguely
where Gwen Stepney had got such an awful hat. Then she noticed
how stout Jack had grown---he would soon be almost as plethoric as
Herbert Melson, who sat a few feet off, breathing puffily as he
leaned his black-gloved hands on his stick.





``I wonder why rich people always grow fat---I suppose it's because
there's nothing to worry them. If I inherit, I shall have to be
careful of my figure,''\ she mused, while the lawyer droned on
through a labyrinth of legacies. The servants came first, then a
few charitable institutions, then several remoter Melsons and
Stepneys, who stirred consciously as their names rang out, and
then subsided into a state of impassiveness befitting the
solemnity of the occasion. Ned Van Alstyne, Jack Stepney, and a
cousin or two followed, each coupled with the mention of a few
thousands: Lily wondered that Grace Stepney was not among them. 
Then she heard her own name---``to my niece Lily Bart ten thousand
dollars---''\ and after that the lawyer again lost himself in a coil
of unintelligible periods, from which the concluding phrase
flashed out with startling distinctness: ``and the residue of my
estate to my dear cousin and name-sake, Grace Julia Stepney.''





There was a subdued gasp of surprise, a rapid turning of heads,
and a surging of sable figures toward the corner in which Miss
Stepney wailed out her sense of unworthiness through the crumpled
ball of a black-edged handkerchief.





Lily stood apart from the general movement, feeling herself for
the first time utterly alone. No one looked at her, no one seemed
aware of her presence; she was probing the very depths of
insignificance. And under her sense of the collective
indifference came the acuter pang of hopes deceived. 
Disinherited---she had been disinherited---and for Grace Stepney! 
She met Gerty's lamentable eyes, fixed on her in a despairing
effort at consolation, and the look brought her to herself. There
was something to be done before she left the house: to be done
with all the nobility she knew how to put into such gestures. She
advanced to the group about Miss Stepney, and holding out her
hand said simply: ``Dear Grace, I am so glad.''





The other ladies had fallen back at her approach, and a space
created itself about her. It widened as she turned to go, and no
one advanced to fill it up. She paused a moment, glancing about
her, calmly taking the measure of her situation. She heard some
one ask a question about the date of the will; she caught a
fragment of the lawyer's answer---something about a sudden
summons, and an ``earlier instrument.'' Then the tide of dispersal
began to drift past her; Mrs.\ Jack Stepney and Mrs.\ Herbert Melson
stood on the doorstep awaiting their motor; a sympathizing group
escorted Grace Stepney to the cab it was felt to be fitting she
should take, though she lived but a street or two away; and
Miss Bart and Gerty found themselves almost alone in the purple
drawing-room, which more than ever, in its stuffy dimness,
resembled a well-kept family vault, in which the last corpse had
just been decently deposited.







In Gerty Farish's sitting-room, whither a hansom had carried the
two friends, Lily dropped into a chair with a faint sound of
laughter: it struck her as a humorous coincidence that her aunt's
legacy should so nearly represent the amount of her debt to
Trenor. The need of discharging that debt had reasserted itself
with increased urgency since her return to America, and she spoke
her first thought in saying to the anxiously hovering Gerty: ``I
wonder when the legacies will be paid.''





But Miss Farish could not pause over the legacies; she broke into
a larger indignation. ``Oh, Lily, it's unjust; it's cruel---Grace
Stepney must \textit{feel} she has no right to all that money!''





``Any one who knew how to please Aunt Julia has a right to her
money,''\ Miss Bart rejoined philosophically.





``But she was devoted to you---she led every one to think---''\ Gerty
checked herself in evident embarrassment, and Miss Bart turned to
her with a direct look. ``Gerty, be honest: this will was made
only six weeks ago. She had heard of my break with the Dorsets?''





``Every one heard, of course, that there had been some
disagreement---some misunderstanding----''





``Did she hear that Bertha turned me off the yacht?''





``Lily!''





``That was what happened, you know. She said I was trying to marry
George Dorset. She did it to make him think she was jealous. 
Isn't that what she told Gwen Stepney?''





``I don't know---I don't listen to such horrors.''





``I \textit{must} listen to them---I must know where I stand.'' She paused,
and again sounded a faint note of derision. ``Did you
notice the women? They were afraid to snub me while they thought
I was going to get the money---afterward they scuttled off as if I
had the plague.'' Gerty remained silent, and she continued: ``I
stayed on to see what would happen. They took their cue from Gwen
Stepney and Lulu Melson---I saw them watching to see what Gwen
would do.---Gerty, I must know just what is being said of me.''





``I tell you I don't listen----''





``One hears such things without listening.'' She rose and laid her
resolute hands on Miss Farish's shoulders. ``Gerty, are people
going to cut me?''





``Your \textit{friends}, Lily---how can you think it?''





``Who are one's friends at such a time? Who, but you, you poor
trustful darling? And heaven knows what \textit{you} suspect me of!''\ She
kissed Gerty with a whimsical murmur. ``You'd never let it make
any difference---but then you're fond of criminals, Gerty! How
about the irreclaimable ones, though? For I'm absolutely
impenitent, you know.''





She drew herself up to the full height of her slender majesty,
towering like some dark angel of defiance above the troubled
Gerty, who could only falter out: ``Lily, Lily---how can you laugh
about such things?''





``So as not to weep, perhaps. But no---I'm not of the tearful
order. I discovered early that crying makes my nose red, and the
knowledge has helped me through several painful episodes.'' She
took a restless turn about the room, and then, reseating herself,
lifted the bright mockery of her eyes to Gerty's anxious
countenance.





``I shouldn't have minded, you know, if I'd got the money---''\ and
at Miss Farish's protesting ``Oh!''\ she repeated calmly: ``Not a
straw, my dear; for, in the first place, they wouldn't have quite
dared to ignore me; and if they had, it wouldn't have mattered,
because I should have been independent of them. But now---!''\ The
irony faded from her eyes, and she bent a clouded face upon her
friend.





``How can you talk so, Lily? Of course the money ought to have
been yours, but after all that makes no difference. The important
thing----''\ Gerty paused, and then continued firmly: ``The important
thing is that you should clear yourself---should tell your friends
the whole truth.''





``The whole truth?''\ Miss Bart laughed. ``What is truth? Where a
woman is concerned, it's the story that's easiest to believe. In
this case it's a great deal easier to believe Bertha Dorset's
story than mine, because she has a big house and an opera box,
and it's convenient to be on good terms with her.''





Miss Farish still fixed her with an anxious gaze. ``But what \textit{is}
your story, Lily? I don't believe any one knows it yet.''





``My story?---I don't believe I know it myself. You see I never
thought of preparing a version in advance as Bertha did---and if I
had, I don't think I should take the trouble to use it now.''





But Gerty continued with her quiet reasonableness: ``I don't want
a version prepared in advance---but I want you to tell me exactly
what happened from the beginning.''





``From the beginning?''\ Miss Bart gently mimicked her. ``Dear Gerty,
how little imagination you good people have! Why, the beginning
was in my cradle, I suppose---in the way I was brought up, and the
things I was taught to care for. Or no---I won't blame anybody for
my faults: I'll say it was in my blood, that I got it from some
wicked pleasure-loving ancestress, who reacted against the homely
virtues of New Amsterdam, and wanted to be back at the court of
the Charleses!''\ And as Miss Farish continued to press her with
troubled eyes, she went on impatiently: ``You asked me just now
for the truth---well, the truth about any girl is that once she's
talked about she's done for; and the more she explains her case
the worse it looks.---My good Gerty, you don't happen to have a
cigarette about you?''







In her stuffy room at the hotel to which she had gone on landing,
Lily Bart that evening reviewed her situation. It was the last
week in June, and none of her friends were in town. The few
relatives who had stayed on, or returned, for the reading of Mrs.
Peniston's will, had taken flight again that afternoon to Newport
or Long Island; and not one of them had made any proffer of
hospitality to Lily. For the first time in her life she found
herself utterly alone except for Gerty Farish. Even at the actual
moment of her break with the Dorsets she had not had so keen a
sense of its consequences, for the Duchess of Beltshire,
hearing of the catastrophe from Lord Hubert, had instantly
offered her protection, and under her sheltering wing Lily had
made an almost triumphant progress to London. There she had been
sorely tempted to linger on in a society which asked of her only
to amuse and charm it, without enquiring too curiously how she
had acquired her gift for doing so; but Selden, before they
parted, had pressed on her the urgent need of returning at once
to her aunt, and Lord Hubert, when he presently reappeared in
London, abounded in the same counsel. Lily did not need to be
told that the Duchess's championship was not the best road to
social rehabilitation, and as she was besides aware that her
noble defender might at any moment drop her in favour of a new
PROTEGEE, she reluctantly decided to return to America. But she
had not been ten minutes on her native shore before she realized
that she had delayed too long to regain it. The Dorsets, the
Stepneys, the Brys---all the actors and witnesses in the miserable
drama---had preceded her with their version of the case; and, even
had she seen the least chance of gaining a hearing for her own,
some obscure disdain and reluctance would have restrained her. 
She knew it was not by explanations and counter-charges that she
could ever hope to recover her lost standing; but even had she
felt the least trust in their efficacy, she would still have been
held back by the feeling which had kept her from defending
herself to Gerty Farish---a feeling that was half pride and half
humiliation. For though she knew she had been ruthlessly
sacrificed to Bertha Dorset's determination to win back her
husband, and though her own relation to Dorset had been that of
the merest good-fellowship, yet she had been perfectly aware from
the outset that her part in the affair was, as Carry Fisher
brutally put it, to distract Dorset's attention from his wife. 
That was what she was ``there for'': it was the price she had
chosen to pay for three months of luxury and freedom from care. 
Her habit of resolutely facing the facts, in her rare moments of
introspection, did not now allow her to put any false gloss on
the situation. She had suffered for the very faithfulness with
which she had carried out her part of the tacit compact, but the
part was not a handsome one at best, and she saw it now in all
the ugliness of failure.





She saw, too, in the same uncompromising light, the train of
consequences resulting from that failure; and these became
clearer to her with every day of her weary lingering in town. She
stayed on partly for the comfort of Gerty Farish's nearness, and
partly for lack of knowing where to go. She understood well
enough the nature of the task before her. She must set out to
regain, little by little, the position she had lost; and the
first step in the tedious task was to find out, as soon as
possible, on how many of her friends she could count. Her hopes
were mainly centred on Mrs.\ Trenor, who had treasures of
easy-going tolerance for those who were amusing or useful to her,
and in the noisy rush of whose existence the still small voice of
detraction was slow to make itself heard. But Judy, though she
must have been apprised of Miss Bart's return, had not even
recognized it by the formal note of condolence which her friend's
bereavement demanded. Any advance on Lily's side might have been
perilous: there was nothing to do but to trust to the happy
chance of an accidental meeting, and Lily knew that, even so late
in the season, there was always a hope of running across her
friends in their frequent passages through town.





To this end she assiduously showed herself at the restaurants
they frequented, where, attended by the troubled Gerty, she
lunched luxuriously, as she said, on her expectations.





``My dear Gerty, you wouldn't have me let the head-waiter see that
I've nothing to live on but Aunt Julia's legacy? Think of Grace
Stepney's satisfaction if she came in and found us lunching on
cold mutton and tea! What sweet shall we have today, dear---\textit{Coupe}
\textit{Jacques} or PECHES A \textit{La} \textit{Melba}?''





She dropped the \textit{menu} abruptly, with a quick heightening of
colour, and Gerty, following her glance, was aware of the
advance, from an inner room, of a party headed by Mrs.\ Trenor and
Carry Fisher. It was impossible for these ladies and their
companions---among whom Lily had at once distinguished both Trenor
and Rosedale---not to pass, in going out, the table at which the
two girls were seated; and Gerty's sense of the fact betrayed
itself in the helpless trepidation of her manner. Miss Bart, on
the contrary, borne forward on the wave of her buoyant grace, and
neither shrinking from her friends nor appearing to lie in wait
for them, gave to the encounter the touch of naturalness
which she could impart to the most strained situations. Such
embarrassment as was shown was on Mrs.\ Trenor's side, and
manifested itself in the mingling of exaggerated warmth with
imperceptible reservations. Her loudly affirmed pleasure at
seeing Miss Bart took the form of a nebulous generalization,
which included neither enquiries as to her future nor the
expression of a definite wish to see her again. Lily, well-versed
in the language of these omissions, knew that they were equally
intelligible to the other members of the party: even Rosedale,
flushed as he was with the importance of keeping such company, at
once took the temperature of Mrs.\ Trenor's cordiality, and
reflected it in his off-hand greeting of Miss Bart. Trenor, red
and uncomfortable, had cut short his salutations on the pretext
of a word to say to the head-waiter; and the rest of the group
soon melted away in Mrs.\ Trenor's wake.





It was over in a moment---the waiter, \textit{menu} in hand, still hung on
the result of the choice between \textit{Coupe} \textit{Jacques} and PECHES A \textit{La}
\textit{Melba}---but Miss Bart, in the interval, had taken the measure of
her fate. Where Judy Trenor led, all the world would follow; and
Lily had the doomed sense of the castaway who has signalled in
vain to fleeing sails.





In a flash she remembered Mrs.\ Trenor's complaints of Carry
Fisher's rapacity, and saw that they denoted an unexpected
acquaintance with her husband's private affairs. In the large
tumultuous disorder of the life at Bellomont, where no one seemed
to have time to observe any one else, and private aims and
personal interests were swept along unheeded in the rush of
collective activities, Lily had fancied herself sheltered from
inconvenient scrutiny; but if Judy knew when Mrs.\ Fisher borrowed
money of her husband, was she likely to ignore the same
transaction on Lily's part? If she was careless of his affections
she was plainly jealous of his pocket; and in that fact Lily read
the explanation of her rebuff. The immediate result of these
conclusions was the passionate resolve to pay back her debt to
Trenor. That obligation discharged, she would have but a thousand
dollars of Mrs.\ Peniston's legacy left, and nothing to live on
but her own small income, which was considerably less than Gerty
Farish's wretched pittance; but this consideration gave way to
the imperative claim of her wounded pride. She must be
quits with the Trenors first; after that she would take thought
for the future.





In her ignorance of legal procrastinations she had supposed that
her legacy would be paid over within a few days of the reading of
her aunt's will; and after an interval of anxious suspense, she
wrote to enquire the cause of the delay. There was another
interval before Mrs.\ Peniston's lawyer, who was also one of the
executors, replied to the effect that, some questions having
arisen relative to the interpretation of the will, he and his
associates might not be in a position to pay the legacies till
the close of the twelvemonth legally allotted for their
settlement. Bewildered and indignant, Lily resolved to try the
effect of a personal appeal; but she returned from her expedition
with a sense of the powerlessness of beauty and charm against the
unfeeling processes of the law. It seemed intolerable to live on
for another year under the weight of her debt; and in her
extremity she decided to turn to Miss Stepney, who still lingered
in town, immersed in the delectable duty of ``going over''\ her
benefactress's effects. It was bitter enough for Lily to ask a
favour of Grace Stepney, but the alternative was bitterer still;
and one morning she presented herself at Mrs.\ Peniston's, where
Grace, for the facilitation of her pious task, had taken up a
provisional abode.





The strangeness of entering as a suppliant the house where she
had so long commanded, increased Lily's desire to shorten the
ordeal; and when Miss Stepney entered the darkened drawing-room,
rustling with the best quality of crape, her visitor went
straight to the point: would she be willing to advance the amount
of the expected legacy?





Grace, in reply, wept and wondered at the request, bemoaned the
inexorableness of the law, and was astonished that Lily had not
realized the exact similarity of their positions. Did she think
that only the payment of the legacies had been delayed? Why, Miss
Stepney herself had not received a penny of her inheritance, and
was paying rent---yes, actually!---for the privilege of living in a
house that belonged to her. She was sure it was not what poor
dear cousin Julia would have wished---she had told the executors
so to their faces; but they were inaccessible to reason, and
there was nothing to do but to wait. Let Lily take example by
her, and be patient---let them both remember how
beautifully patient cousin Julia had always been.





Lily made a movement which showed her imperfect assimilation of
this example. ``But you will have everything, Grace---it would be
easy for you to borrow ten times the amount I am asking for.''





``Borrow---easy for me to borrow?''\ Grace Stepney rose up before her
in sable wrath. ``Do you imagine for a moment that I would raise
money on my expectations from cousin Julia, when I know so well
her unspeakable horror of every transaction of the sort? Why,
Lily, if you must know the truth, it was the idea of your being
in debt that brought on her illness---you remember she had a
slight attack before you sailed. Oh, I don't know the
particulars, of course---I don't \textit{want} to know them---but there were
rumours about your affairs that made her most unhappy---no one
could be with her without seeing that. I can't help it if you are
offended by my telling you this now---if I can do anything to make
you realize the folly of your course, and how deeply \textit{she}
disapproved of it, I shall feel it is the truest way of making up
to you for her loss.''





\chapter*{\raggedright Chapter 5}

\addcontentsline{toc}{chapter}{Chapter 5}

\markboth{HOUSE OF MIRTH}{CHAPTER 5}





It seemed to Lily, as Mrs.\ Peniston's door closed on her, that
she was taking a final leave of her old life. The future
stretched before her dull and bare as the deserted length of
Fifth Avenue, and opportunities showed as meagrely as the few
cabs trailing in quest of fares that did not come. The
completeness of the analogy was, however, disturbed as she
reached the sidewalk by the rapid approach of a hansom which
pulled up at sight of her.





From beneath its luggage-laden top, she caught the wave of a
signalling hand; and the next moment Mrs.\ Fisher, springing to
the street, had folded her in a demonstrative embrace.





``My dear, you don't mean to say you're still in town? When I saw
you the other day at Sherry's I didn't have time to ask----''\ She
broke off, and added with a burst of frankness: ``The truth is I
was \textit{horrid}, Lily, and I've wanted to tell you so ever since.''





``Oh----''\ Miss Bart protested, drawing back from her penitent
clasp; but Mrs.\ Fisher went on with her usual directness: ``Look
here, Lily, don't let's beat about the bush: half the trouble in
life is caused by pretending there isn't any. That's not my way,
and I can only say I'm thoroughly ashamed of myself for following
the other women's lead. But we'll talk of that by and bye---tell
me now where you're staying and what your plans are. I don't
suppose you're keeping house in there with Grace Stepney,
eh?---and it struck me you might be rather at loose ends.''





In Lily's present mood there was no resisting the honest
friendliness of this appeal, and she said with a smile: ``I am at
loose ends for the moment, but Gerty Farish is still in town, and
she's good enough to let me be with her whenever she can spare
the time.''





Mrs.\ Fisher made a slight grimace. ``H'm---that's a temperate joy. 
Oh, I know---Gerty's a trump, and worth all the rest of us put
together; but A \textit{La} \textit{longue} you're used to a little higher
seasoning, aren't you, dear? And besides, I suppose she'll be off
herself before long---the first of August, you say? Well,
look here, you can't spend your summer in town; we'll talk of
that later too. But meanwhile, what do you say to putting a few
things in a trunk and coming down with me to the Sam Gormers'
tonight?''





And as Lily stared at the breathless suddenness of the
suggestion, she continued with her easy laugh: ``You don't know
them and they don't know you; but that don't make a rap of
difference. They've taken the Van Alstyne place at Roslyn, and
I've got \textit{Carte} \textit{Blanche} to bring my friends down there---the more
the merrier. They do things awfully well, and there's to be
rather a jolly party there this week----''\ she broke off, checked
by an undefinable change in Miss Bart's expression. ``Oh, I don't
mean \textit{your} particular set, you know: rather a different crowd, but
very good fun. The fact is, the Gormers have struck out on a line
of their own: what they want is to have a good time, and to have
it in their own way. They gave the other thing a few months'
trial, under my distinguished auspices, and they were really
doing extremely well---getting on a good deal faster than the
Brys, just because they didn't care as much---but suddenly they
decided that the whole business bored them, and that what they
wanted was a crowd they could really feel at home with. Rather
original of them, don't you think so? Mattie Gormer \textit{has} got
aspirations still; women always have; but she's awfully
easy-going, and Sam won't be bothered, and they both like to be
the most important people in sight, so they've started a sort of
continuous performance of their own, a kind of social Coney
Island, where everybody is welcome who can make noise enough and
doesn't put on airs. I think it's awfully good fun myself---some
of the artistic set, you know, any pretty actress that's going,
and so on. This week, for instance, they have Audrey Anstell, who
made such a hit last spring in `The Winning of Winny'; and Paul
Morpeth---he's painting Mattie Gormer---and the Dick Bellingers,
and Kate Corby---well, every one you can think of who's jolly and
makes a row. Now don't stand there with your nose in the air, my
dear---it will be a good deal better than a broiling Sunday in
town, and you'll find clever people as well as noisy
ones---Morpeth, who admires Mattie enormously, always brings one
or two of his set.''





Mrs.\ Fisher drew Lily toward the hansom with friendly authority. 
``Jump in now, there's a dear, and we'll drive round to your hotel
and have your things packed, and then we'll have tea, and the two
maids can meet us at the train.''







It was a good deal better than a broiling Sunday in town---of
that no doubt remained to Lily as, reclining in the shade of a
leafy verandah, she looked seaward across a stretch of greensward
picturesquely dotted with groups of ladies in lace raiment and
men in tennis flannels. The huge Van Alstyne house and its
rambling dependencies were packed to their fullest capacity with
the Gormers'\ week-end guests, who now, in the radiance of the
Sunday forenoon, were dispersing themselves over the grounds in
quest of the various distractions the place afforded: 
distractions ranging from tennis-courts to shooting-galleries,
from bridge and whiskey within doors to motors and steam-launches
without. Lily had the odd sense of having been caught up into the
crowd as carelessly as a passenger is gathered in by an express
train. The blonde and genial Mrs.\ Gormer might, indeed, have
figured the conductor, calmly assigning seats to the rush of
travellers, while Carry Fisher represented the porter pushing
their bags into place, giving them their numbers for the
dining-car, and warning them when their station was at hand. The
train, meanwhile, had scarcely slackened speed---life whizzed on
with a deafening'\ rattle and roar, in which one traveller at
least found a welcome refuge from the sound of her own thoughts. 
The Gormer \textit{milieu} represented a social out-skirt which Lily had
always fastidiously avoided; but it struck her, now that she was
in it, as only a flamboyant copy of her own world, a caricature
approximating the real thing as the ``society play''\ approaches the
manners of the drawing-room. The people about her were doing the
same things as the Trenors, the Van Osburghs and the Dorsets: the
difference lay in a hundred shades of aspect and manner, from the
pattern of the men's waistcoats to the inflexion of the women's
voices. Everything was pitched in a higher key, and there was
more of each thing: more noise, more colour, more champagne, more
familiarity---but also greater good-nature, less rivalry, and a
fresher capacity for enjoyment.





Miss Bart's arrival had been welcomed with an uncritical
friendliness that first irritated her pride and then brought her
to a sharp sense of her own situation---of the place in life
which, for the moment, she must accept and make the best of. 
These people knew her story---of that her first long talk with
Carry Fisher had left no doubt: she was publicly branded as the
heroine of a ``queer''\ episode---but instead of shrinking from her
as her own friends had done, they received her without question
into the easy promiscuity of their lives. They swallowed her past
as easily as they did Miss Anstell's, and with no apparent sense
of any difference in the size of the mouthful: all they asked was
that she should---in her own way, for they recognized a diversity
of gifts---contribute as much to the general amusement as that
graceful actress, whose talents, when off the stage, were of the
most varied order. Lily felt at once that any tendency to be
``stuck-up,''\ to mark a sense of differences and distinctions,
would be fatal to her continuance in the Gormer set. To be taken
in on such terms---and into such a world!---was hard enough to the
lingering pride in her; but she realized, with a pang of
self-contempt, that to be excluded from it would, after all, be
harder still. For, almost at once, she had felt the insidious
charm of slipping back into a life where every material
difficulty was smoothed away. The sudden escape from a stifling
hotel in a dusty deserted city to the space and luxury of a great
country-house fanned by sea breezes, had produced a state of
moral lassitude agreeable enough after the nervous tension and
physical discomfort of the past weeks. For the moment she must
yield to the refreshment her senses craved---after that she would
reconsider her situation, and take counsel with her dignity. Her
enjoyment of her surroundings was, indeed, tinged by the
unpleasant consideration that she was accepting the hospitality
and courting the approval of people she had disdained under other
conditions. But she was growing less sensitive on such points: a
hard glaze of indifference was fast forming over her delicacies
and susceptibilities, and each concession to expediency hardened
the surface a little more.





On the Monday, when the party disbanded with uproarious adieux,
the return to town threw into stronger relief the charms of the
life she was leaving. The other guests were dispersing to take up
the same existence in a different setting: some at Newport, some
at Bar Harbour, some in the elaborate rusticity of an Adirondack
camp. Even Gerty Farish, who welcomed Lily's return with tender
solicitude, would soon be preparing to join the aunt with whom she
spent her summers on Lake George: only Lily herself remained
without plan or purpose, stranded in a backwater of the great
current of pleasure. But Carry Fisher, who had insisted on
transporting her to her own house, where she herself was to perch
for a day or two on the way to the Brys'\ camp, came to the rescue
with a new suggestion.





``Look here, Lily---I'll tell you what it is: I want you to take my
place with Mattie Gormer this summer. They're taking a party out
to Alaska next month in their private car, and Mattie, who is the
laziest woman alive, wants me to go with them, and relieve her of
the bother of arranging things; but the Brys want me too---oh,
yes, we've made it up: didn't I tell you?---and, to put it
frankly, though I like the Gormers best, there's more profit for
me in the Brys. The fact is, they want to try Newport this
summer, and if I can make it a success for them they---well,
they'll make it a success for me.'' Mrs.\ Fisher clasped her hands
enthusiastically. ``Do you know, Lily, the more I think of my idea
the better I like it---quite as much for you as for myself. The
Gormers have both taken a tremendous fancy to you, and the trip
to Alaska is---well---the very thing I should want for you just at
present.''





Miss Bart lifted her eyes with a keen glance. ``To take me out of
my friends'\ way, you mean?''\ she said quietly; and Mrs.\ Fisher
responded with a deprecating kiss: ``To keep you out of their
sight till they realize how much they miss you.''







Miss Bart went with the Gormers to Alaska; and the expedition, if
it did not produce the effect anticipated by her friend, had at
least the negative advantage of removing her from the fiery
centre of criticism and discussion. Gerty Farish had opposed the
plan with all the energy of her somewhat inarticulate nature. She
had even offered to give up her visit to Lake George, and remain
in town with Miss Bart, if the latter would renounce her journey;
but Lily could disguise her real distaste for this plan under a
sufficiently valid reason.





``You dear innocent, don't you see,''\ she protested, ``that Carry is
quite right, and that I must take up my usual life, and go about
among people as much as possible? If my old friends choose to
believe lies about me I shall have to make new ones, that's all;
and you know beggars mustn't be choosers. Not that I don't like
Mattie Gormer---I \textit{do} like her: she's kind and honest and
unaffected; and don't you suppose I feel grateful to her for
making me welcome at a time when, as you've yourself seen, my own
family have unanimously washed their hands of me?''





Gerty shook her head, mutely unconvinced. She felt not only that
Lily was cheapening herself by making use of an intimacy she
would never have cultivated from choice, but that, in drifting
back now to her former manner of life, she was forfeiting her
last chance of ever escaping from it. Gerty had but an obscure
conception of what Lily's actual experience had been: but its
consequences had established a lasting hold on her pity since the
memorable night when she had offered up her own secret hope to
her friend's extremity. To characters like Gerty's such a
sacrifice constitutes a moral claim on the part of the person in
whose behalf it has been made. Having once helped Lily, she must
continue to help her; and helping her, must believe in her,
because faith is the main-spring of such natures. But even if
Miss Bart, after her renewed taste of the amenities of life,
could have returned to the barrenness of a New York August,
mitigated only by poor Gerty's presence, her worldly wisdom would
have counselled her against such an act of abnegation. She knew
that Carry Fisher was right: that an opportune absence might be
the first step toward rehabilitation, and that, at any rate, to
linger on in town out of season was a fatal admission of defeat. 
From the Gormers'\ tumultuous progress across their native
continent, she returned with an altered view of her situation. 
The renewed habit of luxury---the daily waking to an assured
absence of care and presence of material ease---gradually blunted
her appreciation of these values, and left her more conscious of
the void they could not fill. Mattie Gormer's undiscriminating
good-nature, and the slap-dash sociability of her friends, who
treated Lily precisely as they treated each other---all these
characteristic notes of difference began to wear upon her
endurance; and the more she saw to criticize in her companions,
the less justification she found for making use of them. The
longing to get back to her former surroundings hardened to a
fixed idea; but with the strengthening of her purpose came the
inevitable perception that, to attain it, she must exact fresh
concessions from her pride. These, for the moment, took the
unpleasant form of continuing to cling to her hosts after their
return from Alaska. Little as she was in the key of their \textit{milieu},
her immense social facility, her long habit of adapting herself
to others without suffering her own outline to be blurred, the
skilled manipulation of all the polished implements of her craft,
had won for her an important place in the Gormer group. If their
resonant hilarity could never be hers, she contributed a note of
easy elegance more valuable to Mattie Gormer than the louder
passages of the band. Sam Gormer and his special cronies stood
indeed a little in awe of her; but Mattie's following, headed by
Paul Morpeth, made her feel that they prized her for the very
qualities they most conspicuously lacked. If Morpeth, whose
social indolence was as great as his artistic activity, had
abandoned himself to the easy current of the Gormer existence,
where the minor exactions of politeness were unknown or ignored,
and a man could either break his engagements, or keep them in a
painting-jacket and slippers, he still preserved his sense of
differences, and his appreciation of graces he had no time to
cultivate. During the preparations for the Brys'\ \textit{tableaux} he had
been immensely struck by Lily's plastic possibilities---``not the
face: too self-controlled for expression; but the rest of
her---gad, what a model she'd make!''---and though his abhorrence of
the world in which he had seen her was too great for him to think
of seeking her there, he was fully alive to the privilege of
having her to look at and listen to while he lounged in Mattie
Gormer's dishevelled drawing-room.





Lily had thus formed, in the tumult of her surroundings, a little
nucleus of friendly relations which mitigated the crudeness of
her course in lingering with the Gormers after their return. Nor
was she without pale glimpses of her own world, especially since
the breaking-up of the Newport season had set the social current
once more toward Long Island. Kate Corby, whose tastes
made her as promiscuous as Carry Fisher was rendered by her
necessities, occasionally descended on the Gormers, where, after
a first stare of surprise, she took Lily's presence almost too
much as a matter of course. Mrs.\ Fisher, too, appearing
frequently in the neighbourhood, drove over to impart her
experiences and give Lily what she called the latest report from
the weather-bureau; and the latter, who had never directly
invited her confidence, could yet talk with her more freely than
with Gerty Farish, in whose presence it was impossible even to
admit the existence of much that Mrs.\ Fisher conveniently took
for granted.





Mrs.\ Fisher, moreover, had no embarrassing curiosity. She did not
wish to probe the inwardness of Lily's situation, but simply to
view it from the outside, and draw her conclusions accordingly;
and these conclusions, at the end of a confidential talk, she
summed up to her friend in the succinct remark: ``You must marry
as soon as you can.''





Lily uttered a faint laugh---for once Mrs.\ Fisher lacked
originality. ``Do you mean, like Gerty Farish, to recommend the
unfailing panacea of 'a good man's love'?''





``No---I don't think either of my candidates would answer to that
description,''\ said Mrs.\ Fisher after a pause of reflection.





``Either? Are there actually two?''





``Well, perhaps I ought to say one and a half---for the moment.''





Miss Bart received this with increasing amusement. ``Other things
being equal, I think I should prefer a half-husband: who is he?''





``Don't fly out at me till you hear my reasons---George Dorset.''





``Oh----''\ Lily murmured reproachfully; but Mrs.\ Fisher pressed on
unrebuffed. ``Well, why not? They had a few weeks'\ honeymoon when
they first got back from Europe, but now things are going badly
with them again. Bertha has been behaving more than ever like a
madwoman, and George's powers of credulity are very nearly
exhausted. They're at their place here, you know, and I spent
last Sunday with them. It was a ghastly party---no one else but
poor Neddy Silverton, who looks like a galley-slave (they used to
talk of my making that poor boy unhappy!)---and after
luncheon George carried me off on a long walk, and told me the
end would have to come soon.''





Miss Bart made an incredulous gesture. ``As far as that goes, the
end will never come---Bertha will always know how to get him back
when she wants him.''





Mrs.\ Fisher continued to observe her tentatively. ``Not if he has
any one else to turn to! Yes---that's just what it comes to: the
poor creature can't stand alone. And I remember him such a good
fellow, full of life and enthusiasm.'' She paused, and went on,
dropping her glance from Lily's: ``He wouldn't stay with her ten
minutes if he \textit{knew}----''





``Knew----?''\ Miss Bart repeated.





``What \textit{you} must, for instance---with the opportunities you've had! 
If he had positive proof, I mean----''





Lily interrupted her with a deep blush of displeasure. ``Please
let us drop the subject, Carry: it's too odious to me.'' And to
divert her companion's attention she added, with an attempt at
lightness: ``And your second candidate? We must not forget him.''





Mrs.\ Fisher echoed her laugh. ``I wonder if you'll cry out just as
loud if I say---Sim Rosedale?''





Miss Bart did not cry out: she sat silent, gazing thoughtfully at
her friend. The suggestion, in truth, gave expression to a
possibility which, in the last weeks, had more than once recurred
to her; but after a moment she said carelessly: ``Mr.\ Rosedale
wants a wife who can establish him in the bosom of the Van
Osburghs and Trenors.''





Mrs.\ Fisher caught her up eagerly. ``And so \textit{you} could---with his
money! Don't you see how beautifully it would work out for you
both?''





``I don't see any way of making him see it,''\ Lily returned, with a
laugh intended to dismiss the subject.





But in reality it lingered with her long after Mrs.\ Fisher had
taken leave. She had seen very little of Rosedale since her
annexation by the Gormers, for he was still steadily bent on
penetrating to the inner Paradise from which she was now
excluded; but once or twice, when nothing better offered, he had
turned up for a Sunday, and on these occasions he had left her in
no doubt as to his view of her situation. That he still
admired her was, more than ever, offensively evident; for in the
Gormer circle, where he expanded as in his native element, there
were no puzzling conventions to check the full expression of his
approval. But it was in the quality of his admiration that she
read his shrewd estimate of her case. He enjoyed letting the
Gormers see that he had known ``Miss Lily''---she was ``Miss Lily''\ to
him now---before they had had the faintest social existence: 
enjoyed more especially impressing Paul Morpeth with the distance
to which their intimacy dated back. But he let it be felt that
that intimacy was a mere ripple on the surface of a rushing
social current, the kind of relaxation which a man of large
interests and manifold preoccupations permits himself in his
hours of ease.





The necessity of accepting this view of their past relation, and
of meeting it in the key of pleasantry prevalent among her new
friends, was deeply humiliating to Lily. But she dared less than
ever to quarrel with Rosedale. She suspected that her rejection
rankled among the most unforgettable of his rebuffs, and the fact
that he knew something of her wretched transaction with Trenor,
and was sure to put the basest construction on it, seemed to
place her hopelessly in his power. Yet at Carry Fisher's
suggestion a new hope had stirred in her. Much as she disliked
Rosedale, she no longer absolutely despised him. For he was
gradually attaining his object in life, and that, to Lily, was
always less despicable than to miss it. With the slow unalterable
persistency which she had always felt in him, he was making his
way through the dense mass of social antagonisms. Already his
wealth, and the masterly use he had made of it, were giving him
an enviable prominence in the world of affairs, and placing Wall
Street under obligations which only Fifth Avenue could repay. In
response to these claims, his name began to figure on municipal
committees and charitable boards; he appeared at banquets to
distinguished strangers, and his candidacy at one of the
fashionable clubs was discussed with diminishing opposition. He
had figured once or twice at the Trenor dinners, and had learned
to speak with just the right note of disdain of the big Van
Osburgh crushes; and all he now needed was a wife whose
affiliations would shorten the last tedious steps of his ascent. 
It was with that object that, a year earlier, he had fixed
his affections on Miss Bart; but in the interval he had
mounted nearer to the goal, while she had lost the power to
abbreviate the remaining steps of the way. All this she saw with
the clearness of vision that came to her in moments of
despondency. It was success that dazzled her---she could
distinguish facts plainly enough in the twilight of failure. And
the twilight, as she now sought to pierce it, was gradually
lighted by a faint spark of reassurance. Under the utilitarian
motive of Rosedale's wooing she had felt, clearly enough, the
heat of personal inclination. She would not have detested him so
heartily had she not known that he dared to admire her. What,
then, if the passion persisted, though the other motive had
ceased to sustain it? She had never even tried to please him---he
had been drawn to her in spite of her manifest disdain. What if
she now chose to exert the power which, even in its passive
state, he had felt so strongly? What if she made him marry her
for love, now that he had no other reason for marrying her?





\chapter*{\raggedright Chapter 6}

\addcontentsline{toc}{chapter}{Chapter 6}

\markboth{HOUSE OF MIRTH}{CHAPTER 6}





As became persons of their rising consequence, the Gormers were
engaged in building a country-house on Long Island; and it was a
part of Miss Bart's duty to attend her hostess on frequent visits
of inspection to the new estate. There, while Mrs.\ Gormer plunged
into problems of lighting and sanitation, Lily had leisure to
wander, in the bright autumn air, along the tree-fringed bay to
which the land declined. Little as she was addicted to solitude,
there had come to be moments when it seemed a welcome escape from
the empty noises of her life. She was weary of being swept
passively along a current of pleasure and business in which she
had no share; weary of seeing other people pursue amusement and
squander money, while she felt herself of no more account among
them than an expensive toy in the hands of a spoiled child.





It was in this frame of mind that, striking back from the shore
one morning into the windings of an unfamiliar lane, she came
suddenly upon the figure of George Dorset. The Dorset place was
in the immediate neighbourhood of the Gormers'\ newly-acquired
estate, and in her motor-flights thither with Mrs.\ Gormer, Lily
had caught one or two passing glimpses of the couple; but they
moved in so different an orbit that she had not considered the
possibility of a direct encounter.





Dorset, swinging along with bent head, in moody abstraction, did
not see Miss Bart till he was close upon her; but the sight,
instead of bringing him to a halt, as she had half-expected, sent
him toward her with an eagerness which found expression in his
opening words.





``Miss Bart!---You'll shake hands, won't you? I've been hoping to
meet you---I should have written to you if I'd dared.'' His face,
with its tossed red hair and straggling moustache, had a driven
uneasy look, as though life had become an unceasing race between
himself and the thoughts at his heels.





The look drew a word of compassionate greeting from Lily, and he
pressed on, as if encouraged by her tone: ``I wanted to
apologize---to ask you to forgive me for the miserable part I
played----''





She checked him with a quick gesture. ``Don't let us speak of it: 
I was very sorry for you,''\ she said, with a tinge of disdain
which, as she instantly perceived, was not lost on him.





He flushed to his haggard eyes, flushed so cruelly that she
repented the thrust. ``You might well be; you don't know---you must
let me explain. I was deceived: abominably deceived----''





``I am still more sorry for you, then,''\ she interposed, without
irony; ``but you must see that I am not exactly the person with
whom the subject can be discussed.''





He met this with a look of genuine wonder. ``Why not? Isn't it to
you, of all people, that I owe an explanation----''





``No explanation is necessary: the situation was perfectly clear
to me.''





``Ah----''\ he murmured, his head drooping again, and his irresolute
hand switching at the underbrush along the lane. But as Lily made
a movement to pass on, he broke out with fresh vehemence: ``Miss
Bart, for God's sake don't turn from me! We used to be good
friends---you were always kind to me---and you don't know how I
need a friend now.''





The lamentable weakness of the words roused a motion of pity in
Lily's breast. She too needed friends---she had tasted the pang of
loneliness; and her resentment of Bertha Dorset's cruelty
softened her heart to the poor wretch who was after all the chief
of Bertha's victims.





``I still wish to be kind; I feel no ill-will toward you,''\ she
said. ``But you must understand that after what has happened we
can't be friends again---we can't see each other.''





``Ah, you \textit{are} kind---you're merciful---you always were!''\ He fixed
his miserable gaze on her. ``But why can't we be friends---why not,
when I've repented in dust and ashes? Isn't it hard that you
should condemn me to suffer for the falseness, the treachery of
others? I was punished enough at the time---is there to be no
respite for me?''





``I should have thought you had found complete respite in the
reconciliation which was effected at my expense,''\ Lily began,
with renewed impatience; but he broke in imploringly: ``Don't put
it in that way---when that's been the worst of my
punishment. My God!\ what could I do---wasn't I powerless? You were
singled out as a sacrifice: any word I might have said would have
been turned against you----''





``I have told you I don't blame you; all I ask you to understand
is that, after the use Bertha chose to make of me---after all that
her behaviour has since implied---it's impossible that you and I
should meet.''





He continued to stand before her, in his dogged weakness. ``Is
it---need it be? Mightn't there be circumstances----?''\ he checked
himself, slashing at the wayside weeds in a wider radius. Then he
began again: ``Miss Bart, listen---give me a minute. If we're not
to meet again, at least let me have a hearing now. You say we
can't be friends after---after what has happened. But can't I at
least appeal to your pity? Can't I move you if I ask you to think
of me as a prisoner---a prisoner you alone can set free?''





Lily's inward start betrayed itself in a quick blush: was it
possible that this was really the sense of Carry Fisher's
adumbrations?





``I can't see how I can possibly be of any help to you,''\ she
murmured, drawing back a little from the mounting excitement of
his look.





Her tone seemed to sober him, as it had so often done in his
stormiest moments. The stubborn lines of his face relaxed, and he
said, with an abrupt drop to docility: ``You \textit{would} see, if you'd
be as merciful as you used to be: and heaven knows I've never
needed it more!''





She paused a moment, moved in spite of herself by this reminder
of her influence over him. Her fibres had been softened by
suffering, and the sudden glimpse into his mocked and broken life
disarmed her contempt for his weakness.





``I am very sorry for you---I would help you willingly; but you
must have other friends, other advisers.''





``I never had a friend like you,''\ he answered simply. ``And
besides---can't you see?---you're the only person''---his voice
dropped to a whisper---``the only person who knows.''





Again she felt her colour change; again her heart rose in
precipitate throbs to meet what she felt was coming. He lifted
his eyes to her entreatingly. ``You do see, don't you? You
understand? I'm desperate---I'm at the end of my tether. I
want to be free, and you can free me. I know you can. You don't
want to keep me bound fast in hell, do you? You can't want to
take such a vengeance as that. You were always kind---your eyes
are kind now. You say you're sorry for me. Well, it rests with
you to show it; and heaven knows there's nothing to keep you
back. You understand, of course---there wouldn't be a hint of
publicity---not a sound or a syllable to connect you with the
thing. It would never come to that, you know: all I need is to be
able to say definitely: 'I know this---and this---and this'---and the
fight would drop, and the way be cleared, and the whole
abominable business swept out of sight in a second.''





He spoke pantingly, like a tired runner, with breaks of
exhaustion between his words; and through the breaks she caught,
as through the shifting rents of a fog, great golden vistas of
peace and safety. For there was no mistaking the definite
intention behind his vague appeal; she could have filled up the
blanks without the help of Mrs.\ Fisher's insinuations. Here was a
man who turned to her in the extremity of his loneliness and his
humiliation: if she came to him at such a moment he would be hers
with all the force of his deluded faith. And the power to make
him so lay in her hand---lay there in a completeness he could not
even remotely conjecture. Revenge and rehabilitation might be
hers at a stroke---there was something dazzling in the
completeness of the opportunity.





She stood silent, gazing away from him down the autumnal stretch
of the deserted lane. And suddenly fear possessed her---fear of
herself, and of the terrible force of the temptation. All her
past weaknesses were like so many eager accomplices drawing her
toward the path their feet had already smoothed. She turned
quickly, and held out her hand to Dorset.





``Goodbye---I'm sorry; there's nothing in the world that I can do.''





``Nothing? Ah, don't say that,''\ he cried; ``say what's true: that
you abandon me like the others. You, the only creature who could
have saved me!''





``Goodbye---goodbye,''\ she repeated hurriedly; and as she
moved away she heard him cry out on a last note of entreaty: ``At
least you'll let me see you once more?''







Lily, on regaining the Gormer grounds, struck rapidly across the
lawn toward the unfinished house, where she fancied that her
hostess might be speculating, not too resignedly, on the cause of
her delay; for, like many unpunctual persons, Mrs.\ Gormer
disliked to be kept waiting.





As Miss Bart reached the avenue, however, she saw a smart phaeton
with a high-stepping pair disappear behind the shrubbery in the
direction of the gate; and on the doorstep stood Mrs.\ Gormer,
with a glow of retrospective pleasure on her open countenance. At
sight of Lily the glow deepened to an embarrassed red, and she
said with a slight laugh: ``Did you see my visitor? Oh, I thought
you came back by the avenue. It was Mrs.\ George Dorset---she said
she'd dropped in to make a neighbourly call.''





Lily met the announcement with her usual composure, though her
experience of Bertha's idiosyncrasies would not have led her to
include the neighbourly instinct among them; and Mrs.\ Gormer,
relieved to see that she gave no sign of surprise, went on with a
deprecating laugh: ``Of course what really brought her was
curiosity---she made me take her all over the house. But no one
could have been nicer---no airs, you know, and so good-natured: I
can quite see why people think her so fascinating.''





This surprising event, coinciding too completely with her meeting
with Dorset to be regarded as contingent upon it, had yet
immediately struck Lily with a vague sense of foreboding. It was
not in Bertha's habits to be neighbourly, much less to make
advances to any one outside the immediate circle of her
affinities. She had always consistently ignored the world of
outer aspirants, or had recognized its individual members only
when prompted by motives of self-interest; and the very
capriciousness of her condescensions had, as Lily was aware,
given them special value in the eyes of the persons she
distinguished. Lily saw this now in Mrs.\ Gormer's unconcealable
complacency, and in the happy irrelevance with which, for the
next day or two, she quoted Bertha's opinions and
speculated on the origin of her gown. All the secret ambitions
which Mrs.\ Gormer's native indolence, and the attitude of her
companions, kept in habitual abeyance, were now germinating
afresh in the glow of Bertha's advances; and whatever the cause
of the latter, Lily saw that, if they were followed up, they were
likely to have a disturbing effect upon her own future.





She had arranged to break the length of her stay with her new
friends by one or two visits to other acquaintances as recent;
and on her return from this somewhat depressing excursion she was
immediately conscious that Mrs.\ Dorset's influence was still in
the air. There had been another exchange of visits, a tea at a
country-club, an encounter at a hunt ball; there was even a
rumour of an approaching dinner, which Mattie Gormer, with an
unnatural effort at discretion, tried to smuggle out of the
conversation whenever Miss Bart took part in it.





The latter had already planned to return to town after a farewell
Sunday with her friends; and, with Gerty Farish's aid, had
discovered a small private hotel where she might establish
herself for the winter. The hotel being on the edge of a
fashionable neighbourhood, the price of the few square feet she
was to occupy was considerably in excess of her means; but she
found a justification for her dislike of poorer quarters in the
argument that, at this particular juncture, it was of the utmost
importance to keep up a show of prosperity. In reality, it was
impossible for her, while she had the means to pay her way for a
week ahead, to lapse into a form of existence like Gerty
Farish's. She had never been so near the brink of insolvency; but
she could at least manage to meet her weekly hotel bill, and
having settled the heaviest of her previous debts out of the
money she had received from Trenor, she had a still fair margin
of credit to go upon. The situation, however, was not agreeable
enough to lull her to complete unconsciousness of its insecurity. 
Her rooms, with their cramped outlook down a sallow vista of
brick walls and fire-escapes, her lonely meals in the dark
restaurant with its surcharged ceiling and haunting smell of
coffee---all these material discomforts, which were yet to be
accounted as so many privileges soon to be withdrawn, kept
constantly before her the disadvantages of her state; and
her mind reverted the more insistently to Mrs.\ Fisher's counsels. 
Beat about the question as she would, she knew the outcome of it
was that she must try to marry Rosedale; and in this conviction
she was fortified by an unexpected visit from George Dorset.





She found him, on the first Sunday after her return to town,
pacing her narrow sitting-room to the imminent peril of the few
knick-knacks with which she had tried to disguise its plush
exuberances; but the sight of her seemed to quiet him, and he
said meekly that he hadn't come to bother her---that he asked
only to be allowed to sit for half an hour and talk of anything
she liked. In reality, as she knew, he had but one subject: 
himself and his wretchedness; and it was the need of her sympathy
that had drawn him back. But he began with a pretence of
questioning her about herself, and as she replied, she saw that,
for the first time, a faint realization of her plight penetrated
the dense surface of his self-absorption. Was it possible that
her old beast of an aunt had actually cut her off? That she was
living alone like this because there was no one else for her to
go to, and that she really hadn't more than enough to keep alive
on till the wretched little legacy was paid? The fibres of
sympathy were nearly atrophied in him, but he was suffering so
intensely that he had a faint glimpse of what other sufferings
might mean---and, as she perceived, an almost simultaneous
perception of the way in which her particular misfortunes might
serve him.





When at length she dismissed him, on the pretext that she must
dress for dinner, he lingered entreatingly on the threshold to
blurt out: ``It's been such a comfort---do say you'll let me see
you again---''\ But to this direct appeal it was impossible to give
an assent; and she said with friendly decisiveness: ``I'm
sorry---but you know why I can't.''





He coloured to the eyes, pushed the door shut, and stood before
her embarrassed but insistent. ``I know how you might, if you
would---if things were different---and it lies with you to make
them so. It's just a word to say, and you put me out of my
misery!''





Their eyes met, and for a second she trembled again with the
nearness of the temptation. ``You're mistaken; I know nothing; I
saw nothing,''\ she exclaimed, striving, by sheer force of
reiteration, to build a barrier between herself and her peril;
and as he turned away, groaning out ``You sacrifice us both,''\ she
continued to repeat, as if it were a charm: ``I know
nothing---absolutely nothing.''







Lily had seen little of Rosedale since her illuminating talk with
Mrs.\ Fisher, but on the two or three occasions when they had met
she was conscious of having distinctly advanced in his favour. 
There could be no doubt that he admired her as much as ever, and
she believed it rested with herself to raise his admiration to
the point where it should bear down the lingering counsels of
expediency. The task was not an easy one; but neither was it
easy, in her long sleepless nights, to face the thought of what
George Dorset was so clearly ready to offer. Baseness for
baseness, she hated the other least: there were even moments when
a marriage with Rosedale seemed the only honourable solution of
her difficulties. She did not indeed let her imagination range
beyond the day of plighting: after that everything faded into a
haze of material well-being, in which the personality of her
benefactor remained mercifully vague. She had learned, in her
long vigils, that there were certain things not good to think of,
certain midnight images that must at any cost be exorcised---and
one of these was the image of herself as Rosedale's wife.





Carry Fisher, on the strength, as she frankly owned, of the Brys'
Newport success, had taken for the autumn months a small house at
Tuxedo; and thither Lily was bound on the Sunday after Dorset's
visit. Though it was nearly dinner-time when she arrived, her
hostess was still out, and the firelit quiet of the small silent
house descended on her spirit with a sense of peace and
familiarity. It may be doubted if such an emotion had ever before
been evoked by Carry Fisher's surroundings; but, contrasted to
the world in which Lily had lately lived, there was an air of
repose and stability in the very placing of the furniture, and in
the quiet competence of the parlour-maid who led her up to her
room. Mrs.\ Fisher's unconventionality was, after all, a merely
superficial divergence from an inherited social creed, while the
manners of the Gormer circle represented their first attempt to
formulate such a creed for themselves.





It was the first time since her return from Europe that Lily had
found herself in a congenial atmosphere, and the stirring of
familiar associations had almost prepared her, as she descended
the stairs before dinner, to enter upon a group of her old
acquaintances. But this expectation was instantly checked by the
reflection that the friends who remained loyal were precisely
those who would be least willing to expose her to such
encounters; and it was hardly with surprise that she found,
instead, Mr.\ Rosedale kneeling domestically on the drawing-room
hearth before his hostess's little girl.





Rosedale in the paternal r\^{o}le was hardly a figure to soften Lily;
yet she could not but notice a quality of homely goodness in his
advances to the child. They were not, at any rate, the
premeditated and perfunctory endearments of the guest under his
hostess's eye, for he and the little girl had the room to
themselves; and something in his attitude made him seem a simple
and kindly being compared to the small critical creature who
endured his homage. Yes, he would be kind---Lily, from the
threshold, had time to feel---kind in his gross, unscrupulous,
rapacious way, the way of the predatory creature with his mate. 
She had but a moment in which to consider whether this glimpse of
the fireside man mitigated her repugnance, or gave it, rather, a
more concrete and intimate form; for at sight of her he was
immediately on his feet again, the florid and dominant Rosedale
of Mattie Gormer's drawing-room.





It was no surprise to Lily to find that he had been selected as
her only fellow-guest. Though she and her hostess had not met
since the latter's tentative discussion of her future, Lily knew
that the acuteness which enabled Mrs.\ Fisher to lay a safe and
pleasant course through a world of antagonistic forces was not
infrequently exercised for the benefit of her friends. It was, in
fact, characteristic of Carry that, while she actively gleaned
her own stores from the fields of affluence, her real sympathies
were on the other side---with the unlucky, the unpopular, the
unsuccessful, with all her hungry fellow-toilers in the shorn
stubble of success.





Mrs.\ Fisher's experience guarded her against the mistake of
exposing Lily, for the first evening, to the unmitigated
impression of Rosedale's personality. Kate Corby and two
or three men dropped in to dinner, and Lily, alive to every
detail of her friend's method, saw that such opportunities as had
been contrived for her were to be deferred till she had, as it
were, gained courage to make effectual use of them. She had a
sense of acquiescing in this plan with the passiveness of a
sufferer resigned to the surgeon's touch; and this feeling of
almost lethargic helplessness continued when, after the departure
of the guests, Mrs.\ Fisher followed her upstairs.





``May I come in and smoke a cigarette over your fire? If we talk
in my room we shall disturb the child.'' Mrs.\ Fisher looked about
her with the eye of the solicitous hostess. ``I hope you've
managed to make yourself comfortable, dear? Isn't it a jolly
little house? It's such a blessing to have a few quiet weeks with
the baby.''





Carry, in her rare moments of prosperity, became so expansively
maternal that Miss Bart sometimes wondered whether, if she could
ever get time and money enough, she would not end by devoting
them both to her daughter.





``It's a well-earned rest: I'll say that for myself,''\ she
continued, sinking down with a sigh of content on the pillowed
lounge near the fire. ``Louisa Bry is a stern task-master: I often
used to wish myself back with the Gormers. Talk of love making
people jealous and suspicious---it's nothing to social ambition! 
Louisa used to lie awake at night wondering whether the women who
called on us called on \textit{me} because I was with her, or on \textit{her}
because she was with me; and she was always laying traps to find
out what I thought. Of course I had to disown my oldest friends,
rather than let her suspect she owed me the chance of making a
single acquaintance---when, all the while, that was what she had
me there for, and what she wrote me a handsome cheque for when
the season was over!''





Mrs.\ Fisher was not a woman who talked of herself without cause,
and the practice of direct speech, far from precluding in her an
occasional resort to circuitous methods, served rather, at
crucial moments, the purpose of the juggler's chatter while he
shifts the contents of his sleeves. Through the haze of her
cigarette smoke she continued to gaze meditatively at Miss Bart,
who, having dismissed her maid, sat before the
toilet-table shaking out over her shoulders the loosened
undulations of her hair.





``Your hair's wonderful, Lily. Thinner---? What does that matter,
when it's so light and alive? So many women's worries seem to go
straight to their hair---but yours looks as if there had never
been an anxious thought under it. I never saw you look better
than you did this evening. Mattie Gormer told me that Morpeth
wanted to paint you---why don't you let him?''





Miss Bart's immediate answer was to address a critical glance to
the reflection of the countenance under discussion. Then she
said, with a slight touch of irritation: ``I don't care to accept
a portrait from Paul Morpeth.''





Mrs.\ Fisher mused. ``N---no. And just now, especially---well, he
can do you after you're married.'' She waited a moment, and then
went on: ``By the way, I had a visit from Mattie the other day. 
She turned up here last Sunday---and with Bertha Dorset, of all
people in the world!''





She paused again to measure the effect of this announcement on
her hearer, but the brush in Miss Bart's lifted hand maintained
its unwavering stroke from brow to nape.





``I never was more astonished,''\ Mrs.\ Fisher pursued. ``I don't know
two women less predestined to intimacy---from Bertha's standpoint,
that is; for of course poor Mattie thinks it natural enough that
she should be singled out---I've no doubt the rabbit always thinks
it is fascinating the anaconda. Well, you know I've always told
you that Mattie secretly longed to bore herself with the really
fashionable; and now that the chance has come, I see that she's
capable of sacrificing all her old friends to it.''





Lily laid aside her brush and turned a penetrating glance upon
her friend. ``Including \textit{me}?''\ she suggested.





``Ah, my dear,''\ murmured Mrs.\ Fisher, rising to push back a log
from the hearth.





``That's what Bertha means, isn't it?''\ Miss Bart went on steadily. 
``For of course she always means something; and before I left Long
Island I saw that she was beginning to lay her toils for Mattie.''





Mrs.\ Fisher sighed evasively. ``She has her fast now, at any rate. 
To think of that loud independence of Mattie's being only
a subtler form of snobbishness! Bertha can already make her
believe anything she pleases---and I'm afraid she's begun, my poor
child, by insinuating horrors about you.''





Lily flushed under the shadow of her drooping hair. ``The world is
too vile,''\ she murmured, averting herself from Mrs.\ Fisher's
anxious scrutiny.





``It's not a pretty place; and the only way to keep a footing in
it is to fight it on its own terms---and above all, my dear, not
alone!''\ Mrs.\ Fisher gathered up her floating implications in a
resolute grasp. ``You've told me so little that I can only guess
what has been happening; but in the rush we all live in there's
no time to keep on hating any one without a cause, and if Bertha
is still nasty enough to want to injure you with other people it
must be because she's still afraid of you. From her standpoint
there's only one reason for being afraid of you; and my own idea
is that, if you want to punish her, you hold the means in your
hand. I believe you can marry George Dorset tomorrow; but if you
don't care for that particular form of retaliation, the only
thing to save you from Bertha is to marry somebody else.''





\chapter*{\raggedright Chapter 7}

\addcontentsline{toc}{chapter}{Chapter 7}

\markboth{HOUSE OF MIRTH}{CHAPTER 7}





The light projected on the situation by Mrs.\ Fisher had the
cheerless distinctness of a winter dawn. It outlined the facts
with a cold precision unmodified by shade or colour, and
refracted, as it were, from the blank walls of the surrounding
limitations: she had opened windows from which no sky was ever
visible. But the idealist subdued to vulgar necessities must
employ vulgar minds to draw the inferences to which he cannot
stoop; and it was easier for Lily to let Mrs.\ Fisher formulate
her case than to put it plainly to herself. Once confronted with
it, however, she went the full length of its consequences; and
these had never been more clearly present to her than when, the
next afternoon, she set out for a walk with Rosedale.





It was one of those still November days when the air is haunted
with the light of summer, and something in the lines of the
landscape, and in the golden haze which bathed them, recalled to
Miss Bart the September afternoon when she had climbed the slopes
of Bellomont with Selden. The importunate memory was kept before
her by its ironic contrast to her present situation, since her
walk with Selden had represented an irresistible flight from just
such a climax as the present excursion was designed to bring
about. But other memories importuned her also; the recollection
of similar situations, as skillfully led up to, but through some
malice of fortune, or her own unsteadiness of purpose, always
failing of the intended result. Well, her purpose was steady
enough now. She saw that the whole weary work of rehabilitation
must begin again, and against far greater odds, if Bertha Dorset
should succeed in breaking up her friendship with the Gormers;
and her longing for shelter and security was intensified by the
passionate desire to triumph over Bertha, as only wealth and
predominance could triumph over her. As the wife of Rosedale---the
Rosedale she felt it in her power to create---she would at least
present an invulnerable front to her enemy.





She had to draw upon this thought, as upon some fiery stimulant,
to keep up her part in the scene toward which Rosedale
was too frankly tending. As she walked beside him, shrinking in
every nerve from the way in which his look and tone made free of
her, yet telling herself that this momentary endurance of his
mood was the price she must pay for her ultimate power over him,
she tried to calculate the exact point at which concession must
turn to resistance, and the price \textit{he} would have to pay be made
equally clear to him. But his dapper self-confidence seemed
impenetrable to such hints, and she had a sense of something hard
and self-contained behind the superficial warmth of his manner.





They had been seated for some time in the seclusion of a rocky
glen above the lake, when she suddenly cut short the culmination
of an impassioned period by turning upon him the grave loveliness
of her gaze.





``I \textit{do} believe what you say, Mr.\ Rosedale,''\ she said quietly; ``and
I am ready to marry you whenever you wish.''





Rosedale, reddening to the roots of his glossy hair, received
this announcement with a recoil which carried him to his feet,
where he halted before her in an attitude of almost comic
discomfiture.





``For I suppose that is what you do wish,''\ she continued, in the
same quiet tone. ``And, though I was unable to consent when you
spoke to me in this way before, I am ready, now that I know you
so much better, to trust my happiness to your hands.''





She spoke with the noble directness which she could command on
such occasions, and which was like a large steady light thrown
across the tortuous darkness of the situation. In its
inconvenient brightness Rosedale seemed to waver a moment, as
though conscious that every avenue of escape was unpleasantly
illuminated.





Then he gave a short laugh, and drew out a gold cigarette-case,
in which, with plump jewelled fingers, he groped for a
gold-tipped cigarette. Selecting one, he paused to contemplate it
a moment before saying: ``My dear Miss Lily, I'm sorry if there's
been any little misapprehension between us-but you made me feel
my suit was so hopeless that I had really no intention of
renewing it.''





Lily's blood tingled with the grossness of the rebuff; but she checked
the first leap of her anger, and said in a tone of gentle dignity: 
``I have no one but myself to blame if I gave you the impression
that my decision was final.''





Her word-play was always too quick for him, and this reply held
him in puzzled silence while she extended her hand and added,
with the faintest inflection of sadness in her voice: ``Before we
bid each other goodbye, I want at least to thank you for having
once thought of me as you did.''





The touch of her hand, the moving softness of her look, thrilled
a vulnerable fibre in Rosedale. It was her exquisite
inaccessibleness, the sense of distance she could convey without
a hint of disdain, that made it most difficult for him to give
her up.





``Why do you talk of saying goodbye? Ain't we going to be good
friends all the same?''\ he urged, without releasing her hand.





She drew it away quietly. ``What is your idea of being good
friends?''\ she returned with a slight smile. ``Making love to me
without asking me to marry you?''\ Rosedale laughed with a
recovered sense of ease.





``Well, that's about the size of it, I suppose. I can't help
making love to you---I don't see how any man could; but I don't
mean to ask you to marry me as long as I can keep out of it.''





She continued to smile. ``I like your frankness; but I am afraid
our friendship can hardly continue on those terms.'' She turned
away, as though to mark that its final term had in fact been
reached, and he followed her for a few steps with a baffled sense
of her having after all kept the game in her own hands.





``Miss Lily----''\ he began impulsively; but she walked on without
seeming to hear him.





He overtook her in a few quick strides, and laid an entreating
hand on her arm. ``Miss Lily---don't hurry away like that. You're
beastly hard on a fellow; but if you don't mind speaking the
truth I don't see why you shouldn't allow me to do the same.''





She had paused a moment with raised brows, drawing away
instinctively from his touch, though she made no effort to evade
his words.





``I was under the impression,''\ she rejoined, ``that you had done so
without waiting for my permission.''





``Well---why shouldn't you hear my reasons for doing it, then? 
We're neither of us such new hands that a little plain speaking
is going to hurt us. I'm all broken up on you: there's nothing
new in that. I'm more in love with you than I was this time last
year; but I've got to face the fact that the situation is
changed.''





She continued to confront him with the same air of ironic
composure. ``You mean to say that I'm not as desirable a match as
you thought me?''





``Yes; that's what I do mean,''\ he answered resolutely. ``I won't go
into what's happened. I don't believe the stories about you---I
don't \textit{want} to believe them. But they're there, and my not
believing them ain't going to alter the situation.''





She flushed to her temples, but the extremity of her need checked
the retort on her lip and she continued to face him composedly. 
``If they are not true,''\ she said, ``doesn't \textit{that} alter the
situation?''





He met this with a steady gaze of his small stock-taking eyes,
which made her feel herself no more than some superfine human
merchandise. ``I believe it does in novels; but I'm certain it
don't in real life. You know that as well as I do: if we're
speaking the truth, let's speak the whole truth. Last year I was
wild to marry you, and you wouldn't look at me: this year---well,
you appear to be willing. Now, what has changed in the interval? 
Your situation, that's all. Then you thought you could do better;
now----''





``You think you can?''\ broke from her ironically.





``Why, yes, I do: in one way, that is.'' He stood before her, his
hands in his pockets, his chest sturdily expanded under its vivid
waistcoat. ``It's this way, you see: I've had a pretty steady
grind of it these last years, working up my social position. 
Think it's funny I should say that? Why should I mind saying I
want to get into society? A man ain't ashamed to say he wants to
own a racing stable or a picture gallery. Well, a taste for
society's just another kind of hobby. Perhaps I want to get even
with some of the people who cold-shouldered me last year---put it
that way if it sounds better. Anyhow, I want to have the run of
the best houses; and I'm getting it too, little by little. But I
know the quickest way to queer yourself with the right
people is to be seen with the wrong ones; and that's the reason I
want to avoid mistakes.''





Miss Bart continued to stand before him in a silence that might
have expressed either mockery or a half-reluctant respect for his
candour, and after a moment's pause he went on: ``There it is, you
see. I'm more in love with you than ever, but if I married you
now I'd queer myself for good and all, and everything I've worked
for all these years would be wasted.''





She received this with a look from which all tinge of resentment
had faded. After the tissue of social falsehoods in which she had
so long moved it was refreshing to step into the open daylight of
an avowed expediency.





``I understand you,''\ she said. ``A year ago I should have been of
use to you, and now I should be an encumbrance; and I like you
for telling me so quite honestly.'' She extended her hand with a
smile.





Again the gesture had a disturbing effect upon Mr.\ Rosedale's
self-command. ``By George, you're a dead game sport, you are!''\ he
exclaimed; and as she began once more to move away, he broke out
suddenly---``Miss Lily---stop. You know I don't believe those
stories---I believe they were all got up by a woman who didn't
hesitate to sacrifice you to her own convenience----''





Lily drew away with a movement of quick disdain: it was easier to
endure his insolence than his commiseration.





``You are very kind; but I don't think we need discuss the matter
farther.''





But Rosedale's natural imperviousness to hints made it easy for
him to brush such resistance aside. ``I don't want to discuss
anything; I just want to put a plain case before you,''\ he
persisted.





She paused in spite of herself, held by the note of a new purpose
in his look and tone; and he went on, keeping his eyes firmly
upon her: ``The wonder to me is that you've waited so long to get
square with that woman, when you've had the power in your hands.'' 
She continued silent under the rush of astonishment that his
words produced, and he moved a step closer to ask with low-toned
directness: ``Why don't you use those letters of hers you bought
last year?''





Lily stood speechless under the shock of the interrogation. In
the words preceding it she had conjectured, at most, an allusion
to her supposed influence over George Dorset; nor did the
astonishing indelicacy of the reference diminish the likelihood
of Rosedale's resorting to it. But now she saw how far short of
the mark she had fallen; and the surprise of learning that he had
discovered the secret of the letters left her, for the moment,
unconscious of the special use to which he was in the act of
putting his knowledge.





Her temporary loss of self-possession gave him time to press his
point; and he went on quickly, as though to secure completer
control of the situation: ``You see I know where you stand---I know
how completely she's in your power. That sounds like stage-talk,
don't it?---but there's a lot of truth in some of those old gags;
and I don't suppose you bought those letters simply because
you're collecting autographs.''





She continued to look at him with a deepening bewilderment: her
only clear impression resolved itself into a scared sense of his
power.





``You're wondering how I found out about 'em?''\ he went on,
answering her look with a note of conscious pride. ``Perhaps
you've forgotten that I'm the owner of the Benedick-but never
mind about that now. Getting on to things is a mighty useful
accomplishment in business, and I've simply extended it to my
private affairs. For this \textit{is} partly my affair, you see---at least,
it depends on you to make it so. Let's look the situation
straight in the eye. Mrs.\ Dorset, for reasons we needn't go into,
did you a beastly bad turn last spring. Everybody knows what Mrs.
Dorset is, and her best friends wouldn't believe her on oath
where their own interests were concerned; but as long as they're
out of the row it's much easier to follow her lead than to set
themselves against it, and you've simply been sacrificed to their
laziness and selfishness. Isn't that a pretty fair statement of
the case?---Well, some people say you've got the neatest kind of
an answer in your hands: that George Dorset would marry you
tomorrow, if you'd tell him all you know, and give him the chance
to show the lady the door. I daresay he would; but you don't seem
to care for that particular form of getting even, and,
taking a purely business view of the question, I think you're
right. In a deal like that, nobody comes out with perfectly clean
hands, and the only way for you to start fresh is to get Bertha
Dorset to back you up, instead of trying to fight her.''





He paused long enough to draw breath, but not to give her time
for the expression of her gathering resistance; and as he pressed
on, expounding and elucidating his idea with the directness of
the man who has no doubts of his cause, she found the indignation
gradually freezing on her lip, found herself held fast in the
grasp of his argument by the mere cold strength of its
presentation. There was no time now to wonder how he had heard of
her obtaining the letters: all her world was dark outside the
monstrous glare of his scheme for using them. And it was not,
after the first moment, the horror of the idea that held her
spell-bound, subdued to his will; it was rather its subtle
affinity to her own inmost cravings. He would marry her tomorrow
if she could regain Bertha Dorset's friendship; and to induce the
open resumption of that friendship, and the tacit retractation of
all that had caused its withdrawal, she had only to put to the
lady the latent menace contained in the packet so miraculously
delivered into her hands. Lily saw in a flash the advantage of
this course over that which poor Dorset had pressed upon her. The
other plan depended for its success on the infliction of an open
injury, while this reduced the transaction to a private
understanding, of which no third person need have the remotest
hint. Put by Rosedale in terms of business-like give-and-take,
this understanding took on the harmless air of a mutual
accommodation, like a transfer of property or a revision of
boundary lines. It certainly simplified life to view it as a
perpetual adjustment, a play of party politics, in which every
concession had its recognized equivalent: Lily's tired mind was
fascinated by this escape from fluctuating ethical estimates into
a region of concrete weights and measures.





Rosedale, as she listened, seemed to read in her silence not only
a gradual acquiescence in his plan, but a dangerously far-reaching perception of the chances it offered; for as she
continued to stand before him without speaking, he broke out,
with a quick return upon himself: ``You see how simple it is,
don't you? Well, don't be carried away by the idea that it's \textit{too} simple. 
It isn't exactly as if you'd started in with a clean
bill of health. Now we're talking let's call things by
their right names, and clear the whole business up. You know well
enough that Bertha Dorset couldn't have touched you if there
hadn't been---well---questions asked before---little points of
interrogation, eh? Bound to happen to a good-looking girl with
stingy relatives, I suppose; anyhow, they \textit{did} happen, and she
found the ground prepared for her. Do you see where I'm coming
out? You don't want these little questions cropping up again. 
It's one thing to get Bertha Dorset into line---but what you want
is to keep her there. You can frighten her fast enough---but how
are you going to keep her frightened? By showing her that you're
as powerful as she is. All the letters in the world won't do that
for you as you are now; but with a big backing behind you, you'll
keep her just where you want her to be. That's \textit{my} share in the
business---that's what I'm offering you. You can't put the thing
through without me---don't run away with any idea that you can. In
six months you'd be back again among your old worries, or worse
ones; and here I am, ready to lift you out of 'em tomorrow if you
say so. \textit{Do} you say so, Miss Lily?''\ he added, moving suddenly
nearer.





The words, and the movement which accompanied them, combined to
startle Lily out of the state of tranced subservience into which
she had insensibly slipped. Light comes in devious ways to the
groping consciousness, and it came to her now through the
disgusted perception that her would-be accomplice assumed, as a
matter of course, the likelihood of her distrusting him and
perhaps trying to cheat him of his share of the spoils. This
glimpse of his inner mind seemed to present the whole transaction
in a new aspect, and she saw that the essential baseness of the
act lay in its freedom from risk.





She drew back with a quick gesture of rejection, saying, in a
voice that was a surprise to her own ears: ``You are
mistaken---quite mistaken---both in the facts and in what you infer
from them.''





Rosedale stared a moment, puzzled by her sudden dash in a
direction so different from that toward which she had appeared to
be letting him guide her.





``Now what on earth does that mean? I thought we understood each
other!''\ he exclaimed; and to her murmur of ``Ah, we do \textit{now},''\ he
retorted with a sudden burst of violence: ``I suppose it's because
the letters are to \textit{him}, then? Well, I'll be damned if I see what
thanks you've got from him!''





\chapter*{\raggedright Chapter 8}

\addcontentsline{toc}{chapter}{Chapter 8}

\markboth{HOUSE OF MIRTH}{CHAPTER 8}





The autumn days declined to winter. Once more the leisure world
was in transition between country and town, and Fifth Avenue,
still deserted at the week-end, showed from Monday to Friday a
broadening stream of carriages between house-fronts gradually
restored to consciousness.





The Horse Show, some two weeks earlier, had produced a passing
semblance of reanimation, filling the theatres and restaurants
with a human display of the same costly and high-stepping kind as
circled daily about its ring. In Miss Bart's world the Horse
Show, and the public it attracted, had ostensibly come to be
classed among the spectacles disdained of the elect; but, as the
feudal lord might sally forth to join in the dance on his village
green, so society, unofficially and incidentally, still
condescended to look in upon the scene. Mrs.\ Gormer, among the
rest, was not above seizing such an occasion for the display of
herself and her horses; and Lily was given one or two
opportunities of appearing at her friend's side in the most
conspicuous box the house afforded. But this lingering semblance
of intimacy made her only the more conscious of a change in the
relation between Mattie and herself, of a dawning discrimination,
a gradually formed social standard, emerging from Mrs.\ Gormer's
chaotic view of life. It was inevitable that Lily herself should
constitute the first sacrifice to this new ideal, and she knew
that, once the Gormers were established in town, the whole drift
of fashionable life would facilitate Mattie's detachment from
her. She had, in short, failed to make herself indispensable; or
rather, her attempt to do so had been thwarted by an influence
stronger than any she could exert. That influence, in its last
analysis, was simply the power of money: Bertha Dorset's social
credit was based on an impregnable bank-account.





Lily knew that Rosedale had overstated neither the difficulty of
her own position nor the completeness of the vindication he
offered: once Bertha's match in material resources, her superior
gifts would make it easy for her to dominate her adversary. An
understanding of what such domination would mean, and of the
disadvantages accruing from her rejection of it, was
brought home to Lily with increasing clearness during the early
weeks of the winter. Hitherto, she had kept up a semblance of
movement outside the main flow of the social current; but with
the return to town, and the concentrating of scattered
activities, the mere fact of not slipping back naturally into her
old habits of life marked her as being unmistakably excluded from
them. If one were not a part of the season's fixed routine, one
swung unsphered in a void of social non-existence. Lily, for all
her dissatisfied dreaming, had never really conceived the
possibility of revolving about a different centre: it was easy
enough to despise the world, but decidedly difficult to find any
other habitable region. Her sense of irony never quite deserted
her, and she could still note, with self-directed derision, the
abnormal value suddenly acquired by the most tiresome and
insignificant details of her former life. Its very drudgeries had
a charm now that she was involuntarily released from them: 
card-leaving, note-writing, enforced civilities to the dull and
elderly, and the smiling endurance of tedious dinners---how
pleasantly such obligations would have filled the emptiness of
her days! She did indeed leave cards in plenty; she kept herself,
with a smiling and valiant persistence, well in the eye of her
world; nor did she suffer any of those gross rebuffs which
sometimes produce a wholesome reaction of contempt in their
victim. Society did not turn away from her, it simply drifted by,
preoccupied and inattentive, letting her feel, to the full
measure of her humbled pride, how completely she had been the
creature of its favour.





She had rejected Rosedale's suggestion with a promptness of scorn
almost surprising to herself: she had not lost her capacity for
high flashes of indignation. But she could not breathe long on
the heights; there had been nothing in her training to develop
any continuity of moral strength: what she craved, and really
felt herself entitled to, was a situation in which the noblest
attitude should also be the easiest. Hitherto her intermittent
impulses of resistance had sufficed to maintain her self-respect. 
If she slipped she recovered her footing, and it was only
afterward that she was aware of having recovered it each time on
a slightly lower level. She had rejected Rosedale's offer without
conscious effort; her whole being had risen against it;
and she did not yet perceive that, by the mere act of listening
to him, she had learned to live with ideas which would once have
been intolerable to her.








To Gerty Farish, keeping watch over her with a tenderer if less
discerning eye than Mrs.\ Fisher's, the results of the struggle
were already distinctly visible. She did not, indeed, know what
hostages Lily had already given to expediency; but she saw her
passionately and irretrievably pledged to the ruinous policy of
``keeping up.'' Gerty could smile now at her own early dream of her
friend's renovation through adversity: she understood clearly
enough that Lily was not of those to whom privation teaches the
unimportance of what they have lost. But this very fact, to
Gerty, made her friend the more piteously in want of aid, the
more exposed to the claims of a tenderness she was so little
conscious of needing.





Lily, since her return to town, had not often climbed Miss
Farish's stairs. There was something irritating to her in the
mute interrogation of Gerty's sympathy: she felt the real
difficulties of her situation to be incommunicable to any one
whose theory of values was so different from her own, and the
restrictions of Gerty's life, which had once had the charm of
contrast, now reminded her too painfully of the limits to which
her own existence was shrinking. When at length, one afternoon,
she put into execution the belated resolve to visit her friend,
this sense of shrunken opportunities possessed her with unusual
intensity. The walk up Fifth Avenue, unfolding before her, in the
brilliance of the hard winter sunlight, an interminable
procession of fastidiously-equipped carriages---giving her,
through the little squares of brougham-windows, peeps of familiar
profiles bent above visiting-lists, of hurried hands dispensing
notes and cards to attendant footmen---this glimpse of the
ever-revolving wheels of the great social machine made Lily more
than ever conscious of the steepness and narrowness of Gerty's
stairs, and of the cramped blind alley of life to which they led. 
Dull stairs destined to be mounted by dull people: how many
thousands of insignificant figures were going up and down such
stairs all over the world at that very moment---figures as shabby
and uninteresting as that of the middle-aged lady in limp
black who descended Gerty's flight as Lily climbed to it!





``That was poor Miss Jane Silverton---she came to talk things over
with me: she and her sister want to do something to support
themselves,''\ Gerty explained, as Lily followed her into the
sitting-room.





``To support themselves? Are they so hard up?''\ Miss Bart asked
with a touch of irritation: she had not come to listen to the
woes of other people.





``I'm afraid they have nothing left: Ned's debts have swallowed up
everything. They had such hopes, you know, when he broke away
from Carry Fisher; they thought Bertha Dorset would be such a
good influence, because she doesn't care for cards, and---well,
she talked quite beautifully to poor Miss Jane about feeling as
if Ned were her younger brother, and wanting to carry him off on
the yacht, so that he might have a chance to drop cards and
racing, and take up his literary work again.''





Miss Farish paused with a sigh which reflected the perplexity of
her departing visitor. ``But that isn't all; it isn't even the
worst. It seems that Ned has quarrelled with the Dorsets; or at
least Bertha won't allow him to see her, and he is so unhappy
about it that he has taken to gambling again, and going about
with all sorts of queer people. And cousin Grace Van Osburgh
accuses him of having had a very bad influence on Freddy, who
left Harvard last spring, and has been a great deal with Ned ever
since. She sent for Miss Jane, and made a dreadful scene; and
Jack Stepney and Herbert Melson, who were there too, told Miss
Jane that Freddy was threatening to marry some dreadful woman to
whom Ned had introduced him, and that they could do nothing with
him because now he's of age he has his own money. You can fancy
how poor Miss Jane felt---she came to me at once, and seemed to
think that if I could get her something to do she could earn
enough to pay Ned's debts and send him away---I'm afraid she has
no idea how long it would take her to pay for one of his evenings
at bridge. And he was horribly in debt when he came back from the
cruise---I can't see why he should have spent so much more money
under Bertha's influence than Carry's: can you?''





Lily met this query with an impatient gesture. ``My dear Gerty, I
always understand how people can spend much more money---never how
they can spend any less!''





She loosened her furs and settled herself in Gerty's easy-chair,
while her friend busied herself with the tea-cups.





``But what can they do---the Miss Silvertons? How do they mean to
support themselves?''\ she asked, conscious that the note of
irritation still persisted in her voice. It was the very last
topic she had meant to discuss---it really did not interest her in
the least---but she was seized by a sudden perverse curiosity to
know how the two colourless shrinking victims of young
Silverton's sentimental experiments meant to cope with the grim
necessity which lurked so close to her own threshold.





``I don't know---I am trying to find something for them. Miss Jane
reads aloud very nicely---but it's so hard to find any one who is
willing to be read to. And Miss Annie paints a little----''





``Oh, I know---apple-blossoms on blotting-paper; just the kind of
thing I shall be doing myself before long!''\ exclaimed Lily,
starting up with a vehemence of movement that threatened
destruction to Miss Farish's fragile tea-table.





Lily bent over to steady the cups; then she sank back into her
seat. ``I'd forgotten there was no room to dash about in---how
beautifully one does have to behave in a small flat! Oh, Gerty, I
wasn't meant to be good,''\ she sighed out incoherently.





Gerty lifted an apprehensive look to her pale face, in which the
eyes shone with a peculiar sleepless lustre.





``You look horribly tired, Lily; take your tea, and let me give
you this cushion to lean against.''





Miss Bart accepted the cup of tea, but put back the cushion with
an impatient hand.





``Don't give me that! I don't want to lean back---I shall go to
sleep if I do.''





``Well, why not, dear? I'll be as quiet as a mouse,''\ Gerty urged
affectionately.





``No---no; don't be quiet; talk to me---keep me awake! I don't sleep
at night, and in the afternoon a dreadful drowsiness creeps over
me.''





``You don't sleep at night? Since when?''





``I don't know---I can't remember.'' She rose and put the empty cup
on the tea-tray. ``Another, and stronger, please; if I don't keep
awake now I shall see horrors tonight---perfect horrors!''





``But they'll be worse if you drink too much tea.''





``No, no---give it to me; and don't preach, please,''\ Lily returned
imperiously. Her voice had a dangerous edge, and Gerty noticed
that her hand shook as she held it out to receive the second cup.





``But you look so tired: I'm sure you must be ill----''





Miss Bart set down her cup with a start. ``Do I look ill? Does my
face show it?''\ She rose and walked quickly toward the little
mirror above the writing-table. ``What a horrid
looking-glass---it's all blotched and discoloured. Any one would
look ghastly in it!''\ She turned back, fixing her plaintive eyes
on Gerty. ``You stupid dear, why do you say such odious things to
me? It's enough to make one ill to be told one looks so! And
looking ill means looking ugly.'' She caught Gerty's wrists, and
drew her close to the window. ``After all, I'd rather know the
truth. Look me straight in the face, Gerty, and tell me: am I
perfectly frightful?''





``You're perfectly beautiful now, Lily: your eyes are shining, and
your cheeks have grown so pink all of a sudden----''





``Ah, they \textit{were} pale, then---ghastly pale, when I came in? Why
don't you tell me frankly that I'm a wreck? My eyes are bright
now because I'm so nervous---but in the mornings they look like
lead. And I can see the lines coming in my face---the lines of
worry and disappointment and failure! Every sleepless night
leaves a new one---and how can I sleep, when I have such dreadful
things to think about?''





``Dreadful things---what things?''\ asked Gerty, gently detaching her
wrists from her friend's feverish fingers.





``What things? Well, poverty, for one---and I don't know any that's
more dreadful.'' Lily turned away and sank with sudden weariness
into the easy-chair near the tea-table. ``You asked me just now if
I could understand why Ned Silverton spent so much money. Of
course I understand---he spends it on living with the rich. You
think we live \textit{on} the rich, rather than with them: and so we do,
in a sense---but it's a privilege we have to pay for! We eat their
dinners, and drink their wine, and smoke their cigarettes, and use
their carriages and their opera-boxes and their private cars---yes,
but there's a tax to pay on every one of those luxuries. The man
pays it by big tips to the servants, by playing cards beyond his
means, by flowers and presents---and---and---lots of other things
that cost; the girl pays it by tips and cards too---oh, yes, I've
had to take up bridge again---and by going to the best dress-makers,
and having just the right dress for every occasion, and always
keeping herself fresh and exquisite and amusing!''





She leaned back for a moment, closing her eyes, and as she sat
there, her pale lips slightly parted, and the lids dropped above
her fagged brilliant gaze, Gerty had a startled perception of the
change in her face---of the way in which an ashen daylight seemed
suddenly to extinguish its artificial brightness. She looked up,
and the vision vanished.





``It doesn't sound very amusing, does it? And it isn't---I'm sick
to death of it! And yet the thought of giving it all up nearly
kills me---it's what keeps me awake at night, and makes me so
crazy for your strong tea. For I can't go on in this way much
longer, you know---I'm nearly at the end of my tether. And then
what can I do---how on earth am I to keep myself alive? I see
myself reduced to the fate of that poor Silverton woman---slinking
about to employment agencies, and trying to sell painted
blotting-pads to Women's Exchanges! And there are thousands and
thousands of women trying to do the same thing already, and not
one of the number who has less idea how to earn a dollar than I
have!''





She rose again with a hurried glance at the clock. ``It's late,
and I must be off---I have an appointment with Carry Fisher. Don't
look so worried, you dear thing---don't think too much about the
nonsense I've been talking.'' She was before the mirror again,
adjusting her hair with a light hand, drawing down her veil, and
giving a dexterous touch to her furs. ``Of course, you know, it
hasn't come to the employment agencies and the painted
blotting-pads yet; but I'm rather hard-up just for the moment,
and if I could find something to do---notes to write and
visiting-lists to make up, or that kind of thing---it would tide
me over till the legacy is paid. And Carry has promised to find
somebody who wants a kind of social secretary---you know she makes
a specialty of the helpless rich.''








Miss Bart had not revealed to Gerty the full extent of her
anxiety. She was in fact in urgent and immediate need of money: 
money to meet the vulgar weekly claims which could neither be
deferred nor evaded. To give up her apartment, and shrink to the
obscurity of a boarding-house, or the provisional hospitality of
a bed in Gerty Farish's sitting-room, was an expedient which
could only postpone the problem confronting her; and it seemed
wiser as well as more agreeable to remain where she was and find
some means of earning her living. The possibility of having to do
this was one which she had never before seriously considered, and
the discovery that, as a bread-winner, she was likely to prove as
helpless and ineffectual as poor Miss Silverton, was a severe
shock to her self-confidence.





Having been accustomed to take herself at the popular valuation,
as a person of energy and resource, naturally fitted to dominate
any situation in which she found herself, she vaguely imagined
that such gifts would be of value to seekers after social
guidance; but there was unfortunately no specific head under
which the art of saying and doing the right thing could be
offered in the market, and even Mrs.\ Fisher's resourcefulness
failed before the difficulty of discovering a workable vein in
the vague wealth of Lily's graces. Mrs.\ Fisher was full of
indirect expedients for enabling her friends to earn a living,
and could conscientiously assert that she had put several
opportunities of this kind before Lily; but more legitimate
methods of bread-winning were as much out of her line as they
were beyond the capacity of the sufferers she was generally
called upon to assist. Lily's failure to profit by the chances
already afforded her might, moreover, have justified the
abandonment of farther effort on her behalf; but Mrs.\ Fisher's
inexhaustible good-nature made her an adept at creating
artificial demands in response to an actual supply. In the
pursuance of this end she at once started on a voyage of
discovery in Miss Bart's behalf; and as the result of her
explorations she now summoned the latter with the announcement
that she had ``found something.''








Left to herself, Gerty mused distressfully upon her friend's
plight, and her own inability to relieve it. It was clear to her
that Lily, for the present, had no wish for the kind of help she
could give. Miss Farish could see no hope for her friend but in a
life completely reorganized and detached from its old
associations; whereas all Lily's energies were centred in the
determined effort to hold fast to those associations, to keep
herself visibly identified with them, as long as the illusion
could be maintained. Pitiable as such an attitude seemed to
Gerty, she could not judge it as harshly as Selden, for instance,
might have done. She had not forgotten the night of emotion when
she and Lily had lain in each other's arms, and she had seemed to
feel her very heart's blood passing into her friend. The
sacrifice she had made had seemed unavailing enough; no trace
remained in Lily of the subduing influences of that hour; but
Gerty's tenderness, disciplined by long years of contact with
obscure and inarticulate suffering, could wait on its object with
a silent forbearance which took no account of time. She could
not, however, deny herself the solace of taking anxious counsel
with Lawrence Selden, with whom, since his return from Europe,
she had renewed her old relation of cousinly confidence.





Selden himself had never been aware of any change in their
relation. He found Gerty as he had left her, simple, undemanding
and devoted, but with a quickened intelligence of the heart which
he recognized without seeking to explain it. To Gerty herself it
would once have seemed impossible that she should ever again talk
freely with him of Lily Bart; but what had passed in the secrecy
of her own breast seemed to resolve itself, when the mist of the
struggle cleared, into a breaking down of the bounds of self, a
deflecting of the wasted personal emotion into the general
current of human understanding.





It was not till some two weeks after her visit from Lily that
Gerty had the opportunity of communicating her fears to Selden. 
The latter, having presented himself on a Sunday afternoon, had
lingered on through the dowdy animation of his cousin's
tea-hour, conscious of something in her voice and eye which
solicited a word apart; and as soon as the last visitor was gone
Gerty opened her case by asking how lately he had seen Miss Bart.





Selden's perceptible pause gave her time for a slight stir of
surprise.





``I haven't seen her at all---I've perpetually missed seeing her
since she came back.''





This unexpected admission made Gerty pause too; and she was still
hesitating on the brink of her subject when he relieved her by
adding: ``I've wanted to see her---but she seems to have been
absorbed by the Gormer set since her return from Europe.''





``That's all the more reason: she's been very unhappy.''





``Unhappy at being with the Gormers?''





``Oh, I don't defend her intimacy with the Gormers; but that too
is at an end now, I think. You know people have been very unkind
since Bertha Dorset quarrelled with her.''





``Ah----''\ Selden exclaimed, rising abruptly to walk to the window,
where he remained with his eyes on the darkening street while his
cousin continued to explain: ``Judy Trenor and her own family have
deserted her too---and all because Bertha Dorset has said such
horrible things. And she is very poor---you know Mrs.\ Peniston cut
her off with a small legacy, after giving her to understand that
she was to have everything.''





``Yes---I know,''\ Selden assented curtly, turning back into the
room, but only to stir about with restless steps in the
circumscribed space between door and window. ``Yes---she's been
abominably treated; but it's unfortunately the precise thing that
a man who wants to show his sympathy can't say to her.''





His words caused Gerty a slight chill of disappointment. ``There
would be other ways of showing your sympathy,''\ she suggested.





Selden, with a slight laugh, sat down beside her on the little
sofa which projected from the hearth. ``What are you thinking of,
you incorrigible missionary?''\ he asked.





Gerty's colour rose, and her blush was for a moment her only
answer. Then she made it more explicit by saying: ``I am
thinking of the fact that you and she used to be great
friends---that she used to care immensely for what you thought of
her---and that, if she takes your staying away as a sign of what
you think now, I can imagine its adding a great deal to her
unhappiness.''





``My dear child, don't add to it still more---at least to your
conception of it---by attributing to her all sorts of
susceptibilities of your own.'' Selden, for his life, could not
keep a note of dryness out of his voice; but he met Gerty's look
of perplexity by saying more mildly: ``But, though you immensely
exaggerate the importance of anything I could do for Miss Bart,
you can't exaggerate my readiness to do it---if you ask me to.'' He
laid his hand for a moment on hers, and there passed between
them, on the current of the rare contact, one of those exchanges
of meaning which fill the hidden reservoirs of affection. Gerty
had the feeling that he measured the cost of her request as
plainly as she read the significance of his reply; and the sense
of all that was suddenly clear between them made her next words
easier to find.





``I do ask you, then; I ask you because she once told me that you
had been a help to her, and because she needs help now as she has
never needed it before. You know how dependent she has always
been on ease and luxury---how she has hated what was shabby and
ugly and uncomfortable. She can't help it---she was brought up
with those ideas, and has never been able to find her way out of
them. But now all the things she cared for have been taken from
her, and the people who taught her to care for them have
abandoned her too; and it seems to me that if some one could
reach out a hand and show her the other side---show her how much
is left in life and in herself----''\ Gerty broke off, abashed at
the sound of her own eloquence, and impeded by the difficulty of
giving precise expression to her vague yearning for her friend's
retrieval. ``I can't help her myself: she's passed out of my
reach,''\ she continued. ``I think she's afraid of being a burden to
me. When she was last here, two weeks ago, she seemed dreadfully
worried about her future: she said Carry Fisher was trying to
find something for her to do. A few days later she wrote me that
she had taken a position as private secretary, and that I was not
to be anxious, for everything was all right, and she
would come in and tell me about it when she had time; but she has
never come, and I don't like to go to her, because I am afraid of
forcing myself on her when I'm not wanted. Once, when we were
children, and I had rushed up after a long separation, and thrown
my arms about her, she said: 'Please don't kiss me unless I ask
you to, Gerty'---and she \textit{did} ask me, a minute later; but since
then I've always waited to be asked.''





Selden had listened in silence, with the concentrated look which
his thin dark face could assume when he wished to guard it
against any involuntary change of expression. When his cousin
ended, he said with a slight smile: ``Since you've learned the
wisdom of waiting, I don't see why you urge me to rush in---''\ but
the troubled appeal of her eyes made him add, as he rose to take
leave: ``Still, I'll do what you wish, and not hold you
responsible for my failure.''





Selden's avoidance of Miss Bart had not been as unintentional as
he had allowed his cousin to think. At first, indeed, while the
memory of their last hour at Monte Carlo still held the full heat
of his indignation, he had anxiously watched for her return; but
she had disappointed him by lingering in England, and when she
finally reappeared it happened that business had called him to the
West, whence he came back only to learn that she was starting for
Alaska with the Gormers. The revelation of this suddenly-established
intimacy effectually chilled his desire to see her. If, at a
moment when her whole life seemed to be breaking up, she could
cheerfully commit its reconstruction to the Gormers, there was no
reason why such accidents should ever strike her as irreparable. 
Every step she took seemed in fact to carry her farther from the
region where, once or twice, he and she had met for an illumined
moment; and the recognition of this fact, when its first pang had
been surmounted, produced in him a sense of negative relief. It was
much simpler for him to judge Miss Bart by her habitual conduct
than by the rare deviations from it which had thrown her so
disturbingly in his way; and every act of hers which made the
recurrence of such deviations more unlikely, confirmed the sense
of relief with which he returned to the conventional view of her.





But Gerty Farish's words had sufficed to make him see how
little this view was really his, and how impossible it was for
him to live quietly with the thought of Lily Bart. To hear that
she was in need of help---even such vague help as he could
offer---was to be at once repossessed by that thought; and by the
time he reached the street he had sufficiently convinced himself
of the urgency of his cousin's appeal to turn his steps directly
toward Lily's hotel.





There his zeal met a check in the unforeseen news that Miss Bart
had moved away; but, on his pressing his enquiries, the clerk
remembered that she had left an address, for which he presently
began to search through his books.





It was certainly strange that she should have taken this step
without letting Gerty Farish know of her decision; and Selden
waited with a vague sense of uneasiness while the address was
sought for. The process lasted long enough for uneasiness to turn
to apprehension; but when at length a slip of paper was handed
him, and he read on it: ``Care of Mrs.\ Norma Hatch, Emporium
Hotel,''\ his apprehension passed into an incredulous stare, and
this into the gesture of disgust with which he tore the paper in
two, and turned to walk quickly homeward.





\chapter*{\raggedright Chapter 9}

\addcontentsline{toc}{chapter}{Chapter 9}

\markboth{HOUSE OF MIRTH}{CHAPTER 9}





When Lily woke on the morning after her translation to the
Emporium Hotel, her first feeling was one of purely physical
satisfaction. The force of contrast gave an added keenness to the
luxury of lying once more in a soft-pillowed bed, and looking
across a spacious sunlit room at a breakfast-table set invitingly
near the fire. Analysis and introspection might come later; but
for the moment she was not even troubled by the excesses of the
upholstery or the restless convolutions of the furniture. The
sense of being once more lapped and folded in ease, as in some
dense mild medium impenetrable to discomfort, effectually stilled
the faintest note of criticism.





When, the afternoon before, she had presented herself to the lady
to whom Carry Fisher had directed her, she had been conscious of
entering a new world. Carry's vague presentment of Mrs.\ Norma
Hatch (whose reversion to her Christian name was explained as the
result of her latest divorce), left her under the implication of
coming ``from the West,''\ with the not unusual extenuation of
having brought a great deal of money with her. She was, in short,
rich, helpless, unplaced: the very subject for Lily's hand. Mrs.
Fisher had not specified the line her friend was to take; she
owned herself unacquainted with Mrs.\ Hatch, whom she ``knew about''
through Melville Stancy, a lawyer in his leisure moments, and the
Falstaff of a certain section of festive club life. Socially, Mr.
Stancy might have been said to form a connecting link between the
Gormer world and the more dimly-lit region on which Miss Bart now
found herself entering. It was, however, only figuratively that
the illumination of Mrs.\ Hatch's world could be described as dim: 
in actual fact, Lily found her seated in a blaze of electric
light, impartially projected from various ornamental excrescences
on a vast concavity of pink damask and gilding, from which she
rose like Venus from her shell. The analogy was justified by the
appearance of the lady, whose large-eyed prettiness had the
fixity of something impaled and shown under glass. This did not
preclude the immediate discovery that she was some years younger
than her visitor, and that under her showiness, her ease,
the aggression of her dress and voice, there persisted that
ineradicable innocence which, in ladies of her nationality, so
curiously coexists with startling extremes of experience.





The environment in which Lily found herself was as strange to her
as its inhabitants. She was unacquainted with the world of the
fashionable New York hotel---a world over-heated,
over-upholstered, and over-fitted with mechanical appliances for
the gratification of fantastic requirements, while the comforts
of a civilized life were as unattainable as in a desert. Through
this atmosphere of torrid splendour moved wan beings as richly
upholstered as the furniture, beings without definite pursuits or
permanent relations, who drifted on a languid tide of curiosity
from restaurant to concert-hall, from palm-garden to music-room,
from ``art exhibit''\ to dress-maker's opening. High-stepping horses
or elaborately equipped motors waited to carry these ladies into
vague metropolitan distances, whence they returned, still more
wan from the weight of their sables, to be sucked back into the
stifling inertia of the hotel routine. Somewhere behind them, in
the background of their lives, there was doubtless a real past,
peopled by real human activities: they themselves were probably
the product of strong ambitions, persistent energies, diversified
contacts with the wholesome roughness of life; yet they had no
more real existence than the poet's shades in limbo.





Lily had not been long in this pallid world without discovering
that Mrs.\ Hatch was its most substantial figure. That lady,
though still floating in the void, showed faint symptoms of
developing an outline; and in this endeavour she was actively
seconded by Mr.\ Melville Stancy. It was Mr.\ Stancy, a man of
large resounding presence, suggestive of convivial occasions and
of a chivalry finding expression in ``first-night''\ boxes and
thousand dollar bonbonnieres, who had transplanted Mrs.\ Hatch
from the scene of her first development to the higher stage of
hotel life in the metropolis. It was he who had selected the
horses with which she had taken the blue ribbon at the Show, had
introduced her to the photographer whose portraits of her formed
the recurring ornament of ``Sunday Supplements,''\ and had got
together the group which constituted her social world. It was a
small group still, with heterogeneous figures suspended
in large unpeopled spaces; but Lily did not take long to learn
that its regulation was no longer in Mr.\ Stancy's hands. As often
happens, the pupil had outstripped the teacher, and Mrs.\ Hatch
was already aware of heights of elegance as well as depths of
luxury beyond the world of the Emporium. This discovery at once
produced in her a craving for higher guidance, for the adroit
feminine hand which should give the right turn to her
correspondence, the right ``look''\ to her hats, the right
succession to the items of her \textit{menus}. It was, in short, as the
regulator of a germinating social life that Miss Bart's guidance
was required; her ostensible duties as secretary being restricted
by the fact that Mrs.\ Hatch, as yet, knew hardly any one to write
to.





The daily details of Mrs.\ Hatch's existence were as strange to
Lily as its general tenor. The lady's habits were marked by an
Oriental indolence and disorder peculiarly trying to her
companion. Mrs.\ Hatch and her friends seemed to float together
outside the bounds of time and space. No definite hours were
kept; no fixed obligations existed: night and day flowed into one
another in a blur of confused and retarded engagements, so that
one had the impression of lunching at the tea-hour, while dinner
was often merged in the noisy after-theatre supper which
prolonged Mrs.\ Hatch's vigil till daylight.





Through this jumble of futile activities came and went a strange
throng of hangers-on---manicures, beauty-doctors, hair-dressers,
teachers of bridge, of French, of ``physical development'': figures
sometimes indistinguishable, by their appearance, or by Mrs.
Hatch's relation to them, from the visitors constituting her
recognized society. But strangest of all to Lily was the
encounter, in this latter group, of several of her acquaintances. 
She had supposed, and not without relief, that she was passing,
for the moment, completely out of her own circle; but she found
that Mr.\ Stancy, one side of whose sprawling existence overlapped
the edge of Mrs.\ Fisher's world, had drawn several of its
brightest ornaments into the circle of the Emporium. To find Ned
Silverton among the habitual frequenters of Mrs.\ Hatch's
drawing-room was one of Lily's first astonishments; but she soon
discovered that he was not Mr.\ Stancy's most important
recruit. It was on little Freddy Van Osburgh, the small slim heir
of the Van Osburgh millions, that the attention of Mrs.\ Hatch's
group was centred. Freddy, barely out of college, had risen above
the horizon since Lily's eclipse, and she now saw with surprise
what an effulgence he shed on the outer twilight of Mrs.\ Hatch's
existence. This, then, was one of the things that young men ``went
in''\ for when released from the official social routine; this
was the kind of ``previous engagement''\ that so frequently caused
them to disappoint the hopes of anxious hostesses. Lily had an
odd sense of being behind the social tapestry, on the side where
the threads were knotted and the loose ends hung. For a moment
she found a certain amusement in the show, and in her own share
of it: the situation had an ease and unconventionality distinctly
refreshing after her experience of the irony of conventions. But
these flashes of amusement were but brief reactions from the long
disgust of her days. Compared with the vast gilded void of Mrs.
Hatch's existence, the life of Lily's former friends seemed
packed with ordered activities. Even the most irresponsible
pretty woman of her acquaintance had her inherited obligations,
her conventional benevolences, her share in the working of the
great civic machine; and all hung together in the solidarity of
these traditional functions. The performance of specific duties
would have simplified Miss Bart's position; but the vague
attendance on Mrs.\ Hatch was not without its perplexities.





It was not her employer who created these perplexities. Mrs.
Hatch showed from the first an almost touching desire for Lily's
approval. Far from asserting the superiority of wealth, her
beautiful eyes seemed to urge the plea of inexperience: she
wanted to do what was ``nice,''\ to be taught how to be ``lovely.'' 
The difficulty was to find any point of contact between her
ideals and Lily's.





Mrs.\ Hatch swam in a haze of indeterminate enthusiasms, of
aspirations culled from the stage, the newspapers, the fashion
journals, and a gaudy world of sport still more completely beyond
her companion's ken. To separate from these confused conceptions
those most likely to advance the lady on her way, was Lily's
obvious duty; but its performance was hampered by
rapidly-growing doubts. Lily was in fact becoming more and more
aware of a certain ambiguity in her situation. It was not that
she had, in the conventional sense, any doubt of Mrs.\ Hatch's
irreproachableness. The lady's offences were always against taste
rather than conduct; her divorce record seemed due to
geographical rather than ethical conditions; and her worst
laxities were likely to proceed from a wandering and extravagant
good-nature. But if Lily did not mind her detaining her manicure
for luncheon, or offering the ``Beauty-Doctor''\ a seat in Freddy
Van Osburgh's box at the play, she was not equally at ease in
regard to some less apparent lapses from convention. Ned
Silverton's relation to Stancy seemed, for instance, closer and
less clear than any natural affinities would warrant; and both
appeared united in the effort to cultivate Freddy Van Osburgh's
growing taste for Mrs.\ Hatch. There was as yet nothing definable
in the situation, which might well resolve itself into a huge
joke on the part of the other two; but Lily had a vague sense
that the subject of their experiment was too young, too rich and
too credulous. Her embarrassment was increased by the fact that
Freddy seemed to regard her as cooperating with himself in the
social development of Mrs.\ Hatch: a view that suggested, on his
part, a permanent interest in the lady's future. There were
moments when Lily found an ironic amusement in this aspect of the
case. The thought of launching such a missile as Mrs.\ Hatch at
the perfidious bosom of society was not without its charm: Miss
Bart had even beguiled her leisure with visions of the fair Norma
introduced for the first time to a family banquet at the Van
Osburghs'. But the thought of being personally connected with the
transaction was less agreeable; and her momentary flashes of
amusement were followed by increasing periods of doubt.





The sense of these doubts was uppermost when, late one afternoon,
she was surprised by a visit from Lawrence Selden. He found her
alone in the wilderness of pink damask, for in Mrs.\ Hatch's world
the tea-hour was not dedicated to social rites, and the lady was
in the hands of her masseuse.





Selden's entrance had caused Lily an inward start of
embarrassment; but his air of constraint had the effect of
restoring her self-possession, and she took at once the tone of
surprise and pleasure, wondering frankly that he should
have traced her to so unlikely a place, and asking what had
inspired him to make the search.





Selden met this with an unusual seriousness: she had never seen
him so little master of the situation, so plainly at the mercy of
any obstructions she might put in his way. ``I wanted to see you,''
he said; and she could not resist observing in reply that he had
kept his wishes under remarkable control. She had in truth felt
his long absence as one of the chief bitternesses of the last
months: his desertion had wounded sensibilities far below the
surface of her pride.





Selden met the challenge with directness. ``Why should I have
come, unless I thought I could be of use to you? It is my only
excuse for imagining you could want me.''





This struck her as a clumsy evasion, and the thought gave a flash
of keenness to her answer. ``Then you have come now because you
think you can be of use to me?''





He hesitated again. ``Yes: in the modest capacity of a person to
talk things over with.''





For a clever man it was certainly a stupid beginning; and the
idea that his awkwardness was due to the fear of her attaching a
personal significance to his visit, chilled her pleasure in
seeing him. Even under the most adverse conditions, that pleasure
always made itself felt: she might hate him, but she had never
been able to wish him out of the room. She was very near hating
him now; yet the sound of his voice, the way the light fell on
his thin dark hair, the way he sat and moved and wore his
clothes---she was conscious that even these trivial things were
inwoven with her deepest life. In his presence a sudden stillness
came upon her, and the turmoil of her spirit ceased; but an
impulse of resistance to this stealing influence now prompted her
to say: ``It's very good of you to present yourself in that
capacity; but what makes you think I have anything particular to
talk about?''





Though she kept the even tone of light intercourse, the question
was framed in a way to remind him that his good offices were
unsought; and for a moment Selden was checked by it. The
situation between them was one which could have been cleared up
only by a sudden explosion of feeling; and their whole training
and habit of mind were against the chances of such an
explosion. Selden's calmness seemed rather to harden into
resistance, and Miss Bart's into a surface of glittering irony,
as they faced each other from the opposite corners of one of Mrs.
Hatch's elephantine sofas. The sofa in question, and the
apartment peopled by its monstrous mates, served at length to
suggest the turn of Selden's reply.





``Gerty told me that you were acting as Mrs.\ Hatch's secretary;
and I knew she was anxious to hear how you were getting on.''





Miss Bart received this explanation without perceptible
softening. ``Why didn't she look me up herself, then?''\ she asked.





``Because, as you didn't send her your address, she was afraid of
being importunate.'' Selden continued with a smile: ``You see no
such scruples restrained me; but then I haven't as much to risk
if I incur your displeasure.''





Lily answered his smile. ``You haven't incurred it as yet; but I
have an idea that you are going to.''





``That rests with you, doesn't it? You see my initiative doesn't
go beyond putting myself at your disposal.''





``But in what capacity? What am I to do with you?''\ she asked in
the same light tone.





Selden again glanced about Mrs.\ Hatch's drawing-room; then he
said, with a decision which he seemed to have gathered from this
final inspection: ``You are to let me take you away from here.''





Lily flushed at the suddenness of the attack; then she stiffened
under it and said coldly: ``And may I ask where you mean me to
go?''





``Back to Gerty in the first place, if you will; the essential
thing is that it should be away from here.''





The unusual harshness of his tone might have shown her how much
the words cost him; but she was in no state to measure his
feelings while her own were in a flame of revolt. To neglect her,
perhaps even to avoid her, at a time when she had most need of
her friends, and then suddenly and unwarrantably to break into
her life with this strange assumption of authority, was to rouse
in her every instinct of pride and self-defence.





``I am very much obliged to you,''\ she said, ``for taking such
an interest in my plans; but I am quite contented where I am,
and have no intention of leaving.''





Selden had risen, and was standing before her in an attitude of
uncontrollable expectancy.





``That simply means that you don't know where you are!''\ he
exclaimed.





Lily rose also, with a quick flash of anger. ``If you have come
here to say disagreeable things about Mrs.\ Hatch----''





``It is only with your relation to Mrs.\ Hatch that I am
concerned.''





``My relation to Mrs.\ Hatch is one I have no reason to be ashamed
of. She has helped me to earn a living when my old friends were
quite resigned to seeing me starve.''





``Nonsense! Starvation is not the only alternative. You know you
can always find a home with Gerty till you are independent
again.''





``You show such an intimate acquaintance with my affairs that I
suppose you mean---till my aunt's legacy is paid?''





``I do mean that; Gerty told me of it,''\ Selden acknowledged
without embarrassment. He was too much in earnest now to feel any
false constraint in speaking his mind.





``But Gerty does not happen to know,''\ Miss Bart rejoined, ``that I
owe every penny of that legacy.''





``Good God!''\ Selden exclaimed, startled out of his composure by
the abruptness of the statement.





``Every penny of it, and more too,''\ Lily repeated; ``and you now
perhaps see why I prefer to remain with Mrs.\ Hatch rather than
take advantage of Gerty's kindness. I have no money left, except
my small income, and I must earn something more to keep myself
alive.''





Selden hesitated a moment; then he rejoined in a quieter tone: 
``But with your income and Gerty's---since you allow me to go so
far into the details of the situation---you and she could surely
contrive a life together which would put you beyond the need of
having to support yourself. Gerty, I know, is eager to make such
an arrangement, and would be quite happy in it----''





``But I should not,''\ Miss Bart interposed. ``There are many reasons
why it would be neither kind to Gerty nor wise for myself.'' She
paused a moment, and as he seemed to await a farther
explanation, added with a quick lift of her head: ``You will
perhaps excuse me from giving you these reasons.''





``I have no claim to know them,''\ Selden answered, ignoring her
tone; ``no claim to offer any comment or suggestion beyond the one
I have already made. And my right to make that is simply the
universal right of a man to enlighten a woman when he sees her
unconsciously placed in a false position.''





Lily smiled. ``I suppose,''\ she rejoined, ``that by a false position
you mean one outside of what we call society; but you must
remember that I had been excluded from those sacred precincts
long before I met Mrs.\ Hatch. As far as I can see, there is very
little real difference in being inside or out, and I remember
your once telling me that it was only those inside who took the
difference seriously.''





She had not been without intention in making this allusion to
their memorable talk at Bellomont, and she waited with an odd
tremor of the nerves to see what response it would bring; but the
result of the experiment was disappointing. Selden did not allow
the allusion to deflect him from his point; he merely said with
completer fulness of emphasis: ``The question of being inside or
out is, as you say, a small one, and it happens to have nothing
to do with the case, except in so far as Mrs.\ Hatch's desire to
be inside may put you in the position I call false.''





In spite of the moderation of his tone, each word he spoke had
the effect of confirming Lily's resistance. The very
apprehensions he aroused hardened her against him: she had been
on the alert for the note of personal sympathy, for any sign of
recovered power over him; and his attitude of sober impartiality,
the absence of all response to her appeal, turned her hurt pride
to blind resentment of his interference. The conviction that he
had been sent by Gerty, and that, whatever straits he conceived
her to be in, he would never voluntarily have come to her aid,
strengthened her resolve not to admit him a hair's breadth
farther into her confidence. However doubtful she might feel her
situation to be, she would rather persist in darkness than owe
her enlightenment to Selden.





``I don't know,''\ she said, when he had ceased to speak, ``why you
imagine me to be situated as you describe; but as you
have always told me that the sole object of a bringing-up like
mine was to teach a girl to get what she wants, why not assume
that that is precisely what I am doing?''





The smile with which she summed up her case was like a clear
barrier raised against farther confidences: its brightness held
him at such a distance that he had a sense of being almost out of
hearing as he rejoined: ``I am not sure that I have ever called
you a successful example of that kind of bringing-up.''





Her colour rose a little at the implication, but she steeled
herself with a light laugh. ``Ah, wait a little longer---give me a
little more time before you decide!''\ And as he wavered before
her, still watching for a break in the impenetrable front she
presented: ``Don't give me up; I may still do credit to my
training!''\ she affirmed.





\chapter*{\raggedright Chapter 10}

\addcontentsline{toc}{chapter}{Chapter 10}

\markboth{HOUSE OF MIRTH}{CHAPTER 10}





``Look at those spangles, Miss Bart---every one of 'em sewed on
crooked.''





The tall forewoman, a pinched perpendicular figure, dropped the
condemned structure of wire and net on the table at Lily's side,
and passed on to the next figure in the line.





There were twenty of them in the work-room, their fagged
profiles, under exaggerated hair, bowed in the harsh north light
above the utensils of their art; for it was something more than
an industry, surely, this creation of ever-varied settings for
the face of fortunate womanhood. Their own faces were sallow with
the unwholesomeness of hot air and sedentary toil, rather than
with any actual signs of want: they were employed in a
fashionable millinery establishment, and were fairly well clothed
and well paid; but the youngest among them was as dull and
colourless as the middle-aged. In the whole work-room there was
only one skin beneath which the blood still visibly played; and
that now burned with vexation as Miss Bart, under the lash of the
forewoman's comment, began to strip the hat-frame of its
over-lapping spangles.





To Gerty Farish's hopeful spirit a solution appeared to have been
reached when she remembered how beautifully Lily could trim hats. 
Instances of young lady-milliners establishing themselves under
fashionable patronage, and imparting to their ``creations''\ that
indefinable touch which the professional hand can never give, had
flattered Gerty's visions of the future, and convinced even Lily
that her separation from Mrs.\ Norma Hatch need not reduce her to
dependence on her friends.





The parting had occurred a few weeks after Selden's visit, and
would have taken place sooner had it not been for the resistance
set up in Lily by his ill-starred offer of advice. The sense of
being involved in a transaction she would not have cared to
examine too closely had soon afterward defined itself in the
light of a hint from Mr.\ Stancy that, if she ``saw them through,''
she would have no reason to be sorry. The implication that such
loyalty would meet with a direct reward had hastened her flight,
and flung her back, ashamed and penitent, on the broad bosom of
Gerty's sympathy. She did not, however, propose to lie there
prone, and Gerty's inspiration about the hats at once revived her
hopes of profitable activity. Here was, after all, something that
her charming listless hands could really do; she had no doubt of
their capacity for knotting a ribbon or placing a flower to
advantage. And of course only these finishing touches would be
expected of her: subordinate fingers, blunt, grey, needle-pricked
fingers, would prepare the shapes and stitch the linings, while
she presided over the charming little front shop---a shop all white
panels, mirrors, and moss-green hangings---where her finished
creations, hats, wreaths, aigrettes and the rest, perched on their
stands like birds just poising for flight.





But at the very outset of Gerty's campaign this vision of the
green-and-white shop had been dispelled. Other young ladies of
fashion had been thus ``set-up,''\ selling their hats by the mere
attraction of a name and the reputed knack of tying a bow; but
these privileged beings could command a faith in their powers
materially expressed by the readiness to pay their shop-rent and
advance a handsome sum for current expenses. Where was Lily to
find such support? And even could it have been found, how were
the ladies on whose approval she depended to be induced to give
her their patronage? Gerty learned that whatever sympathy her
friend's case might have excited a few months since had been
imperilled, if not lost, by her association with Mrs.\ Hatch. Once
again, Lily had withdrawn from an ambiguous situation in time to
save her self-respect, but too late for public vindication. 
Freddy Van Osburgh was not to marry Mrs.\ Hatch; he had been
rescued at the eleventh hour---some said by the efforts of Gus
Trenor and Rosedale---and despatched to Europe with old Ned Van
Alstyne; but the risk he had run would always be ascribed to Miss
Bart's connivance, and would somehow serve as a summing-up and
corroboration of the vague general distrust of her. It was a
relief to those who had hung back from her to find themselves
thus justified, and they were inclined to insist a little on her
connection with the Hatch case in order to show that they had
been right.





Gerty's quest, at any rate, brought up against a solid wall of
resistance; and even when Carry Fisher, momentarily penitent for
her share in the Hatch affair, joined her efforts to Miss Farish's,
they met with no better success. Gerty had tried to veil her
failure in tender ambiguities; but Carry, always the soul of
candour, put the case squarely to her friend.





``I went straight to Judy Trenor; she has fewer prejudices than
the others, and besides she's always hated Bertha Dorset. But
what \textit{have} you done to her, Lily? At the very first word about
giving you a start she flamed out about some money you'd got from
Gus; I never knew her so hot before. You know she'll let him do
anything but spend money on his friends: the only reason she's
decent to me now is that she knows I'm not hard up.---He
speculated for you, you say? Well, what's the harm? He had no
business to lose. He \textit{didn't} lose? Then what on earth---but I never
\textit{could} understand you, Lily!''





The end of it was that, after anxious enquiry and much
deliberation, Mrs.\ Fisher and Gerty, for once oddly united in
their effort to help their friend, decided on placing her in the
work-room of \textit{Mme}.\ Regina's renowned millinery establishment. Even
this arrangement was not effected without considerable
negotiation, for \textit{Mme}.\ Regina had a strong prejudice against
untrained assistance, and was induced to yield only by the fact
that she owed the patronage of Mrs.\ Bry and Mrs.\ Gormer to Carry
Fisher's influence. She had been willing from the first to employ
Lily in the show-room: as a displayer of hats, a fashionable
beauty might be a valuable asset. But to this suggestion Miss
Bart opposed a negative which Gerty emphatically supported, while
Mrs.\ Fisher, inwardly unconvinced, but resigned to this latest
proof of Lily's unreason, agreed that perhaps in the end it would
be more useful that she should learn the trade. To Regina's
work-room Lily was therefore committed by her friends, and there
Mrs.\ Fisher left her with a sigh of relief, while Gerty's
watchfulness continued to hover over her at a distance.





Lily had taken up her work early in January: it was now two
months later, and she was still being rebuked for her inability
to sew spangles on a hat-frame. As she returned to her work she
heard a titter pass down the tables. She knew she was an object
of criticism and amusement to the other work-women. They were,
of course, aware of her history---the exact situation of
every girl in the room was known and freely discussed by all the
others---but the knowledge did not produce in them any awkward
sense of class distinction: it merely explained why her untutored
fingers were still blundering over the rudiments of the trade. 
Lily had no desire that they should recognize any social
difference in her; but she had hoped to be received as their
equal, and perhaps before long to show herself their superior by
a special deftness of touch, and it was humiliating to find that,
after two months of drudgery, she still betrayed her lack of
early training. Remote was the day when she might aspire to
exercise the talents she felt confident of possessing; only
experienced workers were entrusted with the delicate art of
shaping and trimming the hat, and the forewoman still held her
inexorably to the routine of preparatory work.





She began to rip the spangles from the frame, listening absently
to the buzz of talk which rose and fell with the coming and going
of Miss Haines's active figure. The air was closer than usual,
because Miss Haines, who had a cold, had not allowed a window to
be opened even during the noon recess; and Lily's head was so
heavy with the weight of a sleepless night that the chatter of
her companions had the incoherence of a dream.





``I \textit{told} her he'd never look at her again; and he didn't. I
wouldn't have, either---I think she acted real mean to him. He
took her to the Arion Ball, and had a hack for her both ways.... 
She's taken ten bottles, and her headaches don't seem no
better---but she's written a testimonial to say the first bottle
cured her, and she got five dollars and her picture in the
paper.... Mrs.\ Trenor's hat? The one with the green Paradise? 
Here, Miss Haines---it'll be ready right off.... That was one of
the Trenor girls here yesterday with Mrs.\ George Dorset. How'd I
know? Why, Madam sent for me to alter the flower in that Virot
hat---the blue tulle: she's tall and slight, with her hair fuzzed
out---a good deal like Mamie Leach, on'y thinner....''





On and on it flowed, a current of meaningless sound, on which,
startlingly enough, a familiar name now and then floated to the
surface. It was the strangest part of Lily's strange experience,
the hearing of these names, the seeing the fragmentary
and distorted image of the world she had lived in reflected in
the mirror of the working-girls'\ minds. She had never before
suspected the mixture of insatiable curiosity and contemptuous
freedom with which she and her kind were discussed in this
underworld of toilers who lived on their vanity and
self-indulgence. Every girl in \textit{Mme}.\ Regina's work-room knew to
whom the headgear in her hands was destined, and had her opinion
of its future wearer, and a definite knowledge of the latter's
place in the social system. That Lily was a star fallen from that
sky did not, after the first stir of curiosity had subsided,
materially add to their interest in her. She had fallen, she had
``gone under,''\ and true to the ideal of their race, they were awed
only by success---by the gross tangible image of material
achievement. The consciousness of her different point of view
merely kept them at a little distance from her, as though she
were a foreigner with whom it was an effort to talk.





``Miss Bart, if you can't sew those spangles on more regular I
guess you'd better give the hat to Miss Kilroy.''





Lily looked down ruefully at her handiwork. The forewoman was
right: the sewing on of the spangles was inexcusably bad. What
made her so much more clumsy than usual? Was it a growing
distaste for her task, or actual physical disability? She felt
tired and confused: it was an effort to put her thoughts
together. She rose and handed the hat to Miss Kilroy, who took it
with a suppressed smile.





``I'm sorry; I'm afraid I am not well,''\ she said to the forewoman.





Miss Haines offered no comment. From the first she had augured
ill of \textit{Mme}.\ Regina's consenting to include a fashionable
apprentice among her workers. In that temple of art no raw
beginners were wanted, and Miss Haines would have been more than
human had she not taken a certain pleasure in seeing her
forebodings confirmed.





``You'd better go back to binding edges,''\ she said drily. Lily
slipped out last among the band of liberated work-women. She did
not care to be mingled in their noisy dispersal: once in the
street, she always felt an irresistible return to her old
standpoint, an instinctive shrinking from all that was unpolished
and promiscuous. In the days---how distant they now
seemed!---when she had visited the Girls'\ Club with Gerty Farish,
she had felt an enlightened interest in the working-classes; but
that was because she looked down on them from above, from the
happy altitude of her grace and her beneficence. Now that she was
on a level with them, the point of view was less interesting.





She felt a touch on her arm, and met the penitent eye of Miss
Kilroy. ``Miss Bart, I guess you can sew those spangles on as well
as I can when you're feeling right. Miss Haines didn't act fair
to you.''





Lily's colour rose at the unexpected advance: it was a long time
since real kindness had looked at her from any eyes but Gerty's.





``Oh, thank you: I'm not particularly well, but Miss Haines was
right. I \textit{am} clumsy.''





``Well, it's mean work for anybody with a headache.'' Miss Kilroy
paused irresolutely. ``You ought to go right home and lay down. 
Ever try orangeine?''





``Thank you.'' Lily held out her hand. ``It's very kind of you---I
mean to go home.''





She looked gratefully at Miss Kilroy, but neither knew what more
to say. Lily was aware that the other was on the point of
offering to go home with her, but she wanted to be alone and
silent---even kindness, the sort of kindness that Miss Kilroy
could give, would have jarred on her just then.





``Thank you,''\ she repeated as she turned away.





She struck westward through the dreary March twilight, toward the
street where her boarding-house stood. She had resolutely refused
Gerty's offer of hospitality. Something of her mother's fierce
shrinking from observation and sympathy was beginning to develop
in her, and the promiscuity of small quarters and close intimacy
seemed, on the whole, less endurable than the solitude of a hall
bedroom in a house where she could come and go unremarked among
other workers. For a while she had been sustained by this desire
for privacy and independence; but now, perhaps from increasing
physical weariness, the lassitude brought about by hours of
unwonted confinement, she was beginning to feel acutely the
ugliness and discomfort of her surroundings. The day's task done,
she dreaded to return to her narrow room, with its
blotched wallpaper and shabby paint; and she hated every step of
the walk thither, through the degradation of a New York street in
the last stages of decline from fashion to commerce.





But what she dreaded most of all was having to pass the chemist's
at the corner of Sixth Avenue. She had meant to take another
street: she had usually done so of late. But today her steps were
irresistibly drawn toward the flaring plate-glass corner; she
tried to take the lower crossing, but a laden dray crowded her
back, and she struck across the street obliquely, reaching the
sidewalk just opposite the chemist's door.





Over the counter she caught the eye of the clerk who had waited
on her before, and slipped the prescription into his hand. There
could be no question about the prescription: it was a copy of one
of Mrs.\ Hatch's, obligingly furnished by that lady's chemist. 
Lily was confident that the clerk would fill it without
hesitation; yet the nervous dread of a refusal, or even of an
expression of doubt, communicated itself to her restless hands as
she affected to examine the bottles of perfume stacked on the
glass case before her.





The clerk had read the prescription without comment; but in the
act of handing out the bottle he paused.





``You don't want to increase the dose, you know,''\ he remarked. 
Lily's heart contracted.





What did he mean by looking at her in that way?





``Of course not,''\ she murmured, holding out her hand.





``That's all right: it's a queer-acting drug. A drop or two more,
and off you go---the doctors don't know why.''





The dread lest he should question her, or keep the bottle back,
choked the murmur of acquiescence in her throat; and when at
length she emerged safely from the shop she was almost dizzy with
the intensity of her relief. The mere touch of the packet
thrilled her tired nerves with the delicious promise of a night
of sleep, and in the reaction from her momentary fear she felt as
if the first fumes of drowsiness were already stealing over her.





In her confusion she stumbled against a man who was hurrying down
the last steps of the elevated station. He drew back, and she
heard her name uttered with surprise. It was Rosedale,
fur-coated, glossy and prosperous---but why did she seem to see
him so far off, and as if through a mist of splintered crystals? 
Before she could account for the phenomenon she found herself
shaking hands with him. They had parted with scorn on her side
and anger upon his; but all trace of these emotions seemed to
vanish as their hands met, and she was only aware of a confused
wish that she might continue to hold fast to him.





``Why, what's the matter, Miss Lily? You're not well!''\ he
exclaimed; and she forced her lips into a pallid smile of
reassurance.





``I'm a little tired---it's nothing. Stay with me a moment,
please,''\ she faltered. That she should be asking this service of
Rosedale!





He glanced at the dirty and unpropitious corner on which they
stood, with the shriek of the ``elevated''\ and the tumult of trams
and waggons contending hideously in their ears.





``We can't stay here; but let me take you somewhere for a cup of
tea. The \textit{Longworth} is only a few yards off, and there'll be no
one there at this hour.''





A cup of tea in quiet, somewhere out of the noise and ugliness,
seemed for the moment the one solace she could bear. A few steps
brought them to the ladies'\ door of the hotel he had named, and a
moment later he was seated opposite to her, and the waiter had
placed the tea-tray between them.





``Not a drop of brandy or whiskey first? You look regularly done
up, Miss Lily. Well, take your tea strong, then; and, waiter, get
a cushion for the lady's back.''





Lily smiled faintly at the injunction to take her tea strong. It
was the temptation she was always struggling to resist. Her
craving for the keen stimulant was forever conflicting with that
other craving for sleep---the midnight craving which only the
little phial in her hand could still. But today, at any rate, the
tea could hardly be too strong: she counted on it to pour warmth
and resolution into her empty veins.





As she leaned back before him, her lids drooping in utter
lassitude, though the first warm draught already tinged her face
with returning life, Rosedale was seized afresh by the poignant
surprise of her beauty. The dark pencilling of fatigue under her
eyes, the morbid blue-veined pallour of the temples,
brought out the brightness of her hair and lips, as though all
her ebbing vitality were centred there. Against the dull
chocolate-coloured background of the restaurant, the purity of
her head stood out as it had never done in the most brightly-lit
ball-room. He looked at her with a startled uncomfortable
feeling, as though her beauty were a forgotten enemy that had
lain in ambush and now sprang out on him unawares.





To clear the air he tried to take an easy tone with her. ``Why,
Miss Lily, I haven't seen you for an age. I didn't know what had
become of you.''





As he spoke, he was checked by an embarrassing sense of the
complications to which this might lead. Though he had not seen
her he had heard of her; he knew of her connection with Mrs.
Hatch, and of the talk resulting from it. Mrs.\ Hatch's \textit{milieu} was
one which he had once assiduously frequented, and now as devoutly
shunned.





Lily, to whom the tea had restored her usual clearness of mind,
saw what was in his thoughts and said with a slight smile: ``You
would not be likely to know about me. I have joined the working
classes.''





He stared in genuine wonder. ``You don't mean---? Why, what on earth
are you doing?''





``Learning to be a milliner---at least \textit{trying} to learn,''\ she
hastily qualified the statement.





Rosedale suppressed a low whistle of surprise. ``Come off---you
ain't serious, are you?''





``Perfectly serious. I'm obliged to work for my living.''





``But I understood---I thought you were with Norma Hatch.''





``You heard I had gone to her as her secretary?''





``Something of the kind, I believe.'' He leaned forward to refill
her cup.





Lily guessed the possibilities of embarrassment which the topic
held for him, and raising her eyes to his, she said suddenly: ``I
left her two months ago.''





Rosedale continued to fumble awkwardly with the tea-pot, and she
felt sure that he had heard what had been said of her. But what
was there that Rosedale did not hear?





``Wasn't it a soft berth?''\ he enquired, with an attempt at
lightness.





``Too soft---one might have sunk in too deep.'' Lily rested one arm
on the edge of the table, and sat looking at him more intently
than she had ever looked before. An uncontrollable impulse was
urging her to put her case to this man, from whose curiosity she
had always so fiercely defended herself.





``You know Mrs.\ Hatch, I think? Well, perhaps you can understand
that she might make things too easy for one.''





Rosedale looked faintly puzzled, and she remembered that
allusiveness was lost on him.





``It was no place for you, anyhow,''\ he agreed, so suffused and
immersed in the light of her full gaze that he found himself
being drawn into strange depths of intimacy. He who had had to
subsist on mere fugitive glances, looks winged in flight and
swiftly lost under covert, now found her eyes settling on him
with a brooding intensity that fairly dazzled him.





``I left,''\ Lily continued, ``lest people should say I was helping
Mrs.\ Hatch to marry Freddy Van Osburgh---who is not in the least
too good for her---and as they still continue to say it, I see
that I might as well have stayed where I was.''





``Oh, Freddy----''\ Rosedale brushed aside the topic with an air of
its unimportance which gave a sense of the immense perspective he
had acquired. ``Freddy don't count---but I knew \textit{you} weren't mixed
up in that. It ain't your style.''





Lily coloured slightly: she could not conceal from herself that
the words gave her pleasure. She would have liked to sit there,
drinking more tea, and continuing to talk of herself to Rosedale. 
But the old habit of observing the conventions reminded her that
it was time to bring their colloquy to an end, and she made a
faint motion to push back her chair.





Rosedale stopped her with a protesting gesture. ``Wait a
minute---don't go yet; sit quiet and rest a little longer. You
look thoroughly played out. And you haven't told me----''\ He broke
off, conscious of going farther than he had meant. She saw the
struggle and understood it; understood also the nature of the
spell to which he yielded as, with his eyes on her face, he began
again abruptly: ``What on earth did you mean by saying just now
that you were learning to be a milliner?''





``Just what I said. I am an apprentice at Regina's.''





``Good Lord---\textit{you}? But what for? I knew your aunt had
turned you down: Mrs.\ Fisher told me about it. But I understood
you got a legacy from her----''





``I got ten thousand dollars; but the legacy is not to be paid
till next summer.''





``Well, but---look here: you could \textit{borrow} on it any time you
wanted.''





She shook her head gravely. ``No; for I owe it already.''





``Owe it? The whole ten thousand?''





``Every penny.'' She paused, and then continued abruptly, with her
eyes on his face: ``I think Gus Trenor spoke to you once about
having made some money for me in stocks.''





She waited, and Rosedale, congested with embarrassment, muttered
that he remembered something of the kind.





``He made about nine thousand dollars,''\ Lily pursued, in the same
tone of eager communicativeness. ``At the time, I understood that
he was speculating with my own money: it was incredibly stupid of
me, but I knew nothing of business. Afterward I found out that he
had \textit{not} used my money---that what he said he had made for me he
had really given me. It was meant in kindness, of course; but it
was not the sort of obligation one could remain under. 
Unfortunately I had spent the money before I discovered my
mistake; and so my legacy will have to go to pay it back. That is
the reason why I am trying to learn a trade.''





She made the statement clearly, deliberately, with pauses between
the sentences, so that each should have time to sink deeply into
her hearer's mind. She had a passionate desire that some one
should know the truth about this transaction, and also that the
rumour of her intention to repay the money should reach Judy
Trenor's ears. And it had suddenly occurred to her that Rosedale,
who had surprised Trenor's confidence, was the fitting person to
receive and transmit her version of the facts. She had even felt
a momentary exhilaration at the thought of thus relieving herself
of her detested secret; but the sensation gradually faded in the
telling, and as she ended her pallour was suffused with a deep
blush of misery.





Rosedale continued to stare at her in wonder; but the wonder took
the turn she had least expected.





``But see here---if that's the case, it cleans you out altogether?''





He put it to her as if she had not grasped the consequences of
her act; as if her incorrigible ignorance of business were about
to precipitate her into a fresh act of folly.





``Altogether---yes,''\ she calmly agreed.





He sat silent, his thick hands clasped on the table, his little
puzzled eyes exploring the recesses of the deserted restaurant.





``See here---that's fine,''\ he exclaimed abruptly.





Lily rose from her seat with a deprecating laugh. ``Oh, no---it's
merely a bore,''\ she asserted, gathering together the ends of her
feather scarf.





Rosedale remained seated, too intent on his thoughts to notice
her movement. ``Miss Lily, if you want any backing---I like pluck----''
broke from him disconnectedly.





``Thank you.'' She held out her hand. ``Your tea has given me a
tremendous backing. I feel equal to anything now.''





Her gesture seemed to show a definite intention of dismissal, but
her companion had tossed a bill to the waiter, and was slipping
his short arms into his expensive overcoat.





``Wait a minute---you've got to let me walk home with you,''\ he
said.





Lily uttered no protest, and when he had paused to make sure of
his change they emerged from the hotel and crossed Sixth Avenue
again. As she led the way westward past a long line of areas
which, through the distortion of their paintless rails, revealed
with increasing candour the DISJECTA MEMBRA of bygone dinners,
Lily felt that Rosedale was taking contemptuous note of the
neighbourhood; and before the doorstep at which she finally
paused he looked up with an air of incredulous disgust.





``This isn't the place? Some one told me you were living with Miss
Farish.''





``No: I am boarding here. I have lived too long on my friends.''





He continued to scan the blistered brown stone front, the windows
draped with discoloured lace, and the Pompeian decoration of the
muddy vestibule; then he looked back at her face and said with a
visible effort: ``You'll let me come and see you some day?''





She smiled, recognizing the heroism of the offer to the point of
being frankly touched by it. ``Thank you---I shall be very
glad,''\ she made answer, in the first sincere words she had ever
spoken to him.







That evening in her own room Miss Bart---who had fled early from
the heavy fumes of the basement dinner-table---sat musing upon the
impulse which had led her to unbosom herself to Rosedale. Beneath
it she discovered an increasing sense of loneliness---a dread of
returning to the solitude of her room, while she could be
anywhere else, or in any company but her own. Circumstances, of
late, had combined to cut her off more and more from her few
remaining friends. On Carry Fisher's part the withdrawal was
perhaps not quite involuntary. Having made her final effort on
Lily's behalf, and landed her safely in \textit{Mme}.\ Regina's work-room,
Mrs.\ Fisher seemed disposed to rest from her labours; and Lily,
understanding the reason, could not condemn her. Carry had in
fact come dangerously near to being involved in the episode of
Mrs.\ Norma Hatch, and it had taken some verbal ingenuity to
extricate herself. She frankly owned to having brought Lily and
Mrs.\ Hatch together, but then she did not know Mrs.\ Hatch---she
had expressly warned Lily that she did not know Mrs.\ Hatch---and
besides, she was not Lily's keeper, and really the girl was old
enough to take care of herself. Carry did not put her own case so
brutally, but she allowed it to be thus put for her by her latest
bosom friend, Mrs.\ Jack Stepney: Mrs.\ Stepney, trembling over the
narrowness of her only brother's escape, but eager to vindicate
Mrs.\ Fisher, at whose house she could count on the ``jolly
parties''\ which had become a necessity to her since marriage had
emancipated her from the Van Osburgh point of view.





Lily understood the situation and could make allowances for it. 
Carry had been a good friend to her in difficult days, and
perhaps only a friendship like Gerty's could be proof against
such an increasing strain. Gerty's friendship did indeed hold
fast; yet Lily was beginning to avoid her also. For she could not
go to Gerty's without risk of meeting Selden; and to meet him now
would be pure pain. It was pain enough even to think of him,
whether she considered him in the distinctness of her waking
thoughts, or felt the obsession of his presence through the blur
of her tormented nights. That was one of the reasons why
she had turned again to Mrs.\ Hatch's prescription. In the uneasy
snatches of her natural dreams he came to her sometimes in the
old guise of fellowship and tenderness; and she would rise from
the sweet delusion mocked and emptied of her courage. But in the
sleep which the phial procured she sank far below such
half-waking visitations, sank into depths of dreamless
annihilation from which she woke each morning with an obliterated
past.





Gradually, to be sure, the stress of the old thoughts would
return; but at least they did not importune her waking hour. The
drug gave her a momentary illusion of complete renewal, from
which she drew strength to take up her daily work. The strength
was more and more needed as the perplexities of her future
increased. She knew that to Gerty and Mrs.\ Fisher she was only
passing through a temporary period of probation, since they
believed that the apprenticeship she was serving at \textit{Mme}.\ Regina's
would enable her, when Mrs.\ Peniston's legacy was paid, to
realize the vision of the green-and-white shop with the fuller
competence acquired by her preliminary training. But to Lily
herself, aware that the legacy could not be put to such a use,
the preliminary training seemed a wasted effort. She understood
clearly enough that, even if she could ever learn to compete with
hands formed from childhood for their special work, the small pay
she received would not be a sufficient addition to her income to
compensate her for such drudgery. And the realization of this
fact brought her recurringly face to face with the temptation to
use the legacy in establishing her business. Once installed, and
in command of her own work-women, she believed she had sufficient
tact and ability to attract a fashionable \textit{clientele}; and if the
business succeeded she could gradually lay aside money enough to
discharge her debt to Trenor. But the task might take years to
accomplish, even if she continued to stint herself to the utmost;
and meanwhile her pride would be crushed under the weight of an
intolerable obligation.





These were her superficial considerations; but under them lurked
the secret dread that the obligation might not always remain
intolerable. She knew she could not count on her continuity of
purpose, and what really frightened her was the thought that she
might gradually accommodate herself to remaining
indefinitely in Trenor's debt, as she had accommodated herself to
the part allotted her on the Sabrina, and as she had so nearly
drifted into acquiescing with Stancy's scheme for the advancement
of Mrs.\ Hatch. Her danger lay, as she knew, in her old incurable
dread of discomfort and poverty; in the fear of that mounting
tide of dinginess against which her mother had so passionately
warned her. And now a new vista of peril opened before her. She
understood that Rosedale was ready to lend her money; and the
longing to take advantage of his offer began to haunt her
insidiously. It was of course impossible to accept a loan from
Rosedale; but proximate possibilities hovered temptingly before
her. She was quite sure that he would come and see her again, and
almost sure that, if he did, she could bring him to the point of
offering to marry her on the terms she had previously rejected. 
Would she still reject them if they were offered? More and more,
with every fresh mischance befalling her, did the pursuing furies
seem to take the shape of Bertha Dorset; and close at hand,
safely locked among her papers, lay the means of ending their
pursuit. The temptation, which her scorn of Rosedale had once
enabled her to reject, now insistently returned upon her; and how
much strength was left her to oppose it?





What little there was must at any rate be husbanded to the
utmost; she could not trust herself again to the perils of a
sleepless night. Through the long hours of silence the dark
spirit of fatigue and loneliness crouched upon her breast,
leaving her so drained of bodily strength that her morning
thoughts swam in a haze of weakness. The only hope of renewal lay
in the little bottle at her bed-side; and how much longer that
hope would last she dared not conjecture.





\chapter*{\raggedright Chapter 11}

\addcontentsline{toc}{chapter}{Chapter 11}

\markboth{HOUSE OF MIRTH}{CHAPTER 11}





Lily, lingering for a moment on the corner, looked out on the
afternoon spectacle of Fifth Avenue. It was a day in late April,
and the sweetness of spring was in the air. It mitigated the
ugliness of the long crowded thoroughfare, blurred the gaunt
roof-lines, threw a mauve veil over the discouraging perspective
of the side streets, and gave a touch of poetry to the delicate
haze of green that marked the entrance to the Park.





As Lily stood there, she recognized several familiar faces in the
passing carriages. The season was over, and its ruling forces had
disbanded; but a few still lingered, delaying their departure for
Europe, or passing through town on their return from the South. 
Among them was Mrs.\ Van Osburgh, swaying majestically in her
C-spring barouche, with Mrs.\ Percy Gryce at her side, and the new
heir to the Gryce millions enthroned before them on his nurse's
knees. They were succeeded by Mrs.\ Hatch's electric victoria, in
which that lady reclined in the lonely splendour of a spring
toilet obviously designed for company; and a moment or two later
came Judy Trenor, accompanied by Lady Skiddaw, who had come over
for her annual tarpon fishing and a dip into ``the street.''





This fleeting glimpse of her past served to emphasize the sense
of aimlessness with which Lily at length turned toward home. She
had nothing to do for the rest of the day, nor for the days to
come; for the season was over in millinery as well as in society,
and a week earlier \textit{Mme}.\ Regina had notified her that her services
were no longer required. \textit{Mme}.\ Regina always reduced her staff on
the first of May, and Miss Bart's attendance had of late been so
irregular---she had so often been unwell, and had done so little
work when she came---that it was only as a favour that her
dismissal had hitherto been deferred.





Lily did not question the justice of the decision. She was
conscious of having been forgetful, awkward and slow to learn. It
was bitter to acknowledge her inferiority even to herself, but
the fact had been brought home to her that as a
bread-winner she could never compete with professional ability. 
Since she had been brought up to be ornamental, she could hardly
blame herself for failing to serve any practical purpose; but the
discovery put an end to her consoling sense of universal
efficiency.





As she turned homeward her thoughts shrank in anticipation from
the fact that there would be nothing to get up for the next
morning. The luxury of lying late in bed was a pleasure belonging
to the life of ease; it had no part in the utilitarian existence
of the boarding-house. She liked to leave her room early, and to
return to it as late as possible; and she was walking slowly now
in order to postpone the detested approach to her doorstep.





But the doorstep, as she drew near it, acquired a sudden interest
from the fact that it was occupied---and indeed filled---by the
conspicuous figure of Mr.\ Rosedale, whose presence seemed to take
on an added amplitude from the meanness of his surroundings.





The sight stirred Lily with an irresistible sense of triumph. 
Rosedale, a day or two after their chance meeting, had called to
enquire if she had recovered from her indisposition; but since
then she had not seen or heard from him, and his absence seemed
to betoken a struggle to keep away, to let her pass once more out
of his life. If this were the case, his return showed that the
struggle had been unsuccessful, for Lily knew he was not the man
to waste his time in an ineffectual sentimental dalliance. He was
too busy, too practical, and above all too much preoccupied with
his own advancement, to indulge in such unprofitable asides.





In the peacock-blue parlour, with its bunches of dried pampas
grass, and discoloured steel engravings of sentimental episodes,
he looked about him with unconcealed disgust, laying his hat
distrustfully on the dusty console adorned with a Rogers
statuette.





Lily sat down on one of the plush and rosewood sofas, and he
deposited himself in a rocking-chair draped with a starched
antimacassar which scraped unpleasantly against the pink fold of
skin above his collar.





``My goodness---you can't go on living here!''\ he exclaimed.





Lily smiled at his tone. ``I am not sure that I can; but I have
gone over my expenses very carefully, and I rather think I shall
be able to manage it.''





``Be able to manage it? That's not what I mean---it's no place for
you!''





``It's what I mean; for I have been out of work for the last
week.''





``Out of work---out of work! What a way for you to talk! The idea
of your having to work---it's preposterous.'' He brought out his
sentences in short violent jerks, as though they were forced up
from a deep inner crater of indignation. ``It's a farce---a crazy
farce,''\ he repeated, his eyes fixed on the long vista of the room
reflected in the blotched glass between the windows.





Lily continued to meet his expostulations with a smile. ``I don't
know why I should regard myself as an exception----''\ she began.





``Because you \textit{are}; that's why; and your being in a place like this
is a damnable outrage. I can't talk of it calmly.''





She had in truth never seen him so shaken out of his usual
glibness; and there was something almost moving to her in his
inarticulate struggle with his emotions.





He rose with a start which left the rocking-chair quivering on
its beam ends, and placed himself squarely before her.





``Look here, Miss Lily, I'm going to Europe next week: going over
to Paris and London for a couple of months---and I can't leave you
like this. I can't do it. I know it's none of my business---you've
let me understand that often enough; but things are worse with
you now than they have been before, and you must see that you've
got to accept help from somebody. You spoke to me the other day
about some debt to Trenor. I know what you mean---and I respect
you for feeling as you do about it.''





A blush of surprise rose to Lily's pale face, but before she
could interrupt him he had continued eagerly: ``Well, I'll lend
you the money to pay Trenor; and I won't---I---see here, don't take
me up till I've finished. What I mean is, it'll be a plain
business arrangement, such as one man would make with another. 
Now, what have you got to say against that?''





Lily's blush deepened to a glow in which humiliation and
gratitude were mingled; and both sentiments revealed themselves
in the unexpected gentleness of her reply.





``Only this: that it is exactly what Gus Trenor proposed; and that
I can never again be sure of understanding the plainest business
arrangement.'' Then, realizing that this answer contained a germ
of injustice, she added, even more kindly: ``Not that I don't
appreciate your kindness---that I'm not grateful for it. But a
business arrangement between us would in any case be impossible,
because I shall have no security to give when my debt to Gus
Trenor has been paid.''





Rosedale received this statement in silence: he seemed to feel the
note of finality in her voice, yet to be unable to accept it as
closing the question between them.





In the silence Lily had a clear perception of what was passing
through his mind. Whatever perplexity he felt as to the
inexorableness of her course---however little he penetrated its
motive---she saw that it unmistakably tended to strengthen her
hold over him. It was as though the sense in her of unexplained
scruples and resistances had the same attraction as the delicacy
of feature, the fastidiousness of manner, which gave her an
external rarity, an air of being impossible to match. As he
advanced in social experience this uniqueness had acquired a
greater value for him, as though he were a collector who had
learned to distinguish minor differences of design and quality in
some long-coveted object.





Lily, perceiving all this, understood that he would marry her at
once, on the sole condition of a reconciliation with Mrs.\ Dorset;
and the temptation was the less easy to put aside because, little
by little, circumstances were breaking down her dislike for
Rosedale. The dislike, indeed, still subsisted; but it was
penetrated here and there by the perception of mitigating
qualities in him: of a certain gross kindliness, a rather
helpless fidelity of sentiment, which seemed to be struggling
through the hard surface of his material ambitions.





Reading his dismissal in her eyes, he held out his hand with a
gesture which conveyed something of this inarticulate conflict.





``If you'd only let me, I'd set you up over them all---I'd put you
where you could wipe your feet on 'em!''\ he declared; and
it touched her oddly to see that his new passion had not altered
his old standard of values.







Lily took no sleeping-drops that night. She lay awake viewing her
situation in the crude light which Rosedale's visit had shed on
it. In fending off the offer he was so plainly ready to renew,
had she not sacrificed to one of those abstract notions of honour
that might be called the conventionalities of the moral life? 
What debt did she owe to a social order which had condemned and
banished her without trial? She had never been heard in her own
defence; she was innocent of the charge on which she had been
found guilty; and the irregularity of her conviction might seem
to justify the use of methods as irregular in recovering her lost
rights. Bertha Dorset, to save herself, had not scrupled to ruin
her by an open falsehood; why should she hesitate to make private
use of the facts that chance had put in her way? After all, half
the opprobrium of such an act lies in the name attached to it. 
Call it blackmail and it becomes unthinkable; but explain that it
injures no one, and that the rights regained by it were unjustly
forfeited, and he must be a formalist indeed who can find no plea
in its defence.





The arguments pleading for it with Lily were the old unanswerable
ones of the personal situation: the sense of injury, the sense of
failure, the passionate craving for a fair chance against the
selfish despotism of society. She had learned by experience that
she had neither the aptitude nor the moral constancy to remake
her life on new lines; to become a worker among workers, and let
the world of luxury and pleasure sweep by her unregarded. She
could not hold herself much to blame for this ineffectiveness,
and she was perhaps less to blame than she believed. Inherited
tendencies had combined with early training to make her the
highly specialized product she was: an organism as helpless out
of its narrow range as the sea-anemone torn from the rock. She
had been fashioned to adorn and delight; to what other end does
nature round the rose-leaf and paint the humming-bird's breast? 
And was it her fault that the purely decorative mission is less
easily and harmoniously fulfilled among social beings
than in the world of nature? That it is apt to be hampered by
material necessities or complicated by moral scruples?





These last were the two antagonistic forces which fought out
their battle in her breast during the long watches of the night;
and when she rose the next morning she hardly knew where the
victory lay. She was exhausted by the reaction of a night without
sleep, coming after many nights of rest artificially obtained;
and in the distorting light of fatigue the future stretched out
before her grey, interminable and desolate.





She lay late in bed, refusing the coffee and fried eggs which the
friendly Irish servant thrust through her door, and hating the
intimate domestic noises of the house and the cries and rumblings
of the street. Her week of idleness had brought home to her with
exaggerated force these small aggravations of the boarding-house
world, and she yearned for that other luxurious world, whose
machinery is so carefully concealed that one scene flows into
another without perceptible agency.





At length she rose and dressed. Since she had left \textit{Mme}.\ Regina's
she had spent her days in the streets, partly to escape from the
uncongenial promiscuities of the boarding-house, and partly in
the hope that physical fatigue would help her to sleep. But once
out of the house, she could not decide where to go; for she had
avoided Gerty since her dismissal from the milliner's, and she
was not sure of a welcome anywhere else.





The morning was in harsh contrast to the previous day. A cold
grey sky threatened rain, and a high wind drove the dust in wild
spirals up and down the streets. Lily walked up Fifth Avenue
toward the Park, hoping to find a sheltered nook where she might
sit; but the wind chilled her, and after an hour's wandering
under the tossing boughs she yielded to her increasing weariness,
and took refuge in a little restaurant in Fifty-ninth Street. She
was not hungry, and had meant to go without luncheon; but she was
too tired to return home, and the long perspective of white
tables showed alluringly through the windows.





The room was full of women and girls, all too much engaged in the
rapid absorption of tea and pie to remark her entrance. A hum of
shrill voices reverberated against the low ceiling, leaving Lily
shut out in a little circle of silence. She felt a sudden pang of
profound loneliness. She had lost the sense of time, and
it seemed to her as though she had not spoken to any one for
days. Her eyes sought the faces about her, craving a responsive
glance, some sign of an intuition of her trouble. But the sallow
preoccupied women, with their bags and note-books and rolls of
music, were all engrossed in their own affairs, and even those
who sat by themselves were busy running over proof-sheets or
devouring magazines between their hurried gulps of tea. Lily
alone was stranded in a great waste of disoccupation.





She drank several cups of the tea which was served with her
portion of stewed oysters, and her brain felt clearer and
livelier when she emerged once more into the street. She realized
now that, as she sat in the restaurant, she had unconsciously
arrived at a final decision. The discovery gave her an immediate
illusion of activity: it was exhilarating to think that she had
actually a reason for hurrying home. To prolong her enjoyment of
the sensation she decided to walk; but the distance was so great
that she found herself glancing nervously at the clocks on the
way. One of the surprises of her unoccupied state was the
discovery that time, when it is left to itself and no definite
demands are made on it, cannot be trusted to move at any
recognized pace. Usually it loiters; but just when one has come
to count upon its slowness, it may suddenly break into a wild
irrational gallop.





She found, however, on reaching home, that the hour was still
early enough for her to sit down and rest a few minutes before
putting her plan into execution. The delay did not perceptibly
weaken her resolve. She was frightened and yet stimulated by the
reserved force of resolution which she felt within herself: she
saw it was going to be easier, a great deal easier, than she had
imagined.





At five o'clock she rose, unlocked her trunk, and took out a
sealed packet which she slipped into the bosom of her dress. Even
the contact with the packet did not shake her nerves as she had
half-expected it would. She seemed encased in a strong armour of
indifference, as though the vigorous exertion of her will had
finally benumbed her finer sensibilities.





She dressed herself once more for the street, locked her door and
went out. When she emerged on the pavement, the day was still
high, but a threat of rain darkened the sky and cold
gusts shook the signs projecting from the basement shops along
the street. She reached Fifth Avenue and began to walk slowly
northward. She was sufficiently familiar with Mrs.\ Dorset's
habits to know that she could always be found at home after five. 
She might not, indeed, be accessible to visitors, especially to a
visitor so unwelcome, and against whom it was quite possible that
she had guarded herself by special orders; but Lily had written a
note which she meant to send up with her name, and which she
thought would secure her admission.





She had allowed herself time to walk to Mrs.\ Dorset's, thinking
that the quick movement through the cold evening air would help
to steady her nerves; but she really felt no need of being
tranquillized. Her survey of the situation remained calm and
unwavering.





As she reached Fiftieth Street the clouds broke abruptly, and a
rush of cold rain slanted into her face. She had no umbrella and
the moisture quickly penetrated her thin spring dress. She was
still half a mile from her destination, and she decided to walk
across to Madison Avenue and take the electric car. As she turned
into the side street, a vague memory stirred in her. The row of
budding trees, the new brick and limestone house-fronts, the
Georgian flat-house with flowerboxes on its balconies, were
merged together into the setting of a familiar scene. It was down
this street that she had walked with Selden, that September day
two years ago; a few yards ahead was the doorway they had entered
together. The recollection loosened a throng of benumbed
sensations---longings, regrets, imaginings, the throbbing brood of
the only spring her heart had ever known. It was strange to find
herself passing his house on such an errand. She seemed suddenly
to see her action as he would see it---and the fact of his own
connection with it, the fact that, to attain her end, she must
trade on his name, and profit by a secret of his past, chilled
her blood with shame. What a long way she had travelled since the
day of their first talk together! Even then her feet had been set
in the path she was now following---even then she had resisted the
hand he had held out.





All her resentment of his fancied coldness was swept away in this
overwhelming rush of recollection. Twice he had been
ready to help her---to help her by loving her, as he had said---and
if, the third time, he had seemed to fail her, whom but herself
could she accuse?\ .\ .\ . Well, that part of her life was over; she
did not know why her thoughts still clung to it. But the sudden
longing to see him remained; it grew to hunger as she paused on
the pavement opposite his door. The street was dark and empty,
swept by the rain. She had a vision of his quiet room, of the
bookshelves, and the fire on the hearth. She looked up and saw a
light in his window; then she crossed the street and entered the
house.





\chapter*{\raggedright Chapter 12}

\addcontentsline{toc}{chapter}{Chapter 12}

\markboth{HOUSE OF MIRTH}{CHAPTER 12}





The library looked as she had pictured it. The green-shaded lamps
made tranquil circles of light in the gathering dusk, a little
fire flickered on the hearth, and Selden's easy-chair, which
stood near it, had been pushed aside when he rose to admit her.





He had checked his first movement of surprise, and stood silent,
waiting for her to speak, while she paused a moment on the
threshold, assailed by a rush of memories.





The scene was unchanged. She recognized the row of shelves from
which he had taken down his La Bruyere, and the worn arm of the
chair he had leaned against while she examined the precious
volume. But then the wide September light had filled the room,
making it seem a part of the outer world: now the shaded lamps
and the warm hearth, detaching it from the gathering darkness of
the street, gave it a sweeter touch of intimacy.





Becoming gradually aware of the surprise under Selden's silence,
Lily turned to him and said simply: ``I came to tell you that I
was sorry for the way we parted---for what I said to you that day
at Mrs.\ Hatch's.''





The words rose to her lips spontaneously. Even on her way up the
stairs, she had not thought of preparing a pretext for her visit,
but she now felt an intense longing to dispel the cloud of
misunderstanding that hung between them.





Selden returned her look with a smile. ``I was sorry too that we
should have parted in that way; but I am not sure I didn't bring
it on myself. Luckily I had foreseen the risk I was taking----''





``So that you really didn't care----?''\ broke from her with a flash
of her old irony.





``So that I was prepared for the consequences,''\ he corrected
good-humouredly. ``But we'll talk of all this later. Do come and
sit by the fire. I can recommend that arm-chair, if you'll let me
put a cushion behind you.''





While he spoke she had moved slowly to the middle of the room,
and paused near his writing-table, where the lamp,
striking upward, cast exaggerated shadows on the pallour of her
delicately-hollowed face.





``You look tired---do sit down,''\ he repeated gently.





She did not seem to hear the request. ``I wanted you to know that
I left Mrs.\ Hatch immediately after I saw you,''\ she said, as
though continuing her confession.





``Yes---yes; I know,''\ he assented, with a rising tinge of
embarrassment.





``And that I did so because you told me to. Before you came I had
already begun to see that it would be impossible to remain with
her---for the reasons you gave me; but I wouldn't admit it---I
wouldn't let you see that I understood what you meant.''





``Ah, I might have trusted you to find your own way out---don't
overwhelm me with the sense of my officiousness!''





His light tone, in which, had her nerves been steadier, she would
have recognized the mere effort to bridge over an awkward moment,
jarred on her passionate desire to be understood. In her strange
state of extra-lucidity, which gave her the sense of being
already at the heart of the situation, it seemed incredible that
any one should think it necessary to linger in the conventional
outskirts of word-play and evasion.





``It was not that---I was not ungrateful,''\ she insisted. But the
power of expression failed her suddenly; she felt a tremor in her
throat, and two tears gathered and fell slowly from her eyes.





Selden moved forward and took her hand. ``You are very tired. Why
won't you sit down and let me make you comfortable?''





He drew her to the arm-chair near the fire, and placed a cushion
behind her shoulders.





``And now you must let me make you some tea: you know I always
have that amount of hospitality at my command.''





She shook her head, and two more tears ran over. But she did not
weep easily, and the long habit of self-control reasserted
itself, though she was still too tremulous to speak.





``You know I can coax the water to boil in five minutes,''\ Selden
continued, speaking as though she were a troubled child.





His words recalled the vision of that other afternoon
when they had sat together over his tea-table and talked
jestingly of her future. There were moments when that day seemed
more remote than any other event in her life; and yet she could
always relive it in its minutest detail.





She made a gesture of refusal. ``No: I drink too much tea. I would
rather sit quiet---I must go in a moment,''\ she added confusedly.





Selden continued to stand near her, leaning against the
mantelpiece. The tinge of constraint was beginning to be more
distinctly perceptible under the friendly ease of his manner. Her
self-absorption had not allowed her to perceive it at first; but
now that her consciousness was once more putting forth its eager
feelers, she saw that her presence was becoming an embarrassment
to him. Such a situation can be saved only by an immediate
outrush of feeling; and on Selden's side the determining impulse
was still lacking.





The discovery did not disturb Lily as it might once have done. 
She had passed beyond the phase of well-bred reciprocity, in
which every demonstration must be scrupulously proportioned to
the emotion it elicits, and generosity of feeling is the only
ostentation condemned. But the sense of loneliness returned with
redoubled force as she saw herself forever shut out from Selden's
inmost self. She had come to him with no definite purpose; the
mere longing to see him had directed her; but the secret hope she
had carried with her suddenly revealed itself in its death-pang.





``I must go,''\ she repeated, making a motion to rise from her
chair. ``But I may not see you again for a long time, and I wanted
to tell you that I have never forgotten the things you said to me
at Bellomont, and that sometimes---sometimes when I seemed
farthest from remembering them---they have helped me, and kept me
from mistakes; kept me from really becoming what many people have
thought me.''





Strive as she would to put some order in her thoughts, the words
would not come more clearly; yet she felt that she could not
leave him without trying to make him understand that she had
saved herself whole from the seeming ruin of her life.





A change had come over Selden's face as she spoke. Its guarded
look had yielded to an expression still untinged by personal
emotion, but full of a gentle understanding.





``I am glad to have you tell me that; but nothing I have said has
really made the difference. The difference is in yourself---it
will always be there. And since it \textit{is} there, it can't really
matter to you what people think: you are so sure that your
friends will always understand you.''





``Ah, don't say that---don't say that what you have told me has
made no difference. It seems to shut me out---to leave me all
alone with the other people.'' She had risen and stood before him,
once more completely mastered by the inner urgency of the moment. 
The consciousness of his half-divined reluctance had vanished. 
Whether he wished it or not, he must see her wholly for once
before they parted.





Her voice had gathered strength, and she looked him gravely in
the eyes as she continued. ``Once---twice---you gave me the chance
to escape from my life, and I refused it: refused it because I
was a coward. Afterward I saw my mistake---I saw I could never be
happy with what had contented me before. But it was too late: you
had judged me---I understood. It was too late for happiness---but
not too late to be helped by the thought of what I had missed. 
That is all I have lived on---don't take it from me now! Even in
my worst moments it has been like a little light in the darkness. 
Some women are strong enough to be good by themselves, but I
needed the help of your belief in me. Perhaps I might have
resisted a great temptation, but the little ones would have
pulled me down. And then I remembered---I remembered your saying
that such a life could never satisfy me; and I was ashamed to
admit to myself that it could. That is what you did for me---that
is what I wanted to thank you for. I wanted to tell you that I
have always remembered; and that I have tried---tried hard .\ .\ .''





She broke off suddenly. Her tears had risen again, and in drawing
out her handkerchief her fingers touched the packet in the folds
of her dress. A wave of colour suffused her, and the words died
on her lips. Then she lifted her eyes to his and went on in an
altered voice.





``I have tried hard---but life is difficult, and I am a very
useless person. I can hardly be said to have an independent
existence. I was just a screw or a cog in the great machine
I called life, and when I dropped out of it I found I was
of no use anywhere else. What can one do when one finds that one
only fits into one hole? One must get back to it or be thrown out
into the rubbish heap---and you don't know what it's like in the
rubbish heap!''





Her lips wavered into a smile---she had been distracted by the
whimsical remembrance of the confidences she had made to him, two
years earlier, in that very room. Then she had been planning to
marry Percy Gryce---what was it she was planning now?





The blood had risen strongly under Selden's dark skin, but his
emotion showed itself only in an added seriousness of manner.





``You have something to tell me---do you mean to marry?''\ he said
abruptly.





Lily's eyes did not falter, but a look of wonder, of puzzled
self-interrogation, formed itself slowly in their depths. In the
light of his question, she had paused to ask herself if her
decision had really been taken when she entered the room.





``You always told me I should have to come to it sooner or later!''
she said with a faint smile.





``And you have come to it now?''





``I shall have to come to it---presently. But there is something
else I must come to first.'' She paused again, trying to transmit
to her voice the steadiness of her recovered smile. ``There is
some one I must say goodbye to. Oh, not \textit{you}---we are sure to see
each other again---but the Lily Bart you knew. I have kept her
with me all this time, but now we are going to part, and I have
brought her back to you---I am going to leave her here. When I go
out presently she will not go with me. I shall like to think that
she has stayed with you---and she'll be no trouble, she'll take up
no room.''





She went toward him, and put out her hand, still smiling. ``Will
you let her stay with you?''\ she asked.





He caught her hand, and she felt in his the vibration of feeling
that had not yet risen to his lips. ``Lily---can't I help you?''\ he
exclaimed.





She looked at him gently. ``Do you remember what you said to me
once? That you could help me only by loving me? Well---you did
love me for a moment; and it helped me. It has always
helped me. But the moment is gone---it was I who let it go. And
one must go on living. Goodbye.''





She laid her other hand on his, and they looked at each other
with a kind of solemnity, as though they stood in the presence of
death. Something in truth lay dead between them---the love she had
killed in him and could no longer call to life. But something
lived between them also, and leaped up in her like an
imperishable flame: it was the love his love had kindled, the
passion of her soul for his.





In its light everything else dwindled and fell away from her. She
understood now that she could not go forth and leave her old self
with him: that self must indeed live on in his presence, but it
must still continue to be hers.





Selden had retained her hand, and continued to scrutinize her
with a strange sense of foreboding. The external aspect of the
situation had vanished for him as completely as for her: he felt
it only as one of those rare moments which lift the veil from
their faces as they pass.





``Lily,''\ he said in a low voice, ``you mustn't speak in this way. I
can't let you go without knowing what you mean to do. Things may
change---but they don't pass. You can never go out of my life.''





She met his eyes with an illumined look. ``No,''\ she said. ``I see
that now. Let us always be friends. Then I shall feel safe,
whatever happens.''





``Whatever happens? What do you mean? What is going to happen?''





She turned away quietly and walked toward the hearth.





``Nothing at present---except that I am very cold, and that before
I go you must make up the fire for me.''





She knelt on the hearth-rug, stretching her hands to the embers. 
Puzzled by the sudden change in her tone, he mechanically
gathered a handful of wood from the basket and tossed it on the
fire. As he did so, he noticed how thin her hands looked against
the rising light of the flames. He saw too, under the loose lines
of her dress, how the curves of her figure had shrunk to
angularity; he remembered long afterward how the red play of the
flame sharpened the depression of her nostrils, and intensified
the blackness of the shadows which struck up from her cheekbones
to her eyes. She knelt there for a few moments in
silence; a silence which he dared not break. When she rose he
fancied that he saw her draw something from her dress and drop it
into the fire; but he hardly noticed the gesture at the time. His
faculties seemed tranced, and he was still groping for the word
to break the spell. She went up to him and laid her hands on his
shoulders. ``Goodbye,''\ she said, and as he bent over her she
touched his forehead with her lips.





\chapter*{\raggedright Chapter 13}

\addcontentsline{toc}{chapter}{Chapter 13}

\markboth{HOUSE OF MIRTH}{CHAPTER 13}





The street-lamps were lit, but the rain had ceased, and there was
a momentary revival of light in the upper sky. Lily walked on
unconscious of her surroundings. She was still treading the
buoyant ether which emanates from the high moments of life. But
gradually it shrank away from her and she felt the dull pavement
beneath her feet. The sense of weariness returned with
accumulated force, and for a moment she felt that she could walk
no farther. She had reached the corner of Forty-first Street and
Fifth Avenue, and she remembered that in Bryant Park there were
seats where she might rest.





That melancholy pleasure-ground was almost deserted when she
entered it, and she sank down on an empty bench in the glare of
an electric street-lamp. The warmth of the fire had passed out of
her veins, and she told herself that she must not sit long in the
penetrating dampness which struck up from the wet asphalt. But
her will-power seemed to have spent itself in a last great
effort, and she was lost in the blank reaction which follows on
an unwonted expenditure of energy. And besides, what was there to
go home to? Nothing but the silence of her cheerless room---that
silence of the night which may be more racking to tired nerves
than the most discordant noises: that, and the bottle of chloral
by her bed. The thought of the chloral was the only spot of light
in the dark prospect: she could feel its lulling influence
stealing over her already. But she was troubled by the thought
that it was losing its power---she dared not go back to it too
soon. Of late the sleep it had brought her had been more broken
and less profound; there had been nights when she was perpetually
floating up through it to consciousness. What if the effect of
the drug should gradually fail, as all narcotics were said to
fail? She remembered the chemist's warning against increasing the
dose; and she had heard before of the capricious and incalculable
action of the drug. Her dread of returning to a sleepless night
was so great that she lingered on, hoping that excessive
weariness would reinforce the waning power of the chloral.





Night had now closed in, and the roar of traffic in Forty-second
Street was dying out. As complete darkness fell on the square the
lingering occupants of the benches rose and dispersed; but now
and then a stray figure, hurrying homeward, struck across the
path where Lily sat, looming black for a moment in the white
circle of electric light. One or two of these passers-by
slackened their pace to glance curiously at her lonely figure;
but she was hardly conscious of their scrutiny.





Suddenly, however, she became aware that one of the passing
shadows remained stationary between her line of vision and the
gleaming asphalt; and raising her eyes she saw a young woman
bending over her.





``Excuse me---are you sick?---Why, it's Miss Bart!''\ a half-familiar
voice exclaimed.





Lily looked up. The speaker was a poorly-dressed young woman with
a bundle under her arm. Her face had the air of unwholesome
refinement which ill-health and over-work may produce, but its
common prettiness was redeemed by the strong and generous curve
of the lips.





``You don't remember me,''\ she continued, brightening with the
pleasure of recognition, ``but I'd know you anywhere, I've thought
of you such a lot. I guess my folks all know your name by heart. 
I was one of the girls at Miss Farish's club---you helped me to go
to the country that time I had lung-trouble. My name's Nettie
Struther. It was Nettie Crane then---but I daresay you don't
remember that either.''





Yes: Lily was beginning to remember. The episode of Nettie
Crane's timely rescue from disease had been one of the most
satisfying incidents of her connection with Gerty's charitable
work. She had furnished the girl with the means to go to a
sanatorium in the mountains: it struck her now with a peculiar
irony that the money she had used had been Gus Trenor's.





She tried to reply, to assure the speaker that she had not
forgotten; but her voice failed in the effort, and she felt
herself sinking under a great wave of physical weakness. Nettie
Struther, with a startled exclamation, sat down and slipped a
shabbily-clad arm behind her back.





``Why, Miss Bart, you \textit{are} sick. Just lean on me a little till you
feel better.''





A faint glow of returning strength seemed to pass into Lily from
the pressure of the supporting arm.





``I'm only tired---it is nothing,''\ she found voice to say in a
moment; and then, as she met the timid appeal of her companion's
eyes, she added involuntarily: ``I have been unhappy---in great
trouble.''





``\textit{You} in trouble? I've always thought of you as being so high up,
where everything was just grand. Sometimes, when I felt real
mean, and got to wondering why things were so queerly fixed in
the world, I used to remember that you were having a lovely time,
anyhow, and that seemed to show there was a kind of justice
somewhere. But you mustn't sit here too long---it's fearfully
damp. Don't you feel strong enough to walk on a little ways now?''
she broke off.





``Yes---yes; I must go home,''\ Lily murmured, rising.





Her eyes rested wonderingly on the thin shabby figure at her
side. She had known Nettie Crane as one of the discouraged
victims of over-work and an{\ae}mic parentage: one of the
superfluous fragments of life destined to be swept prematurely
into that social refuse-heap of which Lily had so lately
expressed her dread. But Nettie Struther's frail envelope was now
alive with hope and energy: whatever fate the future reserved for
her, she would not be cast into the refuse-heap without a
struggle.





``I am very glad to have seen you,''\ Lily continued, summoning a
smile to her unsteady lips. ``It'll be my turn to think of you as
happy---and the world will seem a less unjust place to me too.''





``Oh, but I can't leave you like this---you're not fit to go home
alone. And I can't go with you either!''\ Nettie Struther wailed
with a start of recollection. ``You see, it's my husband's
night-shift---he's a motor-man---and the friend I leave the baby
with has to step upstairs to get \textit{her} husband's supper at seven. I
didn't tell you I had a baby, did I? She'll be four months old
day after tomorrow, and to look at her you wouldn't think I'd
ever had a sick day. I'd give anything to show you the baby, Miss
Bart, and we live right down the street here---it's only three
blocks off.'' She lifted her eyes tentatively to Lily's face, and
then added with a burst of courage: ``Why won't you get right into
the cars and come home with me while I get baby's supper? 
It's real warm in our kitchen, and you can rest there, and I'll
take \textit{you} home as soon as ever she drops off to sleep.''





It \textit{was} warm in the kitchen, which, when Nettie Struther's match
had made a flame leap from the gas-jet above the table, revealed
itself to Lily as extraordinarily small and almost miraculously
clean. A fire shone through the polished flanks of the iron
stove, and near it stood a crib in which a baby was sitting
upright, with incipient anxiety struggling for expression on a
countenance still placid with sleep.





Having passionately celebrated her reunion with her offspring,
and excused herself in cryptic language for the lateness of her
return, Nettie restored the baby to the crib and shyly invited
Miss Bart to the rocking-chair near the stove.





``We've got a parlour too,''\ she explained with pardonable pride;
``but I guess it's warmer in here, and I don't want to leave you
alone while I'm getting baby's supper.''





On receiving Lily's assurance that she much preferred the
friendly proximity of the kitchen fire, Mrs.\ Struther proceeded
to prepare a bottle of infantile food, which she tenderly applied
to the baby's impatient lips; and while the ensuing degustation
went on, she seated herself with a beaming countenance beside her
visitor.





``You're sure you won't let me warm up a drop of coffee for you,
Miss Bart? There's some of baby's fresh milk left over---well,
maybe you'd rather just sit quiet and rest a little while. It's
too lovely having you here. I've thought of it so often that I
can't believe it's really come true. I've said to George again
and again: `I just wish Miss Bart could see me \textit{now}---'\ and I used
to watch for your name in the papers, and we'd talk over what you
were doing, and read the descriptions of the dresses you wore. I
haven't seen your name for a long time, though, and I began to be
afraid you were sick, and it worried me so that George said I'd
get sick myself, fretting about it.'' Her lips broke into a
reminiscent smile. ``Well, I can't afford to be sick again, that's
a fact: the last spell nearly finished me. When you sent me off
that time I never thought I'd come back alive, and I didn't much
care if I did. You see I didn't know about George and the baby
then.''





She paused to readjust the bottle to the child's bubbling mouth.





``You precious---don't you be in too much of a hurry! Was it mad
with mommer for getting its supper so late? Marry
Anto'nette---that's what we call her: after the French queen in
that play at the Garden---I told George the actress reminded me of
you, and that made me fancy the name .\ .\ . I never thought I'd
get married, you know, and I'd never have had the heart to go on
working just for myself.''





She broke off again, and meeting the encouragement in Lily's
eyes, went on, with a flush rising under her an{\ae}mic skin: ``You
see I wasn't only just \textit{sick} that time you sent me off---I was
dreadfully unhappy too. I'd known a gentleman where I was
employed---I don't know as you remember I did type-writing in a
big importing firm---and---well---I thought we were to be married: 
he'd gone steady with me six months and given me his mother's
wedding ring. But I presume he was too stylish for me---he
travelled for the firm, and had seen a great deal of society. 
Work girls aren't looked after the way you are, and they don't
always know how to look after themselves. I didn't .\ .\ .\ and it
pretty near killed me when he went away and left off writing .\ .\ . 
It was then I came down sick---I thought it was the end of
everything. I guess it would have been if you hadn't sent me off. 
But when I found I was getting well I began to take heart in
spite of myself. And then, when I got back home, George came
round and asked me to marry him. At first I thought I couldn't,
because we'd been brought up together, and I knew he knew about
me. But after a while I began to see that that made it easier. I
never could have told another man, and I'd never have married
without telling; but if George cared for me enough to have me as
I was, I didn't see why I shouldn't begin over again---and I did.''





The strength of the victory shone forth from her as she lifted
her irradiated face from the child on her knees. ``But, mercy, I
didn't mean to go on like this about myself, with you sitting
there looking so fagged out. Only it's so lovely having you here,
and letting you see just how you've helped me.'' The baby had sunk
back blissfully replete, and Mrs.\ Struther softly rose to
lay the bottle aside. Then she paused before Miss Bart.





``I only wish I could help \textit{you}---but I suppose there's nothing on
earth I could do,''\ she murmured wistfully.





Lily, instead of answering, rose with a smile and held out her
arms; and the mother, understanding the gesture, laid her child
in them.





The baby, feeling herself detached from her habitual anchorage,
made an instinctive motion of resistance; but the soothing
influences of digestion prevailed, and Lily felt the soft weight
sink trustfully against her breast. The child's confidence in its
safety thrilled her with a sense of warmth and returning life,
and she bent over, wondering at the rosy blur of the little face,
the empty clearness of the eyes, the vague tendrilly motions of
the folding and unfolding fingers. At first the burden in her
arms seemed as light as a pink cloud or a heap of down, but as
she continued to hold it the weight increased, sinking deeper,
and penetrating her with a strange sense of weakness, as though
the child entered into her and became a part of herself.





She looked up, and saw Nettie's eyes resting on her with
tenderness and exultation.





``Wouldn't it be too lovely for anything if she could grow up to
be just like you? Of course I know she never \textit{could}---but mothers
are always dreaming the craziest things for their children.''





Lily clasped the child close for a moment and laid her back in
her
mother's arms.





``Oh, she must not do that---I should be afraid to come and see her
too often!''\ she said with a smile; and then, resisting Mrs.
Struther's anxious offer of companionship, and reiterating the
promise that of course she would come back soon, and make
George's acquaintance, and see the baby in her bath, she passed
out of the kitchen and went alone down the tenement stairs.







As she reached the street she realized that she felt stronger and
happier: the little episode had done her good. It was the first
time she had ever come across the results of her spasmodic
benevolence, and the surprised sense of human fellowship took the
mortal chill from her heart.





It was not till she entered her own door that she felt the
reaction of a deeper loneliness. It was long after seven o'clock,
and the light and odours proceeding from the basement made it
manifest that the boarding-house dinner had begun. She hastened
up to her room, lit the gas, and began to dress. She did not mean
to pamper herself any longer, to go without food because her
surroundings made it unpalatable. Since it was her fate to live
in a boarding-house, she must learn to fall in with the
conditions of the life. Nevertheless she was glad that, when she
descended to the heat and glare of the dining-room, the repast
was nearly over.







In her own room again, she was seized with a sudden fever of
activity. For weeks past she had been too listless and
indifferent to set her possessions in order, but now she began to
examine systematically the contents of her drawers and cupboard. 
She had a few handsome dresses left---survivals of her last phase
of splendour, on the Sabrina and in London---but when she had been
obliged to part with her maid she had given the woman a generous
share of her cast-off apparel. The remaining dresses, though they
had lost their freshness, still kept the long unerring lines, the
sweep and amplitude of the great artist's stroke, and as she
spread them out on the bed the scenes in which they had been worn
rose vividly before her. An association lurked in every fold: 
each fall of lace and gleam of embroidery was like a letter in
the record of her past. She was startled to find how the
atmosphere of her old life enveloped her. But, after all, it was
the life she had been made for: every dawning tendency in her had
been carefully directed toward it, all her interests and
activities had been taught to centre around it. She was like some
rare flower grown for exhibition, a flower from which every bud
had been nipped except the crowning blossom of her beauty.





Last of all, she drew forth from the bottom of her trunk a heap
of white drapery which fell shapelessly across her arm. It was
the Reynolds dress she had worn in the Bry \textit{tableaux}. It had been
impossible for her to give it away, but she had never seen it
since that night, and the long flexible folds, as she
shook them out, gave forth an odour of violets which came to her
like a breath from the flower-edged fountain where she had stood
with Lawrence Selden and disowned her fate. She put back the
dresses one by one, laying away with each some gleam of light,
some note of laughter, some stray waft from the rosy shores of
pleasure. She was still in a state of highly-wrought
impressionability, and every hint of the past sent a lingering
tremor along her nerves.





She had just closed her trunk on the white folds of the Reynolds
dress when she heard a tap at her door, and the red fist of the
Irish maid-servant thrust in a belated letter. Carrying it to the
light, Lily read with surprise the address stamped on the upper
corner of the envelope. It was a business communication from the
office of her aunt's executors, and she wondered what unexpected
development had caused them to break silence before the appointed
time. She opened the envelope and a cheque fluttered to the
floor. As she stooped to pick it up the blood rushed to her face. 
The cheque represented the full amount of Mrs.\ Peniston's legacy,
and the letter accompanying it explained that the executors,
having adjusted the business of the estate with less delay than
they had expected, had decided to anticipate the date fixed for
the payment of the bequests.





Lily sat down beside the desk at the foot of her bed, and
spreading out the cheque, read over and over the \textit{ten} \textit{thousand}
\textit{dollars} written across it in a steely business hand. Ten months
earlier the amount it stood for had represented the depths of
penury; but her standard of values had changed in the interval,
and now visions of wealth lurked in every flourish of the pen. As
she continued to gaze at it, she felt the glitter of the visions
mounting to her brain, and after a while she lifted the lid of
the desk and slipped the magic formula out of sight. It was
easier to think without those five figures dancing before her
eyes; and she had a great deal of thinking to do before she
slept.





She opened her cheque-book, and plunged into such anxious
calculations as had prolonged her vigil at Bellomont on the night
when she had decided to marry Percy Gryce. Poverty simplifies
book-keeping, and her financial situation was easier to ascertain
than it had been then; but she had not yet learned the
control of money, and during her transient phase of luxury at the
Emporium she had slipped back into habits of extravagance which
still impaired her slender balance. A careful examination of her
cheque-book, and of the unpaid bills in her desk, showed that,
when the latter had been settled, she would have barely enough to
live on for the next three or four months; and even after that,
if she were to continue her present way of living, without
earning any additional money, all incidental expenses must be
reduced to the vanishing point. She hid her eyes with a shudder,
beholding herself at the entrance of that ever-narrowing
perspective down which she had seen Miss Silverton's dowdy figure
take its despondent way.





It was no longer, however, from the vision of material poverty
that she turned with the greatest shrinking. She had a sense of
deeper empoverishment---of an inner destitution compared to which
outward conditions dwindled into insignificance. It was indeed
miserable to be poor---to look forward to a shabby, anxious
middle-age, leading by dreary degrees of economy and self-denial
to gradual absorption in the dingy communal existence of the
boarding-house. But there was something more miserable still---it
was the clutch of solitude at her heart, the sense of being swept
like a stray uprooted growth down the heedless current of the
years. That was the feeling which possessed her now---the feeling
of being something rootless and ephemeral, mere spin-drift of the
whirling surface of existence, without anything to which the poor
little tentacles of self could cling before the awful flood
submerged them. And as she looked back she saw that there had
never been a time when she had had any real relation to life. Her
parents too had been rootless, blown hither and thither on every
wind of fashion, without any personal existence to shelter them
from its shifting gusts. She herself had grown up without any one
spot of earth being dearer to her than another: there was no
centre of early pieties, of grave endearing traditions, to which
her heart could revert and from which it could draw strength for
itself and tenderness for others. In whatever form a
slowly-accumulated past lives in the blood---whether in the
concrete image of the old house stored with visual memories, or
in the conception of the house not built with hands, but
made up of inherited passions and loyalties---it has the same
power of broadening and deepening the individual existence, of
attaching it by mysterious links of kinship to all the mighty sum
of human striving.





Such a vision of the solidarity of life had never before come to
Lily. She had had a premonition of it in the blind motions of her
mating-instinct; but they had been checked by the disintegrating
influences of the life about her. All the men and women she knew
were like atoms whirling away from each other in some wild
centrifugal dance: her first glimpse of the continuity of life
had come to her that evening in Nettie Struther's kitchen.





The poor little working-girl who had found strength to gather up
the fragments of her life, and build herself a shelter with them,
seemed to Lily to have reached the central truth of existence. It
was a meagre enough life, on the grim edge of poverty, with scant
margin for possibilities of sickness or mischance, but it had the
frail audacious permanence of a bird's nest built on the edge of
a cliff---a mere wisp of leaves and straw, yet so put together
that the lives entrusted to it may hang safely over the abyss.





Yes---but it had taken two to build the nest; the man's faith as
well as the woman's courage. Lily remembered Nettie's words: I
\textit{knew} \textit{he} \textit{knew} \textit{about} \textit{me}. Her husband's faith in her had made her
renewal possible---it is so easy for a woman to become what the
man she loves believes her to be! Well---Selden had twice been
ready to stake his faith on Lily Bart; but the third trial had
been too severe for his endurance. The very quality of his love
had made it the more impossible to recall to life. If it had been
a simple instinct of the blood, the power of her beauty might
have revived it. But the fact that it struck deeper, that it was
inextricably wound up with inherited habits of thought and
feeling, made it as impossible to restore to growth as a
deep-rooted plant torn from its bed. Selden had given her of his
best; but he was as incapable as herself of an uncritical return
to former states of feeling.





There remained to her, as she had told him, the uplifting memory
of his faith in her; but she had not reached the age when a woman
can live on her memories. As she held Nettie Struther's
child in her arms the frozen currents of youth had loosed
themselves and run warm in her veins: the old life-hunger
possessed her, and all her being clamoured for its share of
personal happiness. Yes---it was happiness she still wanted, and
the glimpse she had caught of it made everything else of no
account. One by one she had detached herself from the baser
possibilities, and she saw that nothing now remained to her but
the emptiness of renunciation.





It was growing late, and an immense weariness once more possessed
her. It was not the stealing sense of sleep, but a vivid wakeful
fatigue, a wan lucidity of mind against which all the
possibilities of the future were shadowed forth gigantically. She
was appalled by the intense cleanness of the vision; she seemed
to have broken through the merciful veil which intervenes between
intention and action, and to see exactly what she would do in all
the long days to come. There was the cheque in her desk, for
instance---she meant to use it in paying her debt to Trenor; but
she foresaw that when the morning came she would put off doing
so, would slip into gradual tolerance of the debt. The thought
terrified her---she dreaded to fall from the height of her last
moment with Lawrence Selden. But how could she trust herself to
keep her footing? She knew the strength of the opposing
impulses-she could feel the countless hands of habit dragging her
back into some fresh compromise with fate. She felt an intense
longing to prolong, to perpetuate, the momentary exaltation of
her spirit. If only life could end now---end on this tragic yet
sweet vision of lost possibilities, which gave her a sense of
kinship with all the loving and foregoing in the world!





She reached out suddenly and, drawing the cheque from her
writing-desk, enclosed it in an envelope which she addressed to
her bank. She then wrote out a cheque for Trenor, and placing it,
without an accompanying word, in an envelope inscribed with his
name, laid the two letters side by side on her desk. After that
she continued to sit at the table, sorting her papers and
writing, till the intense silence of the house reminded her of
the lateness of the hour. In the street the noise of wheels had
ceased, and the rumble of the ``elevated''\ came only at long
intervals through the deep unnatural hush. In the mysterious
nocturnal separation from all outward signs of life, she
felt herself more strangely confronted with her fate. The
sensation made her brain reel, and she tried to shut out
consciousness by pressing her hands against her eyes. But the
terrible silence and emptiness seemed to symbolize her
future---she felt as though the house, the street, the world were
all empty, and she alone left sentient in a lifeless universe.





But this was the verge of delirium .\ .\ .\ she had never hung so
near the dizzy brink of the unreal. Sleep was what she
wanted---she remembered that she had not closed her eyes for two
nights. The little bottle was at her bed-side, waiting to lay its
spell upon her. She rose and undressed hastily, hungering now for
the touch of her pillow. She felt so profoundly tired that she
thought she must fall asleep at once; but as soon as she had lain
down every nerve started once more into separate wakefulness. It
was as though a great blaze of electric light had been turned on
in her head, and her poor little anguished self shrank and
cowered in it, without knowing where to take refuge.





She had not imagined that such a multiplication of wakefulness
was possible: her whole past was reenacting itself at a hundred
different points of consciousness. Where was the drug that could
still this legion of insurgent nerves? The sense of exhaustion
would have been sweet compared to this shrill beat of activities;
but weariness had dropped from her as though some cruel stimulant
had been forced into her veins.





She could bear it---yes, she could bear it; but what strength
would be left her the next day? Perspective had disappeared---the
next day pressed close upon her, and on its heels came the days
that were to follow---they swarmed about her like a shrieking mob. 
She must shut them out for a few hours; she must take a brief
bath of oblivion. She put out her hand, and measured the soothing
drops into a glass; but as she did so, she knew they would be
powerless against the supernatural lucidity of her brain. She had
long since raised the dose to its highest limit, but tonight she
felt she must increase it. She knew she took a slight risk in
doing so---she remembered the chemist's warning. If sleep came at
all, it might be a sleep without waking. But after all that was
but one chance in a hundred: the action of the drug was
incalculable, and the addition of a few drops to the regular dose
would probably do no more than procure for her the rest she so
desperately needed....





She did not, in truth, consider the question very closely---the
physical craving for sleep was her only sustained sensation. Her
mind shrank from the glare of thought as instinctively as eyes
contract in a blaze of light---darkness, darkness was what she
must have at any cost. She raised herself in bed and swallowed
the contents of the glass; then she blew out her candle and lay
down.





She lay very still, waiting with a sensuous pleasure for the
first effects of the soporific. She knew in advance what form
they would take---the gradual cessation of the inner throb, the
soft approach of passiveness, as though an invisible hand made
magic passes over her in the darkness. The very slowness and
hesitancy of the effect increased its fascination: it was
delicious to lean over and look down into the dim abysses of
unconsciousness. Tonight the drug seemed to work more slowly than
usual: each passionate pulse had to be stilled in turn, and it
was long before she felt them dropping into abeyance, like
sentinels falling asleep at their posts. But gradually the sense
of complete subjugation came over her, and she wondered languidly
what had made her feel so uneasy and excited. She saw now that
there was nothing to be excited about---she had returned to her
normal view of life. Tomorrow would not be so difficult after
all: she felt sure that she would have the strength to meet it. 
She did not quite remember what it was that she had been afraid
to meet, but the uncertainty no longer troubled her. She had been
unhappy, and now she was happy---she had felt herself alone, and
now the sense of loneliness had vanished.





She stirred once, and turned on her side, and as she did so, she
suddenly understood why she did not feel herself alone. It was
odd---but Nettie Struther's child was lying on her arm: she felt
the pressure of its little head against her shoulder. She did not
know how it had come there, but she felt no great surprise at the
fact, only a gentle penetrating thrill of warmth and pleasure. 
She settled herself into an easier position, hollowing her arm to
pillow the round downy head, and holding her breath lest
a sound should disturb the sleeping child.





As she lay there she said to herself that there was something she
must tell Selden, some word she had found that should make life
clear between them. She tried to repeat the word, which lingered
vague and luminous on the far edge of thought---she was afraid of
not remembering it when she woke; and if she could only remember
it and say it to him, she felt that everything would be well.





Slowly the thought of the word faded, and sleep began to enfold
her. She struggled faintly against it, feeling that she ought to
keep awake on account of the baby; but even this feeling was
gradually lost in an indistinct sense of drowsy peace, through
which, of a sudden, a dark flash of loneliness and terror tore
its way.





She started up again, cold and trembling with the shock: for a
moment she seemed to have lost her hold of the child. But no---she
was mistaken---the tender pressure of its body was still close to
hers: the recovered warmth flowed through her once more, she
yielded to it, sank into it, and slept.





\chapter*{\raggedright Chapter 14}

\addcontentsline{toc}{chapter}{Chapter 14}

\markboth{HOUSE OF MIRTH}{CHAPTER 14}





The next morning rose mild and bright, with a promise of summer
in the air. The sunlight slanted joyously down Lily's street,
mellowed the blistered house-front, gilded the paintless railings
of the door-step, and struck prismatic glories from the panes of
her darkened window.





When such a day coincides with the inner mood there is
intoxication in its breath; and Selden, hastening along the
street through the squalor of its morning confidences, felt
himself thrilling with a youthful sense of adventure. He had cut
loose from the familiar shores of habit, and launched himself on
uncharted seas of emotion; all the old tests and measures were
left behind, and his course was to be shaped by new stars.





That course, for the moment, led merely to Miss Bart's
boarding-house; but its shabby door-step had suddenly become the
threshold of the untried. As he approached he looked up at the
triple row of windows, wondering boyishly which one of them was
hers. It was nine o'clock, and the house, being tenanted by
workers, already showed an awakened front to the street. He
remembered afterward having noticed that only one blind was down. 
He noticed too that there was a pot of pansies on one of the
window sills, and at once concluded that the window must be hers: 
it was inevitable that he should connect her with the one touch
of beauty in the dingy scene.





Nine o'clock was an early hour for a visit, but Selden had passed
beyond all such conventional observances. He only knew that he
must see Lily Bart at once---he had found the word he meant to say
to her, and it could not wait another moment to be said. It was
strange that it had not come to his lips sooner---that he had let
her pass from him the evening before without being able to speak
it. But what did that matter, now that a new day had come? It was
not a word for twilight, but for the morning.





Selden ran eagerly up the steps and pulled the bell; and even in
his state of self-absorption it came as a sharp surprise to him
that the door should open so promptly. It was still more
of a surprise to see, as he entered, that it had been opened by
Gerty Farish---and that behind her, in an agitated blur, several
other figures ominously loomed.





``Lawrence!''\ Gerty cried in a strange voice, ``how could you get
here so quickly?''---and the trembling hand she laid on him seemed
instantly to close about his heart.





He noticed the other faces, vague with fear and conjecture---he
saw the landlady's imposing bulk sway professionally toward him;
but he shrank back, putting up his hand, while his eyes
mechanically mounted the steep black walnut stairs, up which he
was immediately aware that his cousin was about to lead him.





A voice in the background said that the doctor might be back at
any minute---and that nothing, upstairs, was to be disturbed. Some
one else exclaimed: ``It was the greatest mercy---''\ then Selden
felt that Gerty had taken him gently by the hand, and that they
were to be suffered to go up alone.





In silence they mounted the three flights, and walked along the
passage to a closed door. Gerty opened the door, and Selden went
in after her. Though the blind was down, the irresistible
sunlight poured a tempered golden flood into the room, and in its
light Selden saw a narrow bed along the wall, and on the bed,
with motionless hands and calm unrecognizing face, the semblance
of Lily Bart.





That it was her real self, every pulse in him ardently denied. 
Her real self had lain warm on his heart but a few hours
earlier---what had he to do with this estranged and tranquil face
which, for the first time, neither paled nor brightened at his
coming?





Gerty, strangely tranquil too, with the conscious self-control of
one who has ministered to much pain, stood by the bed, speaking
gently, as if transmitting a final message.





``The doctor found a bottle of chloral---she had been sleeping
badly for a long time, and she must have taken an overdose by
mistake.... There is no doubt of that---no doubt---there will be no
question---he has been very kind. I told him that you and I would
like to be left alone with her---to go over her things before any
one else comes. I know it is what she would have wished.''





Selden was hardly conscious of what she said. He stood
looking down on the sleeping face which seemed to lie like a
delicate impalpable mask over the living lineaments he had known. 
He felt that the real Lily was still there, close to him, yet
invisible and inaccessible; and the tenuity of the barrier
between them mocked him with a sense of helplessness. There had
never been more than a little impalpable barrier between
them---and yet he had suffered it to keep them apart! And now,
though it seemed slighter and frailer than ever, it had suddenly
hardened to adamant, and he might beat his life out against it in
vain.





He had dropped on his knees beside the bed, but a touch from
Gerty aroused him. He stood up, and as their eyes met he was
struck by the extraordinary light in his cousin's face.





``You understand what the doctor has gone for? He has promised
that there shall be no trouble---but of course the formalities
must be gone through. And I asked him to give us time to look
through her things first----''





He nodded, and she glanced about the small bare room. ``It won't
take long,''\ she concluded.





``No---it won't take long,''\ he agreed.





She held his hand in hers a moment longer, and then, with a last
look at the bed, moved silently toward the door. On the threshold
she paused to add: ``You will find me downstairs if you want me.''





Selden roused himself to detain her. ``But why are you going? She
would have wished----''





Gerty shook her head with a smile. ``No: this is what she would
have wished----''\ and as she spoke a light broke through Selden's
stony misery, and he saw deep into the hidden things of love.





The door closed on Gerty, and he stood alone with the motionless
sleeper on the bed. His impulse was to return to her side, to
fall on his knees, and rest his throbbing head against the
peaceful cheek on the pillow. They had never been at peace
together, they two; and now he felt himself drawn downward into
the strange mysterious depths of her tranquillity.





But he remembered Gerty's warning words---he knew that, though
time had ceased in this room, its feet were hastening
relentlessly toward the door. Gerty had given him this supreme
half-hour, and he must use it as she willed.





He turned and looked about him, sternly compelling himself to
regain his consciousness of outward things. There was very little
furniture in the room. The shabby chest of drawers was spread
with a lace cover, and set out with a few gold-topped boxes and
bottles, a rose-coloured pin-cushion, a glass tray strewn with
tortoise-shell hair-pins---he shrank from the poignant intimacy of
these trifles, and from the blank surface of the toilet-mirror
above them.





These were the only traces of luxury, of that clinging to the
minute observance of personal seemliness, which showed what her
other renunciations must have cost. There was no other token of
her personality about the room, unless it showed itself in the
scrupulous neatness of the scant articles of furniture: a
washing-stand, two chairs, a small writing-desk, and the little
table near the bed. On this table stood the empty bottle and
glass, and from these also he averted his eyes.





The desk was closed, but on its slanting lid lay two letters
which he took up. One bore the address of a bank, and as it was
stamped and sealed, Selden, after a moment's hesitation, laid it
aside. On the other letter he read Gus Trenor's name; and the
flap of the envelope was still ungummed.





Temptation leapt on him like the stab of a knife. He staggered
under it, steadying himself against the desk. Why had she been
writing to Trenor---writing, presumably, just after their parting
of the previous evening? The thought unhallowed the memory of
that last hour, made a mock of the word he had come to speak, and
defiled even the reconciling silence upon which it fell. He felt
himself flung back on all the ugly uncertainties from which he
thought he had cast loose forever. After all, what did he know of
her life? Only as much as she had chosen to show him, and
measured by the world's estimate, how little that was! By what
right---the letter in his hand seemed to ask---by what right was it
he who now passed into her confidence through the gate which
death had left unbarred? His heart cried out that it was by right
of their last hour together, the hour when she herself had placed
the key in his hand. Yes---but what if the letter to Trenor had
been written afterward?





He put it from him with sudden loathing, and setting his lips,
addressed himself resolutely to what remained of his task. After
all, that task would be easier to perform, now that his personal
stake in it was annulled.





He raised the lid of the desk, and saw within it a cheque-book
and a few packets of bills and letters, arranged with the orderly
precision which characterized all her personal habits. He looked
through the letters first, because it was the most difficult part
of the work. They proved to be few and unimportant, but among
them he found, with a strange commotion of the heart, the note he
had written her the day after the Brys'\ entertainment.





``When may I come to you?''---his words overwhelmed him with a
realization of the cowardice which had driven him from her at the
very moment of attainment. Yes---he had always feared his fate,
and he was too honest to disown his cowardice now; for had not
all his old doubts started to life again at the mere sight of
Trenor's name?





He laid the note in his card-case, folding it away carefully, as
something made precious by the fact that she had held it so;
then, growing once more aware of the lapse of time, he continued
his examination of the papers.





To his surprise, he found that all the bills were receipted;
there was not an unpaid account among them. He opened the
cheque-book, and saw that, the very night before, a cheque of ten
thousand dollars from Mrs.\ Peniston's executors had been entered
in it. The legacy, then, had been paid sooner than Gerty had led
him to expect. But, turning another page or two, he discovered
with astonishment that, in spite of this recent accession of
funds, the balance had already declined to a few dollars. A rapid
glance at the stubs of the last cheques, all of which bore the
date of the previous day, showed that between four or five
hundred dollars of the legacy had been spent in the settlement of
bills, while the remaining thousands were comprehended in one
cheque, made out, at the same time, to Charles Augustus Trenor.





Selden laid the book aside, and sank into the chair beside the
desk. He leaned his elbows on it, and hid his face in his
hands. The bitter waters of life surged high about him, their
sterile taste was on his lips. Did the cheque to Trenor explain
the mystery or deepen it? At first his mind refused to act---he
felt only the taint of such a transaction between a man like
Trenor and a girl like Lily Bart. Then, gradually, his troubled
vision cleared, old hints and rumours came back to him, and out
of the very insinuations he had feared to probe, he constructed
an explanation of the mystery. It was true, then, that she had
taken money from Trenor; but true also, as the contents of the
little desk declared, that the obligation had been intolerable to
her, and that at the first opportunity she had freed herself from
it, though the act left her face to face with bare unmitigated
poverty.





That was all he knew---all he could hope to unravel of the story. 
The mute lips on the pillow refused him more than this---unless
indeed they had told him the rest in the kiss they had left upon
his forehead. Yes, he could now read into that farewell all that
his heart craved to find there; he could even draw from it
courage not to accuse himself for having failed to reach the
height of his opportunity.





He saw that all the conditions of life had conspired to keep them
apart; since his very detachment from the external influences
which swayed her had increased his spiritual fastidiousness, and
made it more difficult for him to live and love uncritically. But
at least he \textit{had} loved her---had been willing to stake his future
on his faith in her---and if the moment had been fated to pass
from them before they could seize it, he saw now that, for both,
it had been saved whole out of the ruin of their lives.





It was this moment of love, this fleeting victory over
themselves, which had kept them from atrophy and extinction;
which, in her, had reached out to him in every struggle against
the influence of her surroundings, and in him, had kept alive the
faith that now drew him penitent and reconciled to her side.





He knelt by the bed and bent over her, draining their last moment
to its lees; and in the silence there passed between them the
word which made all clear.




\chapter*{\raggedright THE END}

\addcontentsline{toc}{chapter}{THE END}

\markboth{HOUSE OF MIRTH}{THE END}






Notes:\  



1.  I have modernized this text by modernizing the contractions: 
do n't becomes don't, \textit{etc}.





2.  I have retained the British spelling of words like favour and
colour.





3.  I found and corrected one instance of the name ``Gertie,''
which I changed to ``Gerty''\ to be consistent with rest of the
book.





Linda Ruoff






\end{document}
